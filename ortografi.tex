\chapter{Ortografi}

% \includesvg[width=\textwidth]{./fig/fig1}

% \begin{tikzpicture}
  \bfseries
  \small
  \node(01_1)[text depth=0.25ex, text height=2.5ex, text centered] at (0   , 0) {\latupalif};
  \node(02_1)[text depth=0.25ex, text height=2.5ex, text centered] at (5   , 0) {\latupbe};
  \node(03_1)[text depth=0.25ex, text height=2.5ex, text centered] at (10  , 0) {\latuppe};
  \node(04_1)[text depth=0.25ex, text height=2.5ex, text centered] at (15  , 0) {\latupthe};
  \node(05_1)[text depth=0.25ex, text height=2.5ex, text centered] at (20  , 0) {\latupte};
  \node(06_1)[text depth=0.25ex, text height=2.5ex, text centered] at (25  , 0) {\latupcim};
  \node(07_1)[text depth=0.25ex, text height=2.5ex, text centered] at (30  , 0) {\latupchim};
  \node(08_1)[text depth=0.25ex, text height=2.5ex, text centered] at (35  , 0) {\latupha};
  \node(09_1)[text depth=0.25ex, text height=2.5ex, text centered] at (40  , 0) {\latupxa};
  \node(10_1)[text depth=0.25ex, text height=2.5ex, text centered] at (45  , 0) {\latupdal};
  \node(11_1)[text depth=0.25ex, text height=2.5ex, text centered] at (50  , 0) {\latupzel};
  \node(12_1)[text depth=0.25ex, text height=2.5ex, text centered] at (55  , 0) {\latupre};
  \node(13_1)[text depth=0.25ex, text height=2.5ex, text centered] at (60  , 0) {\latupze};
  \node(14_1)[text depth=0.25ex, text height=2.5ex, text centered] at (65  , 0) {\latupje};
  \node(15_1)[text depth=0.25ex, text height=2.5ex, text centered] at (70  , 0) {\latupsin};
  \node(16_1)[text depth=0.25ex, text height=2.5ex, text centered] at (75  , 0) {\latupshin};
  \node(17_1)[text depth=0.25ex, text height=2.5ex, text centered] at (80  , 0) {\latupsad};
  \node(18_1)[text depth=0.25ex, text height=2.5ex, text centered] at (85  , 0) {\latupdad};
  \node(19_1)[text depth=0.25ex, text height=2.5ex, text centered] at (90  , 0) {\latupta};
  \node(20_1)[text depth=0.25ex, text height=2.5ex, text centered] at (95  , 0) {\latupza};
  \node(21_1)[text depth=0.25ex, text height=2.5ex, text centered] at (100 , 0) {\latupayn};
  \node(22_1)[text depth=0.25ex, text height=2.5ex, text centered] at (105 , 0) {\latupgayn};
  \node(23_1)[text depth=0.25ex, text height=2.5ex, text centered] at (110 , 0) {\latupfe};
  \node(24_1)[text depth=0.25ex, text height=2.5ex, text centered] at (115 , 0) {\latupkaf};
  \node(25_1)[text depth=0.25ex, text height=2.5ex, text centered] at (120 , 0) {\latupkef};
  \node(26_1)[text depth=0.25ex, text height=2.5ex, text centered] at (125 , 0) {\latupgef};
  \node(27_1)[text depth=0.25ex, text height=2.5ex, text centered] at (130 , 0) {\latupnef};
  \node(28_1)[text depth=0.25ex, text height=2.5ex, text centered] at (135 , 0) {\latuplam};
  \node(29_1)[text depth=0.25ex, text height=2.5ex, text centered] at (140 , 0) {\latupmim};
  \node(30_1)[text depth=0.25ex, text height=2.5ex, text centered] at (145 , 0) {\latupnun};
  \node(31_1)[text depth=0.25ex, text height=2.5ex, text centered] at (150 , 0) {\latupvav};
  \node(32_1)[text depth=0.25ex, text height=2.5ex, text centered] at (155 , 0) {\latuphe};
  \node(33_1)[text depth=0.25ex, text height=2.5ex, text centered] at (160 , 0) {\latupye};

  \scriptsize
  \node(01_2)[text depth=2.25ex, text height=2ex, text centered] at (0   , 20) {\aralif};
  \node(02_2)[text depth=2.25ex, text height=2ex, text centered] at (5   , 20) {\arbe};
  \node(03_2)[text depth=2.25ex, text height=2ex, text centered] at (10  , 20) {\arpe};
  \node(04_2)[text depth=2.25ex, text height=2ex, text centered] at (15  , 20) {\arthe};
  \node(05_2)[text depth=2.25ex, text height=2ex, text centered] at (20  , 20) {\arte};
  \node(06_2)[text depth=2.25ex, text height=2ex, text centered] at (25  , 20) {\arcim};
  \node(07_2)[text depth=2.25ex, text height=2ex, text centered] at (30  , 20) {\archim};
  \node(08_2)[text depth=2.25ex, text height=2ex, text centered] at (35  , 20) {\arha};
  \node(09_2)[text depth=2.25ex, text height=2ex, text centered] at (40  , 20) {\arxa};
  \node(10_2)[text depth=2.25ex, text height=2ex, text centered] at (45  , 20) {\ardal};
  \node(11_2)[text depth=2.25ex, text height=2ex, text centered] at (50  , 20) {\arzel};
  \node(12_2)[text depth=2.25ex, text height=2ex, text centered] at (55  , 20) {\arre};
  \node(13_2)[text depth=2.25ex, text height=2ex, text centered] at (60  , 20) {\arze};
  \node(14_2)[text depth=2.25ex, text height=2ex, text centered] at (65  , 20) {\arje};
  \node(15_2)[text depth=2.25ex, text height=2ex, text centered] at (70  , 20) {\arsin};
  \node(16_2)[text depth=2.25ex, text height=2ex, text centered] at (75  , 20) {\arshin};
  \node(17_2)[text depth=2.25ex, text height=2ex, text centered] at (80  , 20) {\arsad};
  \node(18_2)[text depth=2.25ex, text height=2ex, text centered] at (85  , 20) {\ardad};
  \node(19_2)[text depth=2.25ex, text height=2ex, text centered] at (90  , 20) {\arta};
  \node(20_2)[text depth=2.25ex, text height=2ex, text centered] at (95  , 20) {\arza};
  \node(21_2)[text depth=2.25ex, text height=2ex, text centered] at (100 , 20) {\arayn};
  \node(22_2)[text depth=2.25ex, text height=2ex, text centered] at (105 , 20) {\argayn};
  \node(23_2)[text depth=2.25ex, text height=2ex, text centered] at (110 , 20) {\arfe};
  \node(24_2)[text depth=2.25ex, text height=2ex, text centered] at (115 , 20) {\arkaf};
  \node(25_2)[text depth=2.25ex, text height=2ex, text centered] at (120 , 20) {\arkef};
  \node(26_2)[text depth=2.25ex, text height=2ex, text centered] at (125 , 20) {\argef};
  \node(27_2)[text depth=2.25ex, text height=2ex, text centered] at (130 , 20) {\arnef};
  \node(28_2)[text depth=2.25ex, text height=2ex, text centered] at (135 , 20) {\arlam};
  \node(29_2)[text depth=2.25ex, text height=2ex, text centered] at (140 , 20) {\armim};
  \node(30_2)[text depth=2.25ex, text height=2ex, text centered] at (145 , 20) {\arnun};
  \node(31_2)[text depth=2.25ex, text height=2ex, text centered] at (150 , 20) {\arvav};
  \node(32_2)[text depth=2.25ex, text height=2ex, text centered] at (155 , 20) {\arhe};
  \node(33_2)[text depth=2.25ex, text height=2ex, text centered] at (160 , 20) {\arye};

  \draw[->](01_2) edge (01_1);


\draw[->] (01_2) edge (01_1);
\draw[->] (02_2) edge (02_1);
\draw[->] (03_2) edge (03_1);
\draw[->] (04_2) edge (04_1);
\draw[->] (05_2) edge (05_1);
\draw[->] (06_2) edge (06_1);
\draw[->] (07_2) edge (07_1);
\draw[->] (08_2) edge (08_1);
\draw[->] (09_2) edge (09_1);
\draw[->] (10_2) edge (10_1);
\draw[->] (11_2) edge (11_1);
\draw[->] (12_2) edge (12_1);
\draw[->] (13_2) edge (13_1);
\draw[->] (14_2) edge (14_1);
\draw[->] (15_2) edge (15_1);
\draw[->] (16_2) edge (16_1);
\draw[->] (17_2) edge (17_1);
\draw[->] (18_2) edge (18_1);
\draw[->] (19_2) edge (19_1);
\draw[->] (20_2) edge (20_1);
\draw[->] (21_2) edge (21_1);
\draw[->] (22_2) edge (22_1);
\draw[->] (23_2) edge (23_1);
\draw[->] (24_2) edge (24_1);
\draw[->] (25_2) edge (25_1);
\draw[->] (27_2) edge (27_1);
\draw[->] (26_2) edge (26_1);
\draw[->] (28_2) edge (28_1);
\draw[->] (29_2) edge (29_1);
\draw[->] (30_2) edge (30_1);
\draw[->] (31_2) edge (31_1);
\draw[->] (32_2) edge (32_1);
\draw[->] (33_2) edge (33_1);



































\end{tikzpicture}

\includegraphics[width=\textwidth]{./fig/fig1.tikz}
\section{Türk Diliyle Uyumlu Harfler}

\section{Türk Diline Harici Harfler}

% \section{Eski Yazı}

% sf/tt/rm: \textsf{x}\texttt{x}x

% sf/tt/rm: \textsf{K}\texttt{K}K \textsf{\latupkaf}\texttt{\latupkaf}\latupkaf

% \section{Vekil Alfabe}

\newpage
\begingroup
% \setlength\extrarowh{5pt}
% \begin{table}[htbp]
  % \centering
\renewcommand{\arraystretch}{2.1}
\begin{longtable*}{>{\LARGE}p{1.4\ltw}>{\LARGE}p{1.4\ltw}>{}p{1.9\ltw}>{}p{2.8\ltw}>{\timesfont}p{2\ltw}}
   \small Vekil \newline harf
  & \small Eski yazı \newline harf
  & \small Eski yazı \newline harf ismi
  & \small Yeni yazı\newline karşılığı
  & \small\normalfont Türkçe \newline telaffuz \\
  %
  % \normalsize Eskiyazı Harf & \normalsize Vekil Harf &\normalsize
  % Yeni yazı Mukabili &\normalsize Eskiyazı okunuşu &\normalsize Türkçe \mbox{Okunuşu} \\
  \toprule
  \latupalif \latdownalif & \aralif                   & elif           & *A veya E          & [a, e, æ]       \\
  \latupayn  \latdownayn  & \raisebox{0.6ex}{\arayn}  & ayn            & *gırtlak ünlüsü    & [--]            \\ % ʕſ Ʒʒ
  \latuphemze             & \arhemze                  & hemze          & *gırtlak kapatması & [--]            \\ % ʕſ Ʒʒ
  \latupbe   \latdownbe   & \arbe                     & be             & B                  & [b]             \\
  \latupcim  \latdowncim  & \raisebox{0.8ex}{\arcim}  & cim            & C                  & [d͡ʒ]            \\
  \latupchim \latdownchim & \raisebox{0.8ex}{\archim} & çim            & Ç                  & [t͡ʃ]            \\
  \latupdal  \latdowndal  & \ardal                    & dal            & D                  & [d]             \\
  \latupdad  \latdowndad  & \raisebox{0.8ex}{\ardad}  & dad            & *diş D             & [d, z]          \\
  \latupfe   \latdownfe   & \arfe                     & fe             & F                  & [f]             \\
  \latupgef  \latdowngef  & \argef                    & gef            & G                  & [ɡ]             \\
  \latupgayn \latdowngayn & \raisebox{0.3ex}{\argayn} & gayn           & genelde Ğ          & [ɡ, ɣ]          \\
  \latuphe   \latdownhe   & \arhe                     & he             & *ince H veya E     & [h, æ]          \\
  \latupha   \latdownha   & \raisebox{1.1ex}{\arha}   & ha             & *kalın H           & [h]             \\
  \latupxa   \latdownxa   & \raisebox{0.7ex}{\arxa}   & hı             & *sürtmeli H        & [h]             \\
  \latupye   \latdownye   & \arye                     & ye             & *I, İ veya Y       & [j, ɯ, i]       \\
  \latupje   \latdownje   & \raisebox{0.3ex}{\arje}   & je             & J                  & [ʒ]             \\
  \latupkef  \latdownkef  & \arkef                    & kef            & *ince K            & [k]             \\
  \latupkaf  \latdownkaf  & \arkaf                    & kaf            & *kalın K           & [k]             \\
  \latuplam  \latdownlam  & \arlam                    & lam            & L                  & [l]             \\
  \latupmim  \latdownmim  & \armim                    & mim            & M                  & [m]             \\
  \latupnun  \latdownnun  & \raisebox{0.4ex}{\arnun}  & nun            & N                  & [n]             \\
  \latupnef  \latdownnef  & \raisebox{-0.7ex}{\arnef} & nef, sağır kef & *geniz N           & [ŋ]             \\
  \latuppe   \latdownpe   & \arpe                     & pe             & P                  & [p]             \\
  \latupre   \latdownre   & \raisebox{0.4ex}{\arre}   & re             & R                  & [ɾ]             \\
  \latupsin  \latdownsin  & \raisebox{0.6ex}{\arsin}  & sin            & *ince S            & [s]             \\
  \latupsad  \latdownsad  & \raisebox{0.6ex}{\arsad}  & sad            & *kalın S           & [s]             \\
  \latupthe  \latdownthe  & \arthe                    & peltek s       & *peltek S          & [s]             \\
  \latupshin \latdownshin & \raisebox{0.4ex}{\arshin} & şın            & Ş                  & [ʃ]             \\
  \latupte   \latdownte   & \arte                     & te             & *ince T            & [t]             \\
  \latupta   \latdownta   & \arta                     & tı             & *kalın T           & [t]             \\
  \latupvav  \latdownvav  & \raisebox{0.6ex}{\arvav}  & vav            & *O,Ö,U,Ü veya V    & \small[o, œ, u, y, v] \\
  \latupze   \latdownze   & \raisebox{0.6ex}{\arze}   & ze             & Z                  & [z]             \\
  \latupza   \latdownza   & \arza                     & zı             & *kalın Z           & [z]             \\
  \latupzel  \latdownzel  & \raisebox{0.2ex}{\arzel}  & zel            & *diş Z             & [z]             \\
  \bottomrule
\end{longtable*}
\vspace{-6ex}
\centering
\begin{table}[H]
  \caption{Vekil alfabe. Vekil harfler, yeni yazıda karşılık geldikleri yerlere göre
    sıralanmış olup, rahat karşılaştırma için eski harf isimleri ve konuşmadaki
    telaffuzları ile birlikte verilmiştir. Yeni yazıda birebir karşılığı
    bulunmayan eski harfler yıldız (*) ile işaretlenmiştir.
    Telaffuzda UFA notasyonu kullanılmıştır.}
  \label{tab:vekil1}
\end{table}
\endgroup


