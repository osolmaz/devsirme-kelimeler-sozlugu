
\chapter{Türetme Kuralları}

% \section{Transfiks Morfoloji}
\section{Eklemesiz Morfoloji}

\begin{figure}[htbp]
  \centering
  \includesvg[width=0.6\textwidth]{./fig/fig2}
  \caption{Arapça'da kelime çekimi örneği.
    \fmtkok{\Lnun\Lvav\Lre} kökünü
    \declen{müKeMMeL} kipinde çekmek için \textbf{1.}\ masdar kökünü oluşturan \masdarkok
    seslerini hedef kipten ayırın, \textbf{2.}\ çekilmek istenen kökün seslerini
    kipte karşılık geldikleri yerlere yerleştirin.}
  \label{fig:fig2}
\end{figure}


% \section{Ad Kipleri}

\begin{kip}{\declenlarge{KeML}}{Ad}
\end{kip}

\begin{kip}{\declenlarge{KiML}}{Ad}
\end{kip}

\begin{kip}{\declenlarge{KüML}}{Ad}
\end{kip}

\begin{kip}{\declenlarge{KeMeL}}{Ad}
\end{kip}

\begin{kip}{\declenlarge{KeMLeL}}{Ad}
\end{kip}

\begin{kip}{\declenlarge{KüMLüL}}{Ad}
\end{kip}





\section{Çokluk Ekleri ve Kipleri}

\subsection*{İkilik}
\begin{kip}{\declenlarge{-eyn}}{İkilik}
\end{kip}

\subsection*{Eklemeli Çokluklar}
\begin{kip}{\declenlarge{-în, -ûn}}{Kurallı Eril Çokluk}
\end{kip}

\begin{kip}{\declenlarge{-ât}}{Kurallı Dişil Çokluk}
\end{kip}

\subsection*{Eklemesiz Çokluklar}
\begin{kip}{\declenlarge{eKMâL}}{Kuralsız Çokluk}
\end{kip}

\begin{kip}{\declenlarge{KüMûL}}{Kuralsız Çokluk}
\end{kip}

\begin{kip}{\declenlarge{KüMüL}}{Kuralsız Çokluk}
\end{kip}

\begin{kip}{\declenlarge{KüMeL}}{Kuralsız Çokluk}
\end{kip}

\begin{kip}{\declenlarge{KiMeL}}{Kuralsız Çokluk}
\end{kip}

\begin{kip}{\declenlarge{KiMâL}}{Kuralsız Çokluk}
\end{kip}

\begin{kip}{\declenlarge{KiMMâL}}{Kuralsız Çokluk}
\end{kip}

\begin{kip}{\declenlarge{KeMeLe}}{Kuralsız Çokluk}
\end{kip}

\begin{kip}{\declenlarge{KüMeLâ}}{Kuralsız Çokluk}
\end{kip}

\begin{kip}{\declenlarge{KeMâ'iL}}{Kuralsız Çokluk}
\end{kip}

\begin{kip}{\declenlarge{KevâMiL}}{Kuralsız Çokluk}
\end{kip}

\begin{kip}{\declenlarge{eKâMiL}}{Kuralsız Çokluk}
\end{kip}

\begin{kip}{\declenlarge{eKâMîL}}{Kuralsız Çokluk}
\end{kip}

\begin{kip}{\declenlarge{eKMiLâ}}{Kuralsız Çokluk}
\end{kip}

\begin{kip}{\declenlarge{eKMiLe}}{Kuralsız Çokluk}
\end{kip}

\begin{kip}{\declenlarge{meKâMiL}}{Kuralsız Çokluk}
\end{kip}

\begin{kip}{\declenlarge{meKâMîL}}{Kuralsız Çokluk}
\end{kip}

\begin{kip}{\declenlarge{teKâMîL}}{Kuralsız Çokluk}
\end{kip}

\begin{kip}{\declenlarge{eKMüL}}{Kuralsız Çokluk}
\end{kip}












\section{Eylem Kipleri}

% causative x anticausative: ettirgen x ettirilgen
% transitive x intransitive: geçişli, geçişsiz
% reflexive: dönüşlü
% intensive: güçlü
% stative: durum eylemi


% \subsection*{Kip \rom{1}}
% \subsubsection*{Eylem: \declen{NeCZ}, \declen{NiCZ}, \declen{NüCZ}, vs.}
% \subsubsection*{Etken: \declen{NâCiZ}}
% \subsubsection*{Edilgen: \declen{meNCûZ}}

% % \subsection{İncaz Kipi: Geçişli (Transitif) Eylem}

% \subsection*{Kip \rom{2}}

% \subsubsection*{Eylem: \declen{teNCîZ}}
% \subsubsection*{Etken: \declen{müNeCCiZ}}
% \subsubsection*{Edilgen: \declen{müNeCCeZ}}

\subsection*{Kip \rom{1}}


\begin{kip}{\declenlarge{KeML}, \declenlarge{KiML}, \declenlarge{KüML},vs.}
  {Kip \rom{1}, Eylem Adı}
  Eylem ismi
\end{kip}

\begin{kip}{\declenlarge{KâMiL}}{Kip \rom{1}, Etken Partisip}
  Etken eylem ismi
\end{kip}

\begin{kip}{\declenlarge{meKMûL}}{Kip \rom{1}, Edilgen Partisip}
  Edilgen eylem ismi
\end{kip}


% \subsection{İncaz Kipi: Geçişli (Transitif) Eylem}

\subsection*{Kip \rom{2}}

\begin{kip}{\declenlarge{teKMîL}}{Kip \rom{2}, Eylem Adı}
  Eylem ismi
\end{kip}

\begin{kip}{\declenlarge{müKeMMiL}}{Kip \rom{2}, Etken Partisip}
  Etken eylem ismi
\end{kip}

\begin{kip}{\declenlarge{müKeMMeL}}{Kip \rom{2}, Edilgen Partisip}
  Edilgen eylem ismi
\end{kip}

\subsection*{Kip \rom{3}}

\begin{kip}{\declenlarge{müKâMeLe}}{Kip \rom{3}, Eylem Adı}
  Eylem ismi
\end{kip}

\begin{kip}{\declenlarge{müKâMiL}}{Kip \rom{3}, Etken Partisip}
  Etken eylem ismi
\end{kip}

\begin{kip}{\declenlarge{müKâMeL}}{Kip \rom{3}, Edilgen Partisip}
  Edilgen eylem ismi
\end{kip}


\subsection*{Kip \rom{4}}

\begin{kip}{\declenlarge{iKMâL}}{Kip \rom{4}, Eylem Adı}
  Eylem ismi
\end{kip}

\begin{kip}{\declenlarge{müKMiL}}{Kip \rom{4}, Etken Partisip}
  Etken eylem ismi
\end{kip}

\begin{kip}{\declenlarge{müKMeL}}{Kip \rom{4}, Edilgen Partisip}
  Edilgen eylem ismi
\end{kip}


\subsection*{Kip \rom{5}}

\begin{kip}{\declenlarge{teKeMMüL}}{Kip \rom{5}, Eylem Adı}
  Eylem ismi
\end{kip}

\begin{kip}{\declenlarge{müteKeMMiL}}{Kip \rom{5}, Etken Partisip}
  Etken eylem ismi
\end{kip}

\begin{kip}{\declenlarge{müteKeMMeL}}{Kip \rom{5}, Edilgen Partisip}
  Edilgen eylem ismi
\end{kip}


\subsection*{Kip \rom{6}}

\begin{kip}{\declenlarge{teKâMüL}}{Kip \rom{6}, Eylem Adı}
  Eylem ismi
\end{kip}

\begin{kip}{\declenlarge{müteKâMiL}}{Kip \rom{6}, Etken Partisip}
  Etken eylem ismi
\end{kip}

\begin{kip}{\declenlarge{müteKâMeL}}{Kip \rom{6}, Edilgen Partisip}
  Edilgen eylem ismi
\end{kip}


\subsection*{Kip \rom{7}}

\begin{kip}{\declenlarge{inKiMâL}}{Kip \rom{7}, Eylem Adı}
  Eylem ismi
\end{kip}

\begin{kip}{\declenlarge{münKeMiL}}{Kip \rom{7}, Etken Partisip}
  Etken eylem ismi
\end{kip}

\begin{kip}{\declenlarge{münKeMeL}}{Kip \rom{7}, Edilgen Partisip}
  Edilgen eylem ismi
\end{kip}

\subsection*{Kip \rom{8}}

\begin{kip}{\declenlarge{iKtimâL}}{Kip \rom{8}, Eylem Adı}
  Eylem ismi
\end{kip}

\begin{kip}{\declenlarge{müKteMiL}}{Kip \rom{8}, Etken Partisip}
  Etken eylem ismi
\end{kip}

\begin{kip}{\declenlarge{müKteMeL}}{Kip \rom{8}, Edilgen Partisip}
  Edilgen eylem ismi
\end{kip}


\subsection*{Kip \rom{9}}

\begin{kip}{\declenlarge{iKMiLâL}}{Kip \rom{9}, Eylem Adı}
  Eylem ismi
\end{kip}

\begin{kip}{\declenlarge{müKMeLL}}{Kip \rom{9}, Etken Partisip}
  Etken eylem ismi
\end{kip}


\subsection*{Kip \rom{10}}

\begin{kip}{\declenlarge{istiKMâL}}{Kip \rom{10}, Eylem Adı}
  Eylem ismi
\end{kip}

\begin{kip}{\declenlarge{müsteKMiL}}{Kip \rom{10}, Etken Partisip}
  Etken eylem ismi
\end{kip}

\begin{kip}{\declenlarge{müsteKMeL}}{Kip \rom{10}, Edilgen Partisip}
  Edilgen eylem ismi
\end{kip}


% \section{Muhtelif Kipler ve Ekler}

\section{Nitelik Sıfatı Kipleri}

\section{Karşılaştırma Sıfatı Kipi}

\section{Abartma Sıfatı Kipleri}

\section{İlişki Sıfatı Eki}

\section{Yer Adı Kipleri}

\section{Zaman Adı Kipleri}

\section{Alet Adı Kipleri}

\section{Küçültme Kipi}










\begin{table}[htbp]
  \centering
  \renewcommand{\arraystretch}{1.5}
  \newlength{\tabwidth}
  \setlength{\tabwidth}{12cm}
  \begin{tabular}{p{0.1\tabwidth} >{\raggedright}p{0.3\tabwidth} >{\raggedright}p{0.2\tabwidth} >{\raggedright}p{0.2\tabwidth} p{0.2\tabwidth}}
    Kip & Anlam & Eylem & Etken & Edilgen \\
    \toprule
    \rom{1} & Sade & Keml, Kiml, Küml, vs. &  Kâmil & Mekmûl \\
    % \rom{1} & Sade & Necz, Nücûz, Nicz, Nücz(et), Necâz(et), Nicâz(et), vs. &  Nâciz & Mencuz \\
    \rom{2} & Geçişli, ettirgen, güçlü & Tekmîl &  Mükemmil & Mükemmel \\
    \rom{3} & İşteş & Mükâmele &  Mükâmil & Mükâmel \\
    \rom{4} & Geçişli, ettirgen & İkmâl &  Mükmil & Mükmel \\
    % \rom{5} & Dönüşlü, ettirilgen, güçlü
    \rom{5} & \rom{2}'nin dönüşlüsü, genellikle geçişsiz
                & Tekemmül &  Mütekemmil & Mütekemmel \\
    % \rom{6} & \rom{2}'nin dönüşlüsü, genellikle geçişsiz
    \rom{6} & \rom{3}'deki işteş eylemin muhattabı, dönüşlüsü, genellikle geçişsiz
                & Tekâmül &  Mütekâmil & Mütekâmel \\
    \rom{7} & Dönüşlü, ettirilgen & İnkimâl & Münkemil & Münkemel \\
    \rom{8} & \rom{1}'in dönüşlüsü, geçişsiz & İktimâl & Müktemil & Müktemel \\
    \rom{9} & Durum eylemi, geçişsiz & İkmilâl & Mükmell & --- \\
    \rom{10} & Ettirgen, bazen ettirilgen, vs.  & İstikmâl & Müstekmil & Müstekmel \\
    % \midrule
    % \rom{11} & \rom{9}'un aynısı, şiir dışında nadir & İncîzaz & Müncâzz & --- \\
    % \rom{12} & \multirow{4}{*}{\parbox{0.3\tabwidth}{\raggedright Durum eylemi, çok nadir}} & İncîcaz & Müncavciz & Müncavcez \\
    % \rom{13} &  & İncivvaz & Müncavviz & Müncavvez \\
    % \rom{14} &  & İncinzâz & Müncanziz & Müncanzez \\
    % \rom{15} &  & İncinzâ' & Müncanzin & Müncanzen \\
    \bottomrule
  \end{tabular}
\end{table}



