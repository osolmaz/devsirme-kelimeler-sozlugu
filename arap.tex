
\chapter{Arap Köklü Kelimelerin Kuralları}
\section{Bitişimsiz Morfoloji}

\section{Eylem İsmi Kipleri}

\begin{table}[htbp]
  \footnotesize
  \centering
  \renewcommand{\arraystretch}{1.5}
  \begin{tabular}{p{0.1\tabwidth} >{\raggedright}p{0.3\tabwidth} >{\raggedright}p{0.2\tabwidth} >{\raggedright}p{0.2\tabwidth} p{0.2\tabwidth}}
    Kip No & Anlam & Eylem & Eden & Edilen \\
    \toprule
    \rom{1} & Sade & Necz, Nücûz, Nicz, Nücz(et), Necâz(et), Nicâz(et), vs. &  Nâciz & Mencuz \\
    \rom{2} & Geçişli, ettiren, güçlü & Tenciz &  Münecciz & Müneccez \\
    \rom{3} & İşteş & Münâceze &  Münâciz & Münâcez \\
    \rom{4} & Geçişli, ettiren & İncâz &  Münciz & Müncez \\
    \rom{5} & Dönüşlü, ettiren, güçlü & Teneccüz &  Mütenecciz & Müteneccez \\
    \rom{6} & \rom{3}'ün işteş karşılığı & Tenâcüz &  Mütenâciz & Mütenâcez \\
    \rom{7} & Dönüşlü, ettirilen & İnnicâz & Münneciz & Münnecez \\
    \rom{8} & \rom{1}'in dönüşlüsü, geçişsiz & İnticâz & Münteciz & Müntecez \\
    \rom{9} & Durum eylemi, geçişsiz & İncizâz & Müncezz & --- \\
    \rom{10} & Ettiren, bazen ettirilen, çeşitli anlamlar  & İstincâz & Müstenciz & Müstencez \\
    % \midrule
    % \rom{11} & \rom{9}'un aynısı, şiir dışında nadir & İncîzaz & Müncâzz & --- \\
    % \rom{12} & \multirow{4}{*}{\parbox{0.3\tabwidth}{\raggedright Durum eylemi, çok nadir}} & İncîcaz & Müncavciz & Müncavcez \\
    % \rom{13} &  & İncivvaz & Müncavviz & Müncavvez \\
    % \rom{14} &  & İncinzâz & Müncanziz & Müncanzez \\
    % \rom{15} &  & İncinzâ' & Müncanzin & Müncanzen \\
    \bottomrule
  \end{tabular}
\end{table}

\subsection{Nâciz Kipi: Yapan}

\subsection{Mencuz Kipi: Yapılan}

\subsection{İncaz Kipi: Geçişli (Transitif) Eylem}

