
\chapter{Türetme Kuralları}

% \section{Transfiks Morfoloji}
\section{Eklemesiz Morfoloji}

\begin{figure}[htbp]
  \centering
  \includesvg[width=0.6\textwidth]{./fig/fig2}
  \caption{Arapça'da kelime çekimi örneği.
    \fmtkok{\Lnun\Lvav\Lre} kökünü
    \declen{müKeMMeL} kipinde çekmek için \textbf{1.}\ masdar kökünü oluşturan \masdarkok
    seslerini hedef kipten ayırın, \textbf{2.}\ çekilmek istenen kökün seslerini
    kipte karşılık geldikleri yerlere yerleştirin.}
  \label{fig:fig2}
\end{figure}


% \input{vezinler.tex}



\begin{table}[htbp]
  \centering
  \renewcommand{\arraystretch}{1.5}
  \newlength{\tabwidth}
  \setlength{\tabwidth}{12cm}
  \begin{tabular}{p{0.1\tabwidth} >{\raggedright}p{0.3\tabwidth} >{\raggedright}p{0.2\tabwidth} >{\raggedright}p{0.2\tabwidth} p{0.2\tabwidth}}
    Kip & Anlam & Eylem & Etken & Edilgen \\
    \toprule
    \rom{1} & Sade & Keml, Kiml, Küml, vs. &  Kâmil & Mekmûl \\
    % \rom{1} & Sade & Necz, Nücûz, Nicz, Nücz(et), Necâz(et), Nicâz(et), vs. &  Nâciz & Mencuz \\
    \rom{2} & Geçişli, ettirgen, güçlü & Tekmîl &  Mükemmil & Mükemmel \\
    \rom{3} & İşteş & Mükâmele &  Mükâmil & Mükâmel \\
    \rom{4} & Geçişli, ettirgen & İkmâl &  Mükmil & Mükmel \\
    % \rom{5} & Dönüşlü, ettirilgen, güçlü
    \rom{5} & \rom{2}'nin dönüşlüsü, genellikle geçişsiz
                & Tekemmül &  Mütekemmil & Mütekemmel \\
    % \rom{6} & \rom{2}'nin dönüşlüsü, genellikle geçişsiz
    \rom{6} & \rom{3}'deki işteş eylemin muhattabı, dönüşlüsü, genellikle geçişsiz
                & Tekâmül &  Mütekâmil & Mütekâmel \\
    \rom{7} & Dönüşlü, ettirilgen & İnkimâl & Münkemil & Münkemel \\
    \rom{8} & \rom{1}'in dönüşlüsü, geçişsiz & İktimâl & Müktemil & Müktemel \\
    \rom{9} & Durum eylemi, geçişsiz & İkmilâl & Mükmell & --- \\
    \rom{10} & Ettirgen, bazen ettirilgen, vs.  & İstikmâl & Müstekmil & Müstekmel \\
    % \midrule
    % \rom{11} & \rom{9}'un aynısı, şiir dışında nadir & İncîzaz & Müncâzz & --- \\
    % \rom{12} & \multirow{4}{*}{\parbox{0.3\tabwidth}{\raggedright Durum eylemi, çok nadir}} & İncîcaz & Müncavciz & Müncavcez \\
    % \rom{13} &  & İncivvaz & Müncavviz & Müncavvez \\
    % \rom{14} &  & İncinzâz & Müncanziz & Müncanzez \\
    % \rom{15} &  & İncinzâ' & Müncanzin & Müncanzen \\
    \bottomrule
  \end{tabular}
\end{table}



