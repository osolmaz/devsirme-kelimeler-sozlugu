
\chapter{Arap Köklü Kelimelerin Kuralları}
\section{Bitişimsiz Morfoloji}

\section{Çoğullar}

\section{Eylem İsmi Kipleri}

% \begin{table}[htbp]
%   \footnotesize
%   \centering
%   \renewcommand{\arraystretch}{1.5}
%   \begin{tabular}{p{0.1\tabwidth} >{\raggedright}p{0.3\tabwidth} >{\raggedright}p{0.2\tabwidth} >{\raggedright}p{0.2\tabwidth} p{0.2\tabwidth}}
%     Kip & Anlam & Eylem & Etken & Edilgen \\
%     \toprule
%     \rom{1} & Sade & Necz, Nicz, Nücz, vs. &  Nâciz & Mencûz \\
%     % \rom{1} & Sade & Necz, Nücûz, Nicz, Nücz(et), Necâz(et), Nicâz(et), vs. &  Nâciz & Mencuz \\
%     \rom{2} & Geçişli, ettirilgen, güçlü & Tencîz &  Münecciz & Müneccez \\
%     \rom{3} & İşteş & Münâceze &  Münâciz & Münâcez \\
%     \rom{4} & Geçişli, ettirilgen & İncâz &  Münciz & Müncez \\
%     \rom{5} & Dönüşlü, ettirilgen, güçlü & Teneccüz &  Mütenecciz & Müteneccez \\
%     \rom{6} & \rom{3}'ün işteş karşılığı & Tenâcüz &  Mütenâciz & Mütenâcez \\
%     \rom{7} & Dönüşlü, ettirilgen & İnnicâz & Münneciz & Münnecez \\
%     \rom{8} & \rom{1}'in dönüşlüsü, geçişsiz & İnticâz & Münteciz & Müntecez \\
%     \rom{9} & Durum eylemi, geçişsiz & İncizâz & Müncezz & --- \\
%     \rom{10} & Ettirgen, bazen ettirilgen, vs.  & İstincâz & Müstenciz & Müstencez \\
%     % \midrule
%     % \rom{11} & \rom{9}'un aynısı, şiir dışında nadir & İncîzaz & Müncâzz & --- \\
%     % \rom{12} & \multirow{4}{*}{\parbox{0.3\tabwidth}{\raggedright Durum eylemi, çok nadir}} & İncîcaz & Müncavciz & Müncavcez \\
%     % \rom{13} &  & İncivvaz & Müncavviz & Müncavvez \\
%     % \rom{14} &  & İncinzâz & Müncanziz & Müncanzez \\
%     % \rom{15} &  & İncinzâ' & Müncanzin & Müncanzen \\
%     \bottomrule
%   \end{tabular}
% \end{table}


\begin{table}[htbp]
  \footnotesize
  \centering
  \renewcommand{\arraystretch}{1.5}
  \begin{tabular}{p{0.1\tabwidth} >{\raggedright}p{0.3\tabwidth} >{\raggedright}p{0.2\tabwidth} >{\raggedright}p{0.2\tabwidth} p{0.2\tabwidth}}
    Kip & Anlam & Eylem & Etken & Edilgen \\
    \toprule
    \rom{1} & Sade & Keml, Kiml, Küml, vs. &  Kâmil & Mekmûl \\
    % \rom{1} & Sade & Necz, Nücûz, Nicz, Nücz(et), Necâz(et), Nicâz(et), vs. &  Nâciz & Mencuz \\
    \rom{2} & Geçişli, ettirilgen, güçlü & Tekmîl &  Mükemmil & Mükemmel \\
    \rom{3} & İşteş & Mükâmele &  Mükâmil & Mükâmel \\
    \rom{4} & Geçişli, ettirilgen & İkmâl &  Mükmil & Mükmel \\
    \rom{5} & Dönüşlü, ettirilgen, güçlü & Tekemmül &  Mütekemmil & Mütekemmel \\
    \rom{6} & \rom{3}'ün işteş karşılığı & Tekâmül &  Mütekâmil & Mütekâmel \\
    \rom{7} & Dönüşlü, ettirilgen & İnkimâl & Münkemil & Münkemel \\
    \rom{8} & \rom{1}'in dönüşlüsü, geçişsiz & İktimâl & Müktemil & Müktemel \\
    \rom{9} & Durum eylemi, geçişsiz & İkmilâl & Mükmell & --- \\
    \rom{10} & Ettirgen, bazen ettirilgen, vs.  & İstikmâl & Müstekmil & Müstekmel \\
    % \midrule
    % \rom{11} & \rom{9}'un aynısı, şiir dışında nadir & İncîzaz & Müncâzz & --- \\
    % \rom{12} & \multirow{4}{*}{\parbox{0.3\tabwidth}{\raggedright Durum eylemi, çok nadir}} & İncîcaz & Müncavciz & Müncavcez \\
    % \rom{13} &  & İncivvaz & Müncavviz & Müncavvez \\
    % \rom{14} &  & İncinzâz & Müncanziz & Müncanzez \\
    % \rom{15} &  & İncinzâ' & Müncanzin & Müncanzen \\
    \bottomrule
  \end{tabular}
\end{table}


% \subsection*{Kip \rom{1}}
% \subsubsection*{Eylem: \declen{NeCZ}, \declen{NiCZ}, \declen{NüCZ}, vs.}
% \subsubsection*{Etken: \declen{NâCiZ}}
% \subsubsection*{Edilgen: \declen{meNCûZ}}

% % \subsection{İncaz Kipi: Geçişli (Transitif) Eylem}

% \subsection*{Kip \rom{2}}

% \subsubsection*{Eylem: \declen{teNCîZ}}
% \subsubsection*{Etken: \declen{müNeCCiZ}}
% \subsubsection*{Edilgen: \declen{müNeCCeZ}}

\subsection*{Kip \rom{1}}
\eylem{\declen{KeML}, \declen{KiML}, \declen{KüML}, vs.}
\etken{\declen{KâMiL}}
\edilgen{\declen{meKMûL}}

% \subsection{İncaz Kipi: Geçişli (Transitif) Eylem}

\subsection*{Kip \rom{2}}

\eylem{\declen{teKMîL}}
\etken{\declen{müKeMMiL}}
\edilgen{\declen{müKeMMeL}}


\subsection*{Kip \rom{3}}

\eylem{\declen{müKâMeLe}}
\etken{\declen{müKâMiL}}
\edilgen{\declen{müKâMeL}}

\subsection*{Kip \rom{4}}

\eylem{\declen{iKMâL}}
\etken{\declen{müKMiL}}
\edilgen{\declen{müKMeL}}

\subsection*{Kip \rom{5}}

\eylem{\declen{teKeMMüL}}
\etken{\declen{müteKeMMiL}}
\edilgen{\declen{müteKeMMeL}}

\subsection*{Kip \rom{6}}

\eylem{\declen{teKâMüL}}
\etken{\declen{müteKâMiL}}
\edilgen{\declen{müteKâMeL}}

\subsection*{Kip \rom{7}}

\eylem{\declen{inKiMâL}}
\etken{\declen{münKeMiL}}
\edilgen{\declen{münKeMeL}}

\subsection*{Kip \rom{8}}

\eylem{\declen{iKtimâL}}
\etken{\declen{müKteMiL}}
\edilgen{\declen{müKteMeL}}

\subsection*{Kip \rom{9}}

\eylem{\declen{iKMiLâL}}
\etken{\declen{müKMeLL}}
% \edilgen{\declen{}}

\subsection*{Kip \rom{10}}

\eylem{\declen{istiKMâL}}
\etken{\declen{müsteKMiL}}
\edilgen{\declen{müsteKMeL}}


















