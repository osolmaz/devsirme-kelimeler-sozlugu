
\chapter{Arap Köklü Kelimelerin Kuralları}
\section{Eklemesiz Morfoloji}

\section{Çoğullar}

\section{Eylem İsmi Kipleri}

% causative x anticausative: ettirgen x ettirilgen
% transitive x intransitive: geçişli, geçişsiz
% reflexive: dönüşlü
% intensive: güçlü
% stative: durum eylemi


% \subsection*{Kip \rom{1}}
% \subsubsection*{Eylem: \declen{NeCZ}, \declen{NiCZ}, \declen{NüCZ}, vs.}
% \subsubsection*{Etken: \declen{NâCiZ}}
% \subsubsection*{Edilgen: \declen{meNCûZ}}

% % \subsection{İncaz Kipi: Geçişli (Transitif) Eylem}

% \subsection*{Kip \rom{2}}

% \subsubsection*{Eylem: \declen{teNCîZ}}
% \subsubsection*{Etken: \declen{müNeCCiZ}}
% \subsubsection*{Edilgen: \declen{müNeCCeZ}}

\subsection*{Kip \rom{1}}


\begin{eylemkipi}{Eylem Kipi \rom{1}: \declenlarge{KeML}, \declenlarge{KiML}, \declenlarge{KüML}, vs.}
  Eylem ismi
\end{eylemkipi}

\begin{eylemkipi}{Eylem Kipi \rom{1}, Etkeni: \declenlarge{KâMiL}}
  Etken eylem ismi
\end{eylemkipi}

\begin{eylemkipi}{Eylem Kipi \rom{1}, Edilgeni: \declenlarge{meKMûL}}
  Edilgen eylem ismi
\end{eylemkipi}


% \subsection{İncaz Kipi: Geçişli (Transitif) Eylem}

\subsection*{Kip \rom{2}}

\begin{eylemkipi}{Eylem Kipi \rom{2}: \declenlarge{teKMîL}}
  Eylem ismi
\end{eylemkipi}

\begin{eylemkipi}{Eylem Kipi \rom{2}, Etkeni: \declenlarge{müKeMMiL}}
  Etken eylem ismi
\end{eylemkipi}

\begin{eylemkipi}{Eylem Kipi \rom{2}, Edilgeni: \declenlarge{müKeMMeL}}
  Edilgen eylem ismi
\end{eylemkipi}


\subsection*{Kip \rom{3}}

\begin{eylemkipi}{Eylem Kipi \rom{3}: \declenlarge{müKâMeLe}}
  Eylem ismi
\end{eylemkipi}

\begin{eylemkipi}{Eylem Kipi \rom{3}, Etkeni: \declenlarge{müKâMiL}}
  Etken eylem ismi
\end{eylemkipi}

\begin{eylemkipi}{Eylem Kipi \rom{3}, Edilgeni: \declenlarge{müKâMeL}}
  Edilgen eylem ismi
\end{eylemkipi}


\subsection*{Kip \rom{4}}

\begin{eylemkipi}{Eylem Kipi \rom{4}: \declenlarge{iKMâL}}
  Eylem ismi
\end{eylemkipi}

\begin{eylemkipi}{Eylem Kipi \rom{4}, Etkeni: \declenlarge{müKMiL}}
  Etken eylem ismi
\end{eylemkipi}

\begin{eylemkipi}{Eylem Kipi \rom{4}, Edilgeni: \declenlarge{müKMeL}}
  Edilgen eylem ismi
\end{eylemkipi}


\subsection*{Kip \rom{5}}

\begin{eylemkipi}{Eylem Kipi \rom{5}: \declenlarge{teKeMMüL}}
  Eylem ismi
\end{eylemkipi}

\begin{eylemkipi}{Eylem Kipi \rom{5}, Etkeni: \declenlarge{müteKeMMiL}}
  Etken eylem ismi
\end{eylemkipi}

\begin{eylemkipi}{Eylem Kipi \rom{5}, Edilgeni: \declenlarge{müteKeMMeL}}
  Edilgen eylem ismi
\end{eylemkipi}


\subsection*{Kip \rom{6}}

\begin{eylemkipi}{Eylem Kipi \rom{6}: \declenlarge{teKâMüL}}
  Eylem ismi
\end{eylemkipi}

\begin{eylemkipi}{Eylem Kipi \rom{6}, Etkeni: \declenlarge{müteKâMiL}}
  Etken eylem ismi
\end{eylemkipi}

\begin{eylemkipi}{Eylem Kipi \rom{6}, Edilgeni: \declenlarge{müteKâMeL}}
  Edilgen eylem ismi
\end{eylemkipi}


\subsection*{Kip \rom{7}}

\begin{eylemkipi}{Eylem Kipi \rom{7}: \declenlarge{inKiMâL}}
  Eylem ismi
\end{eylemkipi}

\begin{eylemkipi}{Eylem Kipi \rom{7}, Etkeni: \declenlarge{münKeMiL}}
  Etken eylem ismi
\end{eylemkipi}

\begin{eylemkipi}{Eylem Kipi \rom{7}, Edilgeni: \declenlarge{münKeMeL}}
  Edilgen eylem ismi
\end{eylemkipi}

\subsection*{Kip \rom{8}}

\begin{eylemkipi}{Eylem Kipi \rom{8}: \declenlarge{iKtimâL}}
  Eylem ismi
\end{eylemkipi}

\begin{eylemkipi}{Eylem Kipi \rom{8}, Etkeni: \declenlarge{müKteMiL}}
  Etken eylem ismi
\end{eylemkipi}

\begin{eylemkipi}{Eylem Kipi \rom{8}, Edilgeni: \declenlarge{müKteMeL}}
  Edilgen eylem ismi
\end{eylemkipi}


\subsection*{Kip \rom{9}}

\begin{eylemkipi}{Eylem Kipi \rom{9}: \declenlarge{iKMiLâL}}
  Eylem ismi
\end{eylemkipi}

\begin{eylemkipi}{Eylem Kipi \rom{9}, Etkeni: \declenlarge{müKMeLL}}
  Etken eylem ismi
\end{eylemkipi}


\subsection*{Kip \rom{10}}

\begin{eylemkipi}{Eylem Kipi \rom{10}: \declenlarge{istiKMâL}}
  Eylem ismi
\end{eylemkipi}

\begin{eylemkipi}{Eylem Kipi \rom{10}, Etkeni: \declenlarge{müsteKMiL}}
  Etken eylem ismi
\end{eylemkipi}

\begin{eylemkipi}{Eylem Kipi \rom{10}, Edilgeni: \declenlarge{müsteKMeL}}
  Edilgen eylem ismi
\end{eylemkipi}





\begin{table}[htbp]
  \footnotesize
  \centering
  \renewcommand{\arraystretch}{1.5}
  \begin{tabular}{p{0.1\tabwidth} >{\raggedright}p{0.3\tabwidth} >{\raggedright}p{0.2\tabwidth} >{\raggedright}p{0.2\tabwidth} p{0.2\tabwidth}}
    Kip & Anlam & Eylem & Etken & Edilgen \\
    \toprule
    \rom{1} & Sade & Keml, Kiml, Küml, vs. &  Kâmil & Mekmûl \\
    % \rom{1} & Sade & Necz, Nücûz, Nicz, Nücz(et), Necâz(et), Nicâz(et), vs. &  Nâciz & Mencuz \\
    \rom{2} & Geçişli, ettirgen, güçlü & Tekmîl &  Mükemmil & Mükemmel \\
    \rom{3} & İşteş & Mükâmele &  Mükâmil & Mükâmel \\
    \rom{4} & Geçişli, ettirgen & İkmâl &  Mükmil & Mükmel \\
    % \rom{5} & Dönüşlü, ettirilgen, güçlü
    \rom{5} & \rom{2}'nin dönüşlüsü, genellikle geçişsiz
                & Tekemmül &  Mütekemmil & Mütekemmel \\
    % \rom{6} & \rom{2}'nin dönüşlüsü, genellikle geçişsiz
    \rom{6} & \rom{3}'deki işteş eylemin muhattabı, dönüşlüsü, genellikle geçişsiz
                & Tekâmül &  Mütekâmil & Mütekâmel \\
    \rom{7} & Dönüşlü, ettirilgen & İnkimâl & Münkemil & Münkemel \\
    \rom{8} & \rom{1}'in dönüşlüsü, geçişsiz & İktimâl & Müktemil & Müktemel \\
    \rom{9} & Durum eylemi, geçişsiz & İkmilâl & Mükmell & --- \\
    \rom{10} & Ettirgen, bazen ettirilgen, vs.  & İstikmâl & Müstekmil & Müstekmel \\
    % \midrule
    % \rom{11} & \rom{9}'un aynısı, şiir dışında nadir & İncîzaz & Müncâzz & --- \\
    % \rom{12} & \multirow{4}{*}{\parbox{0.3\tabwidth}{\raggedright Durum eylemi, çok nadir}} & İncîcaz & Müncavciz & Müncavcez \\
    % \rom{13} &  & İncivvaz & Müncavviz & Müncavvez \\
    % \rom{14} &  & İncinzâz & Müncanziz & Müncanzez \\
    % \rom{15} &  & İncinzâ' & Müncanzin & Müncanzen \\
    \bottomrule
  \end{tabular}
\end{table}



\section{Muhtelif Kipler}













