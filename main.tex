\documentclass[a5paper,10pt, twoside]{scrbook}
% \usepackage[turkish]{babel}
\newlength{\tabwidth}
\setlength{\tabwidth}{8.7cm}

% all the local settings defined in /localsettings.sty
\usepackage{localsettings}
\usepackage{arabicletters}

% lazyeqn - math symbols
% Uncomment if you have chosen to clone it to your project
% \usepackage{./lazyeqn/lazyeqn}

% \usepackage{caption}
\usepackage{formatting}

\newcommand*\varhrulefill[1][0.4pt]{\leavevmode\leaders\hrule height#1\hfill\kern0pt}

\usepackage{calc}
\newlength{\linew}
\setlength{\linew}{\textwidth+2em}
\newlength{\ltw}
\setlength{\ltw}{0.08333\linew}
\newlength{\rowh}
\setlength{\rowh}{0.6cm}

\newcommand{\fat}{FarsoArabo\kern -.1em Türkçe}

% \title{\Huge\fat:\\[0.6ex] \large Türkçe İçindeki Farsoarap Unsurların \\
%   Ortografi, Gramer ve Kelime Türetme Esasları}
\title{\LARGE\sffamily Türkçe İçindeki Farsoarap Unsurların \\
  Ortografi, Gramer ve Kelime-Türetme Kuralları}
\author{H.\ O.\ Solmaz}
\date{}

 % Â, â, Î, î, Û, û
\raggedbottom
\begin{document}

% \pagestyle{plain}
\maketitle
\tableofcontents

\newcommand\uyumluharftablo[4]{%
  % \renewcommand{\arraystretch}{3}
  \begin{tabular}{>{\centering}p{0.15\textwidth}%
    >{\centering}p{0.15\textwidth}%
    >{\centering}p{0.25\textwidth}%
    >{\centering}p{0.25\textwidth}}
    \toprule
    \footnotesize Eski yazı harf
    & \footnotesize Vekil harf
    & \footnotesize Eski yazı harf ismi
    & \footnotesize Türkçe telaffuz
      \tabularnewline
      \midrule
    \begin{minipage}[c][10ex]{0.15\textwidth} \centering \Huge #1 \end{minipage}
    & \begin{minipage}[c][10ex]{0.15\textwidth} \centering \Huge #2 \end{minipage}
    & #3
    & \timesfont #4
      \tabularnewline
    \bottomrule
  \end{tabular}%
}

\newcommand\hariciharftablo[5]{%
  % \renewcommand{\arraystretch}{3}
  \begin{tabular}{>{\centering}p{0.15\textwidth}%
    >{\centering}p{0.15\textwidth}%
    >{\centering}p{0.20\textwidth}%
    >{\centering}p{0.17\textwidth}%
    >{\centering}p{0.17\textwidth}%
    }
    \toprule
    \footnotesize Eski yazı harf
    & \footnotesize Vekil harf
    & \footnotesize Eski yazı harf ismi
    & \footnotesize Türkçe telaffuz
    & \footnotesize Arapça telaffuz
      \tabularnewline
      \midrule
    \begin{minipage}[c][10ex]{0.15\textwidth} \centering \Huge #1 \end{minipage}
    & \begin{minipage}[c][10ex]{0.15\textwidth} \centering \Huge #2 \end{minipage}
    & #3
    & \timesfont #4
    & \timesfont #5
      \tabularnewline
    \bottomrule
  \end{tabular}%
}


\chapter{Ortografi}

% \includesvg[width=\textwidth]{./fig/fig1}

% \begin{tikzpicture}
  \bfseries
  \small
  \node(01_1)[text depth=0.25ex, text height=2.5ex, text centered] at (0   , 0) {\latupalif};
  \node(02_1)[text depth=0.25ex, text height=2.5ex, text centered] at (5   , 0) {\latupbe};
  \node(03_1)[text depth=0.25ex, text height=2.5ex, text centered] at (10  , 0) {\latuppe};
  \node(04_1)[text depth=0.25ex, text height=2.5ex, text centered] at (15  , 0) {\latupthe};
  \node(05_1)[text depth=0.25ex, text height=2.5ex, text centered] at (20  , 0) {\latupte};
  \node(06_1)[text depth=0.25ex, text height=2.5ex, text centered] at (25  , 0) {\latupcim};
  \node(07_1)[text depth=0.25ex, text height=2.5ex, text centered] at (30  , 0) {\latupchim};
  \node(08_1)[text depth=0.25ex, text height=2.5ex, text centered] at (35  , 0) {\latupha};
  \node(09_1)[text depth=0.25ex, text height=2.5ex, text centered] at (40  , 0) {\latupxa};
  \node(10_1)[text depth=0.25ex, text height=2.5ex, text centered] at (45  , 0) {\latupdal};
  \node(11_1)[text depth=0.25ex, text height=2.5ex, text centered] at (50  , 0) {\latupzel};
  \node(12_1)[text depth=0.25ex, text height=2.5ex, text centered] at (55  , 0) {\latupre};
  \node(13_1)[text depth=0.25ex, text height=2.5ex, text centered] at (60  , 0) {\latupze};
  \node(14_1)[text depth=0.25ex, text height=2.5ex, text centered] at (65  , 0) {\latupje};
  \node(15_1)[text depth=0.25ex, text height=2.5ex, text centered] at (70  , 0) {\latupsin};
  \node(16_1)[text depth=0.25ex, text height=2.5ex, text centered] at (75  , 0) {\latupshin};
  \node(17_1)[text depth=0.25ex, text height=2.5ex, text centered] at (80  , 0) {\latupsad};
  \node(18_1)[text depth=0.25ex, text height=2.5ex, text centered] at (85  , 0) {\latupdad};
  \node(19_1)[text depth=0.25ex, text height=2.5ex, text centered] at (90  , 0) {\latupta};
  \node(20_1)[text depth=0.25ex, text height=2.5ex, text centered] at (95  , 0) {\latupza};
  \node(21_1)[text depth=0.25ex, text height=2.5ex, text centered] at (100 , 0) {\latupayn};
  \node(22_1)[text depth=0.25ex, text height=2.5ex, text centered] at (105 , 0) {\latupgayn};
  \node(23_1)[text depth=0.25ex, text height=2.5ex, text centered] at (110 , 0) {\latupfe};
  \node(24_1)[text depth=0.25ex, text height=2.5ex, text centered] at (115 , 0) {\latupkaf};
  \node(25_1)[text depth=0.25ex, text height=2.5ex, text centered] at (120 , 0) {\latupkef};
  \node(26_1)[text depth=0.25ex, text height=2.5ex, text centered] at (125 , 0) {\latupgef};
  \node(27_1)[text depth=0.25ex, text height=2.5ex, text centered] at (130 , 0) {\latupnef};
  \node(28_1)[text depth=0.25ex, text height=2.5ex, text centered] at (135 , 0) {\latuplam};
  \node(29_1)[text depth=0.25ex, text height=2.5ex, text centered] at (140 , 0) {\latupmim};
  \node(30_1)[text depth=0.25ex, text height=2.5ex, text centered] at (145 , 0) {\latupnun};
  \node(31_1)[text depth=0.25ex, text height=2.5ex, text centered] at (150 , 0) {\latupvav};
  \node(32_1)[text depth=0.25ex, text height=2.5ex, text centered] at (155 , 0) {\latuphe};
  \node(33_1)[text depth=0.25ex, text height=2.5ex, text centered] at (160 , 0) {\latupye};

  \scriptsize
  \node(01_2)[text depth=2.25ex, text height=2ex, text centered] at (0   , 20) {\aralif};
  \node(02_2)[text depth=2.25ex, text height=2ex, text centered] at (5   , 20) {\arbe};
  \node(03_2)[text depth=2.25ex, text height=2ex, text centered] at (10  , 20) {\arpe};
  \node(04_2)[text depth=2.25ex, text height=2ex, text centered] at (15  , 20) {\arthe};
  \node(05_2)[text depth=2.25ex, text height=2ex, text centered] at (20  , 20) {\arte};
  \node(06_2)[text depth=2.25ex, text height=2ex, text centered] at (25  , 20) {\arcim};
  \node(07_2)[text depth=2.25ex, text height=2ex, text centered] at (30  , 20) {\archim};
  \node(08_2)[text depth=2.25ex, text height=2ex, text centered] at (35  , 20) {\arha};
  \node(09_2)[text depth=2.25ex, text height=2ex, text centered] at (40  , 20) {\arxa};
  \node(10_2)[text depth=2.25ex, text height=2ex, text centered] at (45  , 20) {\ardal};
  \node(11_2)[text depth=2.25ex, text height=2ex, text centered] at (50  , 20) {\arzel};
  \node(12_2)[text depth=2.25ex, text height=2ex, text centered] at (55  , 20) {\arre};
  \node(13_2)[text depth=2.25ex, text height=2ex, text centered] at (60  , 20) {\arze};
  \node(14_2)[text depth=2.25ex, text height=2ex, text centered] at (65  , 20) {\arje};
  \node(15_2)[text depth=2.25ex, text height=2ex, text centered] at (70  , 20) {\arsin};
  \node(16_2)[text depth=2.25ex, text height=2ex, text centered] at (75  , 20) {\arshin};
  \node(17_2)[text depth=2.25ex, text height=2ex, text centered] at (80  , 20) {\arsad};
  \node(18_2)[text depth=2.25ex, text height=2ex, text centered] at (85  , 20) {\ardad};
  \node(19_2)[text depth=2.25ex, text height=2ex, text centered] at (90  , 20) {\arta};
  \node(20_2)[text depth=2.25ex, text height=2ex, text centered] at (95  , 20) {\arza};
  \node(21_2)[text depth=2.25ex, text height=2ex, text centered] at (100 , 20) {\arayn};
  \node(22_2)[text depth=2.25ex, text height=2ex, text centered] at (105 , 20) {\argayn};
  \node(23_2)[text depth=2.25ex, text height=2ex, text centered] at (110 , 20) {\arfe};
  \node(24_2)[text depth=2.25ex, text height=2ex, text centered] at (115 , 20) {\arkaf};
  \node(25_2)[text depth=2.25ex, text height=2ex, text centered] at (120 , 20) {\arkef};
  \node(26_2)[text depth=2.25ex, text height=2ex, text centered] at (125 , 20) {\argef};
  \node(27_2)[text depth=2.25ex, text height=2ex, text centered] at (130 , 20) {\arnef};
  \node(28_2)[text depth=2.25ex, text height=2ex, text centered] at (135 , 20) {\arlam};
  \node(29_2)[text depth=2.25ex, text height=2ex, text centered] at (140 , 20) {\armim};
  \node(30_2)[text depth=2.25ex, text height=2ex, text centered] at (145 , 20) {\arnun};
  \node(31_2)[text depth=2.25ex, text height=2ex, text centered] at (150 , 20) {\arvav};
  \node(32_2)[text depth=2.25ex, text height=2ex, text centered] at (155 , 20) {\arhe};
  \node(33_2)[text depth=2.25ex, text height=2ex, text centered] at (160 , 20) {\arye};

  \draw[->](01_2) edge (01_1);


\draw[->] (01_2) edge (01_1);
\draw[->] (02_2) edge (02_1);
\draw[->] (03_2) edge (03_1);
\draw[->] (04_2) edge (04_1);
\draw[->] (05_2) edge (05_1);
\draw[->] (06_2) edge (06_1);
\draw[->] (07_2) edge (07_1);
\draw[->] (08_2) edge (08_1);
\draw[->] (09_2) edge (09_1);
\draw[->] (10_2) edge (10_1);
\draw[->] (11_2) edge (11_1);
\draw[->] (12_2) edge (12_1);
\draw[->] (13_2) edge (13_1);
\draw[->] (14_2) edge (14_1);
\draw[->] (15_2) edge (15_1);
\draw[->] (16_2) edge (16_1);
\draw[->] (17_2) edge (17_1);
\draw[->] (18_2) edge (18_1);
\draw[->] (19_2) edge (19_1);
\draw[->] (20_2) edge (20_1);
\draw[->] (21_2) edge (21_1);
\draw[->] (22_2) edge (22_1);
\draw[->] (23_2) edge (23_1);
\draw[->] (24_2) edge (24_1);
\draw[->] (25_2) edge (25_1);
\draw[->] (27_2) edge (27_1);
\draw[->] (26_2) edge (26_1);
\draw[->] (28_2) edge (28_1);
\draw[->] (29_2) edge (29_1);
\draw[->] (30_2) edge (30_1);
\draw[->] (31_2) edge (31_1);
\draw[->] (32_2) edge (32_1);
\draw[->] (33_2) edge (33_1);



































\end{tikzpicture}

\includegraphics[width=\textwidth]{./fig/fig1.tikz}
\section{Türk Diliyle Uyumlu Sesler ve Harfler}

\subsection*{Elif}


\uyumluharftablo{\aralif}{\latupalif\latdownalif}{\isimalif}{\trtlfalif}



\subsection*{Be}
\uyumluharftablo{\arbe}{\latupbe\latdownbe}{\isimbe}{\trtlfbe}


\subsection*{Pe}
\uyumluharftablo{\arpe}{\latuppe\latdownpe}{\isimpe}{\trtlfpe}

\subsection*{Te}
\uyumluharftablo{\arte}{\latupte\latdownte}{\isimte}{\trtlfte}

\subsection*{Cim}
\uyumluharftablo{\arcim}{\latupcim\latdowncim}{\isimcim}{\trtlfcim}

\subsection*{Çim}
\uyumluharftablo{\archim}{\latupchim\latdownchim}{\isimchim}{\trtlfchim}


\subsection*{Dal}
\uyumluharftablo{\ardal}{\latupdal\latdowndal}{\isimdal}{\trtlfdal}

\subsection*{Re}
\uyumluharftablo{\arre}{\latupre\latdownre}{\isimre}{\trtlfre}

\subsection*{Ze}
\uyumluharftablo{\arze}{\latupze\latdownze}{\isimze}{\trtlfze}

\subsection*{Je}
\uyumluharftablo{\arje}{\latupje\latdownje}{\isimje}{\trtlfje}

\subsection*{Sin}
\uyumluharftablo{\arsin}{\latupsin\latdownsin}{\isimsin}{\trtlfsin}

\subsection*{Şın}
\uyumluharftablo{\arshin}{\latupshin\latdownshin}{\isimshin}{\trtlfshin}

\subsection*{Gayn}
\uyumluharftablo{\argayn}{\latupgayn\latdowngayn}{\isimgayn}{\trtlfgayn}

\subsection*{Fe}
\uyumluharftablo{\arfe}{\latupfe\latdownfe}{\isimfe}{\trtlffe}

\subsection*{Kef}
\uyumluharftablo{\arkef}{\latupkef\latdownkef}{\isimkef}{\trtlfkef}

\subsection*{Gef}
\uyumluharftablo{\argef}{\latupgef\latdowngef}{\isimgef}{\trtlfgef}

\subsection*{Nef / Sağır Kef}
\uyumluharftablo{\arnef}{\latupnef\latdownnef}{\isimnef}{\trtlfnef}

\subsection*{Lam}
\uyumluharftablo{\arlam}{\latuplam\latdownlam}{\isimlam}{\trtlflam}

\subsection*{Mim}
\uyumluharftablo{\armim}{\latupmim\latdownmim}{\isimmim}{\trtlfmim}

\subsection*{Nun}
\uyumluharftablo{\arnun}{\latupnun\latdownnun}{\isimnun}{\trtlfnun}

\subsection*{Vav}
\uyumluharftablo{\arvav}{\latupvav\latdownvav}{\isimvav}{\trtlfvav}

\subsection*{He}
\uyumluharftablo{\arhe}{\latuphe\latdownhe}{\isimhe}{\trtlfhe}

\subsection*{Ye}
\uyumluharftablo{\arye}{\latupye\latdownye}{\isimye}{\trtlfye}



\section{Türk Diline Harici Sesler ve Harfler}

\subsection*{Peltek Se}
\hariciharftablo{\arthe}{\latupthe\latdownthe}{\isimthe}{\trtlfthe}{\artlfthe}

\subsection*{Ha}
\hariciharftablo{\arha}{\latupha\latdownha}{\isimha}{\trtlfha}{\artlfha}

\subsection*{Hı}
\hariciharftablo{\arxa}{\latupxa\latdownxa}{\isimxa}{\trtlfxa}{\artlfxa}

\subsection*{Zel}
\hariciharftablo{\arzel}{\latupzel\latdownzel}{\isimzel}{\trtlfzel}{\artlfzel}

\subsection*{Sad}
\hariciharftablo{\arsad}{\latupsad\latdownsad}{\isimsad}{\trtlfsad}{\artlfsad}

\subsection*{Dad}
\hariciharftablo{\ardad}{\latupdad\latdowndad}{\isimdad}{\trtlfdad}{\artlfdad}

\subsection*{Ta}
\hariciharftablo{\arta}{\latupta\latdownta}{\isimta}{\trtlfta}{\artlfta}

\subsection*{Za}
\hariciharftablo{\arza}{\latupza\latdownza}{\isimza}{\trtlfza}{\artlfza}

\subsection*{Ayn}
\hariciharftablo{\arayn}{\latupayn\latdownayn}{\isimayn}{\trtlfayn}{\artlfayn}

% \vek{\mim\sinn\te\hemzup{\alif}\cim\re}

% \vek{\kesre{\fethe{\ze}}} \vek{\sukun{\ze}} \vek{\otre{\ze}} \vek{\madde{\ze}}
% \vek{\sedde{\ze}}

\subsection*{Kaf}
\hariciharftablo{\arkaf}{\latupkaf\latdownkaf}{\isimkaf}{\trtlfkaf}{\artlfkaf}



% \section{Eski Yazı}

% sf/tt/rm: \textsf{x}\texttt{x}x

% sf/tt/rm: \textsf{K}\texttt{K}K \textsf{\latupkaf}\texttt{\latupkaf}\latupkaf

% \section{Vekil Alfabe}

\newpage
\begingroup
% \setlength\extrarowh{5pt}
% \begin{table}[htbp]
  % \centering
\renewcommand{\arraystretch}{2.1}
\begin{longtable*}{>{\LARGE}p{1.4\ltw}>{\LARGE}p{1.4\ltw}>{}p{1.9\ltw}>{}p{2.8\ltw}>{\timesfont}p{2\ltw}}
   \small Vekil \newline harf
                              & \small Eski yazı \newline harf
                              & \small Eski yazı \newline harf ismi
                              & \small Yeni yazı\newline karşılığı
                              & \small\normalfont Türkçe \newline telaffuz                                       \\
  %
  % \normalsize Eskiyazı Harf & \normalsize Vekil Harf       & \normalsize
  % Yeni yazı Mukabili        & \normalsize Eskiyazı okunuşu & \normalsize Türkçe \mbox{Okunuşu}                 \\
  \toprule
  \latupalif \latdownalif     & \aralif                      & \isimalif   & *A veya E          & \trtlfalif  \\
  \latupayn  \latdownayn      & \raisebox{0.6ex}{\arayn}     & \isimayn    & *gırtlak ünlüsü    & \trtlfayn   \\ % ʕſ Ʒʒ
  \latupbe   \latdownbe       & \arbe                        & \isimbe     & B                  & \trtlfbe    \\
  \latupcim  \latdowncim      & \raisebox{0.8ex}{\arcim}     & \isimcim    & C                  & \trtlfcim   \\
  \latupchim \latdownchim     & \raisebox{0.8ex}{\archim}    & \isimchim   & Ç                  & \trtlfchim  \\
  \latupdal  \latdowndal      & \ardal                       & \isimdal    & D                  & \trtlfdal   \\
  \latupdad  \latdowndad      & \raisebox{0.8ex}{\ardad}     & \isimdad    & *diş D             & \trtlfdad   \\
  \latupfe   \latdownfe       & \arfe                        & \isimfe     & F                  & \trtlffe    \\
  \latupgef  \latdowngef      & \argef                       & \isimgef    & G                  & \trtlfgef   \\
  \latupgayn \latdowngayn     & \raisebox{0.3ex}{\argayn}    & \isimgayn   & genelde Ğ          & \trtlfgayn  \\
  \latuphe   \latdownhe       & \arhe                        & \isimhe     & *ince H veya E     & \trtlfhe    \\
  \latupha   \latdownha       & \raisebox{1.1ex}{\arha}      & \isimha     & *kalın H           & \trtlfha    \\
  \latupxa   \latdownxa       & \raisebox{0.7ex}{\arxa}      & \isimxa     & *sürtmeli H        & \trtlfxa    \\
  \latupye   \latdownye       & \arye                        & \isimye     & *I, İ veya Y       & \trtlfye    \\
  \latupje   \latdownje       & \raisebox{0.3ex}{\arje}      & \isimje     & J                  & \trtlfje    \\
  \latupkef  \latdownkef      & \arkef                       & \isimkef    & *ince K            & \trtlfkef   \\
  \latupkaf  \latdownkaf      & \arkaf                       & \isimkaf    & *kalın K           & \trtlfkaf   \\
  \latuplam  \latdownlam      & \arlam                       & \isimlam    & L                  & \trtlflam   \\
  \latupmim  \latdownmim      & \armim                       & \isimmim    & M                  & \trtlfmim   \\
  \latupnun  \latdownnun      & \raisebox{0.4ex}{\arnun}     & \isimnun    & N                  & \trtlfnun   \\
  \latupnef  \latdownnef      & \raisebox{-0.7ex}{\arnef}    & \isimnef    & *geniz N           & \trtlfnef   \\
  \latuppe   \latdownpe       & \arpe                        & \isimpe     & P                  & \trtlfpe    \\
  \latupre   \latdownre       & \raisebox{0.4ex}{\arre}      & \isimre     & R                  & \trtlfre    \\
  \latupsin  \latdownsin      & \raisebox{0.6ex}{\arsin}     & \isimsin    & *ince S            & \trtlfsin   \\
  \latupsad  \latdownsad      & \raisebox{0.6ex}{\arsad}     & \isimsad    & *kalın S           & \trtlfsad   \\
  \latupthe  \latdownthe      & \arthe                       & \isimthe    & *peltek S          & \trtlfthe   \\
  \latupshin \latdownshin     & \raisebox{0.4ex}{\arshin}    & \isimshin   & Ş                  & \trtlfshin  \\
  \latupte   \latdownte       & \arte                        & \isimte     & *ince T            & \trtlfte    \\
  \latupta   \latdownta       & \arta                        & \isimta     & *kalın T           & \trtlfta    \\
  \latupvav  \latdownvav      & \raisebox{0.6ex}{\arvav}     & \isimvav    & *O,Ö,U,Ü veya V    & \trtlfvav   \\
  \latupze   \latdownze       & \raisebox{0.6ex}{\arze}      & \isimze     & Z                  & \trtlfze    \\
  \latupza   \latdownza       & \arza                        & \isimza     & *kalın Z           & \trtlfza    \\
  \latupzel  \latdownzel      & \raisebox{0.2ex}{\arzel}     & \isimzel    & *diş Z             & \trtlfzel   \\
  \latuphemze                 & \arhemze                     & \isimhemze  & *gırtlak kapatması & \trtlfhemze \\ % ʕſ Ʒʒ
  \bottomrule
\end{longtable*}
\vspace{-6ex}
\centering
\begin{table}[H]
  \caption{Vekil alfabe. Vekil harfler, yeni yazıda karşılık geldikleri yerlere göre
    sıralanmış olup, rahat karşılaştırma için eski harf isimleri ve konuşmadaki
    telaffuzları ile birlikte verilmiştir. Yeni yazıda birebir karşılığı
    bulunmayan eski harfler yıldız (*) ile işaretlenmiştir.
    Telaffuzda UFA notasyonu kullanılmıştır.}
  \label{tab:vekil1}
\end{table}
\endgroup




% \chapter{Kelime Türetme}



\chapter{Arap Köklü Kelimelerin Kuralları}
\section{Bitişimsiz Morfoloji}

\section{Eylem İsmi Kipleri}

\begin{table}[htbp]
  \footnotesize
  \centering
  \renewcommand{\arraystretch}{1.5}
  \begin{tabular}{p{0.1\tabwidth} >{\raggedright}p{0.3\tabwidth} >{\raggedright}p{0.2\tabwidth} >{\raggedright}p{0.2\tabwidth} p{0.2\tabwidth}}
    Kip No & Anlam & Eylem & Eden & Edilen \\
    \toprule
    \rom{1} & Sade & Necz, Nücûz, Nicz, Nücz(et), Necâz(et), Nicâz(et), vs. &  Nâciz & Mencuz \\
    \rom{2} & Geçişli, ettiren, güçlü & Tenciz &  Münecciz & Müneccez \\
    \rom{3} & İşteş & Münâceze &  Münâciz & Münâcez \\
    \rom{4} & Geçişli, ettiren & İncâz &  Münciz & Müncez \\
    \rom{5} & Dönüşlü, ettiren, güçlü & Teneccüz &  Mütenecciz & Müteneccez \\
    \rom{6} & \rom{3}'ün işteş karşılığı & Tenâcüz &  Mütenâciz & Mütenâcez \\
    \rom{7} & Dönüşlü, ettirilen & İnnicâz & Münneciz & Münnecez \\
    \rom{8} & \rom{1}'in dönüşlüsü, geçişsiz & İnticâz & Münteciz & Müntecez \\
    \rom{9} & Durum eylemi, geçişsiz & İncizâz & Müncezz & --- \\
    \rom{10} & Ettiren, bazen ettirilen, çeşitli anlamlar  & İstincâz & Müstenciz & Müstencez \\
    % \midrule
    % \rom{11} & \rom{9}'un aynısı, şiir dışında nadir & İncîzaz & Müncâzz & --- \\
    % \rom{12} & \multirow{4}{*}{\parbox{0.3\tabwidth}{\raggedright Durum eylemi, çok nadir}} & İncîcaz & Müncavciz & Müncavcez \\
    % \rom{13} &  & İncivvaz & Müncavviz & Müncavvez \\
    % \rom{14} &  & İncinzâz & Müncanziz & Müncanzez \\
    % \rom{15} &  & İncinzâ' & Müncanzin & Müncanzen \\
    \bottomrule
  \end{tabular}
\end{table}

\subsection{Nâciz Kipi: Yapan}

\subsection{Mencuz Kipi: Yapılan}

\subsection{İncaz Kipi: Geçişli (Transitif) Eylem}



\chapter{Fars Köklü Kelimelerin Kuralları}
\section{Fars Köklü Ekler}


\chapter{Kökler Sözlüğü}
\noindent
\setlength{\parindent}{0pt}


\newpage

% \recalctypearea
% \newgeometry{left=4mm,right=15mm,top=18mm, bottom=25mm}


\dictchapter{\latupalif}
\begin{multicols}{2}
\kokkipentry{\latupalif\latuphe\latuplam}{ahâli, ehil, ehlî, ehliyet, teehhül}{kok:alif_he_lam}
\kokkipentry{\latupalif\latupmim\latupre}{âmir, emâre, emir, emîr, memur, umûr, ümerâ}{kok:alif_mim_re}
\end{multicols}
\dictchapter{\latupayn}
\begin{multicols}{2}
\kokkipentry{\latupayn\latupbe\latupre}{ibâre, ibâret, ibret, îtibâr, mûteber, tâbir}{kok:ayn_be_re}
\kokkipentry{\latupayn\latupdal\latuplam}{adâlet, âdil, adl, adliye, îtidâl, muâdelet, muâdil, mûtedil, tâdil, tâdilat}{kok:ayn_dal_lam}
\kokkipentry{\latupayn\latuplam\latupmim}{alem, âlem, âlim, allâme, ilâm, ilim, mâlum, muallim, tâlim, ulema, ulûm}{kok:ayn_lam_mim}
\kokkipentry{\latupayn\latupre\latupdad}{araz, ârıza, ârız, arîza, aruz, arz, avârız, ırz, îtiraz, maraza, mâruz, muârız, taarruz, târiz}{kok:ayn_re_dad}
\kokkipentry{\latupayn\latupre\latupfe}{araf, ârif, arife, irfan, îtiraf, maarif, mârifet, mâruf, mütearife, örf, târif, târife}{kok:ayn_re_fe}
\kokkipentry{\latupayn\latupre\latupkaf}{ırk}{kok:ayn_re_kaf}
\kokkipentry{\latupayn\latupshin\latupre}{âşar, aşîret, aşûre, işret, mâşerî, muâşeret, öşür}{kok:ayn_shin_re}
\kokkipentry{\latupayn\latupzel\latupre}{îtizâr, mâzeret, mâzur, özür}{kok:ayn_zel_re}
\end{multicols}
\dictchapter{\latupbe}
\begin{multicols}{2}
\kokkipentry{\latupbe\latupkaf\latupye}{bakaya, bâki, bakiye, bekâ, ibkâ, mütebâki}{kok:be_kaf_ye}
\kokkipentry{\latupbe\latuplam\latupgayn}{bâliğ, belâgat, belîğ, bülûğ, iblâğ, meblağ, mübalağa, tebellüğ, teblîgat, teblîğ}{kok:be_lam_gayn}
\end{multicols}
\dictchapter{\latupcim}
\begin{multicols}{2}
\kokkipentry{\latupcim\latuphe\latuplam}{câhil, cehâlet, cehl, cühelâ, meçhul, tecâhül}{kok:cim_he_lam}
\kokkipentry{\latupcim\latuplam\latupdal}{celâdet, cellat, cilt, mücellit}{kok:cim_lam_dal}
\kokkipentry{\latupcim\latupmim\latupayn}{câmiâ, câmi, cem, cemaat, cemiyet, cîmâ, cumâ, icmâ, içtimâ, mecmuâ, mecmû}{kok:cim_mim_ayn}
\kokkipentry{\latupcim\latupvav\latupbe}{cevap, icâbet, isticvâp, müstecap}{kok:cim_vav_be}
\end{multicols}
\dictchapter{\latupdal}
\begin{multicols}{2}
\kokkipentry{\latupdal\latupxa\latuplam}{dahil, dâhil, dehâlet, duhûl, ithâl, ithâlat, medhal, müdâhale, müdâhil, tahıl}{kok:dal_xa_lam}
\kokkipentry{\latupdal\latupye\latupre}{deyr}{kok:dal_ye_re}
\kokkipentry{\latupdal\latupre\latupsin}{ders, medrese, müderris, tedris, tedrisat}{kok:dal_re_sin}
\kokkipentry{\latupdal\latupvav\latupre}{dâir, dâire, dâr, dâreyn, devrân, devre, devriye, diyâr, edvar, idâre, medâr, müdevver, müdür, tedvir}{kok:dal_vav_re}
\end{multicols}
\dictchapter{\latupfe}
\begin{multicols}{2}
\kokkipentry{\latupfe\latupayn\latuplam}{efâl, faal, faaliyet, fâil, fiil, infiâl, meful}{kok:fe_ayn_lam}
\kokkipentry{\latupfe\latupha\latupshin}{fâhiş, fâhişe, fuhuş}{kok:fe_ha_shin}
\kokkipentry{\latupfe\latupxa\latupre}{fahrî, iftihâr, müftehir}{kok:fe_xa_re}
\kokkipentry{\latupfe\latupkef\latupre}{efkar, fikir, mefkûre, mütefekkir, tefekkür}{kok:fe_kef_re}
\kokkipentry{\latupfe\latupkaf\latupre}{fakir, fakr, fıkra, fukarâ}{kok:fe_kaf_re}
\kokkipentry{\latupfe\latupre\latupdal}{efrat, fert, infirât, müfredat, münferit}{kok:fe_re_dal}
\kokkipentry{\latupfe\latupre\latupkaf}{fârika, fark, ferik, fırka, firâk, firkat, furkân, müteferrik, tefrika, tefrik}{kok:fe_re_kaf}
\kokkipentry{\latupfe\latupre\latupze}{ifrâz, ifrâzat, müfreze}{kok:fe_re_ze}
\kokkipentry{\latupfe\latupsin\latupdal}{fâsit, fesat, ifsât, müfsit}{kok:fe_sin_dal}
\kokkipentry{\latupfe\latupsin\latupxa}{fesih, infisâh, münfesih, mütefessih, tefessüh}{kok:fe_sin_xa}
\kokkipentry{\latupfe\latupshin\latupvav}{faş, ifşâ}{kok:fe_shin_vav}
\kokkipentry{\latupfe\latupte\latupdad}{fâiz, feyiz}{kok:fe_te_dad}
\kokkipentry{\latupfe\latupte\latupshin}{müfettiş, teftiş}{kok:fe_te_shin}
\kokkipentry{\latupfe\latupta\latupre}{fıtrat, fitre, iftâr}{kok:fe_ta_re}
\kokkipentry{\latupfe\latupvav\latupkaf}{fâik, fevk}{kok:fe_vav_kaf}
\end{multicols}
\dictchapter{\latupgayn}
\begin{multicols}{2}
\kokkipentry{\latupgayn\latupdal\latupre}{gaddar, gadir, mağdur}{kok:gayn_dal_re}
\kokkipentry{\latupgayn\latupfe\latuplam}{gâfil, gaflet, iğfâl, tegâfül}{kok:gayn_fe_lam}
\kokkipentry{\latupgayn\latupye\latupbe}{gâip, gaybûbet, gıyâb, gıybet, kayıp, kayıp}{kok:gayn_ye_be}
\kokkipentry{\latupgayn\latuplam\latupbe}{galebe, gâlibâ, gâlibiyet, gâlip, mağlup, mütegallibe, tegallüp}{kok:gayn_lam_be}
\kokkipentry{\latupgayn\latupnun\latupye}{ganî, gınâ, istiğnâ, müstağnî}{kok:gayn_nun_ye}
\kokkipentry{\latupgayn\latupre\latupbe}{garâib, garibân, garip, garp, gurbet, gurebâ, gurûp, mağrip}{kok:gayn_re_be}
\end{multicols}
\dictchapter{\latuphe}
\begin{multicols}{2}
\kokkipentry{\latuphe\latupmim\latupmim}{ehemmiyet, himmet, ihtimâm, mühim, mühimmat}{kok:he_mim_mim}
\end{multicols}
\dictchapter{\latupha}
\begin{multicols}{2}
\kokkipentry{\latupha\latupkaf\latupkaf}{hak, hakîkat, hakkâk, hokka, hukuk, ihkâk, istihkâk, muhakkak, muhakkik, muhik, müstehak, tahakkuk, tahkik}{kok:ha_kaf_kaf}
\kokkipentry{\latupha\latuplam\latuplam}{hal, helal, hulûl, hülle, inhilâl, mahalle, mahlul, mehel, münhal, tahlil}{kok:ha_lam_lam}
\kokkipentry{\latupha\latupmim\latupdal}{hamd, mahmut, Muhammet}{kok:ha_mim_dal}
\kokkipentry{\latupha\latupmim\latuplam}{hamal, hâmil, hâmile, haml, hamle, hamûle, ihtimâl, muhtemel, mütehammil, tahammül}{kok:ha_mim_lam}
\kokkipentry{\latupha\latupmim\latupmim}{hamam, hummâ}{kok:ha_mim_mim}
\kokkipentry{\latupha\latupmim\latupvav}{hamiyet}{kok:ha_mim_vav}
\kokkipentry{\latupha\latupre\latupbe}{harbi, harp, mihrap, muhârebe, muharip}{kok:ha_re_be}
\kokkipentry{\latupha\latupre\latupkef}{harekât, hareke, hareket, muharrik, müteharrik, tahrik}{kok:ha_re_kef}
\kokkipentry{\latupha\latupre\latupmim}{harâm, harem, hârim, hürmet, ihrâm, ihtirâm, mahrem, mahrum, muhterem}{kok:ha_re_mim}
\kokkipentry{\latupha\latupre\latupsad}{hâris, hırs, ihtirâs, muhteris}{kok:ha_re_sad}
\kokkipentry{\latupha\latupsin\latupbe}{hasep, hâsip, hesap, ihtisâp, mahsup, muhâsebe, muhâsip, muhtesip}{kok:ha_sin_be}
\kokkipentry{\latupha\latupsin\latupsin}{hassas, his, ihsâs, mütehassis}{kok:ha_sin_sin}
\kokkipentry{\latupha\latupsad\latupre}{hasır, hasr, hisar, inhisâr, mahsur, muhâsara, münhâsır}{kok:ha_sad_re}
\kokkipentry{\latupha\latupshin\latupmim}{haşema, haşmet, ihtişâm, muhteşem}{kok:ha_shin_mim}
\kokkipentry{\latupha\latupvav\latuplam}{ahval, hal, hâlâ, hâlen, hâlet, havâle, havâli, hile, ihâle, istihâle, muhâl, mütehavvil, tahavvül, tahvil}{kok:ha_vav_lam}
\end{multicols}
\dictchapter{\latupxa}
\begin{multicols}{2}
\kokkipentry{\latupxa\latupbe\latupre}{haber, ihbâr, istihbârat, muhâberat, muhâbere, muhâbir, muhbir}{kok:xa_be_re}
\kokkipentry{\latupxa\latupye\latuplam}{hayâl, hayâlet, muhayyel, muhayyile, tahayyül}{kok:xa_ye_lam}
\kokkipentry{\latupxa\latupye\latupre}{hayır, ihtiyâr}{kok:xa_ye_re}
\kokkipentry{\latupxa\latuplam\latupfe}{halef, halîfe, hilaf, hilâfet, ihtilâf, kalfa, muhâlefet, muhâlif, muhtelif}{kok:xa_lam_fe}
\kokkipentry{\latupxa\latuplam\latupkaf}{ahlak, halayık, hâlik, halk, hilkat, hulk, mahluk, mahlukat}{kok:xa_lam_kaf}
\kokkipentry{\latupxa\latuplam\latupsad}{halas, hâlis, hulûs, hülâsa, ihlâs, mahlas, muhlis, tahlisiye}{kok:xa_lam_sad}
\kokkipentry{\latupxa\latuplam\latupvav}{hâli, halvet, helâ, tahliye}{kok:xa_lam_vav}
\kokkipentry{\latupxa\latupmim\latupre}{hamr, hamur, mahmur}{kok:xa_mim_re}
\kokkipentry{\latupxa\latupre\latupcim}{harç, hâriciye, hâriç, hurûç, ihrâcat, ihrâç, istihrâç, mahreç}{kok:xa_re_cim}
\kokkipentry{\latupxa\latupsad\latupsad}{has, hassa, havâs, hisse, husus, ihtisâs, mahsus, mütehassıs, tahsis, tahsisat}{kok:xa_sad_sad}
\end{multicols}
\dictchapter{\latupkef}
\begin{multicols}{2}
\kokkipentry{\latupkef\latupbe\latupre}{ekâbir, ekber, kebir, kibar, kibir, kübrâ, mütekebbir, tekbir, tekebbür}{kok:kef_be_re}
\kokkipentry{\latupkef\latupfe\latupre}{kâfir, kefâret, kefere, küffâr, küfrân, küfür, tekfir}{kok:kef_fe_re}
\kokkipentry{\latupkef\latupfe\latupvav}{iktifâ, kâfî, kifâyet, mükâfat}{kok:kef_fe_vav}
\kokkipentry{\latupkef\latuphe\latupnun}{kâhin, kehânet}{kok:kef_he_nun}
\kokkipentry{\latupkef\latupye\latupfe}{keyfiyet, keyif}{kok:kef_ye_fe}
\kokkipentry{\latupkef\latuplam\latupfe}{külfet, mükellef, tekâlif, tekellüf, teklif}{kok:kef_lam_fe}
\kokkipentry{\latupkef\latuplam\latupmim}{kelam, kelime, mükâleme, mütekellim, tekellüm}{kok:kef_lam_mim}
\kokkipentry{\latupkef\latupmim\latuplam}{ekmel, ikmâl, kâmil, kemâl, mükemmel, mütekâmil, tekâmül, tekemmül, tekmil}{kok:kef_mim_lam}
\kokkipentry{\latupkef\latupthe\latupre}{ekser, kesret, kevser, teksir}{kok:kef_the_re}
\kokkipentry{\latupkef\latupte\latupbe}{kâtip, ketebe, kitâbe, kitâbet, kitâbiyat, kitap, küttab, kütüphane, mektep, mektup}{kok:kef_te_be}
\end{multicols}
\dictchapter{\latupkaf}
\begin{multicols}{2}
\kokkipentry{\latupkaf\latupayn\latupdal}{kâide, makat, mütekâit, tekâüt}{kok:kaf_ayn_dal}
\kokkipentry{\latupkaf\latupbe\latupha}{kabahat, takbih}{kok:kaf_be_ha}
\kokkipentry{\latupkaf\latupbe\latuplam}{ikbâl, istikbâl, kabala, kabîl, kâbil, kabîle, kâbiliyet, kabûl, kıble, makbûl, mukâbele, mukâbil, müstakbel, mütekâbil, tekâbül}{kok:kaf_be_lam}
\kokkipentry{\latupkaf\latupdal\latupmim}{akdem, kadem, kademe, kadim, kıdem, kudüm, mukaddem, mukaddime, takaddüm, takdim}{kok:kaf_dal_mim}
\kokkipentry{\latupkaf\latupdal\latupre}{iktidâr, kadar, kader, kadir, kadir, kudret, miktar, mukadder, muktedir, takdir}{kok:kaf_dal_re}
\kokkipentry{\latupkaf\latuphe\latupre}{kahhar, kahır, kâhir}{kok:kaf_he_re}
\kokkipentry{\latupkaf\latupye\latupsin}{kıyas, mikyas, mukâyese}{kok:kaf_ye_sin}
\kokkipentry{\latupkaf\latuplam\latupbe}{inkılâp, kalp, kulp, maklûbe, münkalip}{kok:kaf_lam_be}
\kokkipentry{\latupkaf\latupsin\latupmim}{aksam, kasem, kâsım, kısım, kısmet, maksem, taksim, taksîmat}{kok:kaf_sin_mim}
\kokkipentry{\latupkaf\latupsin\latupre}{esre, inkisâr, kesir, küsûr}{kok:kaf_sin_re}
\kokkipentry{\latupkaf\latupsad\latupbe}{kasaba, kasap}{kok:kaf_sad_be}
\kokkipentry{\latupkaf\latupsad\latupdal}{iktisât, kasıt, kasîde, maksat, maksut}{kok:kaf_sad_dal}
\kokkipentry{\latupkaf\latupte\latuplam}{katil, kâtil, kıtâl, maktul, mukâtele, taktil}{kok:kaf_te_lam}
\kokkipentry{\latupkaf\latupvav\latupmim}{akvam, ikâme, ikâmet, istikâmet, kâim, kamet, kavim, kâyme, kayyım, kayyûm, kıvâm, kıyâm, kıyâmet, kıymet, makâm, mukâvemet, mukâvim, mukîm, müstakîm, takvim}{kok:kaf_vav_mim}
\end{multicols}
\dictchapter{\latuplam}
\begin{multicols}{2}
\kokkipentry{\latuplam\latupfe\latupza}{lafız, telaffuz}{kok:lam_fe_za}
\kokkipentry{\latuplam\latupha\latupkaf}{ilhâk, iltihâk, lâhika, mülhak}{kok:lam_ha_kaf}
\kokkipentry{\latuplam\latupvav\latupha}{lâyiha, levha}{kok:lam_vav_ha}
\kokkipentry{\latuplam\latupze\latupmim}{elzem, iltizâm, ilzâm, istilzâm, lâzım, levâzım, lüzum, mülâzım, mültezim}{kok:lam_ze_mim}
\kokkipentry{\latuplam\latupzel\latupzel}{lezîz, lezzet, mütelezziz, telezzüz}{kok:lam_zel_zel}
\end{multicols}
\dictchapter{\latupmim}
\begin{multicols}{2}
\kokkipentry{\latupmim\latupdal\latupdal}{imdât, istimdât, medet, met, müddet, temdit}{kok:mim_dal_dal}
\kokkipentry{\latupmim\latupdal\latupye}{mütemâdi, temâdi}{kok:mim_dal_ye}
\kokkipentry{\latupmim\latuplam\latupkef}{emlak, istimlâk, mâlik, meleke, melik, memâlik, memleket, memlük, mülk, mülkî, mülkiyet, müstemleke, temellük, temlik}{kok:mim_lam_kef}
\kokkipentry{\latupmim\latupthe\latuplam}{emsal, masal, mesel, meselâ, misal, misil, mümâsil, mümessil, temessül, temsil, timsâl}{kok:mim_the_lam}
\end{multicols}
\dictchapter{\latupnun}
\begin{multicols}{2}
\kokkipentry{\latupnun\latupcim\latupmim}{müneccim, necim, nücûm}{kok:nun_cim_mim}
\kokkipentry{\latupnun\latupkaf\latupmim}{intikâm}{kok:nun_kaf_mim}
\kokkipentry{\latupnun\latupta\latupkaf}{istintâk, mantık, müstantik, nâtıka, nutuk}{kok:nun_ta_kaf}
\kokkipentry{\latupnun\latupvav\latuplam}{minval, nevâle}{kok:nun_vav_lam}
\kokkipentry{\latupnun\latupvav\latupre}{minâre, münevver, nar, nevir, nur, tenevvür, tenvir}{kok:nun_vav_re}
\kokkipentry{\latupnun\latupza\latupmim}{intizâm, manzum, muntazam, nazım, nâzım, nizâm, tanzim, tanzimat}{kok:nun_za_mim}
\kokkipentry{\latupnun\latupza\latupre}{intizâr, manzara, muntazır, münâzara, nazar, nazariyat, nazariye, nâzır, nazîre, nezâret}{kok:nun_za_re}
\end{multicols}
\dictchapter{\latupre}
\begin{multicols}{2}
\kokkipentry{\latupre\latupfe\latupayn}{irtifâ, mürafaa, ref, terfî}{kok:re_fe_ayn}
\kokkipentry{\latupre\latuphe\latupnun}{rehin, rehîne, terhin}{kok:re_he_nun}
\kokkipentry{\latupre\latupha\latupmim}{istirhâm, merhamet, merhum, rahim, rahîm, rahmân, rahmet}{kok:re_ha_mim}
\kokkipentry{\latupre\latupvav\latupha}{ervah, istirahât, müsterih, rahat, râyihâ, reyhân, ruh, terâvih}{kok:re_vav_ha}
\kokkipentry{\latupre\latupze\latupkaf}{erzak, rızk}{kok:re_ze_kaf}
\end{multicols}
\dictchapter{\latupsin}
\begin{multicols}{2}
\kokkipentry{\latupsin\latuphemze\latuplam}{mesele, mesul, sual}{kok:sin_hemze_lam}
\kokkipentry{\latupsin\latupbe\latupkaf}{esbak, müsâbaka, müsâbık, sâbıka, sâbık, sibâk}{kok:sin_be_kaf}
\kokkipentry{\latupsin\latupkef\latupnun}{iskân, meskun, müsekkin, sâkin, sükûn, sükûnet, teskin}{kok:sin_kef_nun}
\kokkipentry{\latupsin\latuplam\latupmim}{islâm, islâmiyet, müsâleme, müsellem, müslim, sâlim, selam, selâmet, selem, selim, tesellüm, teslim}{kok:sin_lam_mim}
\kokkipentry{\latupsin\latupre\latupre}{esrar, meserret, mesrur, sır, sürûr}{kok:sin_re_re}
\end{multicols}
\dictchapter{\latupsad}
\begin{multicols}{2}
\kokkipentry{\latupsad\latupdal\latupkaf}{sadaka, sadâkat, sâdık, sıdk, tasdik}{kok:sad_dal_kaf}
\kokkipentry{\latupsad\latupha\latupbe}{ashap, musâhebe, musâhip, sahabe, sâhip, sohbet}{kok:sad_ha_be}
\kokkipentry{\latupsad\latuplam\latupha}{ıslâh, ıslâhat, ıstılâh, maslahat, musâlaha, salâh, salâhiyet, sâlih, sulh}{kok:sad_lam_ha}
\kokkipentry{\latupsad\latupnun\latupayn}{sanat, sanâyî, sınâî, sunî, zanaat}{kok:sad_nun_ayn}
\kokkipentry{\latupsad\latupnun\latupfe}{esnaf, musannif, sınıf, tasnif}{kok:sad_nun_fe}
\kokkipentry{\latupsad\latupre\latupfe}{masraf, mutasarrıf, sarf, sarraf, sırf, tasarruf, tasrif}{kok:sad_re_fe}
\kokkipentry{\latupsad\latupre\latupha}{sarâhat, sarih, sürâhî, tasrih}{kok:sad_re_ha}
\kokkipentry{\latupsad\latupvav\latupbe}{isâbet, musîbet, tasvip}{kok:sad_vav_be}
\kokkipentry{\latupsad\latupvav\latupfe}{mutasavvıf, sûfî, tasavvuf}{kok:sad_vav_fe}
\end{multicols}
\dictchapter{\latupthe}
\begin{multicols}{2}
\kokkipentry{\latupthe\latuplam\latupthe}{müselles, salı, sâlise, selâse, sülasî, sülüs, teslis}{kok:the_lam_the}
\kokkipentry{\latupthe\latupnun\latupye}{esnâ, istisnâ, mesnevî, müstesnâ, sânî, sâniye, senâ, sene, tesniye}{kok:the_nun_ye}
\end{multicols}
\dictchapter{\latupshin}
\begin{multicols}{2}
\kokkipentry{\latupshin\latupfe\latupkaf}{müşfik, şafak, şefkat}{kok:shin_fe_kaf}
\kokkipentry{\latupshin\latupgayn\latuplam}{işgâl, iştigâl, meşgale, meşgul}{kok:shin_gayn_lam}
\kokkipentry{\latupshin\latuphe\latupdal}{meşhut, müşâhede, müşâhit, şâhit, şehâdet}{kok:shin_he_dal}
\kokkipentry{\latupshin\latupye\latupxa}{meşâyih, meşîhat, şeyh}{kok:shin_ye_xa}
\kokkipentry{\latupshin\latupkef\latuplam$^1$}{işkil, müşkül}{kok:shin_kef_lam1}
\kokkipentry{\latupshin\latupkef\latuplam$^2$}{eşkâl, müteşekkil, şekil, teşekkül, teşkil}{kok:shin_kef_lam2}
\kokkipentry{\latupshin\latupre\latupayn}{meşrû, şer, şeriat, teşrî}{kok:shin_re_ayn}
\kokkipentry{\latupshin\latupre\latupfe}{eşraf, eşref, müşerref, şeref, şerefe, şerîf, teşerrüf, teşrif}{kok:shin_re_fe}
\kokkipentry{\latupshin\latupre\latupre}{şer, şerîr, şirret}{kok:shin_re_re}
\kokkipentry{\latupshin\latupta\latupre}{şatır, şetâret}{kok:shin_ta_re}
\kokkipentry{\latupshin\latupvav\latupkaf}{iştiyak, müşevvik, müştâk, şevk, teşvik}{kok:shin_vav_kaf}
\kokkipentry{\latupshin\latupvav\latupre}{istişâre, işâret, meşveret, müsteşâr, müşâvere, müşâvir, müşîr, şûrâ}{kok:shin_vav_re}
\end{multicols}
\dictchapter{\latupta}
\begin{multicols}{2}
\kokkipentry{\latupta\latupre\latupfe}{etraf, taraf}{kok:ta_re_fe}
\end{multicols}
\dictchapter{\latupvav}
\begin{multicols}{2}
\kokkipentry{\latupvav\latupcim\latupbe}{îcâp, mûcip, vâcip, vecîbe}{kok:vav_cim_be}
\kokkipentry{\latupvav\latupcim\latupdal}{îcat, mevcut, mûcit, vecd, vicdan, vücut}{kok:vav_cim_dal}
\kokkipentry{\latupvav\latupha\latupdal}{ittihât, muvahhit, müttehit, tevhid, vahdet}{kok:vav_ha_dal}
\kokkipentry{\latupvav\latupkaf\latupayn}{îkâ, mevkî, vakâ, vâkıa, vâkî, vekâyî, vukû, vukuat}{kok:vav_kaf_ayn}
\kokkipentry{\latupvav\latuplam\latupdal}{evlat, mevlit, mütevellit, müvellet, tevellüt, tevlit, vâlide, velâdet, velet, velût}{kok:vav_lam_dal}
\kokkipentry{\latupvav\latuplam\latupye}{evlâ, evliyâ, istilâ, mevâli, mevlâ, molla, müstevlî, mütevelli, vâli, velâyet, veli, vilâyet}{kok:vav_lam_ye}
\kokkipentry{\latupvav\latupre\latupxa}{müverrih, târih, tevârih}{kok:vav_re_xa}
\kokkipentry{\latupvav\latupre\latupkaf}{evrak, varak}{kok:vav_re_kaf}
\kokkipentry{\latupvav\latupre\latupthe}{irs, mîras, mûris, tevârüs, vâris, verâset, verese}{kok:vav_re_the}
\kokkipentry{\latupvav\latupsad\latuplam}{ittisâl, muttasıl, muvâsalat, sıla, vâsıl, visâl, vuslat}{kok:vav_sad_lam}
\kokkipentry{\latupvav\latupthe\latupkaf}{mevsuk, mîsak, tevsik, vesâik, vesika}{kok:vav_the_kaf}
\end{multicols}
\dictchapter{\latupze}
\begin{multicols}{2}
\kokkipentry{\latupze\latupye\latupdal}{mezât, müstezât, müzâyede, tezyit, zâit, ziyâde}{kok:ze_ye_dal}
\kokkipentry{\latupze\latupre\latupkaf}{mızrak, zerk}{kok:ze_re_kaf}
\end{multicols}
\dictchapter{\latupza}
\begin{multicols}{2}
\kokkipentry{\latupza\latuphe\latupre}{izhâr, mazhar, müzâheret, müzâhir, tezâhür, zâhir, zehrâ, zevâhir, zuhûr, zührevî}{kok:za_he_re}
\kokkipentry{\latupza\latuplam\latupmim}{mazlum, mezâlim, zâlim, zulmet, zulüm}{kok:za_lam_mim}
\end{multicols}



\chapter{Çekimler Sözlüğü}
\newpage

\dictchapter{A}
\begin{multicols}{3}
\kipkokentry{adâlet}{\latupayn\latupdal\latuplam}{None}{kok:ayn_dal_lam}
\kipkokentry{âdil}{\latupayn\latupdal\latuplam}{K\rom{1}, Ed.}{kok:ayn_dal_lam}
\kipkokentry{adl}{\latupayn\latupdal\latuplam}{None}{kok:ayn_dal_lam}
\kipkokentry{adliye}{\latupayn\latupdal\latuplam}{None}{kok:ayn_dal_lam}
\kipkokentry{ahval}{\latupha\latupvav\latuplam}{None}{kok:ha_vav_lam}
\kipkokentry{akdem}{\latupkaf\latupdal\latupmim}{None}{kok:kaf_dal_mim}
\kipkokentry{aksam}{\latupkaf\latupsin\latupmim}{None}{kok:kaf_sin_mim}
\kipkokentry{akvam}{\latupkaf\latupvav\latupmim}{None}{kok:kaf_vav_mim}
\kipkokentry{alem}{\latupayn\latuplam\latupmim}{None}{kok:ayn_lam_mim}
\kipkokentry{âlem}{\latupayn\latuplam\latupmim}{None}{kok:ayn_lam_mim}
\kipkokentry{âlim}{\latupayn\latuplam\latupmim}{K\rom{1}, Ed.}{kok:ayn_lam_mim}
\kipkokentry{allâme}{\latupayn\latuplam\latupmim}{None}{kok:ayn_lam_mim}
\kipkokentry{araf}{\latupayn\latupre\latupfe}{None}{kok:ayn_re_fe}
\kipkokentry{araz}{\latupayn\latupre\latupdad}{None}{kok:ayn_re_dad}
\kipkokentry{ârıza}{\latupayn\latupre\latupdad}{None}{kok:ayn_re_dad}
\kipkokentry{ârız}{\latupayn\latupre\latupdad}{K\rom{1}, Ed.}{kok:ayn_re_dad}
\kipkokentry{ârif}{\latupayn\latupre\latupfe}{K\rom{1}, Ed.}{kok:ayn_re_fe}
\kipkokentry{arife}{\latupayn\latupre\latupfe}{None}{kok:ayn_re_fe}
\kipkokentry{arîza}{\latupayn\latupre\latupdad}{None}{kok:ayn_re_dad}
\kipkokentry{aruz}{\latupayn\latupre\latupdad}{None}{kok:ayn_re_dad}
\kipkokentry{arz}{\latupayn\latupre\latupdad}{None}{kok:ayn_re_dad}
\kipkokentry{âşar}{\latupayn\latupshin\latupre}{None}{kok:ayn_shin_re}
\kipkokentry{aşîret}{\latupayn\latupshin\latupre}{None}{kok:ayn_shin_re}
\kipkokentry{aşûre}{\latupayn\latupshin\latupre}{None}{kok:ayn_shin_re}
\kipkokentry{avârız}{\latupayn\latupre\latupdad}{None}{kok:ayn_re_dad}
\end{multicols}
\dictchapter{B}
\begin{multicols}{3}
\kipkokentry{bakaya}{\latupbe\latupkaf\latupye}{None}{kok:be_kaf_ye}
\kipkokentry{bâki}{\latupbe\latupkaf\latupye}{K\rom{1}, Ed.}{kok:be_kaf_ye}
\kipkokentry{bakiye}{\latupbe\latupkaf\latupye}{None}{kok:be_kaf_ye}
\kipkokentry{bâliğ}{\latupbe\latuplam\latupgayn}{K\rom{1}, Ed.}{kok:be_lam_gayn}
\kipkokentry{bekâ}{\latupbe\latupkaf\latupye}{None}{kok:be_kaf_ye}
\kipkokentry{belâgat}{\latupbe\latuplam\latupgayn}{None}{kok:be_lam_gayn}
\kipkokentry{belîğ}{\latupbe\latuplam\latupgayn}{None}{kok:be_lam_gayn}
\kipkokentry{bülûğ}{\latupbe\latuplam\latupgayn}{None}{kok:be_lam_gayn}
\end{multicols}
\dictchapter{C}
\begin{multicols}{3}
\kipkokentry{câmiâ}{\latupcim\latupmim\latupayn}{None}{kok:cim_mim_ayn}
\kipkokentry{câmi}{\latupcim\latupmim\latupayn}{K\rom{1}, Ed.}{kok:cim_mim_ayn}
\kipkokentry{celâdet}{\latupcim\latuplam\latupdal}{None}{kok:cim_lam_dal}
\kipkokentry{cellat}{\latupcim\latuplam\latupdal}{None}{kok:cim_lam_dal}
\kipkokentry{cem}{\latupcim\latupmim\latupayn}{None}{kok:cim_mim_ayn}
\kipkokentry{cemaat}{\latupcim\latupmim\latupayn}{None}{kok:cim_mim_ayn}
\kipkokentry{cemiyet}{\latupcim\latupmim\latupayn}{None}{kok:cim_mim_ayn}
\kipkokentry{cevap}{\latupcim\latupvav\latupbe}{None}{kok:cim_vav_be}
\kipkokentry{cilt}{\latupcim\latuplam\latupdal}{None}{kok:cim_lam_dal}
\kipkokentry{cîmâ}{\latupcim\latupmim\latupayn}{None}{kok:cim_mim_ayn}
\kipkokentry{cumâ}{\latupcim\latupmim\latupayn}{None}{kok:cim_mim_ayn}
\end{multicols}
\dictchapter{D}
\begin{multicols}{3}
\kipkokentry{dâir}{\latupdal\latupvav\latupre}{K\rom{1}, Ed.}{kok:dal_vav_re}
\kipkokentry{dâire}{\latupdal\latupvav\latupre}{None}{kok:dal_vav_re}
\kipkokentry{dâr}{\latupdal\latupvav\latupre}{None}{kok:dal_vav_re}
\kipkokentry{dâreyn}{\latupdal\latupvav\latupre}{None}{kok:dal_vav_re}
\kipkokentry{ders}{\latupdal\latupre\latupsin}{None}{kok:dal_re_sin}
\kipkokentry{devir}{\latupdal\latupvav\latupre}{None}{kok:dal_vav_re}
\kipkokentry{devrân}{\latupdal\latupvav\latupre}{None}{kok:dal_vav_re}
\kipkokentry{devre}{\latupdal\latupvav\latupre}{None}{kok:dal_vav_re}
\kipkokentry{devriye}{\latupdal\latupvav\latupre}{None}{kok:dal_vav_re}
\kipkokentry{deyr}{\latupdal\latupye\latupre}{None}{kok:dal_ye_re}
\kipkokentry{diyâr}{\latupdal\latupvav\latupre}{None}{kok:dal_vav_re}
\end{multicols}
\dictchapter{E}
\begin{multicols}{3}
\kipkokentry{edvar}{\latupdal\latupvav\latupre}{None}{kok:dal_vav_re}
\kipkokentry{efâl}{\latupfe\latupayn\latuplam}{None}{kok:fe_ayn_lam}
\kipkokentry{efkar}{\latupfe\latupkef\latupre}{None}{kok:fe_kef_re}
\kipkokentry{efrat}{\latupfe\latupre\latupdal}{None}{kok:fe_re_dal}
\kipkokentry{ehemmiyet}{\latuphe\latupmim\latupmim}{None}{kok:he_mim_mim}
\kipkokentry{ekmel}{\latupkef\latupmim\latuplam}{None}{kok:kef_mim_lam}
\kipkokentry{ekser}{\latupkef\latupthe\latupre}{None}{kok:kef_the_re}
\kipkokentry{emlak}{\latupmim\latuplam\latupkef}{None}{kok:mim_lam_kef}
\kipkokentry{emsal}{\latupmim\latupthe\latuplam}{None}{kok:mim_the_lam}
\kipkokentry{erzak}{\latupre\latupze\latupkaf}{None}{kok:re_ze_kaf}
\kipkokentry{esnâ}{\latupthe\latupnun\latupye}{None}{kok:the_nun_ye}
\kipkokentry{esnaf}{\latupsad\latupnun\latupfe}{None}{kok:sad_nun_fe}
\kipkokentry{esre}{\latupkaf\latupsin\latupre}{None}{kok:kaf_sin_re}
\kipkokentry{eşkâl}{\latupshin\latupkef\latuplam$^2$}{None}{kok:shin_kef_lam2}
\kipkokentry{eşraf}{\latupshin\latupre\latupfe}{None}{kok:shin_re_fe}
\kipkokentry{eşref}{\latupshin\latupre\latupfe}{None}{kok:shin_re_fe}
\kipkokentry{etraf}{\latupta\latupre\latupfe}{None}{kok:ta_re_fe}
\kipkokentry{evlâ}{\latupvav\latuplam\latupye}{None}{kok:vav_lam_ye}
\kipkokentry{evlat}{\latupvav\latuplam\latupdal}{None}{kok:vav_lam_dal}
\kipkokentry{evliyâ}{\latupvav\latuplam\latupye}{None}{kok:vav_lam_ye}
\kipkokentry{evrak}{\latupvav\latupre\latupkaf}{None}{kok:vav_re_kaf}
\end{multicols}
\dictchapter{F}
\begin{multicols}{3}
\kipkokentry{faal}{\latupfe\latupayn\latuplam}{None}{kok:fe_ayn_lam}
\kipkokentry{faaliyet}{\latupfe\latupayn\latuplam}{None}{kok:fe_ayn_lam}
\kipkokentry{fâhiş}{\latupfe\latupha\latupshin}{K\rom{1}, Ed.}{kok:fe_ha_shin}
\kipkokentry{fâhişe}{\latupfe\latupha\latupshin}{None}{kok:fe_ha_shin}
\kipkokentry{fahrî}{\latupfe\latupxa\latupre}{None}{kok:fe_xa_re}
\kipkokentry{fâik}{\latupfe\latupvav\latupkaf}{K\rom{1}, Ed.}{kok:fe_vav_kaf}
\kipkokentry{fâil}{\latupfe\latupayn\latuplam}{K\rom{1}, Ed.}{kok:fe_ayn_lam}
\kipkokentry{fâiz}{\latupfe\latupte\latupdad}{K\rom{1}, Ed.}{kok:fe_te_dad}
\kipkokentry{fakir}{\latupfe\latupkaf\latupre}{None}{kok:fe_kaf_re}
\kipkokentry{fakr}{\latupfe\latupkaf\latupre}{None}{kok:fe_kaf_re}
\kipkokentry{fârika}{\latupfe\latupre\latupkaf}{None}{kok:fe_re_kaf}
\kipkokentry{fark}{\latupfe\latupre\latupkaf}{None}{kok:fe_re_kaf}
\kipkokentry{fâsit}{\latupfe\latupsin\latupdal}{K\rom{1}, Ed.}{kok:fe_sin_dal}
\kipkokentry{faş}{\latupfe\latupshin\latupvav}{None}{kok:fe_shin_vav}
\kipkokentry{ferik}{\latupfe\latupre\latupkaf}{None}{kok:fe_re_kaf}
\kipkokentry{fert}{\latupfe\latupre\latupdal}{None}{kok:fe_re_dal}
\kipkokentry{fesat}{\latupfe\latupsin\latupdal}{None}{kok:fe_sin_dal}
\kipkokentry{fesih}{\latupfe\latupsin\latupxa}{None}{kok:fe_sin_xa}
\kipkokentry{fevk}{\latupfe\latupvav\latupkaf}{None}{kok:fe_vav_kaf}
\kipkokentry{feyiz}{\latupfe\latupte\latupdad}{None}{kok:fe_te_dad}
\kipkokentry{fıkra}{\latupfe\latupkaf\latupre}{None}{kok:fe_kaf_re}
\kipkokentry{fırka}{\latupfe\latupre\latupkaf}{None}{kok:fe_re_kaf}
\kipkokentry{fıtrat}{\latupfe\latupta\latupre}{None}{kok:fe_ta_re}
\kipkokentry{fiil}{\latupfe\latupayn\latuplam}{None}{kok:fe_ayn_lam}
\kipkokentry{fikir}{\latupfe\latupkef\latupre}{None}{kok:fe_kef_re}
\kipkokentry{firâk}{\latupfe\latupre\latupkaf}{None}{kok:fe_re_kaf}
\kipkokentry{firkat}{\latupfe\latupre\latupkaf}{None}{kok:fe_re_kaf}
\kipkokentry{fitre}{\latupfe\latupta\latupre}{None}{kok:fe_ta_re}
\kipkokentry{fuhuş}{\latupfe\latupha\latupshin}{None}{kok:fe_ha_shin}
\kipkokentry{fukarâ}{\latupfe\latupkaf\latupre}{None}{kok:fe_kaf_re}
\kipkokentry{furkân}{\latupfe\latupre\latupkaf}{None}{kok:fe_re_kaf}
\end{multicols}
\dictchapter{G}
\begin{multicols}{3}
\kipkokentry{gaddar}{\latupgayn\latupdal\latupre}{None}{kok:gayn_dal_re}
\kipkokentry{gadir}{\latupgayn\latupdal\latupre}{None}{kok:gayn_dal_re}
\kipkokentry{gâfil}{\latupgayn\latupfe\latuplam}{K\rom{1}, Ed.}{kok:gayn_fe_lam}
\kipkokentry{gaflet}{\latupgayn\latupfe\latuplam}{None}{kok:gayn_fe_lam}
\kipkokentry{galebe}{\latupgayn\latuplam\latupbe}{None}{kok:gayn_lam_be}
\kipkokentry{gâlibâ}{\latupgayn\latuplam\latupbe}{None}{kok:gayn_lam_be}
\kipkokentry{gâlibiyet}{\latupgayn\latuplam\latupbe}{None}{kok:gayn_lam_be}
\kipkokentry{gâlip}{\latupgayn\latuplam\latupbe}{K\rom{1}, Ed.}{kok:gayn_lam_be}
\kipkokentry{ganî}{\latupgayn\latupnun\latupye}{None}{kok:gayn_nun_ye}
\kipkokentry{garâib}{\latupgayn\latupre\latupbe}{None}{kok:gayn_re_be}
\kipkokentry{garibân}{\latupgayn\latupre\latupbe}{None}{kok:gayn_re_be}
\kipkokentry{garîp}{\latupgayn\latupre\latupbe}{None}{kok:gayn_re_be}
\kipkokentry{garp}{\latupgayn\latupre\latupbe}{None}{kok:gayn_re_be}
\kipkokentry{gınâ}{\latupgayn\latupnun\latupye}{None}{kok:gayn_nun_ye}
\kipkokentry{gurbet}{\latupgayn\latupre\latupbe}{None}{kok:gayn_re_be}
\kipkokentry{gurebâ}{\latupgayn\latupre\latupbe}{None}{kok:gayn_re_be}
\kipkokentry{gurûp}{\latupgayn\latupre\latupbe}{None}{kok:gayn_re_be}
\end{multicols}
\dictchapter{H}
\begin{multicols}{3}
\kipkokentry{haber}{\latupxa\latupbe\latupre}{None}{kok:xa_be_re}
\kipkokentry{hak}{\latupha\latupkaf\latupkaf}{None}{kok:ha_kaf_kaf}
\kipkokentry{hakîkat}{\latupha\latupkaf\latupkaf}{None}{kok:ha_kaf_kaf}
\kipkokentry{hakkâk}{\latupha\latupkaf\latupkaf}{None}{kok:ha_kaf_kaf}
\kipkokentry{hal}{\latupha\latupvav\latuplam}{None}{kok:ha_vav_lam}
\kipkokentry{hâlâ}{\latupha\latupvav\latuplam}{None}{kok:ha_vav_lam}
\kipkokentry{halef}{\latupxa\latuplam\latupfe}{None}{kok:xa_lam_fe}
\kipkokentry{hâlen}{\latupha\latupvav\latuplam}{None}{kok:ha_vav_lam}
\kipkokentry{hâlet}{\latupha\latupvav\latuplam}{None}{kok:ha_vav_lam}
\kipkokentry{halîfe}{\latupxa\latuplam\latupfe}{None}{kok:xa_lam_fe}
\kipkokentry{hamal}{\latupha\latupmim\latuplam}{None}{kok:ha_mim_lam}
\kipkokentry{hamam}{\latupha\latupmim\latupmim}{None}{kok:ha_mim_mim}
\kipkokentry{hâmil}{\latupha\latupmim\latuplam}{K\rom{1}, Ed.}{kok:ha_mim_lam}
\kipkokentry{hâmile}{\latupha\latupmim\latuplam}{None}{kok:ha_mim_lam}
\kipkokentry{hamiyet}{\latupha\latupmim\latupvav}{None}{kok:ha_mim_vav}
\kipkokentry{haml}{\latupha\latupmim\latuplam}{None}{kok:ha_mim_lam}
\kipkokentry{hamle}{\latupha\latupmim\latuplam}{None}{kok:ha_mim_lam}
\kipkokentry{hamûle}{\latupha\latupmim\latuplam}{None}{kok:ha_mim_lam}
\kipkokentry{harâm}{\latupha\latupre\latupmim}{None}{kok:ha_re_mim}
\kipkokentry{harbi}{\latupha\latupre\latupbe}{None}{kok:ha_re_be}
\kipkokentry{harem}{\latupha\latupre\latupmim}{None}{kok:ha_re_mim}
\kipkokentry{hârim}{\latupha\latupre\latupmim}{K\rom{1}, Ed.}{kok:ha_re_mim}
\kipkokentry{hâris}{\latupha\latupre\latupsad}{K\rom{1}, Ed.}{kok:ha_re_sad}
\kipkokentry{harp}{\latupha\latupre\latupbe}{None}{kok:ha_re_be}
\kipkokentry{has}{\latupxa\latupsad\latupsad}{K\rom{1}, Ed.}{kok:xa_sad_sad}
\kipkokentry{hasep}{\latupha\latupsin\latupbe}{None}{kok:ha_sin_be}
\kipkokentry{hasır}{\latupha\latupsad\latupre}{None}{kok:ha_sad_re}
\kipkokentry{hâsip}{\latupha\latupsin\latupbe}{K\rom{1}, Ed.}{kok:ha_sin_be}
\kipkokentry{hasr}{\latupha\latupsad\latupre}{None}{kok:ha_sad_re}
\kipkokentry{hassa}{\latupxa\latupsad\latupsad}{None}{kok:xa_sad_sad}
\kipkokentry{haşema}{\latupha\latupshin\latupmim}{None}{kok:ha_shin_mim}
\kipkokentry{haşmet}{\latupha\latupshin\latupmim}{None}{kok:ha_shin_mim}
\kipkokentry{havâle}{\latupha\latupvav\latuplam}{None}{kok:ha_vav_lam}
\kipkokentry{havâli}{\latupha\latupvav\latuplam}{None}{kok:ha_vav_lam}
\kipkokentry{havâs}{\latupxa\latupsad\latupsad}{None}{kok:xa_sad_sad}
\kipkokentry{hayâl}{\latupxa\latupye\latuplam}{None}{kok:xa_ye_lam}
\kipkokentry{hayâlet}{\latupxa\latupye\latuplam}{None}{kok:xa_ye_lam}
\kipkokentry{hayır}{\latupxa\latupye\latupre}{None}{kok:xa_ye_re}
\kipkokentry{hesap}{\latupha\latupsin\latupbe}{None}{kok:ha_sin_be}
\kipkokentry{hırs}{\latupha\latupre\latupsad}{None}{kok:ha_re_sad}
\kipkokentry{hilaf}{\latupxa\latuplam\latupfe}{None}{kok:xa_lam_fe}
\kipkokentry{hilâfet}{\latupxa\latuplam\latupfe}{None}{kok:xa_lam_fe}
\kipkokentry{hile}{\latupha\latupvav\latuplam}{None}{kok:ha_vav_lam}
\kipkokentry{himmet}{\latuphe\latupmim\latupmim}{None}{kok:he_mim_mim}
\kipkokentry{hisar}{\latupha\latupsad\latupre}{None}{kok:ha_sad_re}
\kipkokentry{hisse}{\latupxa\latupsad\latupsad}{None}{kok:xa_sad_sad}
\kipkokentry{hokka}{\latupha\latupkaf\latupkaf}{None}{kok:ha_kaf_kaf}
\kipkokentry{hukuk}{\latupha\latupkaf\latupkaf}{None}{kok:ha_kaf_kaf}
\kipkokentry{hummâ}{\latupha\latupmim\latupmim}{None}{kok:ha_mim_mim}
\kipkokentry{husus}{\latupxa\latupsad\latupsad}{None}{kok:xa_sad_sad}
\kipkokentry{hürmet}{\latupha\latupre\latupmim}{None}{kok:ha_re_mim}
\end{multicols}
\dictchapter{I}
\begin{multicols}{3}
\kipkokentry{ırk}{\latupayn\latupre\latupkaf}{None}{kok:ayn_re_kaf}
\kipkokentry{ırz}{\latupayn\latupre\latupdad}{None}{kok:ayn_re_dad}
\end{multicols}
\dictchapter{İ}
\begin{multicols}{3}
\kipkokentry{ibâre}{\latupayn\latupbe\latupre}{None}{kok:ayn_be_re}
\kipkokentry{ibâret}{\latupayn\latupbe\latupre}{None}{kok:ayn_be_re}
\kipkokentry{ibkâ}{\latupbe\latupkaf\latupye}{None}{kok:be_kaf_ye}
\kipkokentry{iblâğ}{\latupbe\latuplam\latupgayn}{None}{kok:be_lam_gayn}
\kipkokentry{ibret}{\latupayn\latupbe\latupre}{None}{kok:ayn_be_re}
\kipkokentry{icâbet}{\latupcim\latupvav\latupbe}{None}{kok:cim_vav_be}
\kipkokentry{îcâp}{\latupvav\latupcim\latupbe}{None}{kok:vav_cim_be}
\kipkokentry{icmâ}{\latupcim\latupmim\latupayn}{None}{kok:cim_mim_ayn}
\kipkokentry{içtimâ}{\latupcim\latupmim\latupayn}{None}{kok:cim_mim_ayn}
\kipkokentry{idâre}{\latupdal\latupvav\latupre}{None}{kok:dal_vav_re}
\kipkokentry{ifrâz}{\latupfe\latupre\latupze}{None}{kok:fe_re_ze}
\kipkokentry{ifrâzat}{\latupfe\latupre\latupze}{None}{kok:fe_re_ze}
\kipkokentry{ifsât}{\latupfe\latupsin\latupdal}{None}{kok:fe_sin_dal}
\kipkokentry{ifşâ}{\latupfe\latupshin\latupvav}{None}{kok:fe_shin_vav}
\kipkokentry{iftâr}{\latupfe\latupta\latupre}{None}{kok:fe_ta_re}
\kipkokentry{iftihâr}{\latupfe\latupxa\latupre}{None}{kok:fe_xa_re}
\kipkokentry{iğfâl}{\latupgayn\latupfe\latuplam}{None}{kok:gayn_fe_lam}
\kipkokentry{ihâle}{\latupha\latupvav\latuplam}{None}{kok:ha_vav_lam}
\kipkokentry{ihbâr}{\latupxa\latupbe\latupre}{None}{kok:xa_be_re}
\kipkokentry{ihkâk}{\latupha\latupkaf\latupkaf}{None}{kok:ha_kaf_kaf}
\kipkokentry{ihrâm}{\latupha\latupre\latupmim}{None}{kok:ha_re_mim}
\kipkokentry{ihtilâf}{\latupxa\latuplam\latupfe}{None}{kok:xa_lam_fe}
\kipkokentry{ihtimâl}{\latupha\latupmim\latuplam}{None}{kok:ha_mim_lam}
\kipkokentry{ihtimâm}{\latuphe\latupmim\latupmim}{None}{kok:he_mim_mim}
\kipkokentry{ihtirâm}{\latupha\latupre\latupmim}{None}{kok:ha_re_mim}
\kipkokentry{ihtirâs}{\latupha\latupre\latupsad}{None}{kok:ha_re_sad}
\kipkokentry{ihtisâp}{\latupha\latupsin\latupbe}{None}{kok:ha_sin_be}
\kipkokentry{ihtisâs}{\latupxa\latupsad\latupsad}{None}{kok:xa_sad_sad}
\kipkokentry{ihtişâm}{\latupha\latupshin\latupmim}{None}{kok:ha_shin_mim}
\kipkokentry{ihtiyâr}{\latupxa\latupye\latupre}{None}{kok:xa_ye_re}
\kipkokentry{îkâ}{\latupvav\latupkaf\latupayn}{None}{kok:vav_kaf_ayn}
\kipkokentry{ikâme}{\latupkaf\latupvav\latupmim}{None}{kok:kaf_vav_mim}
\kipkokentry{ikâmet}{\latupkaf\latupvav\latupmim}{None}{kok:kaf_vav_mim}
\kipkokentry{ikbâl}{\latupkaf\latupbe\latuplam}{None}{kok:kaf_be_lam}
\kipkokentry{ikmâl}{\latupkef\latupmim\latuplam}{None}{kok:kef_mim_lam}
\kipkokentry{iktidâr}{\latupkaf\latupdal\latupre}{None}{kok:kaf_dal_re}
\kipkokentry{iktifâ}{\latupkef\latupfe\latupvav}{None}{kok:kef_fe_vav}
\kipkokentry{ilâm}{\latupayn\latuplam\latupmim}{None}{kok:ayn_lam_mim}
\kipkokentry{ilhâk}{\latuplam\latupha\latupkaf}{None}{kok:lam_ha_kaf}
\kipkokentry{ilim}{\latupayn\latuplam\latupmim}{None}{kok:ayn_lam_mim}
\kipkokentry{iltihâk}{\latuplam\latupha\latupkaf}{None}{kok:lam_ha_kaf}
\kipkokentry{infiâl}{\latupfe\latupayn\latuplam}{None}{kok:fe_ayn_lam}
\kipkokentry{infirât}{\latupfe\latupre\latupdal}{None}{kok:fe_re_dal}
\kipkokentry{infisâh}{\latupfe\latupsin\latupxa}{None}{kok:fe_sin_xa}
\kipkokentry{inhisâr}{\latupha\latupsad\latupre}{None}{kok:ha_sad_re}
\kipkokentry{inkılâp}{\latupkaf\latuplam\latupbe}{None}{kok:kaf_lam_be}
\kipkokentry{inkisâr}{\latupkaf\latupsin\latupre}{None}{kok:kaf_sin_re}
\kipkokentry{intikâm}{\latupnun\latupkaf\latupmim}{None}{kok:nun_kaf_mim}
\kipkokentry{intizâr}{\latupnun\latupza\latupre}{None}{kok:nun_za_re}
\kipkokentry{irfan}{\latupayn\latupre\latupfe}{None}{kok:ayn_re_fe}
\kipkokentry{irs}{\latupvav\latupre\latupthe}{None}{kok:vav_re_the}
\kipkokentry{irtifâ}{\latupre\latupfe\latupayn}{None}{kok:re_fe_ayn}
\kipkokentry{isâbet}{\latupsad\latupvav\latupbe}{None}{kok:sad_vav_be}
\kipkokentry{iskân}{\latupsin\latupkef\latupnun}{None}{kok:sin_kef_nun}
\kipkokentry{islâm}{\latupsin\latuplam\latupmim}{None}{kok:sin_lam_mim}
\kipkokentry{islâmiyet}{\latupsin\latuplam\latupmim}{None}{kok:sin_lam_mim}
\kipkokentry{isticvâp}{\latupcim\latupvav\latupbe}{None}{kok:cim_vav_be}
\kipkokentry{istiğnâ}{\latupgayn\latupnun\latupye}{None}{kok:gayn_nun_ye}
\kipkokentry{istihâle}{\latupha\latupvav\latuplam}{None}{kok:ha_vav_lam}
\kipkokentry{istihbârat}{\latupxa\latupbe\latupre}{None}{kok:xa_be_re}
\kipkokentry{istihkâk}{\latupha\latupkaf\latupkaf}{None}{kok:ha_kaf_kaf}
\kipkokentry{istikâmet}{\latupkaf\latupvav\latupmim}{None}{kok:kaf_vav_mim}
\kipkokentry{istikbâl}{\latupkaf\latupbe\latuplam}{None}{kok:kaf_be_lam}
\kipkokentry{istilâ}{\latupvav\latuplam\latupye}{None}{kok:vav_lam_ye}
\kipkokentry{istimlâk}{\latupmim\latuplam\latupkef}{None}{kok:mim_lam_kef}
\kipkokentry{istintâk}{\latupnun\latupta\latupkaf}{None}{kok:nun_ta_kaf}
\kipkokentry{istirhâm}{\latupre\latupha\latupmim}{None}{kok:re_ha_mim}
\kipkokentry{istisnâ}{\latupthe\latupnun\latupye}{None}{kok:the_nun_ye}
\kipkokentry{istişâre}{\latupshin\latupvav\latupre}{None}{kok:shin_vav_re}
\kipkokentry{işâret}{\latupshin\latupvav\latupre}{None}{kok:shin_vav_re}
\kipkokentry{işgâl}{\latupshin\latupgayn\latuplam}{None}{kok:shin_gayn_lam}
\kipkokentry{işkil}{\latupshin\latupkef\latuplam$^1$}{None}{kok:shin_kef_lam1}
\kipkokentry{işret}{\latupayn\latupshin\latupre}{None}{kok:ayn_shin_re}
\kipkokentry{iştigâl}{\latupshin\latupgayn\latuplam}{None}{kok:shin_gayn_lam}
\kipkokentry{iştiyak}{\latupshin\latupvav\latupkaf}{None}{kok:shin_vav_kaf}
\kipkokentry{îtibâr}{\latupayn\latupbe\latupre}{None}{kok:ayn_be_re}
\kipkokentry{îtidâl}{\latupayn\latupdal\latuplam}{None}{kok:ayn_dal_lam}
\kipkokentry{îtiraf}{\latupayn\latupre\latupfe}{None}{kok:ayn_re_fe}
\kipkokentry{îtiraz}{\latupayn\latupre\latupdad}{None}{kok:ayn_re_dad}
\kipkokentry{îtizâr}{\latupayn\latupzel\latupre}{None}{kok:ayn_zel_re}
\kipkokentry{ittihât}{\latupvav\latupha\latupdal}{None}{kok:vav_ha_dal}
\kipkokentry{izhâr}{\latupza\latuphe\latupre}{None}{kok:za_he_re}
\end{multicols}
\dictchapter{K}
\begin{multicols}{3}
\kipkokentry{kabahat}{\latupkaf\latupbe\latupha}{None}{kok:kaf_be_ha}
\kipkokentry{kabala}{\latupkaf\latupbe\latuplam}{None}{kok:kaf_be_lam}
\kipkokentry{kabîl}{\latupkaf\latupbe\latuplam}{None}{kok:kaf_be_lam}
\kipkokentry{kâbil}{\latupkaf\latupbe\latuplam}{K\rom{1}, Ed.}{kok:kaf_be_lam}
\kipkokentry{kabîle}{\latupkaf\latupbe\latuplam}{None}{kok:kaf_be_lam}
\kipkokentry{kâbiliyet}{\latupkaf\latupbe\latuplam}{None}{kok:kaf_be_lam}
\kipkokentry{kabûl}{\latupkaf\latupbe\latuplam}{None}{kok:kaf_be_lam}
\kipkokentry{kadar}{\latupkaf\latupdal\latupre}{None}{kok:kaf_dal_re}
\kipkokentry{kadem}{\latupkaf\latupdal\latupmim}{None}{kok:kaf_dal_mim}
\kipkokentry{kademe}{\latupkaf\latupdal\latupmim}{None}{kok:kaf_dal_mim}
\kipkokentry{kader}{\latupkaf\latupdal\latupre}{None}{kok:kaf_dal_re}
\kipkokentry{kadim}{\latupkaf\latupdal\latupmim}{None}{kok:kaf_dal_mim}
\kipkokentry{kadir}{\latupkaf\latupdal\latupre}{None}{kok:kaf_dal_re}
\kipkokentry{kadir}{\latupkaf\latupdal\latupre}{K\rom{1}, Ed.}{kok:kaf_dal_re}
\kipkokentry{kâfî}{\latupkef\latupfe\latupvav}{K\rom{1}, Ed.}{kok:kef_fe_vav}
\kipkokentry{kâfir}{\latupkef\latupfe\latupre}{K\rom{1}, Ed.}{kok:kef_fe_re}
\kipkokentry{kâim}{\latupkaf\latupvav\latupmim}{K\rom{1}, Ed.}{kok:kaf_vav_mim}
\kipkokentry{kalfa}{\latupxa\latuplam\latupfe}{None}{kok:xa_lam_fe}
\kipkokentry{kalp}{\latupkaf\latuplam\latupbe}{None}{kok:kaf_lam_be}
\kipkokentry{kamet}{\latupkaf\latupvav\latupmim}{None}{kok:kaf_vav_mim}
\kipkokentry{kâmil}{\latupkef\latupmim\latuplam}{K\rom{1}, Ed.}{kok:kef_mim_lam}
\kipkokentry{kasem}{\latupkaf\latupsin\latupmim}{None}{kok:kaf_sin_mim}
\kipkokentry{kâsım}{\latupkaf\latupsin\latupmim}{K\rom{1}, Ed.}{kok:kaf_sin_mim}
\kipkokentry{kâtil}{\latupkaf\latupte\latuplam}{K\rom{1}, Ed.}{kok:kaf_te_lam}
\kipkokentry{katil}{\latupkaf\latupte\latuplam}{None}{kok:kaf_te_lam}
\kipkokentry{kâtip}{\latupkef\latupte\latupbe}{K\rom{1}, Ed.}{kok:kef_te_be}
\kipkokentry{kavim}{\latupkaf\latupvav\latupmim}{None}{kok:kaf_vav_mim}
\kipkokentry{kâyme}{\latupkaf\latupvav\latupmim}{None}{kok:kaf_vav_mim}
\kipkokentry{kayyım}{\latupkaf\latupvav\latupmim}{None}{kok:kaf_vav_mim}
\kipkokentry{kayyûm}{\latupkaf\latupvav\latupmim}{None}{kok:kaf_vav_mim}
\kipkokentry{kefâret}{\latupkef\latupfe\latupre}{None}{kok:kef_fe_re}
\kipkokentry{kefere}{\latupkef\latupfe\latupre}{None}{kok:kef_fe_re}
\kipkokentry{kelam}{\latupkef\latuplam\latupmim}{None}{kok:kef_lam_mim}
\kipkokentry{kelime}{\latupkef\latuplam\latupmim}{None}{kok:kef_lam_mim}
\kipkokentry{kemâl}{\latupkef\latupmim\latuplam}{None}{kok:kef_mim_lam}
\kipkokentry{kesir}{\latupkaf\latupsin\latupre}{None}{kok:kaf_sin_re}
\kipkokentry{kesret}{\latupkef\latupthe\latupre}{None}{kok:kef_the_re}
\kipkokentry{ketebe}{\latupkef\latupte\latupbe}{None}{kok:kef_te_be}
\kipkokentry{kevser}{\latupkef\latupthe\latupre}{None}{kok:kef_the_re}
\kipkokentry{kıble}{\latupkaf\latupbe\latuplam}{None}{kok:kaf_be_lam}
\kipkokentry{kıdem}{\latupkaf\latupdal\latupmim}{None}{kok:kaf_dal_mim}
\kipkokentry{kısım}{\latupkaf\latupsin\latupmim}{None}{kok:kaf_sin_mim}
\kipkokentry{kısmet}{\latupkaf\latupsin\latupmim}{None}{kok:kaf_sin_mim}
\kipkokentry{kıtâl}{\latupkaf\latupte\latuplam}{None}{kok:kaf_te_lam}
\kipkokentry{kıvâm}{\latupkaf\latupvav\latupmim}{None}{kok:kaf_vav_mim}
\kipkokentry{kıyâm}{\latupkaf\latupvav\latupmim}{None}{kok:kaf_vav_mim}
\kipkokentry{kıyâmet}{\latupkaf\latupvav\latupmim}{None}{kok:kaf_vav_mim}
\kipkokentry{kıyas}{\latupkaf\latupye\latupsin}{None}{kok:kaf_ye_sin}
\kipkokentry{kıymet}{\latupkaf\latupvav\latupmim}{None}{kok:kaf_vav_mim}
\kipkokentry{kifâyet}{\latupkef\latupfe\latupvav}{None}{kok:kef_fe_vav}
\kipkokentry{kitâbe}{\latupkef\latupte\latupbe}{None}{kok:kef_te_be}
\kipkokentry{kitâbet}{\latupkef\latupte\latupbe}{None}{kok:kef_te_be}
\kipkokentry{kitâbiyat}{\latupkef\latupte\latupbe}{None}{kok:kef_te_be}
\kipkokentry{kitap}{\latupkef\latupte\latupbe}{None}{kok:kef_te_be}
\kipkokentry{kudret}{\latupkaf\latupdal\latupre}{None}{kok:kaf_dal_re}
\kipkokentry{kudüm}{\latupkaf\latupdal\latupmim}{None}{kok:kaf_dal_mim}
\kipkokentry{kulp}{\latupkaf\latuplam\latupbe}{None}{kok:kaf_lam_be}
\kipkokentry{küffâr}{\latupkef\latupfe\latupre}{None}{kok:kef_fe_re}
\kipkokentry{küfrân}{\latupkef\latupfe\latupre}{None}{kok:kef_fe_re}
\kipkokentry{küfür}{\latupkef\latupfe\latupre}{None}{kok:kef_fe_re}
\kipkokentry{külfet}{\latupkef\latuplam\latupfe}{None}{kok:kef_lam_fe}
\kipkokentry{küsûr}{\latupkaf\latupsin\latupre}{None}{kok:kaf_sin_re}
\kipkokentry{küttab}{\latupkef\latupte\latupbe}{None}{kok:kef_te_be}
\kipkokentry{kütüphane}{\latupkef\latupte\latupbe}{None}{kok:kef_te_be}
\end{multicols}
\dictchapter{L}
\begin{multicols}{3}
\kipkokentry{lafız}{\latuplam\latupfe\latupza}{None}{kok:lam_fe_za}
\kipkokentry{lâhika}{\latuplam\latupha\latupkaf}{None}{kok:lam_ha_kaf}
\kipkokentry{lezîz}{\latuplam\latupzel\latupzel}{None}{kok:lam_zel_zel}
\kipkokentry{lezzet}{\latuplam\latupzel\latupzel}{None}{kok:lam_zel_zel}
\end{multicols}
\dictchapter{M}
\begin{multicols}{3}
\kipkokentry{maarif}{\latupayn\latupre\latupfe}{None}{kok:ayn_re_fe}
\kipkokentry{mağdur}{\latupgayn\latupdal\latupre}{K\rom{1}, Edl.}{kok:gayn_dal_re}
\kipkokentry{mağlup}{\latupgayn\latuplam\latupbe}{K\rom{1}, Edl.}{kok:gayn_lam_be}
\kipkokentry{mağrip}{\latupgayn\latupre\latupbe}{None}{kok:gayn_re_be}
\kipkokentry{mahrem}{\latupha\latupre\latupmim}{None}{kok:ha_re_mim}
\kipkokentry{mahrum}{\latupha\latupre\latupmim}{K\rom{1}, Edl.}{kok:ha_re_mim}
\kipkokentry{mahsup}{\latupha\latupsin\latupbe}{K\rom{1}, Edl.}{kok:ha_sin_be}
\kipkokentry{mahsur}{\latupha\latupsad\latupre}{K\rom{1}, Edl.}{kok:ha_sad_re}
\kipkokentry{mahsus}{\latupxa\latupsad\latupsad}{K\rom{1}, Edl.}{kok:xa_sad_sad}
\kipkokentry{makâm}{\latupkaf\latupvav\latupmim}{None}{kok:kaf_vav_mim}
\kipkokentry{makbûl}{\latupkaf\latupbe\latuplam}{K\rom{1}, Edl.}{kok:kaf_be_lam}
\kipkokentry{maklûbe}{\latupkaf\latuplam\latupbe}{None}{kok:kaf_lam_be}
\kipkokentry{maksem}{\latupkaf\latupsin\latupmim}{None}{kok:kaf_sin_mim}
\kipkokentry{maktul}{\latupkaf\latupte\latuplam}{K\rom{1}, Edl.}{kok:kaf_te_lam}
\kipkokentry{mâlik}{\latupmim\latuplam\latupkef}{K\rom{1}, Ed.}{kok:mim_lam_kef}
\kipkokentry{mâlum}{\latupayn\latuplam\latupmim}{K\rom{1}, Edl.}{kok:ayn_lam_mim}
\kipkokentry{mantık}{\latupnun\latupta\latupkaf}{None}{kok:nun_ta_kaf}
\kipkokentry{manzara}{\latupnun\latupza\latupre}{None}{kok:nun_za_re}
\kipkokentry{maraza}{\latupayn\latupre\latupdad}{None}{kok:ayn_re_dad}
\kipkokentry{mârifet}{\latupayn\latupre\latupfe}{None}{kok:ayn_re_fe}
\kipkokentry{mâruf}{\latupayn\latupre\latupfe}{K\rom{1}, Edl.}{kok:ayn_re_fe}
\kipkokentry{mâruz}{\latupayn\latupre\latupdad}{K\rom{1}, Edl.}{kok:ayn_re_dad}
\kipkokentry{masal}{\latupmim\latupthe\latuplam}{None}{kok:mim_the_lam}
\kipkokentry{masraf}{\latupsad\latupre\latupfe}{None}{kok:sad_re_fe}
\kipkokentry{mâşerî}{\latupayn\latupshin\latupre}{None}{kok:ayn_shin_re}
\kipkokentry{mâzeret}{\latupayn\latupzel\latupre}{None}{kok:ayn_zel_re}
\kipkokentry{mazhar}{\latupza\latuphe\latupre}{None}{kok:za_he_re}
\kipkokentry{mâzur}{\latupayn\latupzel\latupre}{K\rom{1}, Edl.}{kok:ayn_zel_re}
\kipkokentry{meblağ}{\latupbe\latuplam\latupgayn}{None}{kok:be_lam_gayn}
\kipkokentry{mecmû}{\latupcim\latupmim\latupayn}{K\rom{1}, Edl.}{kok:cim_mim_ayn}
\kipkokentry{mecmuâ}{\latupcim\latupmim\latupayn}{None}{kok:cim_mim_ayn}
\kipkokentry{medâr}{\latupdal\latupvav\latupre}{None}{kok:dal_vav_re}
\kipkokentry{medrese}{\latupdal\latupre\latupsin}{None}{kok:dal_re_sin}
\kipkokentry{mefkûre}{\latupfe\latupkef\latupre}{None}{kok:fe_kef_re}
\kipkokentry{meful}{\latupfe\latupayn\latuplam}{K\rom{1}, Edl.}{kok:fe_ayn_lam}
\kipkokentry{mektep}{\latupkef\latupte\latupbe}{None}{kok:kef_te_be}
\kipkokentry{mektup}{\latupkef\latupte\latupbe}{K\rom{1}, Edl.}{kok:kef_te_be}
\kipkokentry{meleke}{\latupmim\latuplam\latupkef}{None}{kok:mim_lam_kef}
\kipkokentry{melik}{\latupmim\latuplam\latupkef}{K\rom{1}, Ed.}{kok:mim_lam_kef}
\kipkokentry{memâlik}{\latupmim\latuplam\latupkef}{None}{kok:mim_lam_kef}
\kipkokentry{memleket}{\latupmim\latuplam\latupkef}{None}{kok:mim_lam_kef}
\kipkokentry{memlük}{\latupmim\latuplam\latupkef}{K\rom{1}, Edl.}{kok:mim_lam_kef}
\kipkokentry{merhamet}{\latupre\latupha\latupmim}{None}{kok:re_ha_mim}
\kipkokentry{merhum}{\latupre\latupha\latupmim}{K\rom{1}, Edl.}{kok:re_ha_mim}
\kipkokentry{mesel}{\latupmim\latupthe\latuplam}{None}{kok:mim_the_lam}
\kipkokentry{meselâ}{\latupmim\latupthe\latuplam}{None}{kok:mim_the_lam}
\kipkokentry{mesele}{\latupsin\latuphemze\latuplam}{None}{kok:sin_hemze_lam}
\kipkokentry{meskun}{\latupsin\latupkef\latupnun}{K\rom{1}, Edl.}{kok:sin_kef_nun}
\kipkokentry{mesnevî}{\latupthe\latupnun\latupye}{None}{kok:the_nun_ye}
\kipkokentry{mesul}{\latupsin\latuphemze\latuplam}{K\rom{1}, Edl.}{kok:sin_hemze_lam}
\kipkokentry{meşgale}{\latupshin\latupgayn\latuplam}{None}{kok:shin_gayn_lam}
\kipkokentry{meşgul}{\latupshin\latupgayn\latuplam}{K\rom{1}, Edl.}{kok:shin_gayn_lam}
\kipkokentry{meşhut}{\latupshin\latuphe\latupdal}{K\rom{1}, Edl.}{kok:shin_he_dal}
\kipkokentry{meşveret}{\latupshin\latupvav\latupre}{None}{kok:shin_vav_re}
\kipkokentry{mevâli}{\latupvav\latuplam\latupye}{None}{kok:vav_lam_ye}
\kipkokentry{mevkî}{\latupvav\latupkaf\latupayn}{None}{kok:vav_kaf_ayn}
\kipkokentry{mevlâ}{\latupvav\latuplam\latupye}{None}{kok:vav_lam_ye}
\kipkokentry{mevlit}{\latupvav\latuplam\latupdal}{None}{kok:vav_lam_dal}
\kipkokentry{mevsuk}{\latupvav\latupthe\latupkaf}{K\rom{1}, Edl.}{kok:vav_the_kaf}
\kipkokentry{mezât}{\latupze\latupye\latupdal}{None}{kok:ze_ye_dal}
\kipkokentry{mızrak}{\latupze\latupre\latupkaf}{None}{kok:ze_re_kaf}
\kipkokentry{mihrap}{\latupha\latupre\latupbe}{None}{kok:ha_re_be}
\kipkokentry{miktar}{\latupkaf\latupdal\latupre}{None}{kok:kaf_dal_re}
\kipkokentry{mikyas}{\latupkaf\latupye\latupsin}{None}{kok:kaf_ye_sin}
\kipkokentry{minâre}{\latupnun\latupvav\latupre}{None}{kok:nun_vav_re}
\kipkokentry{minval}{\latupnun\latupvav\latuplam}{None}{kok:nun_vav_lam}
\kipkokentry{mîras}{\latupvav\latupre\latupthe}{None}{kok:vav_re_the}
\kipkokentry{mîsak}{\latupvav\latupthe\latupkaf}{None}{kok:vav_the_kaf}
\kipkokentry{misal}{\latupmim\latupthe\latuplam}{None}{kok:mim_the_lam}
\kipkokentry{misil}{\latupmim\latupthe\latuplam}{None}{kok:mim_the_lam}
\kipkokentry{molla}{\latupvav\latuplam\latupye}{None}{kok:vav_lam_ye}
\kipkokentry{muâdelet}{\latupayn\latupdal\latuplam}{None}{kok:ayn_dal_lam}
\kipkokentry{muâdil}{\latupayn\latupdal\latuplam}{None}{kok:ayn_dal_lam}
\kipkokentry{muallim}{\latupayn\latuplam\latupmim}{None}{kok:ayn_lam_mim}
\kipkokentry{muârız}{\latupayn\latupre\latupdad}{None}{kok:ayn_re_dad}
\kipkokentry{muâşeret}{\latupayn\latupshin\latupre}{None}{kok:ayn_shin_re}
\kipkokentry{mûcip}{\latupvav\latupcim\latupbe}{None}{kok:vav_cim_be}
\kipkokentry{muhâberat}{\latupxa\latupbe\latupre}{None}{kok:xa_be_re}
\kipkokentry{muhâbere}{\latupxa\latupbe\latupre}{None}{kok:xa_be_re}
\kipkokentry{muhâbir}{\latupxa\latupbe\latupre}{None}{kok:xa_be_re}
\kipkokentry{muhakkak}{\latupha\latupkaf\latupkaf}{None}{kok:ha_kaf_kaf}
\kipkokentry{muhakkik}{\latupha\latupkaf\latupkaf}{None}{kok:ha_kaf_kaf}
\kipkokentry{muhâl}{\latupha\latupvav\latuplam}{None}{kok:ha_vav_lam}
\kipkokentry{muhâlefet}{\latupxa\latuplam\latupfe}{None}{kok:xa_lam_fe}
\kipkokentry{muhâlif}{\latupxa\latuplam\latupfe}{None}{kok:xa_lam_fe}
\kipkokentry{muhârebe}{\latupha\latupre\latupbe}{None}{kok:ha_re_be}
\kipkokentry{muharip}{\latupha\latupre\latupbe}{None}{kok:ha_re_be}
\kipkokentry{muhâsara}{\latupha\latupsad\latupre}{None}{kok:ha_sad_re}
\kipkokentry{muhâsebe}{\latupha\latupsin\latupbe}{None}{kok:ha_sin_be}
\kipkokentry{muhâsip}{\latupha\latupsin\latupbe}{None}{kok:ha_sin_be}
\kipkokentry{muhayyel}{\latupxa\latupye\latuplam}{None}{kok:xa_ye_lam}
\kipkokentry{muhayyile}{\latupxa\latupye\latuplam}{None}{kok:xa_ye_lam}
\kipkokentry{muhbir}{\latupxa\latupbe\latupre}{None}{kok:xa_be_re}
\kipkokentry{muhik}{\latupha\latupkaf\latupkaf}{None}{kok:ha_kaf_kaf}
\kipkokentry{muhtelif}{\latupxa\latuplam\latupfe}{None}{kok:xa_lam_fe}
\kipkokentry{muhtemel}{\latupha\latupmim\latuplam}{None}{kok:ha_mim_lam}
\kipkokentry{muhterem}{\latupha\latupre\latupmim}{None}{kok:ha_re_mim}
\kipkokentry{muhteris}{\latupha\latupre\latupsad}{None}{kok:ha_re_sad}
\kipkokentry{muhtesip}{\latupha\latupsin\latupbe}{None}{kok:ha_sin_be}
\kipkokentry{muhteşem}{\latupha\latupshin\latupmim}{None}{kok:ha_shin_mim}
\kipkokentry{mukâbele}{\latupkaf\latupbe\latuplam}{None}{kok:kaf_be_lam}
\kipkokentry{mukâbil}{\latupkaf\latupbe\latuplam}{None}{kok:kaf_be_lam}
\kipkokentry{mukaddem}{\latupkaf\latupdal\latupmim}{None}{kok:kaf_dal_mim}
\kipkokentry{mukadder}{\latupkaf\latupdal\latupre}{None}{kok:kaf_dal_re}
\kipkokentry{mukaddime}{\latupkaf\latupdal\latupmim}{None}{kok:kaf_dal_mim}
\kipkokentry{mukâtele}{\latupkaf\latupte\latuplam}{None}{kok:kaf_te_lam}
\kipkokentry{mukâvemet}{\latupkaf\latupvav\latupmim}{None}{kok:kaf_vav_mim}
\kipkokentry{mukâvim}{\latupkaf\latupvav\latupmim}{None}{kok:kaf_vav_mim}
\kipkokentry{mukâyese}{\latupkaf\latupye\latupsin}{None}{kok:kaf_ye_sin}
\kipkokentry{mukîm}{\latupkaf\latupvav\latupmim}{None}{kok:kaf_vav_mim}
\kipkokentry{muktedir}{\latupkaf\latupdal\latupre}{None}{kok:kaf_dal_re}
\kipkokentry{muntazır}{\latupnun\latupza\latupre}{None}{kok:nun_za_re}
\kipkokentry{mûris}{\latupvav\latupre\latupthe}{None}{kok:vav_re_the}
\kipkokentry{musannif}{\latupsad\latupnun\latupfe}{None}{kok:sad_nun_fe}
\kipkokentry{musîbet}{\latupsad\latupvav\latupbe}{None}{kok:sad_vav_be}
\kipkokentry{mutasarrıf}{\latupsad\latupre\latupfe}{None}{kok:sad_re_fe}
\kipkokentry{mutasavvıf}{\latupsad\latupvav\latupfe}{None}{kok:sad_vav_fe}
\kipkokentry{mûteber}{\latupayn\latupbe\latupre}{None}{kok:ayn_be_re}
\kipkokentry{mûtedil}{\latupayn\latupdal\latuplam}{None}{kok:ayn_dal_lam}
\kipkokentry{muvahhit}{\latupvav\latupha\latupdal}{None}{kok:vav_ha_dal}
\kipkokentry{mübalağa}{\latupbe\latuplam\latupgayn}{None}{kok:be_lam_gayn}
\kipkokentry{mücellit}{\latupcim\latuplam\latupdal}{None}{kok:cim_lam_dal}
\kipkokentry{müderris}{\latupdal\latupre\latupsin}{None}{kok:dal_re_sin}
\kipkokentry{müdevver}{\latupdal\latupvav\latupre}{None}{kok:dal_vav_re}
\kipkokentry{müdür}{\latupdal\latupvav\latupre}{None}{kok:dal_vav_re}
\kipkokentry{müfettiş}{\latupfe\latupte\latupshin}{None}{kok:fe_te_shin}
\kipkokentry{müfredat}{\latupfe\latupre\latupdal}{None}{kok:fe_re_dal}
\kipkokentry{müfreze}{\latupfe\latupre\latupze}{None}{kok:fe_re_ze}
\kipkokentry{müfsit}{\latupfe\latupsin\latupdal}{None}{kok:fe_sin_dal}
\kipkokentry{müftehir}{\latupfe\latupxa\latupre}{None}{kok:fe_xa_re}
\kipkokentry{mühim}{\latuphe\latupmim\latupmim}{None}{kok:he_mim_mim}
\kipkokentry{mühimmat}{\latuphe\latupmim\latupmim}{None}{kok:he_mim_mim}
\kipkokentry{mükâfat}{\latupkef\latupfe\latupvav}{None}{kok:kef_fe_vav}
\kipkokentry{mükâleme}{\latupkef\latuplam\latupmim}{None}{kok:kef_lam_mim}
\kipkokentry{mükellef}{\latupkef\latuplam\latupfe}{None}{kok:kef_lam_fe}
\kipkokentry{mükemmel}{\latupkef\latupmim\latuplam}{None}{kok:kef_mim_lam}
\kipkokentry{mülhak}{\latuplam\latupha\latupkaf}{None}{kok:lam_ha_kaf}
\kipkokentry{mülk}{\latupmim\latuplam\latupkef}{None}{kok:mim_lam_kef}
\kipkokentry{mülkî}{\latupmim\latuplam\latupkef}{None}{kok:mim_lam_kef}
\kipkokentry{mülkiyet}{\latupmim\latuplam\latupkef}{None}{kok:mim_lam_kef}
\kipkokentry{mümâsil}{\latupmim\latupthe\latuplam}{None}{kok:mim_the_lam}
\kipkokentry{mümessil}{\latupmim\latupthe\latuplam}{None}{kok:mim_the_lam}
\kipkokentry{münâzara}{\latupnun\latupza\latupre}{None}{kok:nun_za_re}
\kipkokentry{müneccim}{\latupnun\latupcim\latupmim}{None}{kok:nun_cim_mim}
\kipkokentry{münevver}{\latupnun\latupvav\latupre}{None}{kok:nun_vav_re}
\kipkokentry{münferit}{\latupfe\latupre\latupdal}{None}{kok:fe_re_dal}
\kipkokentry{münfesih}{\latupfe\latupsin\latupxa}{None}{kok:fe_sin_xa}
\kipkokentry{münhâsır}{\latupha\latupsad\latupre}{None}{kok:ha_sad_re}
\kipkokentry{münkalip}{\latupkaf\latuplam\latupbe}{None}{kok:kaf_lam_be}
\kipkokentry{mürafaa}{\latupre\latupfe\latupayn}{None}{kok:re_fe_ayn}
\kipkokentry{müsâleme}{\latupsin\latuplam\latupmim}{None}{kok:sin_lam_mim}
\kipkokentry{müsekkin}{\latupsin\latupkef\latupnun}{None}{kok:sin_kef_nun}
\kipkokentry{müsellem}{\latupsin\latuplam\latupmim}{None}{kok:sin_lam_mim}
\kipkokentry{müslim}{\latupsin\latuplam\latupmim}{None}{kok:sin_lam_mim}
\kipkokentry{müstağnî}{\latupgayn\latupnun\latupye}{None}{kok:gayn_nun_ye}
\kipkokentry{müstakbel}{\latupkaf\latupbe\latuplam}{None}{kok:kaf_be_lam}
\kipkokentry{müstakîm}{\latupkaf\latupvav\latupmim}{None}{kok:kaf_vav_mim}
\kipkokentry{müstantik}{\latupnun\latupta\latupkaf}{None}{kok:nun_ta_kaf}
\kipkokentry{müstecap}{\latupcim\latupvav\latupbe}{None}{kok:cim_vav_be}
\kipkokentry{müstehak}{\latupha\latupkaf\latupkaf}{None}{kok:ha_kaf_kaf}
\kipkokentry{müstemleke}{\latupmim\latuplam\latupkef}{None}{kok:mim_lam_kef}
\kipkokentry{müstesnâ}{\latupthe\latupnun\latupye}{None}{kok:the_nun_ye}
\kipkokentry{müsteşâr}{\latupshin\latupvav\latupre}{None}{kok:shin_vav_re}
\kipkokentry{müstevlî}{\latupvav\latuplam\latupye}{None}{kok:vav_lam_ye}
\kipkokentry{müstezât}{\latupze\latupye\latupdal}{None}{kok:ze_ye_dal}
\kipkokentry{müşâhede}{\latupshin\latuphe\latupdal}{None}{kok:shin_he_dal}
\kipkokentry{müşâhit}{\latupshin\latuphe\latupdal}{None}{kok:shin_he_dal}
\kipkokentry{müşâvere}{\latupshin\latupvav\latupre}{None}{kok:shin_vav_re}
\kipkokentry{müşâvir}{\latupshin\latupvav\latupre}{None}{kok:shin_vav_re}
\kipkokentry{müşerref}{\latupshin\latupre\latupfe}{None}{kok:shin_re_fe}
\kipkokentry{müşevvik}{\latupshin\latupvav\latupkaf}{None}{kok:shin_vav_kaf}
\kipkokentry{müşfik}{\latupshin\latupfe\latupkaf}{None}{kok:shin_fe_kaf}
\kipkokentry{müşîr}{\latupshin\latupvav\latupre}{None}{kok:shin_vav_re}
\kipkokentry{müşkül}{\latupshin\latupkef\latuplam$^1$}{None}{kok:shin_kef_lam1}
\kipkokentry{müştâk}{\latupshin\latupvav\latupkaf}{None}{kok:shin_vav_kaf}
\kipkokentry{mütearife}{\latupayn\latupre\latupfe}{None}{kok:ayn_re_fe}
\kipkokentry{mütebâki}{\latupbe\latupkaf\latupye}{None}{kok:be_kaf_ye}
\kipkokentry{mütefekkir}{\latupfe\latupkef\latupre}{None}{kok:fe_kef_re}
\kipkokentry{müteferrik}{\latupfe\latupre\latupkaf}{None}{kok:fe_re_kaf}
\kipkokentry{mütefessih}{\latupfe\latupsin\latupxa}{None}{kok:fe_sin_xa}
\kipkokentry{mütegallibe}{\latupgayn\latuplam\latupbe}{None}{kok:gayn_lam_be}
\kipkokentry{mütehammil}{\latupha\latupmim\latuplam}{None}{kok:ha_mim_lam}
\kipkokentry{mütehassıs}{\latupxa\latupsad\latupsad}{None}{kok:xa_sad_sad}
\kipkokentry{mütehavvil}{\latupha\latupvav\latuplam}{None}{kok:ha_vav_lam}
\kipkokentry{mütekâbil}{\latupkaf\latupbe\latuplam}{None}{kok:kaf_be_lam}
\kipkokentry{mütekâmil}{\latupkef\latupmim\latuplam}{None}{kok:kef_mim_lam}
\kipkokentry{mütekellim}{\latupkef\latuplam\latupmim}{None}{kok:kef_lam_mim}
\kipkokentry{mütelezziz}{\latuplam\latupzel\latupzel}{None}{kok:lam_zel_zel}
\kipkokentry{müteşekkil}{\latupshin\latupkef\latuplam$^2$}{None}{kok:shin_kef_lam2}
\kipkokentry{mütevelli}{\latupvav\latuplam\latupye}{None}{kok:vav_lam_ye}
\kipkokentry{mütevellit}{\latupvav\latuplam\latupdal}{None}{kok:vav_lam_dal}
\kipkokentry{müttehit}{\latupvav\latupha\latupdal}{None}{kok:vav_ha_dal}
\kipkokentry{müvellet}{\latupvav\latuplam\latupdal}{None}{kok:vav_lam_dal}
\kipkokentry{müverrih}{\latupvav\latupre\latupxa}{None}{kok:vav_re_xa}
\kipkokentry{müzâheret}{\latupza\latuphe\latupre}{None}{kok:za_he_re}
\kipkokentry{müzâhir}{\latupza\latuphe\latupre}{None}{kok:za_he_re}
\kipkokentry{müzâyede}{\latupze\latupye\latupdal}{None}{kok:ze_ye_dal}
\end{multicols}
\dictchapter{N}
\begin{multicols}{3}
\kipkokentry{nar}{\latupnun\latupvav\latupre}{None}{kok:nun_vav_re}
\kipkokentry{nâtıka}{\latupnun\latupta\latupkaf}{None}{kok:nun_ta_kaf}
\kipkokentry{nazar}{\latupnun\latupza\latupre}{None}{kok:nun_za_re}
\kipkokentry{nazariyat}{\latupnun\latupza\latupre}{None}{kok:nun_za_re}
\kipkokentry{nazariye}{\latupnun\latupza\latupre}{None}{kok:nun_za_re}
\kipkokentry{nâzır}{\latupnun\latupza\latupre}{K\rom{1}, Ed.}{kok:nun_za_re}
\kipkokentry{nazîre}{\latupnun\latupza\latupre}{None}{kok:nun_za_re}
\kipkokentry{necim}{\latupnun\latupcim\latupmim}{None}{kok:nun_cim_mim}
\kipkokentry{nevâle}{\latupnun\latupvav\latuplam}{None}{kok:nun_vav_lam}
\kipkokentry{nevir}{\latupnun\latupvav\latupre}{None}{kok:nun_vav_re}
\kipkokentry{nezâret}{\latupnun\latupza\latupre}{None}{kok:nun_za_re}
\kipkokentry{nur}{\latupnun\latupvav\latupre}{None}{kok:nun_vav_re}
\kipkokentry{nutuk}{\latupnun\latupta\latupkaf}{None}{kok:nun_ta_kaf}
\kipkokentry{nücûm}{\latupnun\latupcim\latupmim}{None}{kok:nun_cim_mim}
\end{multicols}
\dictchapter{Ö}
\begin{multicols}{3}
\kipkokentry{örf}{\latupayn\latupre\latupfe}{None}{kok:ayn_re_fe}
\kipkokentry{öşür}{\latupayn\latupshin\latupre}{None}{kok:ayn_shin_re}
\kipkokentry{özür}{\latupayn\latupzel\latupre}{None}{kok:ayn_zel_re}
\end{multicols}
\dictchapter{R}
\begin{multicols}{3}
\kipkokentry{rahim}{\latupre\latupha\latupmim}{None}{kok:re_ha_mim}
\kipkokentry{rahîm}{\latupre\latupha\latupmim}{None}{kok:re_ha_mim}
\kipkokentry{rahmân}{\latupre\latupha\latupmim}{None}{kok:re_ha_mim}
\kipkokentry{rahmet}{\latupre\latupha\latupmim}{None}{kok:re_ha_mim}
\kipkokentry{ref}{\latupre\latupfe\latupayn}{None}{kok:re_fe_ayn}
\kipkokentry{rehin}{\latupre\latuphe\latupnun}{None}{kok:re_he_nun}
\kipkokentry{rehîne}{\latupre\latuphe\latupnun}{None}{kok:re_he_nun}
\kipkokentry{rızk}{\latupre\latupze\latupkaf}{None}{kok:re_ze_kaf}
\end{multicols}
\dictchapter{S}
\begin{multicols}{3}
\kipkokentry{sâkin}{\latupsin\latupkef\latupnun}{K\rom{1}, Ed.}{kok:sin_kef_nun}
\kipkokentry{sâlim}{\latupsin\latuplam\latupmim}{K\rom{1}, Ed.}{kok:sin_lam_mim}
\kipkokentry{sanat}{\latupsad\latupnun\latupayn}{None}{kok:sad_nun_ayn}
\kipkokentry{sanâyî}{\latupsad\latupnun\latupayn}{None}{kok:sad_nun_ayn}
\kipkokentry{sânî}{\latupthe\latupnun\latupye}{K\rom{1}, Ed.}{kok:the_nun_ye}
\kipkokentry{sâniye}{\latupthe\latupnun\latupye}{None}{kok:the_nun_ye}
\kipkokentry{sarâhat}{\latupsad\latupre\latupha}{None}{kok:sad_re_ha}
\kipkokentry{sarf}{\latupsad\latupre\latupfe}{None}{kok:sad_re_fe}
\kipkokentry{sarih}{\latupsad\latupre\latupha}{None}{kok:sad_re_ha}
\kipkokentry{sarraf}{\latupsad\latupre\latupfe}{None}{kok:sad_re_fe}
\kipkokentry{selam}{\latupsin\latuplam\latupmim}{None}{kok:sin_lam_mim}
\kipkokentry{selâmet}{\latupsin\latuplam\latupmim}{None}{kok:sin_lam_mim}
\kipkokentry{selem}{\latupsin\latuplam\latupmim}{None}{kok:sin_lam_mim}
\kipkokentry{selim}{\latupsin\latuplam\latupmim}{None}{kok:sin_lam_mim}
\kipkokentry{senâ}{\latupthe\latupnun\latupye}{None}{kok:the_nun_ye}
\kipkokentry{sene}{\latupthe\latupnun\latupye}{None}{kok:the_nun_ye}
\kipkokentry{sınâî}{\latupsad\latupnun\latupayn}{None}{kok:sad_nun_ayn}
\kipkokentry{sınıf}{\latupsad\latupnun\latupfe}{None}{kok:sad_nun_fe}
\kipkokentry{sırf}{\latupsad\latupre\latupfe}{None}{kok:sad_re_fe}
\kipkokentry{sual}{\latupsin\latuphemze\latuplam}{None}{kok:sin_hemze_lam}
\kipkokentry{sûfî}{\latupsad\latupvav\latupfe}{None}{kok:sad_vav_fe}
\kipkokentry{sunî}{\latupsad\latupnun\latupayn}{None}{kok:sad_nun_ayn}
\kipkokentry{sükûn}{\latupsin\latupkef\latupnun}{None}{kok:sin_kef_nun}
\kipkokentry{sükûnet}{\latupsin\latupkef\latupnun}{None}{kok:sin_kef_nun}
\kipkokentry{sürâhî}{\latupsad\latupre\latupha}{None}{kok:sad_re_ha}
\end{multicols}
\dictchapter{Ş}
\begin{multicols}{3}
\kipkokentry{şafak}{\latupshin\latupfe\latupkaf}{None}{kok:shin_fe_kaf}
\kipkokentry{şâhit}{\latupshin\latuphe\latupdal}{K\rom{1}, Ed.}{kok:shin_he_dal}
\kipkokentry{şefkat}{\latupshin\latupfe\latupkaf}{None}{kok:shin_fe_kaf}
\kipkokentry{şehâdet}{\latupshin\latuphe\latupdal}{None}{kok:shin_he_dal}
\kipkokentry{şekil}{\latupshin\latupkef\latuplam$^2$}{None}{kok:shin_kef_lam2}
\kipkokentry{şeref}{\latupshin\latupre\latupfe}{None}{kok:shin_re_fe}
\kipkokentry{şerefe}{\latupshin\latupre\latupfe}{None}{kok:shin_re_fe}
\kipkokentry{şerîf}{\latupshin\latupre\latupfe}{None}{kok:shin_re_fe}
\kipkokentry{şevk}{\latupshin\latupvav\latupkaf}{None}{kok:shin_vav_kaf}
\kipkokentry{şûrâ}{\latupshin\latupvav\latupre}{None}{kok:shin_vav_re}
\end{multicols}
\dictchapter{T}
\begin{multicols}{3}
\kipkokentry{taarruz}{\latupayn\latupre\latupdad}{None}{kok:ayn_re_dad}
\kipkokentry{tâbir}{\latupayn\latupbe\latupre}{K\rom{2}, Eyl.}{kok:ayn_be_re}
\kipkokentry{tâdil}{\latupayn\latupdal\latuplam}{K\rom{2}, Eyl.}{kok:ayn_dal_lam}
\kipkokentry{tâdilat}{\latupayn\latupdal\latuplam}{None}{kok:ayn_dal_lam}
\kipkokentry{tahakkuk}{\latupha\latupkaf\latupkaf}{None}{kok:ha_kaf_kaf}
\kipkokentry{tahammül}{\latupha\latupmim\latuplam}{None}{kok:ha_mim_lam}
\kipkokentry{tahavvül}{\latupha\latupvav\latuplam}{None}{kok:ha_vav_lam}
\kipkokentry{tahayyül}{\latupxa\latupye\latuplam}{None}{kok:xa_ye_lam}
\kipkokentry{tahkik}{\latupha\latupkaf\latupkaf}{K\rom{2}, Eyl.}{kok:ha_kaf_kaf}
\kipkokentry{tahsis}{\latupxa\latupsad\latupsad}{K\rom{2}, Eyl.}{kok:xa_sad_sad}
\kipkokentry{tahsisat}{\latupxa\latupsad\latupsad}{None}{kok:xa_sad_sad}
\kipkokentry{tahvil}{\latupha\latupvav\latuplam}{K\rom{2}, Eyl.}{kok:ha_vav_lam}
\kipkokentry{takaddüm}{\latupkaf\latupdal\latupmim}{None}{kok:kaf_dal_mim}
\kipkokentry{takbih}{\latupkaf\latupbe\latupha}{K\rom{2}, Eyl.}{kok:kaf_be_ha}
\kipkokentry{takdim}{\latupkaf\latupdal\latupmim}{K\rom{2}, Eyl.}{kok:kaf_dal_mim}
\kipkokentry{takdir}{\latupkaf\latupdal\latupre}{K\rom{2}, Eyl.}{kok:kaf_dal_re}
\kipkokentry{taksim}{\latupkaf\latupsin\latupmim}{K\rom{2}, Eyl.}{kok:kaf_sin_mim}
\kipkokentry{taksîmat}{\latupkaf\latupsin\latupmim}{None}{kok:kaf_sin_mim}
\kipkokentry{taktil}{\latupkaf\latupte\latuplam}{K\rom{2}, Eyl.}{kok:kaf_te_lam}
\kipkokentry{takvim}{\latupkaf\latupvav\latupmim}{K\rom{2}, Eyl.}{kok:kaf_vav_mim}
\kipkokentry{tâlim}{\latupayn\latuplam\latupmim}{K\rom{2}, Eyl.}{kok:ayn_lam_mim}
\kipkokentry{taraf}{\latupta\latupre\latupfe}{None}{kok:ta_re_fe}
\kipkokentry{târif}{\latupayn\latupre\latupfe}{K\rom{2}, Eyl.}{kok:ayn_re_fe}
\kipkokentry{târife}{\latupayn\latupre\latupfe}{None}{kok:ayn_re_fe}
\kipkokentry{târih}{\latupvav\latupre\latupxa}{K\rom{2}, Eyl.}{kok:vav_re_xa}
\kipkokentry{târiz}{\latupayn\latupre\latupdad}{K\rom{2}, Eyl.}{kok:ayn_re_dad}
\kipkokentry{tasarruf}{\latupsad\latupre\latupfe}{None}{kok:sad_re_fe}
\kipkokentry{tasavvuf}{\latupsad\latupvav\latupfe}{None}{kok:sad_vav_fe}
\kipkokentry{tasnif}{\latupsad\latupnun\latupfe}{K\rom{2}, Eyl.}{kok:sad_nun_fe}
\kipkokentry{tasrif}{\latupsad\latupre\latupfe}{K\rom{2}, Eyl.}{kok:sad_re_fe}
\kipkokentry{tasrih}{\latupsad\latupre\latupha}{K\rom{2}, Eyl.}{kok:sad_re_ha}
\kipkokentry{tasvip}{\latupsad\latupvav\latupbe}{K\rom{2}, Eyl.}{kok:sad_vav_be}
\kipkokentry{tebellüğ}{\latupbe\latuplam\latupgayn}{None}{kok:be_lam_gayn}
\kipkokentry{teblîgat}{\latupbe\latuplam\latupgayn}{None}{kok:be_lam_gayn}
\kipkokentry{teblîğ}{\latupbe\latuplam\latupgayn}{K\rom{2}, Eyl.}{kok:be_lam_gayn}
\kipkokentry{tedris}{\latupdal\latupre\latupsin}{K\rom{2}, Eyl.}{kok:dal_re_sin}
\kipkokentry{tedrisat}{\latupdal\latupre\latupsin}{None}{kok:dal_re_sin}
\kipkokentry{tedvir}{\latupdal\latupvav\latupre}{K\rom{2}, Eyl.}{kok:dal_vav_re}
\kipkokentry{tefekkür}{\latupfe\latupkef\latupre}{None}{kok:fe_kef_re}
\kipkokentry{tefessüh}{\latupfe\latupsin\latupxa}{None}{kok:fe_sin_xa}
\kipkokentry{tefrika}{\latupfe\latupre\latupkaf}{None}{kok:fe_re_kaf}
\kipkokentry{tefrik}{\latupfe\latupre\latupkaf}{K\rom{2}, Eyl.}{kok:fe_re_kaf}
\kipkokentry{teftiş}{\latupfe\latupte\latupshin}{K\rom{2}, Eyl.}{kok:fe_te_shin}
\kipkokentry{tegâfül}{\latupgayn\latupfe\latuplam}{None}{kok:gayn_fe_lam}
\kipkokentry{tegallüp}{\latupgayn\latuplam\latupbe}{None}{kok:gayn_lam_be}
\kipkokentry{tekâbül}{\latupkaf\latupbe\latuplam}{None}{kok:kaf_be_lam}
\kipkokentry{tekâlif}{\latupkef\latuplam\latupfe}{None}{kok:kef_lam_fe}
\kipkokentry{tekâmül}{\latupkef\latupmim\latuplam}{None}{kok:kef_mim_lam}
\kipkokentry{tekellüf}{\latupkef\latuplam\latupfe}{None}{kok:kef_lam_fe}
\kipkokentry{tekellüm}{\latupkef\latuplam\latupmim}{None}{kok:kef_lam_mim}
\kipkokentry{tekemmül}{\latupkef\latupmim\latuplam}{None}{kok:kef_mim_lam}
\kipkokentry{tekfir}{\latupkef\latupfe\latupre}{K\rom{2}, Eyl.}{kok:kef_fe_re}
\kipkokentry{teklif}{\latupkef\latuplam\latupfe}{K\rom{2}, Eyl.}{kok:kef_lam_fe}
\kipkokentry{tekmil}{\latupkef\latupmim\latuplam}{K\rom{2}, Eyl.}{kok:kef_mim_lam}
\kipkokentry{teksir}{\latupkef\latupthe\latupre}{K\rom{2}, Eyl.}{kok:kef_the_re}
\kipkokentry{telaffuz}{\latuplam\latupfe\latupza}{None}{kok:lam_fe_za}
\kipkokentry{telezzüz}{\latuplam\latupzel\latupzel}{None}{kok:lam_zel_zel}
\kipkokentry{temellük}{\latupmim\latuplam\latupkef}{None}{kok:mim_lam_kef}
\kipkokentry{temessül}{\latupmim\latupthe\latuplam}{None}{kok:mim_the_lam}
\kipkokentry{temlik}{\latupmim\latuplam\latupkef}{K\rom{2}, Eyl.}{kok:mim_lam_kef}
\kipkokentry{temsil}{\latupmim\latupthe\latuplam}{K\rom{2}, Eyl.}{kok:mim_the_lam}
\kipkokentry{tenevvür}{\latupnun\latupvav\latupre}{None}{kok:nun_vav_re}
\kipkokentry{tenvir}{\latupnun\latupvav\latupre}{K\rom{2}, Eyl.}{kok:nun_vav_re}
\kipkokentry{terfî}{\latupre\latupfe\latupayn}{K\rom{2}, Eyl.}{kok:re_fe_ayn}
\kipkokentry{terhin}{\latupre\latuphe\latupnun}{K\rom{2}, Eyl.}{kok:re_he_nun}
\kipkokentry{tesellüm}{\latupsin\latuplam\latupmim}{None}{kok:sin_lam_mim}
\kipkokentry{teskin}{\latupsin\latupkef\latupnun}{K\rom{2}, Eyl.}{kok:sin_kef_nun}
\kipkokentry{teslim}{\latupsin\latuplam\latupmim}{K\rom{2}, Eyl.}{kok:sin_lam_mim}
\kipkokentry{tesniye}{\latupthe\latupnun\latupye}{None}{kok:the_nun_ye}
\kipkokentry{teşekkül}{\latupshin\latupkef\latuplam$^2$}{None}{kok:shin_kef_lam2}
\kipkokentry{teşerrüf}{\latupshin\latupre\latupfe}{None}{kok:shin_re_fe}
\kipkokentry{teşkil}{\latupshin\latupkef\latuplam$^2$}{K\rom{2}, Eyl.}{kok:shin_kef_lam2}
\kipkokentry{teşrif}{\latupshin\latupre\latupfe}{K\rom{2}, Eyl.}{kok:shin_re_fe}
\kipkokentry{teşvik}{\latupshin\latupvav\latupkaf}{K\rom{2}, Eyl.}{kok:shin_vav_kaf}
\kipkokentry{tevârih}{\latupvav\latupre\latupxa}{None}{kok:vav_re_xa}
\kipkokentry{tevârüs}{\latupvav\latupre\latupthe}{None}{kok:vav_re_the}
\kipkokentry{tevellüt}{\latupvav\latuplam\latupdal}{None}{kok:vav_lam_dal}
\kipkokentry{tevhid}{\latupvav\latupha\latupdal}{K\rom{2}, Eyl.}{kok:vav_ha_dal}
\kipkokentry{tevlit}{\latupvav\latuplam\latupdal}{K\rom{2}, Eyl.}{kok:vav_lam_dal}
\kipkokentry{tevsik}{\latupvav\latupthe\latupkaf}{K\rom{2}, Eyl.}{kok:vav_the_kaf}
\kipkokentry{tezâhür}{\latupza\latuphe\latupre}{None}{kok:za_he_re}
\kipkokentry{tezyit}{\latupze\latupye\latupdal}{K\rom{2}, Eyl.}{kok:ze_ye_dal}
\kipkokentry{timsâl}{\latupmim\latupthe\latuplam}{None}{kok:mim_the_lam}
\end{multicols}
\dictchapter{U}
\begin{multicols}{3}
\kipkokentry{ulema}{\latupayn\latuplam\latupmim}{None}{kok:ayn_lam_mim}
\kipkokentry{ulûm}{\latupayn\latuplam\latupmim}{None}{kok:ayn_lam_mim}
\end{multicols}
\dictchapter{V}
\begin{multicols}{3}
\kipkokentry{vâcip}{\latupvav\latupcim\latupbe}{K\rom{1}, Ed.}{kok:vav_cim_be}
\kipkokentry{vahdet}{\latupvav\latupha\latupdal}{None}{kok:vav_ha_dal}
\kipkokentry{vakâ}{\latupvav\latupkaf\latupayn}{None}{kok:vav_kaf_ayn}
\kipkokentry{vâkıa}{\latupvav\latupkaf\latupayn}{None}{kok:vav_kaf_ayn}
\kipkokentry{vâkî}{\latupvav\latupkaf\latupayn}{K\rom{1}, Ed.}{kok:vav_kaf_ayn}
\kipkokentry{vâli}{\latupvav\latuplam\latupye}{K\rom{1}, Ed.}{kok:vav_lam_ye}
\kipkokentry{vâlide}{\latupvav\latuplam\latupdal}{None}{kok:vav_lam_dal}
\kipkokentry{varak}{\latupvav\latupre\latupkaf}{None}{kok:vav_re_kaf}
\kipkokentry{vâris}{\latupvav\latupre\latupthe}{K\rom{1}, Ed.}{kok:vav_re_the}
\kipkokentry{vecîbe}{\latupvav\latupcim\latupbe}{None}{kok:vav_cim_be}
\kipkokentry{vekâyî}{\latupvav\latupkaf\latupayn}{None}{kok:vav_kaf_ayn}
\kipkokentry{velâdet}{\latupvav\latuplam\latupdal}{None}{kok:vav_lam_dal}
\kipkokentry{velâyet}{\latupvav\latuplam\latupye}{None}{kok:vav_lam_ye}
\kipkokentry{velet}{\latupvav\latuplam\latupdal}{None}{kok:vav_lam_dal}
\kipkokentry{veli}{\latupvav\latuplam\latupye}{None}{kok:vav_lam_ye}
\kipkokentry{velût}{\latupvav\latuplam\latupdal}{None}{kok:vav_lam_dal}
\kipkokentry{verâset}{\latupvav\latupre\latupthe}{None}{kok:vav_re_the}
\kipkokentry{verese}{\latupvav\latupre\latupthe}{None}{kok:vav_re_the}
\kipkokentry{vesâik}{\latupvav\latupthe\latupkaf}{None}{kok:vav_the_kaf}
\kipkokentry{vesika}{\latupvav\latupthe\latupkaf}{None}{kok:vav_the_kaf}
\kipkokentry{vilâyet}{\latupvav\latuplam\latupye}{None}{kok:vav_lam_ye}
\kipkokentry{vukû}{\latupvav\latupkaf\latupayn}{None}{kok:vav_kaf_ayn}
\kipkokentry{vukuat}{\latupvav\latupkaf\latupayn}{None}{kok:vav_kaf_ayn}
\end{multicols}
\dictchapter{Z}
\begin{multicols}{3}
\kipkokentry{zâhir}{\latupza\latuphe\latupre}{K\rom{1}, Ed.}{kok:za_he_re}
\kipkokentry{zâit}{\latupze\latupye\latupdal}{K\rom{1}, Ed.}{kok:ze_ye_dal}
\kipkokentry{zanaat}{\latupsad\latupnun\latupayn}{None}{kok:sad_nun_ayn}
\kipkokentry{zehrâ}{\latupza\latuphe\latupre}{None}{kok:za_he_re}
\kipkokentry{zerk}{\latupze\latupre\latupkaf}{None}{kok:ze_re_kaf}
\kipkokentry{zevâhir}{\latupza\latuphe\latupre}{None}{kok:za_he_re}
\kipkokentry{ziyâde}{\latupze\latupye\latupdal}{None}{kok:ze_ye_dal}
\kipkokentry{zuhûr}{\latupza\latuphe\latupre}{None}{kok:za_he_re}
\kipkokentry{zührevî}{\latupza\latuphe\latupre}{None}{kok:za_he_re}
\end{multicols}



% \restoregeometry
% \recalctypearea




\end{document}
