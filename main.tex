\documentclass[a5paper,12pt]{scrbook}
\usepackage[turkish]{babel}
\newlength{\tabwidth}
\setlength{\tabwidth}{8.7cm}

% all the local settings defined in /localsettings.sty
\usepackage{localsettings}
\usepackage{arabicletters}
% lazyeqn - math symbols
% Uncomment if you have chosen to clone it to your project
% \usepackage{./lazyeqn/lazyeqn}

% \usepackage{caption}

\newcommand*\varhrulefill[1][0.4pt]{\leavevmode\leaders\hrule height#1\hfill\kern0pt}

\usepackage{calc}
\newlength{\linew}
\setlength{\linew}{\textwidth+2em}
\newlength{\ltw}
\setlength{\ltw}{0.08333\linew}
\newlength{\rowh}
\setlength{\rowh}{0.6cm}

\newcommand{\fat}{FarsoArabo\kern -.1em Türkçe}

\title{\Huge\fat:\\[0.6ex] \large Türkçe İçindeki FarsoArap Unsurların \\
  Ortografi, Gramer ve Kelime Türetme Esasları}
\author{H\kern -0.05em .\kern 0.1em O\kern -0.05em .\kern 0.2em Solmaz}
\date{}


\newcommand{\mprow}[5]{%
  \noindent
  % \vspace{2.1ex}
  \begin{minipage}[b][\rowh]{\linewidth}%\vspace{0.5ex}
  \begin{minipage}{0.15\linew}%
    \Large #2
  \end{minipage}
  \begin{minipage}{0.15\linewidth}
    \large #1
  \end{minipage}
  \begin{minipage}{0.20\linew}%
    #3
  \end{minipage}
  \begin{minipage}{0.20\linew}
    #4
  \end{minipage}
  \begin{minipage}{0.20\linew}
    #5
  \end{minipage}
  \end{minipage}%
  \newline\noindent %\vspace{0.5ex}
}

\begin{document}

\maketitle
\tableofcontents

\chapter{Ortografi}

\section{Eski Alfabe}

\section{Vekil Alfabe}

% \begin{longtable}{p{2em}p{2em}p{2em}p{2em}p{2em}p{2em}p{2em}p{2em}}
% Isolated & Final & Medial & Initial & Name & Modern Turkish & ALA-LC[7] & IPA \\
% \toprule
% \textarabic{ ا‎ } & \textarabic{ ـا‎ } & -- & -- & elif & a, e & --, ā & a, e \\
% \textarabic{ ء‎ } & -- & -- & -- & hemze  & -- &  -- & -- \\
% \textarabic{ ب‎ } & \textarabic{ ـب‎ } & \textarabic{ ـبـ‎ } & \textarabic{ بـ‎ } & be & b & b & b \\
% \textarabic{ پ‎ } & \textarabic{ ـپ‎ } & \textarabic{ ـپـ‎ } & \textarabic{ پـ‎ } & pe & p & p & p \\
% \textarabic{ ت‎ } & \textarabic{ ـت‎ } & \textarabic{ ـتـ‎ } & \textarabic{ تـ‎ } & te & t & t & t \\
% \textarabic{ ث‎ } & \textarabic{ ـث‎ } & \textarabic{ ـثـ‎ } & \textarabic{ ثـ‎ } & se & s & s & s \\
% \textarabic{ ج‎ } & \textarabic{ ـج‎ } & \textarabic{ ـجـ‎ } & \textarabic{ جـ‎ } & cim & c & c & d͡ʒ \\
% \textarabic{ چ‎ } & \textarabic{ ـچ‎ } & \textarabic{ ـچـ‎ } & \textarabic{ چـ‎ } & çim & ç & ç & t͡ʃ \\
% \textarabic{ ح‎ } & \textarabic{ ـح‎ } & \textarabic{ ـحـ‎ } & \textarabic{ حـ‎ } & ha & h & ḥ & h \\
% \textarabic{ خ‎ } & \textarabic{ ـخ‎ } & \textarabic{ ـخـ‎ } & \textarabic{ خـ‎ } & hı & h & ḫ & h \\
% \textarabic{ د‎ } & \textarabic{ ـد‎ } & -- & -- & dal & d & d & d \\
% \textarabic{ ذ‎ } & \textarabic{ ـذ‎ } & -- & -- & zel & z & z & z \\
% \textarabic{ ر‎ } & \textarabic{ ـر‎ } & -- & -- & re & r & r & ɾ \\
% \textarabic{ ز‎ } & \textarabic{ ـز‎ } & -- & -- & ze & z & z & z \\
% \textarabic{ ژ‎ } & \textarabic{ ـژ‎ } & -- & -- & je & j & j & ʒ \\
% \textarabic{ س‎ } & \textarabic{ ـس‎ } & \textarabic{ ـسـ‎ } & \textarabic{ سـ‎ } & sin & s & s & s \\
% \textarabic{ ش‎ } & \textarabic{ ـش‎ } & \textarabic{ ـشـ‎ } & \textarabic{ شـ‎ } & şın & ş & ș & ʃ \\
% \textarabic{ ص‎ } & \textarabic{ ـص‎ } & \textarabic{ ـصـ‎ } & \textarabic{ صـ‎ } & sad & s & ṣ & s \\
% \textarabic{ ض‎ } & \textarabic{ ـض‎ } & \textarabic{ ـضـ‎ } & \textarabic{ ضـ‎ } & dad & d, z & ż & d \\
% \textarabic{ ط‎ } & \textarabic{ ـط‎ } & \textarabic{ ـطـ‎ } & \textarabic{ طـ‎ } & tı & t & ṭ & t \\
% \textarabic{ ظ‎ } & \textarabic{ ـظ‎ } & \textarabic{ ـظـ‎ } & \textarabic{ ظـ‎ } & zı & z & ẓ & z \\
% \textarabic{ ع‎ } & \textarabic{ ـع‎ } & \textarabic{ ـعـ‎ } & \textarabic{ عـ‎ } & ayn & ', h (or omitted) & ‘ & ʔ \\
% \textarabic{ غ‎ } & \textarabic{ ـغ‎ } & \textarabic{ ـغـ‎ } & \textarabic{ غـ‎ } & gayn & g, ğ & ġ & ɡ, ɣ \\
% \textarabic{ ف‎ } & \textarabic{ ـف‎ } & \textarabic{ ـفـ‎ } & \textarabic{ فـ‎ } & fe & f & f & f \\
% \textarabic{ ق‎ } & \textarabic{ ـق‎ } & \textarabic{ ـقـ‎ } & \textarabic{ قـ‎ } & kaf & k & ḳ & k \\
% \textarabic{ ك‎ } & \textarabic{ ـك‎ } & \textarabic{ ـكـ‎ } & \textarabic{ كـ‎ } & kef & k, g, ğ, n & k & k \\
% \textarabic{ گ‎ } & \textarabic{ ـگ‎ } & \textarabic{ ـگـ‎ } & \textarabic{ گـ‎ } & gef (1) & g, ğ & g & ɡ \\
% \textarabic{ ڭ‎ } & \textarabic{ ـڭ‎ } & \textarabic{ ـڭـ‎ } & \textarabic{ ڭـ‎ } & nef, sağır kef & n & ñ & ŋ \\
% \textarabic{ ل‎ } & \textarabic{ ـل‎ } & \textarabic{ ـلـ‎ } & \textarabic{ لـ‎ } & lam & l & l & l \\
% \textarabic{ م‎ } & \textarabic{ ـم‎ } & \textarabic{ ـمـ‎ } & \textarabic{ مـ‎ } & mim & m & m & m \\
% \textarabic{ ن‎ } & \textarabic{ ـن‎ } & \textarabic{ ـنـ‎ } & \textarabic{ نـ‎ } & nun & n & n & n \\
% \textarabic{ و‎ } & \textarabic{ ـو‎ } &  -- & -- & vav & v, o, ö, u, ü & v, ū, aw, avv, ūv & v, o, œ, u, y \\
% \textarabic{ ه‎ } & \textarabic{ ـه‎ } & \textarabic{ ـهـ‎ } & \textarabic{ هـ‎ } & he & h, e, a & h (2) & h, æ \\
% \textarabic{ ی‎ } & \textarabic{ ـی‎ } & \textarabic{ ـیـ‎ } & \textarabic{ یـ‎ } & ye & y, ı, i & y, ī, ay, á, īy & j, ɯ, i \\
% \bottomrule
% \end{longtable}
% %p{\ltw}p{2\ltw}p{2\ltw}p{2\ltw}{2\ltw}


% \begingroup
% \setlength\extrarowh{5pt}
% % \begin{table}[htbp]
%   % \centering
% \begin{longtable}{>{\large}p{1.5\ltw}>{\Large}p{1.5\ltw}p{2.5\ltw}p{2.5\ltw}p{4\ltw}}
%   \normalsize Eskiyazı Harf & \normalsize Vekil Harf &\normalsize
%   Yeniyazı Mukabili &\normalsize Eskiyazı okunuşu &\normalsize Türkçe \mbox{Okunuşu} \\
%   \toprule
%   \arelif    & Ää     & açık ünlü      & elif           & [a], [e]          \\
%   % \arhemze &        & hemze          & hemze          & --            \\
%   \arbe      & Bb     & b              & be             & [b]             \\
%   \arpe      & Pp     & p              & pe             & [p]             \\
%   \arte      & Tt     & t              & te             & [t]             \\
%   \arthe     & Þþ     & peltek s       & peltek s       & [s]             \\
%   \arcim     & Cc     & c              & cim            & [d͡ʒ]           \\
%   \archim    & Çç     & ç              & çim            & [t͡ʃ]           \\
%   \arha      & \kH\kh & kalın h        & ha             & [h]             \\
%   \arxa      & Xx     & sürten h       & hı             & [h]             \\
%   \ardal     & Dd     & d              & dal            & [d]             \\
%   \arzel     & Ⱬⱬ     & diş z          & zel            & [z]             \\
%   \arre      & Rr     & r              & re             & [ɾ]             \\
%   \arze      & Zz     & z              & ze             & [z]             \\
%   \arje      & Jj     & j              & je             & [ʒ]             \\
%   \arsin     & Ss     & s              & sin            & [s]             \\
%   \arshin    & Şş     & ş              & şın            & [ʃ]             \\
%   \arsad     & \kS\ks & kalın s        & sad            & [s]             \\
%   \ardad     & Ɖɖ     & diş d          & dad            & [d]             \\
%   \arta      & \kT\kt & kalın t        & tı             & [t]             \\
%   \arza      & \kZ\kz & kalın z        & zı             & [z]             \\
%   \arayn     & Ʒʒ     & gırtlak a      & ayn            & [ʔ]             \\ % ʕſ
%   \argayn    & Ğğ     & yumuşak g      & gayn           & [ɡ, ɣ]          \\
%   \arfe      & Ff     & f              & fe             & [f]             \\
%   \arkaf     & \kK\kk & kalın k        & kaf            & [k]             \\
%   \arkef     & Kk     & k              & kef            & [k]             \\
%   \argef     & Gg     & g              & gef            & [ɡ]             \\
%   \arnef     & Ŋŋ     & geniz n        & nef, sağır kef & [ŋ]             \\
%   \arlam     & Ll     & l              & lam            & [l]             \\
%   \armim     & Mm     & m              & mim            & [m]             \\
%   \arnun     & Nn     & n              & nun            & [n]             \\
%   \arvav     & Vv     & v              & vav            & [v, o, œ, u, y] \\
%   \arhe      & Hh     & h              & he             & [h, æ]          \\
%   \arye      & Ii     & y              & ye             & [j, ɯ, i]       \\
%   \bottomrule
%   \caption{Lol}
%   \label{tab:vekil1}
% \end{longtable}
% % \end{table}
% \endgroup

\begingroup
% \centering
\mprow{\normalsize Eskiyazı \\harf}{\normalsize Vekil \\harf}{Vekil
  \\harf ismi}{Eskiyazı \\harf ismi}{Türkçe \\telaffuz}
\vspace{1ex} \hrule \vspace{1ex}

\mprow{\arelif   }{Ää     }{açık e       }{elif           }{[a, e]       }
\mprow{\arayn    }{Øø     }{gırtlak a    }{ayn            }{[ʔ]            } % ʕſ Ʒʒ
\mprow{\arbe     }{Bb     }{b            }{be             }{[b]            }
\mprow{\arcim    }{Cc     }{c            }{cim            }{[d͡ʒ]          }
\mprow{\archim   }{Çç     }{ç            }{çim            }{[t͡ʃ]          }
\mprow{\ardal    }{Dd     }{d            }{dal            }{[d]            }
\mprow{\ardad    }{Ɖɖ     }{diş d        }{dad            }{[d, z]          }
% \mprow{---       }{E      }{e            }{                }{               }
\mprow{\arfe     }{Ff     }{f            }{fe             }{[f]            }
\mprow{\argef    }{Gg     }{g            }{gef            }{[ɡ]            }
\mprow{\argayn   }{Ğğ     }{yumuşak g    }{gayn           }{[ɡ, ɣ]         }
\mprow{\arhe     }{Hh     }{h            }{he             }{[h, æ]         }
\mprow{\arxa     }{Xx     }{sürtmeli h     }{hı             }{[h]            }
\mprow{\arye     }{İi     }{i            }{ye             }{[j, ɯ, i]      }
\mprow{\arje     }{Jj     }{j            }{je             }{[ʒ]            }
\mprow{\arkef    }{Kk     }{k            }{kef            }{[k]            }
\mprow{\arlam    }{Ll     }{l            }{lam            }{[l]            }
\mprow{\armim    }{Mm     }{m            }{mim            }{[m]            }
\mprow{\arnun    }{Nn     }{n            }{nun            }{[n]            }
\mprow{\arnef    }{Ŋŋ     }{geniz n      }{nef, sağır kef }{[ŋ]            }
\mprow{\arpe     }{Pp     }{p            }{pe             }{[p]            }
\mprow{\arre     }{Rr     }{r            }{re             }{[ɾ]            }
\mprow{\arsin    }{Ss     }{s            }{sin            }{[s]            }
\mprow{\arthe    }{Þþ     }{peltek s     }{peltek s       }{[s]            }
\mprow{\arshin   }{Şş     }{ş            }{şın            }{[ʃ]            }
\mprow{\arte     }{Tt     }{t            }{te             }{[t]            }
\mprow{\arvav    }{Vv     }{v            }{vav            }{[v, o, œ, u, y]}
\mprow{\arze     }{Zz     }{z            }{ze             }{[z]            }
\mprow{\arzel    }{Ⱬⱬ     }{diş z        }{zel            }{[z]            }
\vspace{-1ex} \hrule \vspace{-3ex}
\begin{table}[H]
\caption{Vekil alfabe. Vekil harfler yeniyazıda tekabül ettikleri yerlere göre
  sıralanmış olup rahat karşılaştırma için eski harf isimleri ve konuşmadaki
  telaffuzları ile birlikte verilmiştir. Telaffuzda UFA notasyonu kullanılmıştır.}
\end{table}
\endgroup

\mprow{\arha     }{\kH\kh }{kalın h      }{ha             }{[h]            }
\mprow{\arkaf    }{\kK\kk }{kalın k      }{kaf            }{[k]            }
\mprow{\arsad    }{\kS\ks }{kalın s      }{sad            }{[s]            }
\mprow{\arta     }{\kT\kt }{kalın t      }{tı             }{[t]            }
\mprow{\arza     }{\kZ\kz }{kalın z      }{zı             }{[z]            }


% \chapter{Kelime Türetme}

\chapter{Arap Köklü Kelimelerin Kuralları}
\section{Bitişimsiz Morfologi}

\section{Eylem İsmi Kipleri}

\begin{table}[htbp]
  \footnotesize
  \centering

  \begin{tabular}{p{0.1\tabwidth} >{\raggedright}p{0.3\tabwidth} >{\raggedright}p{0.2\tabwidth} >{\raggedright}p{0.2\tabwidth} p{0.2\tabwidth}}
    Kip No & Anlam & Eylem & Eden & Edilen \\
    \toprule
    \rom{1} & Sade & Necz, Nücûz, Nicz, Nücz(et), Necâz(et), Nicâz(et), vs. &  Nâciz & Mencuz \\
    \rom{2} & Geçişli, ettiren, güçlü & Tenciz &  Münecciz & Müneccez \\
    \rom{3} & İşteş & Münâceze, Nicaz &  Münâciz & Münâcez \\
    \rom{4} & Geçişli, ettiren & İncaz &  Münciz & Müncez \\
    \rom{5} & Dönüşlü, ettiren, güçlü & Tenaccüz &  Mütenecciz & Müteneccez \\
    \rom{6} & \rom{3}'ün işteşlik mukabili & Tenâcüz &  Mütenâciz & Mütenâcez \\
    \rom{7} & Dönüşlü, ettirilen & İnnicaz & Münnaciz & Münnacez \\
    \rom{8} & \rom{1}'in dönüşlüsü, geçişsiz & İnticaz & Müntaciz & Müntacez \\
    \rom{9} & Durum eylemi, geçişsiz & İncizaz & Müncezz & --- \\
    \rom{10} & Ettiren, bazen ettirilen, çeşitli anlamlar  & İstincaz & Müstanciz & Müstancez \\
    \midrule
    \rom{11} & \rom{9}'un aynısı, şiir dışında nadir & İncîzaz & Müncâzz & --- \\
    \rom{12} & \multirow{4}{*}{\parbox{0.3\tabwidth}{\raggedright Durum eylemi, çok nadir}} & İncîcaz & Müncavciz & Müncavcez \\
    \rom{13} &  & İncivvaz & Müncavviz & Müncavvez \\
    \rom{14} &  & İncinzaz & Müncanziz & Müncanzez \\
    \rom{15} &  & İncinzâ' & Müncanzin & Müncanzen \\
    \bottomrule
  \end{tabular}
\end{table}

\subsection{Nâciz Kipi: Yapan}

\subsection{Mencuz Kipi: Yapılan}

\subsection{İncaz Kipi: Geçişli (Transitif) Eylem}


\chapter{Fars Köklü Kelimelerin Kuralları}
\section{Fars Köklü NeoOsmanlıca Ekler}
\fat

\end{document}
