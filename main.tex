\documentclass[a5paper,12pt, oneside]{scrbook}
\usepackage[turkish]{babel}
\newlength{\tabwidth}
\setlength{\tabwidth}{8.7cm}

% all the local settings defined in /localsettings.sty
\usepackage{localsettings}
\usepackage{arabicletters}
% lazyeqn - math symbols
% Uncomment if you have chosen to clone it to your project
% \usepackage{./lazyeqn/lazyeqn}

% \usepackage{caption}

\newcommand*\varhrulefill[1][0.4pt]{\leavevmode\leaders\hrule height#1\hfill\kern0pt}

\usepackage{calc}
\newlength{\linew}
\setlength{\linew}{\textwidth+2em}
\newlength{\ltw}
\setlength{\ltw}{0.08333\linew}
\newlength{\rowh}
\setlength{\rowh}{0.6cm}

\newcommand{\fat}{FarsoArabo\kern -.1em Türkçe}

% \title{\Huge\fat:\\[0.6ex] \large Türkçe İçindeki Farsoarap Unsurların \\
%   Ortografi, Gramer ve Kelime Türetme Esasları}
\title{\Large Türkçe İçindeki Farsoarap Unsurların \\
  Ortografi, Gramer ve Kelime-Türetme Kuralları}
\author{H.\ O.\ Solmaz}
\date{}


\newcommand{\mprow}[5]{%
  \noindent
  % \vspace{2.1ex}
  \begin{minipage}[b][\rowh]{\linewidth}%\vspace{0.5ex}
  \begin{minipage}{0.14\linew}%
    \Large #2
  \end{minipage}
  \begin{minipage}{0.14\linewidth}
    \large #1
  \end{minipage}
  \begin{minipage}{0.19\linew}
    \small #4
  \end{minipage}
  \begin{minipage}{0.25\linew}%
    \small#3
  \end{minipage}
  \begin{minipage}{0.20\linew}
    \small#5
  \end{minipage}
  \end{minipage}%
  \newline\noindent %\vspace{0.5ex}
}

\begin{document}

% \pagestyle{plain}
\maketitle
\tableofcontents

\chapter{Ortografi}

\section{Eski Alfabe}

\section{Vekil Alfabe}

% \begin{longtable}{p{2em}p{2em}p{2em}p{2em}p{2em}p{2em}p{2em}p{2em}}
% Isolated & Final & Medial & Initial & Name & Modern Turkish & ALA-LC[7] & IPA \\
% \toprule
% \textarabic{ ا‎ } & \textarabic{ ـا‎ } & -- & -- & elif & a, e & --, ā & a, e \\
% \textarabic{ ء‎ } & -- & -- & -- & hemze  & -- &  -- & -- \\
% \textarabic{ ب‎ } & \textarabic{ ـب‎ } & \textarabic{ ـبـ‎ } & \textarabic{ بـ‎ } & be & b & b & b \\
% \textarabic{ پ‎ } & \textarabic{ ـپ‎ } & \textarabic{ ـپـ‎ } & \textarabic{ پـ‎ } & pe & p & p & p \\
% \textarabic{ ت‎ } & \textarabic{ ـت‎ } & \textarabic{ ـتـ‎ } & \textarabic{ تـ‎ } & te & t & t & t \\
% \textarabic{ ث‎ } & \textarabic{ ـث‎ } & \textarabic{ ـثـ‎ } & \textarabic{ ثـ‎ } & se & s & s & s \\
% \textarabic{ ج‎ } & \textarabic{ ـج‎ } & \textarabic{ ـجـ‎ } & \textarabic{ جـ‎ } & cim & c & c & d͡ʒ \\
% \textarabic{ چ‎ } & \textarabic{ ـچ‎ } & \textarabic{ ـچـ‎ } & \textarabic{ چـ‎ } & çim & ç & ç & t͡ʃ \\
% \textarabic{ ح‎ } & \textarabic{ ـح‎ } & \textarabic{ ـحـ‎ } & \textarabic{ حـ‎ } & ha & h & ḥ & h \\
% \textarabic{ خ‎ } & \textarabic{ ـخ‎ } & \textarabic{ ـخـ‎ } & \textarabic{ خـ‎ } & hı & h & ḫ & h \\
% \textarabic{ د‎ } & \textarabic{ ـد‎ } & -- & -- & dal & d & d & d \\
% \textarabic{ ذ‎ } & \textarabic{ ـذ‎ } & -- & -- & zel & z & z & z \\
% \textarabic{ ر‎ } & \textarabic{ ـر‎ } & -- & -- & re & r & r & ɾ \\
% \textarabic{ ز‎ } & \textarabic{ ـز‎ } & -- & -- & ze & z & z & z \\
% \textarabic{ ژ‎ } & \textarabic{ ـژ‎ } & -- & -- & je & j & j & ʒ \\
% \textarabic{ س‎ } & \textarabic{ ـس‎ } & \textarabic{ ـسـ‎ } & \textarabic{ سـ‎ } & sin & s & s & s \\
% \textarabic{ ش‎ } & \textarabic{ ـش‎ } & \textarabic{ ـشـ‎ } & \textarabic{ شـ‎ } & şın & ş & ș & ʃ \\
% \textarabic{ ص‎ } & \textarabic{ ـص‎ } & \textarabic{ ـصـ‎ } & \textarabic{ صـ‎ } & sad & s & ṣ & s \\
% \textarabic{ ض‎ } & \textarabic{ ـض‎ } & \textarabic{ ـضـ‎ } & \textarabic{ ضـ‎ } & dad & d, z & ż & d \\
% \textarabic{ ط‎ } & \textarabic{ ـط‎ } & \textarabic{ ـطـ‎ } & \textarabic{ طـ‎ } & tı & t & ṭ & t \\
% \textarabic{ ظ‎ } & \textarabic{ ـظ‎ } & \textarabic{ ـظـ‎ } & \textarabic{ ظـ‎ } & zı & z & ẓ & z \\
% \textarabic{ ع‎ } & \textarabic{ ـع‎ } & \textarabic{ ـعـ‎ } & \textarabic{ عـ‎ } & ayn & ', h (or omitted) & ‘ & ʔ \\
% \textarabic{ غ‎ } & \textarabic{ ـغ‎ } & \textarabic{ ـغـ‎ } & \textarabic{ غـ‎ } & gayn & g, ğ & ġ & ɡ, ɣ \\
% \textarabic{ ف‎ } & \textarabic{ ـف‎ } & \textarabic{ ـفـ‎ } & \textarabic{ فـ‎ } & fe & f & f & f \\
% \textarabic{ ق‎ } & \textarabic{ ـق‎ } & \textarabic{ ـقـ‎ } & \textarabic{ قـ‎ } & kaf & k & ḳ & k \\
% \textarabic{ ك‎ } & \textarabic{ ـك‎ } & \textarabic{ ـكـ‎ } & \textarabic{ كـ‎ } & kef & k, g, ğ, n & k & k \\
% \textarabic{ گ‎ } & \textarabic{ ـگ‎ } & \textarabic{ ـگـ‎ } & \textarabic{ گـ‎ } & gef (1) & g, ğ & g & ɡ \\
% \textarabic{ ڭ‎ } & \textarabic{ ـڭ‎ } & \textarabic{ ـڭـ‎ } & \textarabic{ ڭـ‎ } & nef, sağır kef & n & ñ & ŋ \\
% \textarabic{ ل‎ } & \textarabic{ ـل‎ } & \textarabic{ ـلـ‎ } & \textarabic{ لـ‎ } & lam & l & l & l \\
% \textarabic{ م‎ } & \textarabic{ ـم‎ } & \textarabic{ ـمـ‎ } & \textarabic{ مـ‎ } & mim & m & m & m \\
% \textarabic{ ن‎ } & \textarabic{ ـن‎ } & \textarabic{ ـنـ‎ } & \textarabic{ نـ‎ } & nun & n & n & n \\
% \textarabic{ و‎ } & \textarabic{ ـو‎ } &  -- & -- & vav & v, o, ö, u, ü & v, ū, aw, avv, ūv & v, o, œ, u, y \\
% \textarabic{ ه‎ } & \textarabic{ ـه‎ } & \textarabic{ ـهـ‎ } & \textarabic{ هـ‎ } & he & h, e, a & h (2) & h, æ \\
% \textarabic{ ی‎ } & \textarabic{ ـی‎ } & \textarabic{ ـیـ‎ } & \textarabic{ یـ‎ } & ye & y, ı, i & y, ī, ay, á, īy & j, ɯ, i \\
% \bottomrule
% \end{longtable}
% %p{\ltw}p{2\ltw}p{2\ltw}p{2\ltw}{2\ltw}

\newpage
\begingroup
% \setlength\extrarowh{5pt}
% \begin{table}[htbp]
  % \centering
\renewcommand{\arraystretch}{1.62}
\begin{longtable}{>{\large}p{1.5\ltw}>{\Large}p{1.5\ltw}>{\small}p{2\ltw}>{\small}p{2.8\ltw}>{\small}p{2\ltw}}
   \normalsize Vekil \newline harf
  & \normalsize Eski yazı \newline harf
  & \normalsize Eski yazı \newline harf ismi
  & \normalsize Yeni yazı\newline karşılığı
  & \normalsize Türkçe \newline telaffuz \\
  %
  % \normalsize Eskiyazı Harf & \normalsize Vekil Harf &\normalsize
  % Yeni yazı Mukabili &\normalsize Eskiyazı okunuşu &\normalsize Türkçe \mbox{Okunuşu} \\
  \toprule
   Ää     & \arelif                   & elif           & *A veya E          & [a, e, æ]       \\
   Øø     & \raisebox{0.6ex}{\arayn}  & ayn            & *gırtlak ünlüsü    & [--]            \\ % ʕſ Ʒʒ
   Bb     & \arbe                     & be             & B                  & [b]             \\
   Cc     & \raisebox{0.8ex}{\arcim}  & cim            & C                  & [d͡ʒ]            \\
   Çç     & \raisebox{0.8ex}{\archim} & çim            & Ç                  & [t͡ʃ]            \\
   Dd     & \ardal                    & dal            & D                  & [d]             \\
   Ɖɖ     & \raisebox{0.8ex}{\ardad}  & dad            & *diş D             & [d, z]          \\
   Ff     & \arfe                     & fe             & F                  & [f]             \\
   Gg     & \argef                    & gef            & G                  & [ɡ]             \\
   Ğğ     & \raisebox{0.6ex}{\argayn} & gayn           & genelde Ğ          & [ɡ, ɣ]          \\
   Hh     & \arhe                     & he             & *ince H veya E     & [h, æ]          \\
   \kH\kh & \raisebox{1.1ex}{\arha}   & ha             & *kalın H           & [h]             \\
   Xx     & \raisebox{0.7ex}{\arxa}   & hı             & *sürtmeli H        & [h]             \\
   İi     & \arye                     & ye             & *I, İ veya Y       & [j, ɯ, i]       \\
   Jj     & \raisebox{0.3ex}{\arje}   & je             & J                  & [ʒ]             \\
   Kk     & \arkef                    & kef            & *ince K            & [k]             \\
   \kK\kk & \arkaf                    & kaf            & *kalın K           & [k]             \\
   Ll     & \arlam                    & lam            & L                  & [l]             \\
   Mm     & \armim                    & mim            & M                  & [m]             \\
   Nn     & \raisebox{0.4ex}{\arnun}  & nun            & N                  & [n]             \\
   Ŋŋ     & \raisebox{-0.7ex}{\arnef} & nef, sağır kef & *geniz N           & [ŋ]             \\
   Pp     & \arpe                     & pe             & P                  & [p]             \\
   Rr     & \raisebox{0.4ex}{\arre}   & re             & R                  & [ɾ]             \\
   Ss     & \raisebox{0.6ex}{\arsin}  & sin            & *ince S            & [s]             \\
   \kS\ks & \raisebox{0.6ex}{\arsad}  & sad            & *kalın S           & [s]             \\
   Þþ     & \arthe                    & peltek s       & *peltek S          & [s]             \\
   Şş     & \raisebox{0.4ex}{\arshin} & şın            & Ş                  & [ʃ]             \\
   Tt     & \arte                     & te             & *ince T            & [t]             \\
   \kT\kt & \arta                     & tı             & *kalın T           & [t]             \\
   Vv     & \raisebox{0.6ex}{\arvav}  & vav            & *O, Ö, U, Ü veya V & [o, œ, u, y, v] \\
   Zz     & \raisebox{0.6ex}{\arze}   & ze             & Z                  & [z]             \\
   \kZ\kz & \arza                     & zı             & *kalın Z           & [z]             \\
   Ⱬⱬ     & \arzel                    & zel            & *diş Z             & [z]             \\
  \bottomrule
\end{longtable}
\vspace{-5ex}
\centering
\begin{table}[H]
  \caption{Vekil alfabe. Vekil harfler, yeni yazıda karşılık geldikleri yerlere göre
    sıralanmış olup, rahat karşılaştırma için eski harf isimleri ve konuşmadaki
    telaffuzları ile birlikte verilmiştir. Yeni yazıda birebir karşılığı
    bulunmayan eski harfler yıldız (*) ile işaretlenmiştir.
    Telaffuzda UFA notasyonu kullanılmıştır.}
  \label{tab:vekil1}
\end{table}
\endgroup


% \newpage
% \begingroup
% % \centering
% \mprow{\normalsize Eski yazı \\harf}
% {\normalsize Vekil \\harf}
% {\normalsize Yeni yazı\\karşılığı}
% {\normalsize Eski yazı \\harf ismi}
% {\normalsize Türkçe \\telaffuz}
% \vspace{1ex} \hrule \vspace{1ex}

% \mprow{\arelif   }{Ää     }{*A veya E     }{elif           }{[a, e, æ]       }
% \mprow{\arayn    }{Øø     }{*gırtlak ünlüsü}{ayn            }{[--]            } % ʕſ Ʒʒ
% \mprow{\arbe     }{Bb     }{B            }{be             }{[b]            }
% \mprow{\arcim    }{Cc     }{C            }{cim            }{[d͡ʒ]          }
% \mprow{\archim   }{Çç     }{Ç            }{çim            }{[t͡ʃ]          }
% \mprow{\ardal    }{Dd     }{D            }{dal            }{[d]            }
% \mprow{\ardad    }{Ɖɖ     }{*diş D       }{dad            }{[d, z]          }
% % \mprow{---       }{E      }{e           }{                }{               }
% \mprow{\arfe     }{Ff     }{F            }{fe             }{[f]            }
% \mprow{\argef    }{Gg     }{G            }{gef            }{[ɡ]            }
% \mprow{\argayn   }{Ğğ     }{Ğ            }{gayn           }{[ɡ, ɣ]         }
% \mprow{\arhe     }{Hh     }{*ince H veya E}{he             }{[h, æ]         }
% \mprow{\arha     }{\kH\kh }{*kalın H     }{ha             }{[h]            }
% \mprow{\arxa     }{Xx     }{*sürtmeli H  }{hı             }{[h]            }
% \mprow{\arye     }{İi     }{*I, İ veya Y }{ye             }{[j, ɯ, i]      }
% \mprow{\arje     }{Jj     }{J            }{je             }{[ʒ]            }
% \mprow{\arkef    }{Kk     }{*ince K      }{kef            }{[k]            }
% \mprow{\arkaf    }{\kK\kk }{*kalın K     }{kaf            }{[k]            }
% \mprow{\arlam    }{Ll     }{L            }{lam            }{[l]            }
% \mprow{\armim    }{Mm     }{M            }{mim            }{[m]            }
% \mprow{\arnun    }{Nn     }{N            }{nun            }{[n]            }
% \mprow{\arnef    }{Ŋŋ     }{*geniz N     }{nef, sağır kef }{[ŋ]            }
% \mprow{\arpe     }{Pp     }{P            }{pe             }{[p]            }
% \mprow{\arre     }{Rr     }{R            }{re             }{[ɾ]            }
% \mprow{\arsin    }{Ss     }{*ince S      }{sin            }{[s]            }
% \mprow{\arsad    }{\kS\ks }{*kalın S     }{sad            }{[s]            }
% \mprow{\arthe    }{Þþ     }{*peltek S    }{peltek s       }{[s]            }
% \mprow{\arshin   }{Şş     }{Ş            }{şın            }{[ʃ]            }
% \mprow{\arte     }{Tt     }{*ince T      }{te             }{[t]            }
% \mprow{\arta     }{\kT\kt }{*kalın T     }{tı             }{[t]            }
% \mprow{\arvav    }{Vv     }{*O, Ö, U, Ü veya V}{vav            }{[o, œ, u, y, v]}
% \mprow{\arze     }{Zz     }{Z           }{ze             }{[z]            }
% \mprow{\arza     }{\kZ\kz }{*kalın Z     }{zı             }{[z]            }
% \mprow{\arzel    }{Ⱬⱬ     }{*diş Z       }{zel            }{[z]            }
% \vspace{-1ex} \hrule \vspace{-3ex}
% \begin{table}[H]
% \caption{Vekil alfabe. Vekil harfler, yeni yazıda karşılık geldikleri yerlere göre
%   sıralanmış olup, rahat karşılaştırma için eski harf isimleri ve konuşmadaki
%   telaffuzları ile birlikte verilmiştir. Yeni yazıda birebir karşılığı
%   bulunmayan eski harfler yıldız (*) ile işaretlenmiştir.
%   Telaffuzda UFA notasyonu kullanılmıştır.}
% \end{table}
% \endgroup



% \chapter{Kelime Türetme}

\chapter{Arap Köklü Kelimelerin Kuralları}
\section{Bitişimsiz Morfoloji}

\section{Eylem İsmi Kipleri}

\begin{table}[htbp]
  \footnotesize
  \centering
  \renewcommand{\arraystretch}{1.5}
  \begin{tabular}{p{0.1\tabwidth} >{\raggedright}p{0.3\tabwidth} >{\raggedright}p{0.2\tabwidth} >{\raggedright}p{0.2\tabwidth} p{0.2\tabwidth}}
    Kip No & Anlam & Eylem & Eden & Edilen \\
    \toprule
    \rom{1} & Sade & Necz, Nücûz, Nicz, Nücz(et), Necâz(et), Nicâz(et), vs. &  Nâciz & Mencuz \\
    \rom{2} & Geçişli, ettiren, güçlü & Tenciz &  Münecciz & Müneccez \\
    \rom{3} & İşteş & Münâceze &  Münâciz & Münâcez \\
    \rom{4} & Geçişli, ettiren & İncâz &  Münciz & Müncez \\
    \rom{5} & Dönüşlü, ettiren, güçlü & Teneccüz &  Mütenecciz & Müteneccez \\
    \rom{6} & \rom{3}'ün işteş karşılığı & Tenâcüz &  Mütenâciz & Mütenâcez \\
    \rom{7} & Dönüşlü, ettirilen & İnnicâz & Münneciz & Münnecez \\
    \rom{8} & \rom{1}'in dönüşlüsü, geçişsiz & İnticâz & Münteciz & Müntecez \\
    \rom{9} & Durum eylemi, geçişsiz & İncizâz & Müncezz & --- \\
    \rom{10} & Ettiren, bazen ettirilen, çeşitli anlamlar  & İstincâz & Müstenciz & Müstencez \\
    % \midrule
    % \rom{11} & \rom{9}'un aynısı, şiir dışında nadir & İncîzaz & Müncâzz & --- \\
    % \rom{12} & \multirow{4}{*}{\parbox{0.3\tabwidth}{\raggedright Durum eylemi, çok nadir}} & İncîcaz & Müncavciz & Müncavcez \\
    % \rom{13} &  & İncivvaz & Müncavviz & Müncavvez \\
    % \rom{14} &  & İncinzâz & Müncanziz & Müncanzez \\
    % \rom{15} &  & İncinzâ' & Müncanzin & Müncanzen \\
    \bottomrule
  \end{tabular}
\end{table}

\subsection{Nâciz Kipi: Yapan}

\subsection{Mencuz Kipi: Yapılan}

\subsection{İncaz Kipi: Geçişli (Transitif) Eylem}


\chapter{Fars Köklü Kelimelerin Kuralları}
\section{Fars Köklü Neo-osmanlıca Ekler}
\fat

\end{document}
