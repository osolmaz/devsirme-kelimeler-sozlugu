\documentclass[b5paper,11pt, twoside]{scrbook}
% \usepackage[turkish]{babel}
% \newlength{\tabwidth}
% \setlength{\tabwidth}{8.7cm}

% all the local settings defined in /localsettings.sty
\usepackage{localsettings}
\usepackage{arabicletters}

% lazyeqn - math symbols
% Uncomment if you have chosen to clone it to your project
% \usepackage{./lazyeqn/lazyeqn}

% \usepackage{caption}
\usepackage{formatting}
\usepackage{kipdesc}

\newcommand*\varhrulefill[1][0.4pt]{\leavevmode\leaders\hrule height#1\hfill\kern0pt}

\usepackage{calc}
\newlength{\linew}
\setlength{\linew}{\textwidth+2em}
\newlength{\ltw}
\setlength{\ltw}{0.08333\linew}
\newlength{\rowh}
\setlength{\rowh}{0.6cm}


%  Türkçe İçindeki Farsoarap Unsurların Ortografi, Gramer ve Kelime Türetme Esasları
  %   Ortografi, Gramer ve Kelime Türetme Kuralları
\title{DEVŞİRME KELİMELER SÖZLÜĞÜ
  \\[3em]
  \large CİLT I
  \\[1.5em]
  ARAP KÖKENLİ KELİMELERİN TÜRETME KURALLARI}
\author{Onur Solmaz}
\date{Versiyon: 25 Temmuz 2018}

 % Â, â, Î, î, Û, û
\raggedbottom
\begin{document}

% \pagestyle{plain}
\maketitle
\tableofcontents

\newpage

\section*{Önsöz}


\thispagestyle{empty}
\cleardoublepage



\newpage


\chapter{Giriş}


\newcommand\uyumluharftablo[4]{%
  % \renewcommand{\arraystretch}{3}
  \begin{tabular}{>{\centering}p{0.15\textwidth}%
    >{\centering}p{0.15\textwidth}%
    >{\centering}p{0.25\textwidth}%
    >{\centering}p{0.25\textwidth}}
    \toprule
    \footnotesize Eski yazı harf
    & \footnotesize Vekil harf
    & \footnotesize Eski yazı harf ismi
    & \footnotesize Türkçe telaffuz
      \tabularnewline
      \midrule
    \begin{minipage}[c][10ex]{0.15\textwidth} \centering \Huge #1 \end{minipage}
    & \begin{minipage}[c][10ex]{0.15\textwidth} \centering \Huge #2 \end{minipage}
    & #3
    & \timesfont #4
      \tabularnewline
    \bottomrule
  \end{tabular}%
}

\newcommand\hariciharftablo[5]{%
  % \renewcommand{\arraystretch}{3}
  \begin{tabular}{>{\centering}p{0.15\textwidth}%
    >{\centering}p{0.15\textwidth}%
    >{\centering}p{0.20\textwidth}%
    >{\centering}p{0.17\textwidth}%
    >{\centering}p{0.17\textwidth}%
    }
    \toprule
    \footnotesize Eski yazı harf
    & \footnotesize Vekil harf
    & \footnotesize Eski yazı harf ismi
    & \footnotesize Türkçe telaffuz
    & \footnotesize Arapça telaffuz
      \tabularnewline
      \midrule
    \begin{minipage}[c][10ex]{0.15\textwidth} \centering \Huge #1 \end{minipage}
    & \begin{minipage}[c][10ex]{0.15\textwidth} \centering \Huge #2 \end{minipage}
    & #3
    & \timesfont #4
    & \timesfont #5
      \tabularnewline
    \bottomrule
  \end{tabular}%
}


\chapter{Ortografi}


% \begin{tikzpicture}
  \bfseries
  \small
  \node(01_1)[text depth=0.25ex, text height=2.5ex, text centered] at (0   , 0) {\latupalif};
  \node(02_1)[text depth=0.25ex, text height=2.5ex, text centered] at (5   , 0) {\latupbe};
  \node(03_1)[text depth=0.25ex, text height=2.5ex, text centered] at (10  , 0) {\latuppe};
  \node(04_1)[text depth=0.25ex, text height=2.5ex, text centered] at (15  , 0) {\latupthe};
  \node(05_1)[text depth=0.25ex, text height=2.5ex, text centered] at (20  , 0) {\latupte};
  \node(06_1)[text depth=0.25ex, text height=2.5ex, text centered] at (25  , 0) {\latupcim};
  \node(07_1)[text depth=0.25ex, text height=2.5ex, text centered] at (30  , 0) {\latupchim};
  \node(08_1)[text depth=0.25ex, text height=2.5ex, text centered] at (35  , 0) {\latupha};
  \node(09_1)[text depth=0.25ex, text height=2.5ex, text centered] at (40  , 0) {\latupxa};
  \node(10_1)[text depth=0.25ex, text height=2.5ex, text centered] at (45  , 0) {\latupdal};
  \node(11_1)[text depth=0.25ex, text height=2.5ex, text centered] at (50  , 0) {\latupzel};
  \node(12_1)[text depth=0.25ex, text height=2.5ex, text centered] at (55  , 0) {\latupre};
  \node(13_1)[text depth=0.25ex, text height=2.5ex, text centered] at (60  , 0) {\latupze};
  \node(14_1)[text depth=0.25ex, text height=2.5ex, text centered] at (65  , 0) {\latupje};
  \node(15_1)[text depth=0.25ex, text height=2.5ex, text centered] at (70  , 0) {\latupsin};
  \node(16_1)[text depth=0.25ex, text height=2.5ex, text centered] at (75  , 0) {\latupshin};
  \node(17_1)[text depth=0.25ex, text height=2.5ex, text centered] at (80  , 0) {\latupsad};
  \node(18_1)[text depth=0.25ex, text height=2.5ex, text centered] at (85  , 0) {\latupdad};
  \node(19_1)[text depth=0.25ex, text height=2.5ex, text centered] at (90  , 0) {\latupta};
  \node(20_1)[text depth=0.25ex, text height=2.5ex, text centered] at (95  , 0) {\latupza};
  \node(21_1)[text depth=0.25ex, text height=2.5ex, text centered] at (100 , 0) {\latupayn};
  \node(22_1)[text depth=0.25ex, text height=2.5ex, text centered] at (105 , 0) {\latupgayn};
  \node(23_1)[text depth=0.25ex, text height=2.5ex, text centered] at (110 , 0) {\latupfe};
  \node(24_1)[text depth=0.25ex, text height=2.5ex, text centered] at (115 , 0) {\latupkaf};
  \node(25_1)[text depth=0.25ex, text height=2.5ex, text centered] at (120 , 0) {\latupkef};
  \node(26_1)[text depth=0.25ex, text height=2.5ex, text centered] at (125 , 0) {\latupgef};
  \node(27_1)[text depth=0.25ex, text height=2.5ex, text centered] at (130 , 0) {\latupnef};
  \node(28_1)[text depth=0.25ex, text height=2.5ex, text centered] at (135 , 0) {\latuplam};
  \node(29_1)[text depth=0.25ex, text height=2.5ex, text centered] at (140 , 0) {\latupmim};
  \node(30_1)[text depth=0.25ex, text height=2.5ex, text centered] at (145 , 0) {\latupnun};
  \node(31_1)[text depth=0.25ex, text height=2.5ex, text centered] at (150 , 0) {\latupvav};
  \node(32_1)[text depth=0.25ex, text height=2.5ex, text centered] at (155 , 0) {\latuphe};
  \node(33_1)[text depth=0.25ex, text height=2.5ex, text centered] at (160 , 0) {\latupye};

  \scriptsize
  \node(01_2)[text depth=2.25ex, text height=2ex, text centered] at (0   , 20) {\aralif};
  \node(02_2)[text depth=2.25ex, text height=2ex, text centered] at (5   , 20) {\arbe};
  \node(03_2)[text depth=2.25ex, text height=2ex, text centered] at (10  , 20) {\arpe};
  \node(04_2)[text depth=2.25ex, text height=2ex, text centered] at (15  , 20) {\arthe};
  \node(05_2)[text depth=2.25ex, text height=2ex, text centered] at (20  , 20) {\arte};
  \node(06_2)[text depth=2.25ex, text height=2ex, text centered] at (25  , 20) {\arcim};
  \node(07_2)[text depth=2.25ex, text height=2ex, text centered] at (30  , 20) {\archim};
  \node(08_2)[text depth=2.25ex, text height=2ex, text centered] at (35  , 20) {\arha};
  \node(09_2)[text depth=2.25ex, text height=2ex, text centered] at (40  , 20) {\arxa};
  \node(10_2)[text depth=2.25ex, text height=2ex, text centered] at (45  , 20) {\ardal};
  \node(11_2)[text depth=2.25ex, text height=2ex, text centered] at (50  , 20) {\arzel};
  \node(12_2)[text depth=2.25ex, text height=2ex, text centered] at (55  , 20) {\arre};
  \node(13_2)[text depth=2.25ex, text height=2ex, text centered] at (60  , 20) {\arze};
  \node(14_2)[text depth=2.25ex, text height=2ex, text centered] at (65  , 20) {\arje};
  \node(15_2)[text depth=2.25ex, text height=2ex, text centered] at (70  , 20) {\arsin};
  \node(16_2)[text depth=2.25ex, text height=2ex, text centered] at (75  , 20) {\arshin};
  \node(17_2)[text depth=2.25ex, text height=2ex, text centered] at (80  , 20) {\arsad};
  \node(18_2)[text depth=2.25ex, text height=2ex, text centered] at (85  , 20) {\ardad};
  \node(19_2)[text depth=2.25ex, text height=2ex, text centered] at (90  , 20) {\arta};
  \node(20_2)[text depth=2.25ex, text height=2ex, text centered] at (95  , 20) {\arza};
  \node(21_2)[text depth=2.25ex, text height=2ex, text centered] at (100 , 20) {\arayn};
  \node(22_2)[text depth=2.25ex, text height=2ex, text centered] at (105 , 20) {\argayn};
  \node(23_2)[text depth=2.25ex, text height=2ex, text centered] at (110 , 20) {\arfe};
  \node(24_2)[text depth=2.25ex, text height=2ex, text centered] at (115 , 20) {\arkaf};
  \node(25_2)[text depth=2.25ex, text height=2ex, text centered] at (120 , 20) {\arkef};
  \node(26_2)[text depth=2.25ex, text height=2ex, text centered] at (125 , 20) {\argef};
  \node(27_2)[text depth=2.25ex, text height=2ex, text centered] at (130 , 20) {\arnef};
  \node(28_2)[text depth=2.25ex, text height=2ex, text centered] at (135 , 20) {\arlam};
  \node(29_2)[text depth=2.25ex, text height=2ex, text centered] at (140 , 20) {\armim};
  \node(30_2)[text depth=2.25ex, text height=2ex, text centered] at (145 , 20) {\arnun};
  \node(31_2)[text depth=2.25ex, text height=2ex, text centered] at (150 , 20) {\arvav};
  \node(32_2)[text depth=2.25ex, text height=2ex, text centered] at (155 , 20) {\arhe};
  \node(33_2)[text depth=2.25ex, text height=2ex, text centered] at (160 , 20) {\arye};

  \draw[->](01_2) edge (01_1);


\draw[->] (01_2) edge (01_1);
\draw[->] (02_2) edge (02_1);
\draw[->] (03_2) edge (03_1);
\draw[->] (04_2) edge (04_1);
\draw[->] (05_2) edge (05_1);
\draw[->] (06_2) edge (06_1);
\draw[->] (07_2) edge (07_1);
\draw[->] (08_2) edge (08_1);
\draw[->] (09_2) edge (09_1);
\draw[->] (10_2) edge (10_1);
\draw[->] (11_2) edge (11_1);
\draw[->] (12_2) edge (12_1);
\draw[->] (13_2) edge (13_1);
\draw[->] (14_2) edge (14_1);
\draw[->] (15_2) edge (15_1);
\draw[->] (16_2) edge (16_1);
\draw[->] (17_2) edge (17_1);
\draw[->] (18_2) edge (18_1);
\draw[->] (19_2) edge (19_1);
\draw[->] (20_2) edge (20_1);
\draw[->] (21_2) edge (21_1);
\draw[->] (22_2) edge (22_1);
\draw[->] (23_2) edge (23_1);
\draw[->] (24_2) edge (24_1);
\draw[->] (25_2) edge (25_1);
\draw[->] (27_2) edge (27_1);
\draw[->] (26_2) edge (26_1);
\draw[->] (28_2) edge (28_1);
\draw[->] (29_2) edge (29_1);
\draw[->] (30_2) edge (30_1);
\draw[->] (31_2) edge (31_1);
\draw[->] (32_2) edge (32_1);
\draw[->] (33_2) edge (33_1);



































\end{tikzpicture}

\includegraphics[width=\textwidth]{./fig/fig1.tikz}
\section{Türk Diliyle Uyumlu Sesler ve Harfler}

\subsection*{Elif}


\uyumluharftablo{\aralif}{\Lalif\latdownalif}{\isimalif}{\trtlfalif}



\subsection*{Be}
\uyumluharftablo{\arbe}{\Lbe\latdownbe}{\isimbe}{\trtlfbe}


\subsection*{Pe}
\uyumluharftablo{\arpe}{\Lpe\latdownpe}{\isimpe}{\trtlfpe}

\subsection*{Te}
\uyumluharftablo{\arte}{\Lte\latdownte}{\isimte}{\trtlfte}

\subsection*{Cim}
\uyumluharftablo{\arcim}{\Lcim\latdowncim}{\isimcim}{\trtlfcim}

\subsection*{Çim}
\uyumluharftablo{\archim}{\Lchim\latdownchim}{\isimchim}{\trtlfchim}


\subsection*{Dal}
\uyumluharftablo{\ardal}{\Ldal\latdowndal}{\isimdal}{\trtlfdal}

\subsection*{Re}
\uyumluharftablo{\arre}{\Lre\latdownre}{\isimre}{\trtlfre}

\subsection*{Ze}
\uyumluharftablo{\arze}{\Lze\latdownze}{\isimze}{\trtlfze}

\subsection*{Je}
\uyumluharftablo{\arje}{\Lje\latdownje}{\isimje}{\trtlfje}

\subsection*{Sin}
\uyumluharftablo{\arsin}{\Lsin\latdownsin}{\isimsin}{\trtlfsin}

\subsection*{Şın}
\uyumluharftablo{\arshin}{\Lshin\latdownshin}{\isimshin}{\trtlfshin}

\subsection*{Gayn}
\uyumluharftablo{\argayn}{\Lgayn\latdowngayn}{\isimgayn}{\trtlfgayn}

\subsection*{Fe}
\uyumluharftablo{\arfe}{\Lfe\latdownfe}{\isimfe}{\trtlffe}

\subsection*{Kef}
\uyumluharftablo{\arkef}{\Lkef\latdownkef}{\isimkef}{\trtlfkef}

\subsection*{Gef}
\uyumluharftablo{\argef}{\Lgef\latdowngef}{\isimgef}{\trtlfgef}

\subsection*{Nef / Sağır Kef}
\uyumluharftablo{\arnef}{\Lnef\latdownnef}{\isimnef}{\trtlfnef}

\subsection*{Lam}
\uyumluharftablo{\arlam}{\Llam\latdownlam}{\isimlam}{\trtlflam}

\subsection*{Mim}
\uyumluharftablo{\armim}{\Lmim\latdownmim}{\isimmim}{\trtlfmim}

\subsection*{Nun}
\uyumluharftablo{\arnun}{\Lnun\latdownnun}{\isimnun}{\trtlfnun}

\subsection*{Vav}
\uyumluharftablo{\arvav}{\Lvav\latdownvav}{\isimvav}{\trtlfvav}

\subsection*{He}
\uyumluharftablo{\arhe}{\Lhe\latdownhe}{\isimhe}{\trtlfhe}

\subsection*{Ye}
\uyumluharftablo{\arye}{\Lye\latdownye}{\isimye}{\trtlfye}



\section{Türk Diline Yabancı Sesler ve Harfler}

\subsection*{Peltek Se}
\hariciharftablo{\arthe}{\Lthe\latdownthe}{\isimthe}{\trtlfthe}{\artlfthe}

\subsection*{Ha}
\hariciharftablo{\arha}{\Lha\latdownha}{\isimha}{\trtlfha}{\artlfha}

\subsection*{Hı}
\hariciharftablo{\arxa}{\Lxa\latdownxa}{\isimxa}{\trtlfxa}{\artlfxa}

\subsection*{Zel}
\hariciharftablo{\arzel}{\Lzel\latdownzel}{\isimzel}{\trtlfzel}{\artlfzel}

\subsection*{Sad}
\hariciharftablo{\arsad}{\Lsad\latdownsad}{\isimsad}{\trtlfsad}{\artlfsad}

\subsection*{Dad}
\hariciharftablo{\ardad}{\Ldad\latdowndad}{\isimdad}{\trtlfdad}{\artlfdad}

\subsection*{Ta}
\hariciharftablo{\arta}{\Lta\latdownta}{\isimta}{\trtlfta}{\artlfta}

\subsection*{Za}
\hariciharftablo{\arza}{\Lza\latdownza}{\isimza}{\trtlfza}{\artlfza}

\subsection*{Ayn}
\hariciharftablo{\arayn}{\Layn\latdownayn}{\isimayn}{\trtlfayn}{\artlfayn}

% \vek{\mim\sinn\te\hemzup{\alif}\cim\re}

% \vek{\kesre{\fethe{\ze}}} \vek{\sukun{\ze}} \vek{\otre{\ze}} \vek{\madde{\ze}}
% \vek{\sedde{\ze}}

\subsection*{Kaf}
\hariciharftablo{\arkaf}{\Lkaf\latdownkaf}{\isimkaf}{\trtlfkaf}{\artlfkaf}



% \section{Eski Yazı}

% sf/tt/rm: \textsf{x}\texttt{x}x

% sf/tt/rm: \textsf{K}\texttt{K}K \textsf{\Lkaf}\texttt{\Lkaf}\Lkaf

% \section{Vekil Alfabe}

\newpage
\begingroup
% \setlength\extrarowh{5pt}
% \begin{table}[htbp]
  % \centering
\renewcommand{\arraystretch}{2.1}
\begin{longtable*}{>{\LARGE}p{1.4\ltw}>{\LARGE}p{1.4\ltw}>{}p{1.9\ltw}>{}p{2.8\ltw}>{\timesfont}p{2\ltw}}
   \small Vekil \newline harf
                              & \small Eski yazı \newline harf
                              & \small Eski yazı \newline harf ismi
                              & \small Yeni yazı\newline karşılığı
                              & \small\normalfont Türkçe \newline telaffuz                                       \\
  %
  % \normalsize Eskiyazı Harf & \normalsize Vekil Harf       & \normalsize
  % Yeni yazı Mukabili        & \normalsize Eskiyazı okunuşu & \normalsize Türkçe \mbox{Okunuşu}                 \\
  \toprule
  \Lalif \latdownalif     & \aralif                      & \isimalif   & *A veya E          & \trtlfalif  \\
  \Layn  \latdownayn      & \raisebox{0.6ex}{\arayn}     & \isimayn    & *gırtlak ünlüsü    & \trtlfayn   \\ % ʕſ Ʒʒ
  \Lbe   \latdownbe       & \arbe                        & \isimbe     & B                  & \trtlfbe    \\
  \Lcim  \latdowncim      & \raisebox{0.8ex}{\arcim}     & \isimcim    & C                  & \trtlfcim   \\
  \Lchim \latdownchim     & \raisebox{0.8ex}{\archim}    & \isimchim   & Ç                  & \trtlfchim  \\
  \Ldal  \latdowndal      & \ardal                       & \isimdal    & D                  & \trtlfdal   \\
  \Ldad  \latdowndad      & \raisebox{0.8ex}{\ardad}     & \isimdad    & *diş D             & \trtlfdad   \\
  \Lfe   \latdownfe       & \arfe                        & \isimfe     & F                  & \trtlffe    \\
  \Lgef  \latdowngef      & \argef                       & \isimgef    & G                  & \trtlfgef   \\
  \Lgayn \latdowngayn     & \raisebox{0.3ex}{\argayn}    & \isimgayn   & genelde Ğ          & \trtlfgayn  \\
  \Lhe   \latdownhe       & \arhe                        & \isimhe     & *ince H veya E     & \trtlfhe    \\
  \Lha   \latdownha       & \raisebox{1.1ex}{\arha}      & \isimha     & *kalın H           & \trtlfha    \\
  \Lxa   \latdownxa       & \raisebox{0.7ex}{\arxa}      & \isimxa     & *sürtmeli H        & \trtlfxa    \\
  \Lye   \latdownye       & \arye                        & \isimye     & *I, İ veya Y       & \trtlfye    \\
  \Lje   \latdownje       & \raisebox{0.3ex}{\arje}      & \isimje     & J                  & \trtlfje    \\
  \Lkef  \latdownkef      & \arkef                       & \isimkef    & *ince K            & \trtlfkef   \\
  \Lkaf  \latdownkaf      & \arkaf                       & \isimkaf    & *kalın K           & \trtlfkaf   \\
  \Llam  \latdownlam      & \arlam                       & \isimlam    & L                  & \trtlflam   \\
  \Lmim  \latdownmim      & \armim                       & \isimmim    & M                  & \trtlfmim   \\
  \Lnun  \latdownnun      & \raisebox{0.4ex}{\arnun}     & \isimnun    & N                  & \trtlfnun   \\
  \Lnef  \latdownnef      & \raisebox{-0.7ex}{\arnef}    & \isimnef    & *geniz N           & \trtlfnef   \\
  \Lpe   \latdownpe       & \arpe                        & \isimpe     & P                  & \trtlfpe    \\
  \Lre   \latdownre       & \raisebox{0.4ex}{\arre}      & \isimre     & R                  & \trtlfre    \\
  \Lsin  \latdownsin      & \raisebox{0.6ex}{\arsin}     & \isimsin    & *ince S            & \trtlfsin   \\
  \Lsad  \latdownsad      & \raisebox{0.6ex}{\arsad}     & \isimsad    & *kalın S           & \trtlfsad   \\
  \Lthe  \latdownthe      & \arthe                       & \isimthe    & *peltek S          & \trtlfthe   \\
  \Lshin \latdownshin     & \raisebox{0.4ex}{\arshin}    & \isimshin   & Ş                  & \trtlfshin  \\
  \Lte   \latdownte       & \arte                        & \isimte     & *ince T            & \trtlfte    \\
  \Lta   \latdownta       & \arta                        & \isimta     & *kalın T           & \trtlfta    \\
  \Lvav  \latdownvav      & \raisebox{0.6ex}{\arvav}     & \isimvav    & *O,Ö,U,Ü veya V    & \trtlfvav   \\
  \Lze   \latdownze       & \raisebox{0.6ex}{\arze}      & \isimze     & Z                  & \trtlfze    \\
  \Lza   \latdownza       & \arza                        & \isimza     & *kalın Z           & \trtlfza    \\
  \Lzel  \latdownzel      & \raisebox{0.2ex}{\arzel}     & \isimzel    & *diş Z             & \trtlfzel   \\
  \Lhemze                 & \arhemze                     & \isimhemze  & *gırtlak kapatması & \trtlfhemze \\ % ʕſ Ʒʒ
  \bottomrule
\end{longtable*}
\vspace{-6ex}
\centering
\begin{table}[H]
  \caption{Vekil alfabe. Vekil harfler, yeni yazıda karşılık geldikleri yerlere göre
    sıralanmış olup, rahat karşılaştırma için eski harf isimleri ve konuşmadaki
    telaffuzları ile birlikte verilmiştir. Yeni yazıda birebir karşılığı
    bulunmayan eski harfler yıldız (*) ile işaretlenmiştir.
    Telaffuzda UFA notasyonu kullanılmıştır.}
  \label{tab:vekil1}
\end{table}
\endgroup





\chapter{Arap Köklü Kelimelerin Kuralları}
\section{Eklemesiz Morfoloji}

\section{Çokluk Kipleri}

\section{Eylem Kipleri}

% causative x anticausative: ettirgen x ettirilgen
% transitive x intransitive: geçişli, geçişsiz
% reflexive: dönüşlü
% intensive: güçlü
% stative: durum eylemi


% \subsection*{Kip \rom{1}}
% \subsubsection*{Eylem: \declen{NeCZ}, \declen{NiCZ}, \declen{NüCZ}, vs.}
% \subsubsection*{Etken: \declen{NâCiZ}}
% \subsubsection*{Edilgen: \declen{meNCûZ}}

% % \subsection{İncaz Kipi: Geçişli (Transitif) Eylem}

% \subsection*{Kip \rom{2}}

% \subsubsection*{Eylem: \declen{teNCîZ}}
% \subsubsection*{Etken: \declen{müNeCCiZ}}
% \subsubsection*{Edilgen: \declen{müNeCCeZ}}

\subsection*{Kip \rom{1}}


\begin{kip}{\declenlarge{KeML}, \declenlarge{KiML}, \declenlarge{KüML},
    vs. \hfill Kip \rom{1}, Eylem Adı}
  Eylem ismi
\end{kip}

\begin{kip}{\declenlarge{KâMiL} \hfill Kip \rom{1}, Etken Partisip}
  Etken eylem ismi
\end{kip}

\begin{kip}{\declenlarge{meKMûL} \hfill Kip \rom{1}, Edilgen Partisip}
  Edilgen eylem ismi
\end{kip}


% \subsection{İncaz Kipi: Geçişli (Transitif) Eylem}

\subsection*{Kip \rom{2}}

\begin{kip}{\declenlarge{teKMîL} \hfill Kip \rom{2}, Eylem Adı}
  Eylem ismi
\end{kip}

\begin{kip}{\declenlarge{müKeMMiL} \hfill Kip \rom{2}, Etken Partisip}
  Etken eylem ismi
\end{kip}

\begin{kip}{\declenlarge{müKeMMeL} \hfill Kip \rom{2}, Edilgen Partisip}
  Edilgen eylem ismi
\end{kip}


\subsection*{Kip \rom{3}}

\begin{kip}{\declenlarge{müKâMeLe} \hfill Kip \rom{3}, Eylem Adı}
  Eylem ismi
\end{kip}

\begin{kip}{\declenlarge{müKâMiL} \hfill Kip \rom{3}, Etken Partisip}
  Etken eylem ismi
\end{kip}

\begin{kip}{\declenlarge{müKâMeL} \hfill Kip \rom{3}, Edilgen Partisip}
  Edilgen eylem ismi
\end{kip}


\subsection*{Kip \rom{4}}

\begin{kip}{\declenlarge{iKMâL} \hfill Kip \rom{4}, Eylem Adı}
  Eylem ismi
\end{kip}

\begin{kip}{\declenlarge{müKMiL} \hfill Kip \rom{4}, Etken Partisip}
  Etken eylem ismi
\end{kip}

\begin{kip}{\declenlarge{müKMeL} \hfill Kip \rom{4}, Edilgen Partisip}
  Edilgen eylem ismi
\end{kip}


\subsection*{Kip \rom{5}}

\begin{kip}{\declenlarge{teKeMMüL} \hfill Kip \rom{5}, Eylem Adı}
  Eylem ismi
\end{kip}

\begin{kip}{\declenlarge{müteKeMMiL} \hfill Kip \rom{5}, Etken Partisip}
  Etken eylem ismi
\end{kip}

\begin{kip}{\declenlarge{müteKeMMeL} \hfill Kip \rom{5}, Edilgen Partisip}
  Edilgen eylem ismi
\end{kip}


\subsection*{Kip \rom{6}}

\begin{kip}{\declenlarge{teKâMüL} \hfill Kip \rom{6}, Eylem Adı}
  Eylem ismi
\end{kip}

\begin{kip}{\declenlarge{müteKâMiL} \hfill Kip \rom{6}, Etken Partisip}
  Etken eylem ismi
\end{kip}

\begin{kip}{\declenlarge{müteKâMeL} \hfill Kip \rom{6}, Edilgen Partisip}
  Edilgen eylem ismi
\end{kip}


\subsection*{Kip \rom{7}}

\begin{kip}{\declenlarge{inKiMâL} \hfill Kip \rom{7}, Eylem Adı}
  Eylem ismi
\end{kip}

\begin{kip}{\declenlarge{münKeMiL} \hfill Kip \rom{7}, Etken Partisip}
  Etken eylem ismi
\end{kip}

\begin{kip}{\declenlarge{münKeMeL} \hfill Kip \rom{7}, Edilgen Partisip}
  Edilgen eylem ismi
\end{kip}

\subsection*{Kip \rom{8}}

\begin{kip}{\declenlarge{iKtimâL} \hfill Kip \rom{8}, Eylem Adı}
  Eylem ismi
\end{kip}

\begin{kip}{\declenlarge{müKteMiL} \hfill Kip \rom{8}, Etken Partisip}
  Etken eylem ismi
\end{kip}

\begin{kip}{\declenlarge{müKteMeL} \hfill Kip \rom{8}, Edilgen Partisip}
  Edilgen eylem ismi
\end{kip}


\subsection*{Kip \rom{9}}

\begin{kip}{\declenlarge{iKMiLâL} \hfill Kip \rom{9}, Eylem Adı}
  Eylem ismi
\end{kip}

\begin{kip}{\declenlarge{müKMeLL} \hfill Kip \rom{9}, Etken Partisip}
  Etken eylem ismi
\end{kip}


\subsection*{Kip \rom{10}}

\begin{kip}{\declenlarge{istiKMâL} \hfill Kip \rom{10}, Eylem Adı}
  Eylem ismi
\end{kip}

\begin{kip}{\declenlarge{müsteKMiL} \hfill Kip \rom{10}, Etken Partisip}
  Etken eylem ismi
\end{kip}

\begin{kip}{\declenlarge{müsteKMeL} \hfill Kip \rom{10}, Edilgen Partisip}
  Edilgen eylem ismi
\end{kip}





\begin{table}[htbp]
  \footnotesize
  \centering
  \renewcommand{\arraystretch}{1.5}
  \begin{tabular}{p{0.1\tabwidth} >{\raggedright}p{0.3\tabwidth} >{\raggedright}p{0.2\tabwidth} >{\raggedright}p{0.2\tabwidth} p{0.2\tabwidth}}
    Kip & Anlam & Eylem & Etken & Edilgen \\
    \toprule
    \rom{1} & Sade & Keml, Kiml, Küml, vs. &  Kâmil & Mekmûl \\
    % \rom{1} & Sade & Necz, Nücûz, Nicz, Nücz(et), Necâz(et), Nicâz(et), vs. &  Nâciz & Mencuz \\
    \rom{2} & Geçişli, ettirgen, güçlü & Tekmîl &  Mükemmil & Mükemmel \\
    \rom{3} & İşteş & Mükâmele &  Mükâmil & Mükâmel \\
    \rom{4} & Geçişli, ettirgen & İkmâl &  Mükmil & Mükmel \\
    % \rom{5} & Dönüşlü, ettirilgen, güçlü
    \rom{5} & \rom{2}'nin dönüşlüsü, genellikle geçişsiz
                & Tekemmül &  Mütekemmil & Mütekemmel \\
    % \rom{6} & \rom{2}'nin dönüşlüsü, genellikle geçişsiz
    \rom{6} & \rom{3}'deki işteş eylemin muhattabı, dönüşlüsü, genellikle geçişsiz
                & Tekâmül &  Mütekâmil & Mütekâmel \\
    \rom{7} & Dönüşlü, ettirilgen & İnkimâl & Münkemil & Münkemel \\
    \rom{8} & \rom{1}'in dönüşlüsü, geçişsiz & İktimâl & Müktemil & Müktemel \\
    \rom{9} & Durum eylemi, geçişsiz & İkmilâl & Mükmell & --- \\
    \rom{10} & Ettirgen, bazen ettirilgen, vs.  & İstikmâl & Müstekmil & Müstekmel \\
    % \midrule
    % \rom{11} & \rom{9}'un aynısı, şiir dışında nadir & İncîzaz & Müncâzz & --- \\
    % \rom{12} & \multirow{4}{*}{\parbox{0.3\tabwidth}{\raggedright Durum eylemi, çok nadir}} & İncîcaz & Müncavciz & Müncavcez \\
    % \rom{13} &  & İncivvaz & Müncavviz & Müncavvez \\
    % \rom{14} &  & İncinzâz & Müncanziz & Müncanzez \\
    % \rom{15} &  & İncinzâ' & Müncanzin & Müncanzen \\
    \bottomrule
  \end{tabular}
\end{table}



\section{Muhtelif Kipler}
















\chapter{Arap Köklü Kelimeler Sözlüğü}

\newpage

\recalctypearea
\newgeometry{left=8mm,right=8mm,top=18mm, bottom=25mm}


\setlength{\parindent}{0pt}
% \begin{multicols}{3}
% \sffamily
\noindent
\small
\dictchapter{A}
\begin{multicols}{2}
\kipkokentry{abd}{\latupayn\latupbe\latupdal}{None}{kok:ayn_be_dal}
\kipkokentry{abdal}{\latupbe\latupdal\latuplam}{None}{kok:be_dal_lam}
\kipkokentry{abes}{\latupayn\latupbe\latupthe}{None}{kok:ayn_be_the}
\kipkokentry{âbide}{\latupalif\latupbe\latupdal}{None}{kok:alif_be_dal}
\kipkokentry{âbit}{\latupayn\latupbe\latupdal}{K\rom{1}, Ed.}{kok:ayn_be_dal}
\kipkokentry{ablak}{\latupbe\latuplam\latupkaf}{None}{kok:be_lam_kaf}
\kipkokentry{abraş}{\latupbe\latupre\latupshin}{None}{kok:be_re_shin}
\kipkokentry{abus}{\latupayn\latupbe\latupsin}{None}{kok:ayn_be_sin}
\kipkokentry{acabâ}{\latupayn\latupcim\latupbe}{None}{kok:ayn_cim_be}
\kipkokentry{acar}{\latupayn\latupcim\latupre}{None}{kok:ayn_cim_re}
\kipkokentry{acayip}{\latupayn\latupcim\latupbe}{None}{kok:ayn_cim_be}
\kipkokentry{acele}{\latupayn\latupcim\latuplam}{None}{kok:ayn_cim_lam}
\kipkokentry{acem}{\latupayn\latupcim\latupmim}{None}{kok:ayn_cim_mim}
\kipkokentry{acep}{\latupayn\latupcim\latupbe}{None}{kok:ayn_cim_be}
\kipkokentry{âcil}{\latupayn\latupcim\latuplam}{K\rom{1}, Ed.}{kok:ayn_cim_lam}
\kipkokentry{âciz}{\latupayn\latupcim\latupze}{K\rom{1}, Ed.}{kok:ayn_cim_ze}
\kipkokentry{acul}{\latupayn\latupcim\latuplam}{None}{kok:ayn_cim_lam}
\kipkokentry{acûze}{\latupayn\latupcim\latupze}{None}{kok:ayn_cim_ze}
\kipkokentry{acz}{\latupayn\latupcim\latupze}{None}{kok:ayn_cim_ze}
\kipkokentry{adale}{\latupayn\latupdad\latuplam}{None}{kok:ayn_dad_lam}
\kipkokentry{adâlet}{\latupayn\latupdal\latuplam}{None}{kok:ayn_dal_lam}
\kipkokentry{adam}{\latupalif\latupdal\latupmim}{None}{kok:alif_dal_mim}
\kipkokentry{âdap}{\latupalif\latupdal\latupbe}{None}{kok:alif_dal_be}
\kipkokentry{adâvet}{\latupayn\latupdal\latupvav}{None}{kok:ayn_dal_vav}
\kipkokentry{add}{\latupayn\latupdal\latupdal}{None}{kok:ayn_dal_dal}
\kipkokentry{âdem}{\latupalif\latupdal\latupmim}{None}{kok:alif_dal_mim}
\kipkokentry{adem}{\latupayn\latupdal\latupmim}{None}{kok:ayn_dal_mim}
\kipkokentry{adese}{\latupayn\latupdal\latupsin}{None}{kok:ayn_dal_sin}
\kipkokentry{adet}{\latupayn\latupdal\latupdal}{None}{kok:ayn_dal_dal}
\kipkokentry{âdet}{\latupayn\latupvav\latupdal}{None}{kok:ayn_vav_dal}
\kipkokentry{âdî}{\latupayn\latupvav\latupdal}{None}{kok:ayn_vav_dal}
\kipkokentry{âdil}{\latupayn\latupdal\latuplam}{K\rom{1}, Ed.}{kok:ayn_dal_lam}
\kipkokentry{adl}{\latupayn\latupdal\latuplam}{None}{kok:ayn_dal_lam}
\kipkokentry{adliye}{\latupayn\latupdal\latuplam}{None}{kok:ayn_dal_lam}
\kipkokentry{af}{\latupayn\latupfe\latupvav}{None}{kok:ayn_fe_vav}
\kipkokentry{âfak}{\latupalif\latupfe\latupkaf}{None}{kok:alif_fe_kaf}
\kipkokentry{âfet}{\latupalif\latupvav\latupfe}{None}{kok:alif_vav_fe}
\kipkokentry{afif}{\latupayn\latupfe\latupfe}{None}{kok:ayn_fe_fe}
\kipkokentry{âfiyet}{\latupayn\latupfe\latupvav}{None}{kok:ayn_fe_vav}
\kipkokentry{ağnam}{\latupgayn\latupnun\latupmim}{None}{kok:gayn_nun_mim}
\kipkokentry{ağyar}{\latupgayn\latupye\latupre}{None}{kok:gayn_ye_re}
\kipkokentry{ahâli}{\latupalif\latuphe\latuplam}{None}{kok:alif_he_lam}
\kipkokentry{ahbap}{\latupha\latupbe\latupbe$^2$}{None}{kok:ha_be_be2}
\kipkokentry{ahfat}{\latupha\latupfe\latupdal}{None}{kok:ha_fe_dal}
\kipkokentry{ahi}{\latupalif\latupxa\latupvav}{None}{kok:alif_xa_vav}
\kipkokentry{âhir}{\latupalif\latupxa\latupre}{K\rom{1}, Ed.}{kok:alif_xa_re}
\kipkokentry{âhiret}{\latupalif\latupxa\latupre}{None}{kok:alif_xa_re}
\kipkokentry{âhit}{\latupayn\latuphe\latupdal}{None}{kok:ayn_he_dal}
\kipkokentry{ahize}{\latupalif\latupxa\latupzel}{None}{kok:alif_xa_zel}
\kipkokentry{ahkam}{\latupha\latupkef\latupmim}{None}{kok:ha_kef_mim}
\kipkokentry{ahlak}{\latupxa\latuplam\latupkaf}{None}{kok:xa_lam_kaf}
\kipkokentry{ahmak}{\latupha\latupmim\latupkaf}{None}{kok:ha_mim_kaf}
\kipkokentry{ahmer}{\latupha\latupmim\latupre}{None}{kok:ha_mim_re}
\kipkokentry{ahmet}{\latupha\latupmim\latupdal}{None}{kok:ha_mim_dal}
\kipkokentry{ahrar}{\latupha\latupre\latupre$^1$}{None}{kok:ha_re_re1}
\kipkokentry{ahsen}{\latupha\latupsin\latupnun}{None}{kok:ha_sin_nun}
\kipkokentry{ahşap}{\latupxa\latupshin\latupbe}{None}{kok:xa_shin_be}
\kipkokentry{ahval}{\latupha\latupvav\latuplam}{None}{kok:ha_vav_lam}
\kipkokentry{ahz}{\latupalif\latupxa\latupzel}{None}{kok:alif_xa_zel}
\kipkokentry{âidat}{\latupayn\latupvav\latupdal}{None}{kok:ayn_vav_dal}
\kipkokentry{âile}{\latupayn\latupvav\latuplam}{None}{kok:ayn_vav_lam}
\kipkokentry{âit}{\latupayn\latupvav\latupdal}{K\rom{1}, Ed.}{kok:ayn_vav_dal}
\kipkokentry{akâit}{\latupayn\latupkaf\latupdal}{None}{kok:ayn_kaf_dal}
\kipkokentry{akâmet}{\latupayn\latupkaf\latupmim}{None}{kok:ayn_kaf_mim}
\kipkokentry{âkap}{\latupayn\latupkaf\latupbe}{None}{kok:ayn_kaf_be}
\kipkokentry{akar}{\latupayn\latupkaf\latupre}{None}{kok:ayn_kaf_re}
\kipkokentry{akâret}{\latupayn\latupkaf\latupre}{None}{kok:ayn_kaf_re}
\kipkokentry{akdem}{\latupkaf\latupdal\latupmim}{None}{kok:kaf_dal_mim}
\kipkokentry{âkıbet}{\latupayn\latupkaf\latupbe}{None}{kok:ayn_kaf_be}
\kipkokentry{akıl}{\latupayn\latupkaf\latuplam}{None}{kok:ayn_kaf_lam}
\kipkokentry{akîde}{\latupayn\latupkaf\latupdal}{None}{kok:ayn_kaf_dal}
\kipkokentry{âkif}{\latupayn\latupkef\latupfe}{K\rom{1}, Ed.}{kok:ayn_kef_fe}
\kipkokentry{akik}{\latupayn\latupkaf\latupkaf}{None}{kok:ayn_kaf_kaf}
\kipkokentry{âkil}{\latupayn\latupkaf\latuplam}{K\rom{1}, Ed.}{kok:ayn_kaf_lam}
\kipkokentry{akim}{\latupayn\latupkaf\latupmim}{None}{kok:ayn_kaf_mim}
\kipkokentry{akis}{\latupayn\latupkef\latupsin}{None}{kok:ayn_kef_sin}
\kipkokentry{âkit}{\latupayn\latupkaf\latupdal}{K\rom{1}, Ed.}{kok:ayn_kaf_dal}
\kipkokentry{akit}{\latupayn\latupkaf\latupdal}{None}{kok:ayn_kaf_dal}
\kipkokentry{akrabâ}{\latupkaf\latupre\latupbe$^1$}{None}{kok:kaf_re_be1}
\kipkokentry{akran}{\latupkaf\latupre\latupnun}{None}{kok:kaf_re_nun}
\kipkokentry{aksam}{\latupkaf\latupsin\latupmim}{None}{kok:kaf_sin_mim}
\kipkokentry{aksi}{\latupayn\latupkef\latupsin}{None}{kok:ayn_kef_sin}
\kipkokentry{aktar}{\latupayn\latupta\latupre}{None}{kok:ayn_ta_re}
\kipkokentry{akvam}{\latupkaf\latupvav\latupmim}{None}{kok:kaf_vav_mim}
\kipkokentry{âlâ}{\latupayn\latuplam\latupvav}{None}{kok:ayn_lam_vav}
\kipkokentry{alâka}{\latupayn\latuplam\latupkaf}{None}{kok:ayn_lam_kaf}
\kipkokentry{alem}{\latupayn\latuplam\latupmim}{None}{kok:ayn_lam_mim}
\kipkokentry{âlem}{\latupayn\latuplam\latupmim}{None}{kok:ayn_lam_mim}
\kipkokentry{alenî}{\latupayn\latuplam\latupnun}{None}{kok:ayn_lam_nun}
\kipkokentry{âlet}{\latupalif\latupvav\latuplam}{None}{kok:alif_vav_lam}
\kipkokentry{ali}{\latupayn\latuplam\latupvav}{K\rom{1}, Ed.}{kok:ayn_lam_vav}
\kipkokentry{alil}{\latupayn\latuplam\latuplam}{None}{kok:ayn_lam_lam}
\kipkokentry{âlim}{\latupayn\latuplam\latupmim}{K\rom{1}, Ed.}{kok:ayn_lam_mim}
\kipkokentry{Allah}{\latupalif\latuplam\latuphe}{None}{kok:alif_lam_he}
\kipkokentry{allâme}{\latupayn\latuplam\latupmim}{None}{kok:ayn_lam_mim}
\kipkokentry{âmâ}{\latupayn\latupmim\latupye}{None}{kok:ayn_mim_ye}
\kipkokentry{âmak}{\latupalif\latupmim\latupkaf}{None}{kok:alif_mim_kaf}
\kipkokentry{amal}{\latupayn\latupmim\latuplam}{None}{kok:ayn_mim_lam}
\kipkokentry{aman}{\latupalif\latupmim\latupnun}{None}{kok:alif_mim_nun}
\kipkokentry{amel}{\latupayn\latupmim\latuplam}{None}{kok:ayn_mim_lam}
\kipkokentry{amele}{\latupayn\latupmim\latuplam}{None}{kok:ayn_mim_lam}
\kipkokentry{ameliyat}{\latupayn\latupmim\latuplam}{None}{kok:ayn_mim_lam}
\kipkokentry{âmil}{\latupayn\latupmim\latuplam}{K\rom{1}, Ed.}{kok:ayn_mim_lam}
\kipkokentry{âmir}{\latupalif\latupmim\latupre}{K\rom{1}, Ed.}{kok:alif_mim_re}
\kipkokentry{amme}{\latupayn\latupmim\latupmim}{None}{kok:ayn_mim_mim}
\kipkokentry{amut}{\latupayn\latupmim\latupdal}{None}{kok:ayn_mim_dal}
\kipkokentry{an}{\latupalif\latupvav\latupnun}{None}{kok:alif_vav_nun}
\kipkokentry{anane}{\latupayn\latupnun}{None}{kok:ayn_nun}
\kipkokentry{anâsır}{\latupayn\latupnun\latupsad\latupre}{None}{kok:ayn_nun_sad_re}
\kipkokentry{ânî}{\latupalif\latupvav\latupnun}{None}{kok:alif_vav_nun}
\kipkokentry{Ankâ}{\latupayn\latupnun\latupkaf}{None}{kok:ayn_nun_kaf}
\kipkokentry{anut}{\latupayn\latupnun\latupdal}{None}{kok:ayn_nun_dal}
\kipkokentry{aptal}{\latupbe\latupdal\latuplam}{None}{kok:be_dal_lam}
\kipkokentry{ar}{\latupayn\latupvav\latupre}{None}{kok:ayn_vav_re}
\kipkokentry{araf}{\latupayn\latupre\latupfe}{None}{kok:ayn_re_fe}
\kipkokentry{arak}{\latupayn\latupre\latupkaf$^2$}{None}{kok:ayn_re_kaf2}
\kipkokentry{arap}{\latupayn\latupre\latupbe}{None}{kok:ayn_re_be}
\kipkokentry{araz}{\latupayn\latupre\latupdad}{None}{kok:ayn_re_dad}
\kipkokentry{arâzi}{\latupalif\latupre\latupdad}{None}{kok:alif_re_dad}
\kipkokentry{arbede}{\latupayn\latupre\latupbe\latupdal}{None}{kok:ayn_re_be_dal}
\kipkokentry{ardiye}{\latupalif\latupre\latupdad}{None}{kok:alif_re_dad}
\kipkokentry{ârıza}{\latupayn\latupre\latupdad}{None}{kok:ayn_re_dad}
\kipkokentry{ârız}{\latupayn\latupre\latupdad}{K\rom{1}, Ed.}{kok:ayn_re_dad}
\kipkokentry{âri}{\latupayn\latupre\latupye}{K\rom{1}, Ed.}{kok:ayn_re_ye}
\kipkokentry{ârif}{\latupayn\latupre\latupfe}{K\rom{1}, Ed.}{kok:ayn_re_fe}
\kipkokentry{arife}{\latupayn\latupre\latupfe}{None}{kok:ayn_re_fe}
\kipkokentry{âriyet}{\latupayn\latupvav\latupre}{None}{kok:ayn_vav_re}
\kipkokentry{ariyet}{\latupayn\latupre\latupye}{None}{kok:ayn_re_ye}
\kipkokentry{arîza}{\latupayn\latupre\latupdad}{None}{kok:ayn_re_dad}
\kipkokentry{arsa}{\latupayn\latupre\latupsad}{None}{kok:ayn_re_sad}
\kipkokentry{arş}{\latupayn\latupre\latupshin}{None}{kok:ayn_re_shin}
\kipkokentry{arus}{\latupayn\latupre\latupsin}{None}{kok:ayn_re_sin}
\kipkokentry{aruz}{\latupayn\latupre\latupdad}{None}{kok:ayn_re_dad}
\kipkokentry{arz}{\latupayn\latupre\latupdad}{None}{kok:ayn_re_dad}
\kipkokentry{arz}{\latupalif\latupre\latupdad}{None}{kok:alif_re_dad}
\kipkokentry{âsâ}{\latupayn\latupsad\latupvav}{None}{kok:ayn_sad_vav}
\kipkokentry{asabî}{\latupayn\latupsad\latupbe}{None}{kok:ayn_sad_be}
\kipkokentry{asabiye}{\latupayn\latupsad\latupbe}{None}{kok:ayn_sad_be}
\kipkokentry{asâlet}{\latupalif\latupsad\latuplam}{None}{kok:alif_sad_lam}
\kipkokentry{âsap}{\latupayn\latupsad\latupbe}{None}{kok:ayn_sad_be}
\kipkokentry{âsar}{\latupalif\latupthe\latupre}{None}{kok:alif_the_re}
\kipkokentry{ases}{\latupayn\latupsin\latupsin}{None}{kok:ayn_sin_sin}
\kipkokentry{asgarî}{\latupsad\latupgayn\latupre}{None}{kok:sad_gayn_re}
\kipkokentry{ashap}{\latupsad\latupha\latupbe}{None}{kok:sad_ha_be}
\kipkokentry{asıl}{\latupalif\latupsad\latuplam}{None}{kok:alif_sad_lam}
\kipkokentry{âsım}{\latupayn\latupsad\latupmim}{K\rom{1}, Ed.}{kok:ayn_sad_mim}
\kipkokentry{asır}{\latupayn\latupsad\latupre$^1$}{None}{kok:ayn_sad_re1}
\kipkokentry{âsî}{\latupayn\latupsad\latupye}{K\rom{1}, Ed.}{kok:ayn_sad_ye}
\kipkokentry{asil}{\latupalif\latupsad\latuplam}{None}{kok:alif_sad_lam}
\kipkokentry{aslâ}{\latupalif\latupsad\latuplam}{None}{kok:alif_sad_lam}
\kipkokentry{asliye}{\latupalif\latupsad\latuplam}{None}{kok:alif_sad_lam}
\kipkokentry{âşar}{\latupayn\latupshin\latupre}{None}{kok:ayn_shin_re}
\kipkokentry{âşık}{\latupayn\latupshin\latupkaf}{K\rom{1}, Ed.}{kok:ayn_shin_kaf}
\kipkokentry{aşîret}{\latupayn\latupshin\latupre}{None}{kok:ayn_shin_re}
\kipkokentry{aşk}{\latupayn\latupshin\latupkaf}{None}{kok:ayn_shin_kaf}
\kipkokentry{aşûre}{\latupayn\latupshin\latupre}{None}{kok:ayn_shin_re}
\kipkokentry{atâlet}{\latupayn\latupta\latuplam}{None}{kok:ayn_ta_lam}
\kipkokentry{atebe}{\latupayn\latupte\latupbe}{None}{kok:ayn_te_be}
\kipkokentry{atıf}{\latupayn\latupta\latupfe}{None}{kok:ayn_ta_fe}
\kipkokentry{âtıfet}{\latupayn\latupta\latupfe}{None}{kok:ayn_ta_fe}
\kipkokentry{âtıl}{\latupayn\latupta\latuplam}{K\rom{1}, Ed.}{kok:ayn_ta_lam}
\kipkokentry{âti}{\latupalif\latupte\latupye}{K\rom{1}, Ed.}{kok:alif_te_ye}
\kipkokentry{atik}{\latupayn\latupte\latupkaf}{None}{kok:ayn_te_kaf}
\kipkokentry{atiye}{\latupayn\latupta\latupvav}{None}{kok:ayn_ta_vav}
\kipkokentry{atlas}{\latupta\latuplam\latupsin}{None}{kok:ta_lam_sin}
\kipkokentry{attar}{\latupayn\latupta\latupre}{None}{kok:ayn_ta_re}
\kipkokentry{âvam}{\latupayn\latupmim\latupmim}{None}{kok:ayn_mim_mim}
\kipkokentry{avârız}{\latupayn\latupre\latupdad}{None}{kok:ayn_re_dad}
\kipkokentry{avene}{\latupayn\latupvav\latupnun}{None}{kok:ayn_vav_nun}
\kipkokentry{avret}{\latupayn\latupvav\latupre}{None}{kok:ayn_vav_re}
\kipkokentry{ayal}{\latupayn\latupvav\latuplam}{None}{kok:ayn_vav_lam}
\kipkokentry{ayan}{\latupayn\latupye\latupnun}{None}{kok:ayn_ye_nun}
\kipkokentry{âyan}{\latupayn\latupye\latupnun}{None}{kok:ayn_ye_nun}
\kipkokentry{ayar}{\latupayn\latupye\latupre$^1$}{None}{kok:ayn_ye_re1}
\kipkokentry{ayıp}{\latupayn\latupye\latupbe}{None}{kok:ayn_ye_be}
\kipkokentry{ayn}{\latupayn\latupye\latupnun}{None}{kok:ayn_ye_nun}
\kipkokentry{aynı}{\latupayn\latupye\latupnun}{None}{kok:ayn_ye_nun}
\kipkokentry{aynî}{\latupayn\latupye\latupnun}{None}{kok:ayn_ye_nun}
\kipkokentry{ayvaz}{\latupayn\latupvav\latupdad}{None}{kok:ayn_vav_dad}
\kipkokentry{ayyar}{\latupayn\latupye\latupre$^2$}{None}{kok:ayn_ye_re2}
\kipkokentry{ayyaş}{\latupayn\latupye\latupshin}{None}{kok:ayn_ye_shin}
\kipkokentry{ayyuk}{\latupayn\latupye\latupkaf}{None}{kok:ayn_ye_kaf}
\kipkokentry{âzâ}{\latupayn\latupdad\latupvav}{None}{kok:ayn_dad_vav}
\kipkokentry{azâmet}{\latupayn\latupza\latupmim}{None}{kok:ayn_za_mim}
\kipkokentry{azamî}{\latupayn\latupza\latupmim}{None}{kok:ayn_za_mim}
\kipkokentry{âzap}{\latupayn\latupzel\latupbe}{None}{kok:ayn_zel_be}
\kipkokentry{azil}{\latupayn\latupze\latuplam}{None}{kok:ayn_ze_lam}
\kipkokentry{azim}{\latupayn\latupze\latupmim}{None}{kok:ayn_ze_mim}
\kipkokentry{azim}{\latupayn\latupza\latupmim}{None}{kok:ayn_za_mim}
\kipkokentry{azîmet}{\latupayn\latupze\latupmim}{None}{kok:ayn_ze_mim}
\kipkokentry{aziz}{\latupayn\latupze\latupze}{None}{kok:ayn_ze_ze}
\end{multicols}
\dictchapter{B}
\begin{multicols}{2}
\kipkokentry{badana}{\latupbe\latupta\latupnun}{None}{kok:be_ta_nun}
\kipkokentry{bâdire}{\latupbe\latupdal\latupre}{None}{kok:be_dal_re}
\kipkokentry{bâği}{\latupbe\latupgayn\latupye}{K\rom{1}, Ed.}{kok:be_gayn_ye}
\kipkokentry{bahir}{\latupbe\latupha\latupre$^1$}{None}{kok:be_ha_re1}
\kipkokentry{bahis}{\latupbe\latupha\latupthe}{None}{kok:be_ha_the}
\kipkokentry{bahriye}{\latupbe\latupha\latupre$^1$}{None}{kok:be_ha_re1}
\kipkokentry{bâis}{\latupbe\latupayn\latupthe}{K\rom{1}, Ed.}{kok:be_ayn_the}
\kipkokentry{bakaya}{\latupbe\latupkaf\latupye}{None}{kok:be_kaf_ye}
\kipkokentry{bâki}{\latupbe\latupkaf\latupye}{K\rom{1}, Ed.}{kok:be_kaf_ye}
\kipkokentry{bâkir}{\latupbe\latupkef\latupre}{K\rom{1}, Ed.}{kok:be_kef_re}
\kipkokentry{bâkire}{\latupbe\latupkef\latupre}{None}{kok:be_kef_re}
\kipkokentry{bakiye}{\latupbe\latupkaf\latupye}{None}{kok:be_kaf_ye}
\kipkokentry{bakkal}{\latupbe\latupkaf\latuplam}{None}{kok:be_kaf_lam}
\kipkokentry{bakla}{\latupbe\latupkaf\latuplam}{None}{kok:be_kaf_lam}
\kipkokentry{bakliyat}{\latupbe\latupkaf\latuplam}{None}{kok:be_kaf_lam}
\kipkokentry{bâliğ}{\latupbe\latuplam\latupgayn}{K\rom{1}, Ed.}{kok:be_lam_gayn}
\kipkokentry{bâni}{\latupbe\latupnun\latupye}{K\rom{1}, Ed.}{kok:be_nun_ye}
\kipkokentry{bap}{\latupbe\latupvav\latupbe}{None}{kok:be_vav_be}
\kipkokentry{barîka}{\latupbe\latupre\latupkaf}{None}{kok:be_re_kaf}
\kipkokentry{bâriz}{\latupbe\latupre\latupze}{K\rom{1}, Ed.}{kok:be_re_ze}
\kipkokentry{basîret}{\latupbe\latupsad\latupre}{None}{kok:be_sad_re}
\kipkokentry{basit}{\latupbe\latupsin\latupta}{None}{kok:be_sin_ta}
\kipkokentry{bâtıl}{\latupbe\latupta\latuplam}{K\rom{1}, Ed.}{kok:be_ta_lam}
\kipkokentry{bâtın}{\latupbe\latupta\latupnun}{K\rom{1}, Ed.}{kok:be_ta_nun}
\kipkokentry{batın}{\latupbe\latupta\latupnun}{None}{kok:be_ta_nun}
\kipkokentry{bâtınî}{\latupbe\latupta\latupnun}{None}{kok:be_ta_nun}
\kipkokentry{battal}{\latupbe\latupta\latuplam}{None}{kok:be_ta_lam}
\kipkokentry{battâniye}{\latupbe\latupta\latupnun}{None}{kok:be_ta_nun}
\kipkokentry{bayat}{\latupbe\latupye\latupte}{K\rom{1}, Ed.}{kok:be_ye_te}
\kipkokentry{bayır}{\latupbe\latupvav\latupre}{K\rom{1}, Ed.}{kok:be_vav_re}
\kipkokentry{bâyi}{\latupbe\latupye\latupayn}{K\rom{1}, Ed.}{kok:be_ye_ayn}
\kipkokentry{bâzan}{\latupbe\latupayn\latupdad}{None}{kok:be_ayn_dad}
\kipkokentry{bâzı}{\latupbe\latupayn\latupdad}{None}{kok:be_ayn_dad}
\kipkokentry{bedâyi}{\latupbe\latupdal\latupayn}{None}{kok:be_dal_ayn}
\kipkokentry{bedel}{\latupbe\latupdal\latuplam}{None}{kok:be_dal_lam}
\kipkokentry{beden}{\latupbe\latupdal\latupnun}{None}{kok:be_dal_nun}
\kipkokentry{bedevî}{\latupbe\latupdal\latupvav}{None}{kok:be_dal_vav}
\kipkokentry{bedîhî}{\latupbe\latupdal\latuphe}{None}{kok:be_dal_he}
\kipkokentry{bedii}{\latupbe\latupdal\latupayn}{None}{kok:be_dal_ayn}
\kipkokentry{bedir}{\latupbe\latupdal\latupre}{None}{kok:be_dal_re}
\kipkokentry{behîmiyet}{\latupbe\latuphe\latupmim}{None}{kok:be_he_mim}
\kipkokentry{beis}{\latupbe\latupalif\latupsin}{None}{kok:be_alif_sin}
\kipkokentry{bekâ}{\latupbe\latupkaf\latupye}{None}{kok:be_kaf_ye}
\kipkokentry{bekar}{\latupbe\latupkef\latupre}{None}{kok:be_kef_re}
\kipkokentry{bekâret}{\latupbe\latupkef\latupre}{None}{kok:be_kef_re}
\kipkokentry{belâ}{\latupbe\latuplam\latupvav}{None}{kok:be_lam_vav}
\kipkokentry{belâgat}{\latupbe\latuplam\latupgayn}{None}{kok:be_lam_gayn}
\kipkokentry{belde}{\latupbe\latuplam\latupdal}{None}{kok:be_lam_dal}
\kipkokentry{belediye}{\latupbe\latuplam\latupdal}{None}{kok:be_lam_dal}
\kipkokentry{beliğ}{\latupbe\latuplam\latupgayn}{None}{kok:be_lam_gayn}
\kipkokentry{beraat}{\latupbe\latupre\latupalif}{None}{kok:be_re_alif}
\kipkokentry{berd}{\latupbe\latupre\latupdal}{None}{kok:be_re_dal}
\kipkokentry{bereket}{\latupbe\latupre\latupkef}{None}{kok:be_re_kef}
\kipkokentry{berî}{\latupbe\latupre\latupalif}{None}{kok:be_re_alif}
\kipkokentry{berk}{\latupbe\latupre\latupkaf}{None}{kok:be_re_kaf}
\kipkokentry{berrak}{\latupbe\latupre\latupkaf}{None}{kok:be_re_kaf}
\kipkokentry{beşâret}{\latupbe\latupshin\latupre}{None}{kok:be_shin_re}
\kipkokentry{beşer}{\latupbe\latupshin\latupre}{None}{kok:be_shin_re}
\kipkokentry{beşuş}{\latupbe\latupshin\latupshin}{None}{kok:be_shin_shin}
\kipkokentry{bevl}{\latupbe\latupvav\latuplam}{None}{kok:be_vav_lam}
\kipkokentry{bevliye}{\latupbe\latupvav\latuplam}{None}{kok:be_vav_lam}
\kipkokentry{bevvap}{\latupbe\latupvav\latupbe}{None}{kok:be_vav_be}
\kipkokentry{bey}{\latupbe\latupye\latupayn}{None}{kok:be_ye_ayn}
\kipkokentry{beyan}{\latupbe\latupye\latupnun}{None}{kok:be_ye_nun}
\kipkokentry{beyaz}{\latupbe\latupye\latupdad}{None}{kok:be_ye_dad}
\kipkokentry{beyit}{\latupbe\latupye\latupte}{None}{kok:be_ye_te}
\kipkokentry{beyn}{\latupbe\latupye\latupnun}{None}{kok:be_ye_nun}
\kipkokentry{beyzâ}{\latupbe\latupye\latupdad}{None}{kok:be_ye_dad}
\kipkokentry{beyzî}{\latupbe\latupye\latupdad}{None}{kok:be_ye_dad}
\kipkokentry{bez}{\latupbe\latupze\latupze}{None}{kok:be_ze_ze}
\kipkokentry{bezir}{\latupbe\latupze\latupre}{None}{kok:be_ze_re}
\kipkokentry{bezzaz}{\latupbe\latupze\latupze}{None}{kok:be_ze_ze}
\kipkokentry{bızır}{\latupbe\latupza\latupre}{None}{kok:be_za_re}
\kipkokentry{biat}{\latupbe\latupye\latupayn}{None}{kok:be_ye_ayn}
\kipkokentry{bidat}{\latupbe\latupdal\latupayn}{None}{kok:be_dal_ayn}
\kipkokentry{bidâyet}{\latupbe\latupdal\latupalif}{None}{kok:be_dal_alif}
\kipkokentry{bikir}{\latupbe\latupkef\latupre}{None}{kok:be_kef_re}
\kipkokentry{binâ}{\latupbe\latupnun\latupye}{None}{kok:be_nun_ye}
\kipkokentry{bin}{\latupbe\latupnun}{None}{kok:be_nun}
\kipkokentry{budala}{\latupbe\latupdal\latuplam}{None}{kok:be_dal_lam}
\kipkokentry{buğz}{\latupbe\latupgayn\latupdad}{None}{kok:be_gayn_dad}
\kipkokentry{buhar}{\latupbe\latupxa\latupre}{None}{kok:be_xa_re}
\kipkokentry{buhran}{\latupbe\latupha\latupre$^2$}{None}{kok:be_ha_re2}
\kipkokentry{buhur}{\latupbe\latupxa\latupre}{None}{kok:be_xa_re}
\kipkokentry{butlan}{\latupbe\latupta\latuplam}{None}{kok:be_ta_lam}
\kipkokentry{buut}{\latupbe\latupayn\latupdal}{None}{kok:be_ayn_dal}
\kipkokentry{bühtan}{\latupbe\latuphe\latupte}{None}{kok:be_he_te}
\kipkokentry{büluğ}{\latupbe\latuplam\latupgayn}{None}{kok:be_lam_gayn}
\kipkokentry{bünyan}{\latupbe\latupnun\latupye}{None}{kok:be_nun_ye}
\kipkokentry{bünye}{\latupbe\latupnun\latupye}{None}{kok:be_nun_ye}
\kipkokentry{bürhan}{\latupbe\latupre\latuphe}{None}{kok:be_re_he}
\kipkokentry{bürûdet}{\latupbe\latupre\latupdal}{None}{kok:be_re_dal}
\kipkokentry{büşrâ}{\latupbe\latupshin\latupre}{None}{kok:be_shin_re}
\end{multicols}
\dictchapter{C}
\begin{multicols}{2}
\kipkokentry{caba}{\latupcim\latupbe\latupye}{None}{kok:cim_be_ye}
\kipkokentry{cabbar}{\latupcim\latupbe\latupre}{None}{kok:cim_be_re}
\kipkokentry{cadde}{\latupcim\latupdal\latupdal$^1$}{None}{kok:cim_dal_dal1}
\kipkokentry{câhil}{\latupcim\latuphe\latuplam}{K\rom{1}, Ed.}{kok:cim_he_lam}
\kipkokentry{câiz}{\latupcim\latupvav\latupze}{K\rom{1}, Ed.}{kok:cim_vav_ze}
\kipkokentry{câlip}{\latupcim\latuplam\latupbe}{K\rom{1}, Ed.}{kok:cim_lam_be}
\kipkokentry{câmi}{\latupcim\latupmim\latupayn}{K\rom{1}, Ed.}{kok:cim_mim_ayn}
\kipkokentry{câmiâ}{\latupcim\latupmim\latupayn}{None}{kok:cim_mim_ayn}
\kipkokentry{cânî}{\latupcim\latupnun\latupye}{K\rom{1}, Ed.}{kok:cim_nun_ye}
\kipkokentry{cânip}{\latupcim\latupnun\latupbe}{K\rom{1}, Ed.}{kok:cim_nun_be}
\kipkokentry{cârî}{\latupcim\latupre\latupye}{K\rom{1}, Ed.}{kok:cim_re_ye}
\kipkokentry{câriye}{\latupcim\latupre\latupye}{None}{kok:cim_re_ye}
\kipkokentry{câsus}{\latupcim\latupsin\latupsin}{None}{kok:cim_sin_sin}
\kipkokentry{câvit}{\latupcim\latupvav\latupdal}{K\rom{1}, Ed.}{kok:cim_vav_dal}
\kipkokentry{câzibe}{\latupcim\latupzel\latupbe}{None}{kok:cim_zel_be}
\kipkokentry{câzip}{\latupcim\latupzel\latupbe}{K\rom{1}, Ed.}{kok:cim_zel_be}
\kipkokentry{cebel}{\latupcim\latupbe\latuplam$^1$}{None}{kok:cim_be_lam1}
\kipkokentry{cebin}{\latupcim\latupbe\latupnun}{None}{kok:cim_be_nun}
\kipkokentry{cebir}{\latupcim\latupbe\latupre}{None}{kok:cim_be_re}
\kipkokentry{cedel}{\latupcim\latupdal\latuplam}{None}{kok:cim_dal_lam}
\kipkokentry{cedit}{\latupcim\latupdal\latupdal$^1$}{None}{kok:cim_dal_dal1}
\kipkokentry{cefâ}{\latupcim\latupfe\latupvav}{None}{kok:cim_fe_vav}
\kipkokentry{cehâlet}{\latupcim\latuphe\latuplam}{None}{kok:cim_he_lam}
\kipkokentry{cehd}{\latupcim\latuphe\latupdal}{None}{kok:cim_he_dal}
\kipkokentry{cehl}{\latupcim\latuphe\latuplam}{None}{kok:cim_he_lam}
\kipkokentry{celâdet}{\latupcim\latuplam\latupdal}{None}{kok:cim_lam_dal}
\kipkokentry{celal}{\latupcim\latuplam\latuplam}{None}{kok:cim_lam_lam}
\kipkokentry{celep}{\latupcim\latuplam\latupbe}{None}{kok:cim_lam_be}
\kipkokentry{cellat}{\latupcim\latuplam\latupdal}{None}{kok:cim_lam_dal}
\kipkokentry{celp}{\latupcim\latuplam\latupbe}{None}{kok:cim_lam_be}
\kipkokentry{celse}{\latupcim\latuplam\latupsin}{None}{kok:cim_lam_sin}
\kipkokentry{cem}{\latupcim\latupmim\latupayn}{None}{kok:cim_mim_ayn}
\kipkokentry{cemaat}{\latupcim\latupmim\latupayn}{None}{kok:cim_mim_ayn}
\kipkokentry{cemal}{\latupcim\latupmim\latuplam}{None}{kok:cim_mim_lam}
\kipkokentry{cemîle}{\latupcim\latupmim\latuplam}{None}{kok:cim_mim_lam}
\kipkokentry{cemiyet}{\latupcim\latupmim\latupayn}{None}{kok:cim_mim_ayn}
\kipkokentry{cemre}{\latupcim\latupmim\latupre}{None}{kok:cim_mim_re}
\kipkokentry{cenâbet}{\latupcim\latupnun\latupbe}{None}{kok:cim_nun_be}
\kipkokentry{cenah}{\latupcim\latupnun\latupha}{None}{kok:cim_nun_ha}
\kipkokentry{cenap}{\latupcim\latupnun\latupbe}{None}{kok:cim_nun_be}
\kipkokentry{cenin}{\latupcim\latupnun\latupnun}{None}{kok:cim_nun_nun}
\kipkokentry{cennet}{\latupcim\latupnun\latupnun}{None}{kok:cim_nun_nun}
\kipkokentry{cenup}{\latupcim\latupnun\latupbe}{None}{kok:cim_nun_be}
\kipkokentry{cep}{\latupcim\latupye\latupbe}{None}{kok:cim_ye_be}
\kipkokentry{cephe}{\latupcim\latupbe\latuphe}{None}{kok:cim_be_he}
\kipkokentry{cer}{\latupcim\latupre\latupre}{None}{kok:cim_re_re}
\kipkokentry{cerahat}{\latupcim\latupre\latupha}{None}{kok:cim_re_ha}
\kipkokentry{cerbeze}{\latupcim\latupre\latupbe\latupze}{None}{kok:cim_re_be_ze}
\kipkokentry{cereme}{\latupcim\latupre\latupmim}{None}{kok:cim_re_mim}
\kipkokentry{cereyan}{\latupcim\latupre\latupye}{None}{kok:cim_re_ye}
\kipkokentry{cerh}{\latupcim\latupre\latupha}{None}{kok:cim_re_ha}
\kipkokentry{cerîde}{\latupcim\latupre\latupdal}{None}{kok:cim_re_dal}
\kipkokentry{ceriha}{\latupcim\latupre\latupha}{None}{kok:cim_re_ha}
\kipkokentry{cerrah}{\latupcim\latupre\latupha}{None}{kok:cim_re_ha}
\kipkokentry{cesâmet}{\latupcim\latupsin\latupmim}{None}{kok:cim_sin_mim}
\kipkokentry{cesâret}{\latupcim\latupsin\latupre}{None}{kok:cim_sin_re}
\kipkokentry{ceset}{\latupcim\latupsin\latupdal}{None}{kok:cim_sin_dal}
\kipkokentry{cesim}{\latupcim\latupsin\latupmim}{None}{kok:cim_sin_mim}
\kipkokentry{cesur}{\latupcim\latupsin\latupre}{None}{kok:cim_sin_re}
\kipkokentry{cet}{\latupcim\latupdal\latupdal$^2$}{None}{kok:cim_dal_dal2}
\kipkokentry{cevâhir}{\latupcim\latupvav\latuphe\latupre}{None}{kok:cim_vav_he_re}
\kipkokentry{cevap}{\latupcim\latupvav\latupbe}{None}{kok:cim_vav_be}
\kipkokentry{cevat}{\latupcim\latupvav\latupdal}{None}{kok:cim_vav_dal}
\kipkokentry{cevaz}{\latupcim\latupvav\latupze}{None}{kok:cim_vav_ze}
\kipkokentry{cevdet}{\latupcim\latupvav\latupdal}{None}{kok:cim_vav_dal}
\kipkokentry{cevelan}{\latupcim\latupvav\latuplam}{None}{kok:cim_vav_lam}
\kipkokentry{cevher}{\latupcim\latupvav\latuphe\latupre}{None}{kok:cim_vav_he_re}
\kipkokentry{cevir}{\latupcim\latupvav\latupre$^2$}{None}{kok:cim_vav_re2}
\kipkokentry{cevval}{\latupcim\latupvav\latuplam}{None}{kok:cim_vav_lam}
\kipkokentry{cezâ}{\latupcim\latupze\latupye}{None}{kok:cim_ze_ye}
\kipkokentry{cezb}{\latupcim\latupzel\latupbe}{None}{kok:cim_zel_be}
\kipkokentry{cezbe}{\latupcim\latupzel\latupbe}{None}{kok:cim_zel_be}
\kipkokentry{cezerye}{\latupcim\latupzel\latupre}{None}{kok:cim_zel_re}
\kipkokentry{cezir}{\latupcim\latupze\latupre}{None}{kok:cim_ze_re}
\kipkokentry{cezîre}{\latupcim\latupze\latupre}{None}{kok:cim_ze_re}
\kipkokentry{cezrî}{\latupcim\latupzel\latupre}{None}{kok:cim_zel_re}
\kipkokentry{cezve}{\latupcim\latupzel\latupvav}{None}{kok:cim_zel_vav}
\kipkokentry{cibâyet}{\latupcim\latupbe\latupye}{None}{kok:cim_be_ye}
\kipkokentry{cibilliyet}{\latupcim\latupbe\latuplam$^2$}{None}{kok:cim_be_lam2}
\kipkokentry{cidal}{\latupcim\latupdal\latuplam}{None}{kok:cim_dal_lam}
\kipkokentry{ciddi}{\latupcim\latupdal\latupdal$^1$}{None}{kok:cim_dal_dal1}
\kipkokentry{cihat}{\latupcim\latuphe\latupdal}{None}{kok:cim_he_dal}
\kipkokentry{cihaz}{\latupcim\latuphe\latupze}{None}{kok:cim_he_ze}
\kipkokentry{cihet}{\latupvav\latupcim\latuphe}{None}{kok:vav_cim_he}
\kipkokentry{cilâ}{\latupcim\latuplam\latupvav}{None}{kok:cim_lam_vav}
\kipkokentry{cilbap}{\latupcim\latuplam\latupbe}{None}{kok:cim_lam_be}
\kipkokentry{cilt}{\latupcim\latuplam\latupdal}{None}{kok:cim_lam_dal}
\kipkokentry{cilve}{\latupcim\latuplam\latupvav}{None}{kok:cim_lam_vav}
\kipkokentry{cîmâ}{\latupcim\latupmim\latupayn}{None}{kok:cim_mim_ayn}
\kipkokentry{cimcime}{\latupcim\latupmim}{None}{kok:cim_mim}
\kipkokentry{cin}{\latupcim\latupnun\latupnun}{None}{kok:cim_nun_nun}
\kipkokentry{cinâî}{\latupcim\latupnun\latupye}{None}{kok:cim_nun_ye}
\kipkokentry{cinas}{\latupcim\latupnun\latupsin}{None}{kok:cim_nun_sin}
\kipkokentry{cinâyet}{\latupcim\latupnun\latupye}{None}{kok:cim_nun_ye}
\kipkokentry{cinnet}{\latupcim\latupnun\latupnun}{None}{kok:cim_nun_nun}
\kipkokentry{cins}{\latupcim\latupnun\latupsin}{None}{kok:cim_nun_sin}
\kipkokentry{cinsiyet}{\latupcim\latupnun\latupsin}{None}{kok:cim_nun_sin}
\kipkokentry{cirit}{\latupcim\latupre\latupdal}{None}{kok:cim_re_dal}
\kipkokentry{cirm}{\latupcim\latupre\latupmim}{None}{kok:cim_re_mim}
\kipkokentry{cisim}{\latupcim\latupsin\latupmim}{None}{kok:cim_sin_mim}
\kipkokentry{civar}{\latupcim\latupvav\latupre}{None}{kok:cim_vav_re}
\kipkokentry{cizye}{\latupcim\latupze\latupye}{None}{kok:cim_ze_ye}
\kipkokentry{cumâ}{\latupcim\latupmim\latupayn}{None}{kok:cim_mim_ayn}
\kipkokentry{cumhur}{\latupcim\latupmim\latuphe\latupre}{None}{kok:cim_mim_he_re}
\kipkokentry{cumhûriyet}{\latupcim\latupmim\latuphe\latupre}{None}{kok:cim_mim_he_re}
\kipkokentry{cübbe}{\latupcim\latupbe\latupbe}{None}{kok:cim_be_be}
\kipkokentry{cühelâ}{\latupcim\latuphe\latuplam}{None}{kok:cim_he_lam}
\kipkokentry{cülus}{\latupcim\latuplam\latupsin}{None}{kok:cim_lam_sin}
\kipkokentry{cümle}{\latupcim\latupmim\latuplam}{None}{kok:cim_mim_lam}
\kipkokentry{cünüp}{\latupcim\latupnun\latupbe}{None}{kok:cim_nun_be}
\kipkokentry{cüret}{\latupcim\latupre\latupalif}{None}{kok:cim_re_alif}
\kipkokentry{cüruf}{\latupcim\latupre\latupfe}{None}{kok:cim_re_fe}
\kipkokentry{cürüm}{\latupcim\latupre\latupmim}{None}{kok:cim_re_mim}
\kipkokentry{cüsse}{\latupcim\latupthe\latupthe}{None}{kok:cim_the_the}
\kipkokentry{cüz}{\latupcim\latupze\latupalif}{None}{kok:cim_ze_alif}
\kipkokentry{cüzam}{\latupcim\latupzel\latupmim}{None}{kok:cim_zel_mim}
\kipkokentry{cüzî}{\latupcim\latupze\latupalif}{None}{kok:cim_ze_alif}
\end{multicols}
\dictchapter{Ç}
\begin{multicols}{2}
\kipkokentry{çam}{\latupshin\latupmim\latupayn}{None}{kok:shin_mim_ayn}
\kipkokentry{çeyiz}{\latupcim\latuphe\latupze}{None}{kok:cim_he_ze}
\end{multicols}
\dictchapter{D}
\begin{multicols}{2}
\kipkokentry{dabbe}{\latupdal\latupbe\latupbe}{None}{kok:dal_be_be}
\kipkokentry{dağdağa}{\latupdal\latupgayn}{None}{kok:dal_gayn}
\kipkokentry{dâhî}{\latupdal\latuphe\latupye}{K\rom{1}, Ed.}{kok:dal_he_ye}
\kipkokentry{dâhil}{\latupdal\latupxa\latuplam}{K\rom{1}, Ed.}{kok:dal_xa_lam}
\kipkokentry{dahil}{\latupdal\latupxa\latuplam}{None}{kok:dal_xa_lam}
\kipkokentry{dâimâ}{\latupdal\latupvav\latupmim}{None}{kok:dal_vav_mim}
\kipkokentry{dâim}{\latupdal\latupvav\latupmim}{K\rom{1}, Ed.}{kok:dal_vav_mim}
\kipkokentry{dâir}{\latupdal\latupvav\latupre}{K\rom{1}, Ed.}{kok:dal_vav_re}
\kipkokentry{dâire}{\latupdal\latupvav\latupre}{None}{kok:dal_vav_re}
\kipkokentry{dakâyık}{\latupdal\latupkaf\latupkaf}{None}{kok:dal_kaf_kaf}
\kipkokentry{dakik}{\latupdal\latupkaf\latupkaf}{None}{kok:dal_kaf_kaf}
\kipkokentry{dakîka}{\latupdal\latupkaf\latupkaf}{None}{kok:dal_kaf_kaf}
\kipkokentry{dalâlet}{\latupdad\latuplam\latuplam}{None}{kok:dad_lam_lam}
\kipkokentry{dâr}{\latupdal\latupvav\latupre}{None}{kok:dal_vav_re}
\kipkokentry{dara}{\latupta\latupre\latupha}{None}{kok:ta_re_ha}
\kipkokentry{darbe}{\latupdad\latupre\latupbe}{None}{kok:dad_re_be}
\kipkokentry{dâreyn}{\latupdal\latupvav\latupre}{None}{kok:dal_vav_re}
\kipkokentry{darp}{\latupdad\latupre\latupbe}{None}{kok:dad_re_be}
\kipkokentry{dâvâ}{\latupdal\latupayn\latupvav}{None}{kok:dal_ayn_vav}
\kipkokentry{dâvet}{\latupdal\latupayn\latupvav}{None}{kok:dal_ayn_vav}
\kipkokentry{dâvetiye}{\latupdal\latupayn\latupvav}{None}{kok:dal_ayn_vav}
\kipkokentry{davul}{\latupta\latupbe\latuplam}{None}{kok:ta_be_lam}
\kipkokentry{debbağ}{\latupdal\latupbe\latupgayn}{None}{kok:dal_be_gayn}
\kipkokentry{debdebe}{\latupdal\latupbe}{None}{kok:dal_be}
\kipkokentry{deccal}{\latupdal\latupcim\latuplam}{None}{kok:dal_cim_lam}
\kipkokentry{defâ}{\latupdal\latupfe\latupayn}{None}{kok:dal_fe_ayn}
\kipkokentry{def}{\latupdal\latupfe\latupayn}{None}{kok:dal_fe_ayn}
\kipkokentry{defaat}{\latupdal\latupfe\latupayn}{None}{kok:dal_fe_ayn}
\kipkokentry{defâten}{\latupdal\latupfe\latupayn}{None}{kok:dal_fe_ayn}
\kipkokentry{defin}{\latupdal\latupfe\latupnun}{None}{kok:dal_fe_nun}
\kipkokentry{defîne}{\latupdal\latupfe\latupnun}{None}{kok:dal_fe_nun}
\kipkokentry{dehâ}{\latupdal\latuphe\latupye}{None}{kok:dal_he_ye}
\kipkokentry{dehâlet}{\latupdal\latupxa\latuplam}{None}{kok:dal_xa_lam}
\kipkokentry{dehâvet}{\latupdal\latuphe\latupye}{None}{kok:dal_he_ye}
\kipkokentry{dehrî}{\latupdal\latuphe\latupre}{None}{kok:dal_he_re}
\kipkokentry{dehşet}{\latupdal\latuphe\latupshin}{None}{kok:dal_he_shin}
\kipkokentry{delâlet}{\latupdal\latuplam\latuplam}{None}{kok:dal_lam_lam}
\kipkokentry{delil}{\latupdal\latuplam\latuplam}{None}{kok:dal_lam_lam}
\kipkokentry{delk}{\latupdal\latuplam\latupkaf}{None}{kok:dal_lam_kaf}
\kipkokentry{dem}{\latupdal\latupmim}{None}{kok:dal_mim}
\kipkokentry{denâet}{\latupdal\latupnun\latupvav}{None}{kok:dal_nun_vav}
\kipkokentry{denî}{\latupdal\latupnun\latupvav}{None}{kok:dal_nun_vav}
\kipkokentry{derç}{\latupdal\latupre\latupcim$^2$}{None}{kok:dal_re_cim2}
\kipkokentry{derece}{\latupdal\latupre\latupcim$^1$}{None}{kok:dal_re_cim1}
\kipkokentry{dereke}{\latupdal\latupre\latupkef}{None}{kok:dal_re_kef}
\kipkokentry{ders}{\latupdal\latupre\latupsin}{None}{kok:dal_re_sin}
\kipkokentry{desîse}{\latupdal\latupsin\latupsin}{None}{kok:dal_sin_sin}
\kipkokentry{devâ}{\latupdal\latupvav\latupye}{None}{kok:dal_vav_ye}
\kipkokentry{devam}{\latupdal\latupvav\latupmim}{None}{kok:dal_vav_mim}
\kipkokentry{devlet}{\latupdal\latupvav\latuplam}{None}{kok:dal_vav_lam}
\kipkokentry{devrân}{\latupdal\latupvav\latupre}{None}{kok:dal_vav_re}
\kipkokentry{devre}{\latupdal\latupvav\latupre}{None}{kok:dal_vav_re}
\kipkokentry{devriye}{\latupdal\latupvav\latupre}{None}{kok:dal_vav_re}
\kipkokentry{deyr}{\latupdal\latupye\latupre}{None}{kok:dal_ye_re}
\kipkokentry{deyyus}{\latupdal\latupye\latupthe}{None}{kok:dal_ye_the}
\kipkokentry{dikkat}{\latupdal\latupkaf\latupkaf}{None}{kok:dal_kaf_kaf}
\kipkokentry{din}{\latupdal\latupye\latupnun$^2$}{None}{kok:dal_ye_nun2}
\kipkokentry{dirâyet}{\latupdal\latupre\latupye}{None}{kok:dal_re_ye}
\kipkokentry{diyânet}{\latupdal\latupye\latupnun$^2$}{None}{kok:dal_ye_nun2}
\kipkokentry{diyâr}{\latupdal\latupvav\latupre}{None}{kok:dal_vav_re}
\kipkokentry{duâ}{\latupdal\latupayn\latupvav}{None}{kok:dal_ayn_vav}
\kipkokentry{duhul}{\latupdal\latupxa\latuplam}{None}{kok:dal_xa_lam}
\kipkokentry{dumur}{\latupdad\latupmim\latupre}{None}{kok:dad_mim_re}
\kipkokentry{dun}{\latupdal\latupnun\latupvav}{None}{kok:dal_nun_vav}
\kipkokentry{dübür}{\latupdal\latupbe\latupre}{None}{kok:dal_be_re}
\kipkokentry{dükkan}{\latupdal\latupkef\latupnun}{None}{kok:dal_kef_nun}
\kipkokentry{dünyâ}{\latupdal\latupnun\latupvav}{None}{kok:dal_nun_vav}
\kipkokentry{düvel}{\latupdal\latupvav\latuplam}{None}{kok:dal_vav_lam}
\kipkokentry{düyun}{\latupdal\latupye\latupnun$^1$}{None}{kok:dal_ye_nun1}
\end{multicols}
\dictchapter{E}
\begin{multicols}{2}
\kipkokentry{ebat}{\latupbe\latupayn\latupdal}{None}{kok:be_ayn_dal}
\kipkokentry{ebedî}{\latupalif\latupbe\latupdal}{None}{kok:alif_be_dal}
\kipkokentry{ebediyet}{\latupalif\latupbe\latupdal}{None}{kok:alif_be_dal}
\kipkokentry{ebet}{\latupalif\latupbe\latupdal}{None}{kok:alif_be_dal}
\kipkokentry{ebeveyn}{\latupalif\latupbe\latupvav}{None}{kok:alif_be_vav}
\kipkokentry{ebleh}{\latupbe\latuplam\latuphe}{None}{kok:be_lam_he}
\kipkokentry{ebrar}{\latupbe\latupre\latupre}{None}{kok:be_re_re}
\kipkokentry{ebu}{\latupalif\latupbe\latupvav}{None}{kok:alif_be_vav}
\kipkokentry{ecdat}{\latupcim\latupdal\latupdal$^2$}{None}{kok:cim_dal_dal2}
\kipkokentry{ecel}{\latupalif\latupcim\latuplam}{None}{kok:alif_cim_lam}
\kipkokentry{ecir}{\latupalif\latupcim\latupre}{None}{kok:alif_cim_re}
\kipkokentry{ecnebî}{\latupcim\latupnun\latupbe}{None}{kok:cim_nun_be}
\kipkokentry{ecved}{\latupcim\latupvav\latupdal}{None}{kok:cim_vav_dal}
\kipkokentry{eczâ}{\latupcim\latupze\latupalif}{None}{kok:cim_ze_alif}
\kipkokentry{edâ}{\latupalif\latupdal\latupye}{None}{kok:alif_dal_ye}
\kipkokentry{edat}{\latupalif\latupdal\latupvav}{None}{kok:alif_dal_vav}
\kipkokentry{edebiyat}{\latupalif\latupdal\latupbe}{None}{kok:alif_dal_be}
\kipkokentry{edep}{\latupalif\latupdal\latupbe}{None}{kok:alif_dal_be}
\kipkokentry{edevat}{\latupalif\latupdal\latupvav}{None}{kok:alif_dal_vav}
\kipkokentry{edip}{\latupalif\latupdal\latupbe}{None}{kok:alif_dal_be}
\kipkokentry{edna}{\latupdal\latupnun\latupvav}{None}{kok:dal_nun_vav}
\kipkokentry{edvar}{\latupdal\latupvav\latupre}{None}{kok:dal_vav_re}
\kipkokentry{edyan}{\latupdal\latupye\latupnun$^2$}{None}{kok:dal_ye_nun2}
\kipkokentry{efâl}{\latupfe\latupayn\latuplam}{None}{kok:fe_ayn_lam}
\kipkokentry{efdal}{\latupfe\latupdad\latuplam}{None}{kok:fe_dad_lam}
\kipkokentry{efkar}{\latupfe\latupkef\latupre}{None}{kok:fe_kef_re}
\kipkokentry{eflak}{\latupfe\latuplam\latupkef}{None}{kok:fe_lam_kef}
\kipkokentry{efrat}{\latupfe\latupre\latupdal}{None}{kok:fe_re_dal}
\kipkokentry{eğreti}{\latupayn\latupvav\latupre}{None}{kok:ayn_vav_re}
\kipkokentry{ehemmiyet}{\latuphe\latupmim\latupmim}{None}{kok:he_mim_mim}
\kipkokentry{ehibbâ}{\latupha\latupbe\latupbe$^2$}{None}{kok:ha_be_be2}
\kipkokentry{ehil}{\latupalif\latuphe\latuplam}{None}{kok:alif_he_lam}
\kipkokentry{ehlî}{\latupalif\latuphe\latuplam}{None}{kok:alif_he_lam}
\kipkokentry{ehliyet}{\latupalif\latuphe\latuplam}{None}{kok:alif_he_lam}
\kipkokentry{ehram}{\latuphe\latupre\latupmim}{None}{kok:he_re_mim}
\kipkokentry{ehven}{\latuphe\latupvav\latupnun}{None}{kok:he_vav_nun}
\kipkokentry{ekâbir}{\latupkef\latupbe\latupre}{None}{kok:kef_be_re}
\kipkokentry{ekalliyet}{\latupkaf\latuplam\latuplam}{None}{kok:kaf_lam_lam}
\kipkokentry{ekber}{\latupkef\latupbe\latupre}{None}{kok:kef_be_re}
\kipkokentry{ekmel}{\latupkef\latupmim\latuplam}{None}{kok:kef_mim_lam}
\kipkokentry{ekrat}{\latupkef\latupre\latupdal}{None}{kok:kef_re_dal}
\kipkokentry{ekrem}{\latupkef\latupre\latupmim}{None}{kok:kef_re_mim}
\kipkokentry{ekser}{\latupkef\latupthe\latupre}{None}{kok:kef_the_re}
\kipkokentry{elbette}{\latupbe\latupte\latupte}{None}{kok:be_te_te}
\kipkokentry{elbise}{\latuplam\latupbe\latupsin}{None}{kok:lam_be_sin}
\kipkokentry{elem}{\latupalif\latuplam\latupmim}{None}{kok:alif_lam_mim}
\kipkokentry{elhan}{\latuplam\latupha\latupnun}{None}{kok:lam_ha_nun}
\kipkokentry{elif}{\latupalif\latuplam\latupfe}{None}{kok:alif_lam_fe}
\kipkokentry{elim}{\latupalif\latuplam\latupmim}{None}{kok:alif_lam_mim}
\kipkokentry{elsine}{\latuplam\latupsin\latupnun}{None}{kok:lam_sin_nun}
\kipkokentry{elvan}{\latuplam\latupvav\latupnun}{None}{kok:lam_vav_nun}
\kipkokentry{elviye}{\latuplam\latupvav\latupye}{None}{kok:lam_vav_ye}
\kipkokentry{elyaf}{\latuplam\latupye\latupfe}{None}{kok:lam_ye_fe}
\kipkokentry{elzem}{\latuplam\latupze\latupmim}{None}{kok:lam_ze_mim}
\kipkokentry{emânet}{\latupalif\latupmim\latupnun}{None}{kok:alif_mim_nun}
\kipkokentry{emâre}{\latupalif\latupmim\latupre}{None}{kok:alif_mim_re}
\kipkokentry{emel}{\latupalif\latupmim\latuplam}{None}{kok:alif_mim_lam}
\kipkokentry{emin}{\latupalif\latupmim\latupnun}{None}{kok:alif_mim_nun}
\kipkokentry{emir}{\latupalif\latupmim\latupre}{None}{kok:alif_mim_re}
\kipkokentry{emir}{\latupalif\latupmim\latupre}{None}{kok:alif_mim_re}
\kipkokentry{emlak}{\latupmim\latuplam\latupkef}{None}{kok:mim_lam_kef}
\kipkokentry{emniyet}{\latupalif\latupmim\latupnun}{None}{kok:alif_mim_nun}
\kipkokentry{emsal}{\latupmim\latupthe\latuplam}{None}{kok:mim_the_lam}
\kipkokentry{emtia}{\latupmim\latupte\latupayn}{None}{kok:mim_te_ayn}
\kipkokentry{emval}{\latupmim\latupvav\latuplam}{None}{kok:mim_vav_lam}
\kipkokentry{enam}{\latupnun\latupayn\latupmim}{None}{kok:nun_ayn_mim}
\kipkokentry{enbiyâ}{\latupnun\latupbe\latupvav}{None}{kok:nun_be_vav}
\kipkokentry{ender}{\latupnun\latupdal\latupre}{None}{kok:nun_dal_re}
\kipkokentry{enfes}{\latupnun\latupfe\latupsin}{None}{kok:nun_fe_sin}
\kipkokentry{enfiye}{\latupalif\latupnun\latupfe}{None}{kok:alif_nun_fe}
\kipkokentry{enkaz}{\latupnun\latupkaf\latupdad}{None}{kok:nun_kaf_dad}
\kipkokentry{ensar}{\latupnun\latupsad\latupre$^1$}{None}{kok:nun_sad_re1}
\kipkokentry{envâ}{\latupnun\latupvav\latupayn}{None}{kok:nun_vav_ayn}
\kipkokentry{erbap}{\latupre\latupbe\latupbe}{None}{kok:re_be_be}
\kipkokentry{erkan}{\latupre\latupkef\latupnun}{None}{kok:re_kef_nun}
\kipkokentry{ervah}{\latupre\latupvav\latupha}{None}{kok:re_vav_ha}
\kipkokentry{erzak}{\latupre\latupze\latupkaf}{None}{kok:re_ze_kaf}
\kipkokentry{esâmî}{\latupsin\latupmim\latupye}{None}{kok:sin_mim_ye}
\kipkokentry{esâret}{\latupalif\latupsin\latupre}{None}{kok:alif_sin_re}
\kipkokentry{esas}{\latupalif\latupsin\latupsin}{None}{kok:alif_sin_sin}
\kipkokentry{esâtir}{\latupsin\latupta\latupre}{None}{kok:sin_ta_re}
\kipkokentry{esbak}{\latupsin\latupbe\latupkaf}{None}{kok:sin_be_kaf}
\kipkokentry{esbap}{\latupsin\latupbe\latupbe}{None}{kok:sin_be_be}
\kipkokentry{esef}{\latupalif\latupsin\latupfe}{None}{kok:alif_sin_fe}
\kipkokentry{eser}{\latupalif\latupthe\latupre}{None}{kok:alif_the_re}
\kipkokentry{esham}{\latupsin\latuphe\latupmim}{None}{kok:sin_he_mim}
\kipkokentry{esir}{\latupalif\latupsin\latupre}{None}{kok:alif_sin_re}
\kipkokentry{eslaf}{\latupsin\latuplam\latupfe}{None}{kok:sin_lam_fe}
\kipkokentry{esmâ}{\latupsin\latupmim\latupye}{None}{kok:sin_mim_ye}
\kipkokentry{esmer}{\latupsin\latupmim\latupre$^1$}{None}{kok:sin_mim_re1}
\kipkokentry{esnâ}{\latupthe\latupnun\latupye}{None}{kok:the_nun_ye}
\kipkokentry{esnaf}{\latupsad\latupnun\latupfe}{None}{kok:sad_nun_fe}
\kipkokentry{esrar}{\latupsin\latupre\latupre}{None}{kok:sin_re_re}
\kipkokentry{esre}{\latupkef\latupsin\latupre}{None}{kok:kef_sin_re}
\kipkokentry{esvap}{\latupthe\latupvav\latupbe}{None}{kok:the_vav_be}
\kipkokentry{esvet}{\latupsin\latupvav\latupdal}{None}{kok:sin_vav_dal}
\kipkokentry{eşhas}{\latupshin\latupxa\latupsad}{None}{kok:shin_xa_sad}
\kipkokentry{eşkal}{\latupshin\latupkef\latuplam$^2$}{None}{kok:shin_kef_lam2}
\kipkokentry{eşkıyâ}{\latupshin\latupkaf\latupvav$^1$}{None}{kok:shin_kaf_vav1}
\kipkokentry{eşraf}{\latupshin\latupre\latupfe}{None}{kok:shin_re_fe}
\kipkokentry{eşref}{\latupshin\latupre\latupfe}{None}{kok:shin_re_fe}
\kipkokentry{eşyâ}{\latupshin\latupye\latupalif}{None}{kok:shin_ye_alif}
\kipkokentry{etfal}{\latupta\latupfe\latuplam}{None}{kok:ta_fe_lam}
\kipkokentry{etıbbâ}{\latupta\latupbe\latupbe}{None}{kok:ta_be_be}
\kipkokentry{etraf}{\latupta\latupre\latupfe}{None}{kok:ta_re_fe}
\kipkokentry{etvar}{\latupta\latupvav\latupre}{None}{kok:ta_vav_re}
\kipkokentry{eûzu}{\latupayn\latupvav\latupzel}{None}{kok:ayn_vav_zel}
\kipkokentry{evham}{\latupvav\latuphe\latupmim}{None}{kok:vav_he_mim}
\kipkokentry{eviç}{\latupalif\latupvav\latupcim}{None}{kok:alif_vav_cim}
\kipkokentry{evkaf}{\latupvav\latupkaf\latupfe}{None}{kok:vav_kaf_fe}
\kipkokentry{evlâ}{\latupvav\latuplam\latupye}{None}{kok:vav_lam_ye}
\kipkokentry{evlat}{\latupvav\latuplam\latupdal}{None}{kok:vav_lam_dal}
\kipkokentry{evliyâ}{\latupvav\latuplam\latupye}{None}{kok:vav_lam_ye}
\kipkokentry{evrak}{\latupvav\latupre\latupkaf}{None}{kok:vav_re_kaf}
\kipkokentry{evsaf}{\latupvav\latupsad\latupfe}{None}{kok:vav_sad_fe}
\kipkokentry{evvel}{\latupalif\latupvav\latuplam}{None}{kok:alif_vav_lam}
\kipkokentry{eytam}{\latupye\latupte\latupmim}{None}{kok:ye_te_mim}
\kipkokentry{eyyam}{\latupye\latupvav\latupmim}{None}{kok:ye_vav_mim}
\kipkokentry{ezâ}{\latupalif\latupzel\latupye}{None}{kok:alif_zel_ye}
\kipkokentry{ezan}{\latupalif\latupzel\latupnun}{None}{kok:alif_zel_nun}
\kipkokentry{ezel}{\latupalif\latupze\latuplam}{None}{kok:alif_ze_lam}
\kipkokentry{ezelî}{\latupalif\latupze\latuplam}{None}{kok:alif_ze_lam}
\kipkokentry{eziyet}{\latupalif\latupzel\latupye}{None}{kok:alif_zel_ye}
\end{multicols}
\dictchapter{F}
\begin{multicols}{2}
\kipkokentry{faal}{\latupfe\latupayn\latuplam}{None}{kok:fe_ayn_lam}
\kipkokentry{faaliyet}{\latupfe\latupayn\latuplam}{None}{kok:fe_ayn_lam}
\kipkokentry{fâcia}{\latupfe\latupcim\latupayn}{None}{kok:fe_cim_ayn}
\kipkokentry{fâhiş}{\latupfe\latupha\latupshin}{K\rom{1}, Ed.}{kok:fe_ha_shin}
\kipkokentry{fâhişe}{\latupfe\latupha\latupshin}{None}{kok:fe_ha_shin}
\kipkokentry{fahrî}{\latupfe\latupxa\latupre}{None}{kok:fe_xa_re}
\kipkokentry{fâik}{\latupfe\latupvav\latupkaf}{K\rom{1}, Ed.}{kok:fe_vav_kaf}
\kipkokentry{fâil}{\latupfe\latupayn\latuplam}{K\rom{1}, Ed.}{kok:fe_ayn_lam}
\kipkokentry{fâiz}{\latupfe\latupte\latupdad}{K\rom{1}, Ed.}{kok:fe_te_dad}
\kipkokentry{fak}{\latupfe\latupxa\latupxa}{None}{kok:fe_xa_xa}
\kipkokentry{fakih}{\latupfe\latupkaf\latuphe}{None}{kok:fe_kaf_he}
\kipkokentry{fakir}{\latupfe\latupkaf\latupre}{None}{kok:fe_kaf_re}
\kipkokentry{fakr}{\latupfe\latupkaf\latupre}{None}{kok:fe_kaf_re}
\kipkokentry{fal}{\latupfe\latupalif\latuplam}{None}{kok:fe_alif_lam}
\kipkokentry{fânî}{\latupfe\latupnun\latupye}{K\rom{1}, Ed.}{kok:fe_nun_ye}
\kipkokentry{faraş}{\latupfe\latupre\latupshin}{None}{kok:fe_re_shin}
\kipkokentry{faraza}{\latupfe\latupre\latupdad}{None}{kok:fe_re_dad}
\kipkokentry{farazî}{\latupfe\latupre\latupdad}{None}{kok:fe_re_dad}
\kipkokentry{fâriğ}{\latupfe\latupre\latupgayn}{K\rom{1}, Ed.}{kok:fe_re_gayn}
\kipkokentry{fârika}{\latupfe\latupre\latupkaf}{None}{kok:fe_re_kaf}
\kipkokentry{farîza}{\latupfe\latupre\latupdad}{None}{kok:fe_re_dad}
\kipkokentry{fark}{\latupfe\latupre\latupkaf}{None}{kok:fe_re_kaf}
\kipkokentry{farz}{\latupfe\latupre\latupdad}{None}{kok:fe_re_dad}
\kipkokentry{fâsık}{\latupfe\latupsin\latupkaf}{K\rom{1}, Ed.}{kok:fe_sin_kaf}
\kipkokentry{fâsıla}{\latupfe\latupsad\latuplam}{None}{kok:fe_sad_lam}
\kipkokentry{fasıl}{\latupfe\latupsad\latuplam}{None}{kok:fe_sad_lam}
\kipkokentry{fasih}{\latupfe\latupsin\latupha}{None}{kok:fe_sin_ha}
\kipkokentry{fasîle}{\latupfe\latupsad\latuplam}{None}{kok:fe_sad_lam}
\kipkokentry{fâsit}{\latupfe\latupsin\latupdal}{K\rom{1}, Ed.}{kok:fe_sin_dal}
\kipkokentry{faş}{\latupfe\latupshin\latupvav}{None}{kok:fe_shin_vav}
\kipkokentry{fâtiha}{\latupfe\latupte\latupha}{None}{kok:fe_te_ha}
\kipkokentry{fâtih}{\latupfe\latupte\latupha}{K\rom{1}, Ed.}{kok:fe_te_ha}
\kipkokentry{fayda}{\latupfe\latupye\latupdal}{None}{kok:fe_ye_dal}
\kipkokentry{fâzıl}{\latupfe\latupdad\latuplam}{K\rom{1}, Ed.}{kok:fe_dad_lam}
\kipkokentry{fazîlet}{\latupfe\latupdad\latuplam}{None}{kok:fe_dad_lam}
\kipkokentry{fazla}{\latupfe\latupdad\latuplam}{None}{kok:fe_dad_lam}
\kipkokentry{fecaat}{\latupfe\latupcim\latupayn}{None}{kok:fe_cim_ayn}
\kipkokentry{fecî}{\latupfe\latupcim\latupayn}{None}{kok:fe_cim_ayn}
\kipkokentry{fecir}{\latupfe\latupcim\latupre}{None}{kok:fe_cim_re}
\kipkokentry{fedâ}{\latupfe\latupdal\latupye}{None}{kok:fe_dal_ye}
\kipkokentry{fehim}{\latupfe\latuphe\latupmim}{None}{kok:fe_he_mim}
\kipkokentry{fehvâ}{\latupfe\latupha\latupvav}{None}{kok:fe_ha_vav}
\kipkokentry{fek}{\latupfe\latupkef\latupkef}{None}{kok:fe_kef_kef}
\kipkokentry{felah}{\latupfe\latuplam\latupha}{None}{kok:fe_lam_ha}
\kipkokentry{felç}{\latupfe\latuplam\latupcim}{None}{kok:fe_lam_cim}
\kipkokentry{felek}{\latupfe\latuplam\latupkef}{None}{kok:fe_lam_kef}
\kipkokentry{fellah}{\latupfe\latuplam\latupha}{None}{kok:fe_lam_ha}
\kipkokentry{felsefe}{\latupfe\latuplam\latupsin\latupfe}{None}{kok:fe_lam_sin_fe}
\kipkokentry{fen}{\latupfe\latupnun\latupnun}{None}{kok:fe_nun_nun}
\kipkokentry{fenâ}{\latupfe\latupnun\latupye}{None}{kok:fe_nun_ye}
\kipkokentry{ferâgat}{\latupfe\latupre\latupgayn}{None}{kok:fe_re_gayn}
\kipkokentry{ferağ}{\latupfe\latupre\latupgayn}{None}{kok:fe_re_gayn}
\kipkokentry{ferah}{\latupfe\latupre\latupha}{None}{kok:fe_re_ha}
\kipkokentry{ferâset}{\latupfe\latupre\latupsin}{None}{kok:fe_re_sin}
\kipkokentry{ferç}{\latupfe\latupre\latupcim}{None}{kok:fe_re_cim}
\kipkokentry{feri}{\latupfe\latupre\latupayn}{None}{kok:fe_re_ayn}
\kipkokentry{ferik}{\latupfe\latupre\latupkaf}{None}{kok:fe_re_kaf}
\kipkokentry{fert}{\latupfe\latupre\latupdal}{None}{kok:fe_re_dal}
\kipkokentry{fesat}{\latupfe\latupsin\latupdal}{None}{kok:fe_sin_dal}
\kipkokentry{fesih}{\latupfe\latupsin\latupxa}{None}{kok:fe_sin_xa}
\kipkokentry{fetha}{\latupfe\latupte\latupha}{None}{kok:fe_te_ha}
\kipkokentry{fetih}{\latupfe\latupte\latupha}{None}{kok:fe_te_ha}
\kipkokentry{fetret}{\latupfe\latupte\latupre}{None}{kok:fe_te_re}
\kipkokentry{fettah}{\latupfe\latupte\latupha}{None}{kok:fe_te_ha}
\kipkokentry{fettan}{\latupfe\latupte\latupnun}{None}{kok:fe_te_nun}
\kipkokentry{fetvâ}{\latupfe\latupte\latupvav}{None}{kok:fe_te_vav}
\kipkokentry{fevç}{\latupfe\latupvav\latupcim}{None}{kok:fe_vav_cim}
\kipkokentry{feveran}{\latupfe\latupvav\latupre}{None}{kok:fe_vav_re}
\kipkokentry{fevk}{\latupfe\latupvav\latupkaf}{None}{kok:fe_vav_kaf}
\kipkokentry{fevrî}{\latupfe\latupvav\latupre}{None}{kok:fe_vav_re}
\kipkokentry{fevt}{\latupfe\latupvav\latupte}{None}{kok:fe_vav_te}
\kipkokentry{feyiz}{\latupfe\latupte\latupdad}{None}{kok:fe_te_dad}
\kipkokentry{fezâ}{\latupfe\latupdad\latupvav}{None}{kok:fe_dad_vav}
\kipkokentry{fıkdan}{\latupfe\latupkaf\latupdal}{None}{kok:fe_kaf_dal}
\kipkokentry{fıkıh}{\latupfe\latupkaf\latuphe}{None}{kok:fe_kaf_he}
\kipkokentry{fıkra}{\latupfe\latupkaf\latupre}{None}{kok:fe_kaf_re}
\kipkokentry{fırın}{\latupfe\latupre\latupnun}{None}{kok:fe_re_nun}
\kipkokentry{fırka}{\latupfe\latupre\latupkaf}{None}{kok:fe_re_kaf}
\kipkokentry{fırsat}{\latupfe\latupre\latupsad}{None}{kok:fe_re_sad}
\kipkokentry{fısk}{\latupfe\latupsin\latupkaf}{None}{kok:fe_sin_kaf}
\kipkokentry{fıtık}{\latupfe\latupte\latupkaf}{None}{kok:fe_te_kaf}
\kipkokentry{fıtrat}{\latupfe\latupta\latupre}{None}{kok:fe_ta_re}
\kipkokentry{fidye}{\latupfe\latupdal\latupye}{None}{kok:fe_dal_ye}
\kipkokentry{fiil}{\latupfe\latupayn\latuplam}{None}{kok:fe_ayn_lam}
\kipkokentry{fikir}{\latupfe\latupkef\latupre}{None}{kok:fe_kef_re}
\kipkokentry{fil}{\latupfe\latupye\latuplam}{None}{kok:fe_ye_lam}
\kipkokentry{firak}{\latupfe\latupre\latupkaf}{None}{kok:fe_re_kaf}
\kipkokentry{firar}{\latupfe\latupre\latupre}{None}{kok:fe_re_re}
\kipkokentry{firik}{\latupfe\latupre\latupkef}{None}{kok:fe_re_kef}
\kipkokentry{firkat}{\latupfe\latupre\latupkaf}{None}{kok:fe_re_kaf}
\kipkokentry{fistan}{\latupfe\latupsin\latupte}{None}{kok:fe_sin_te}
\kipkokentry{fitil}{\latupfe\latupte\latuplam}{None}{kok:fe_te_lam}
\kipkokentry{fitne}{\latupfe\latupte\latupnun}{None}{kok:fe_te_nun}
\kipkokentry{fitre}{\latupfe\latupta\latupre}{None}{kok:fe_ta_re}
\kipkokentry{fodul}{\latupfe\latupdad\latuplam}{None}{kok:fe_dad_lam}
\kipkokentry{fuhuş}{\latupfe\latupha\latupshin}{None}{kok:fe_ha_shin}
\kipkokentry{fukarâ}{\latupfe\latupkaf\latupre}{None}{kok:fe_kaf_re}
\kipkokentry{ful}{\latupfe\latupvav\latuplam}{None}{kok:fe_vav_lam}
\kipkokentry{furkan}{\latupfe\latupre\latupkaf}{None}{kok:fe_re_kaf}
\kipkokentry{fuzûlî}{\latupfe\latupdad\latuplam}{None}{kok:fe_dad_lam}
\kipkokentry{fücceten}{\latupfe\latupcim\latupalif}{None}{kok:fe_cim_alif}
\kipkokentry{fücur}{\latupfe\latupcim\latupre}{None}{kok:fe_cim_re}
\kipkokentry{fünun}{\latupfe\latupnun\latupnun}{None}{kok:fe_nun_nun}
\kipkokentry{fürû}{\latupfe\latupre\latupayn}{None}{kok:fe_re_ayn}
\kipkokentry{fütuhat}{\latupfe\latupte\latupha}{None}{kok:fe_te_ha}
\kipkokentry{fütur}{\latupfe\latupte\latupre}{None}{kok:fe_te_re}
\kipkokentry{fütüvvet}{\latupfe\latupte\latupvav}{None}{kok:fe_te_vav}
\end{multicols}
\dictchapter{G}
\begin{multicols}{2}
\kipkokentry{gabâvet}{\latupgayn\latupbe\latupvav}{None}{kok:gayn_be_vav}
\kipkokentry{gabî}{\latupgayn\latupbe\latupvav}{None}{kok:gayn_be_vav}
\kipkokentry{gabin}{\latupgayn\latupbe\latupnun}{None}{kok:gayn_be_nun}
\kipkokentry{gaddar}{\latupgayn\latupdal\latupre}{None}{kok:gayn_dal_re}
\kipkokentry{gadir}{\latupgayn\latupdal\latupre}{None}{kok:gayn_dal_re}
\kipkokentry{gaffar}{\latupgayn\latupfe\latupre}{None}{kok:gayn_fe_re}
\kipkokentry{gâfil}{\latupgayn\latupfe\latuplam}{K\rom{1}, Ed.}{kok:gayn_fe_lam}
\kipkokentry{gaflet}{\latupgayn\latupfe\latuplam}{None}{kok:gayn_fe_lam}
\kipkokentry{gâile}{\latupgayn\latupvav\latuplam}{None}{kok:gayn_vav_lam}
\kipkokentry{gâip}{\latupgayn\latupye\latupbe}{K\rom{1}, Ed.}{kok:gayn_ye_be}
\kipkokentry{gaita}{\latupgayn\latupvav\latupta}{None}{kok:gayn_vav_ta}
\kipkokentry{galat}{\latupgayn\latuplam\latupta}{None}{kok:gayn_lam_ta}
\kipkokentry{galebe}{\latupgayn\latuplam\latupbe}{None}{kok:gayn_lam_be}
\kipkokentry{galeyan}{\latupgayn\latuplam\latupye}{None}{kok:gayn_lam_ye}
\kipkokentry{gâlibâ}{\latupgayn\latuplam\latupbe}{None}{kok:gayn_lam_be}
\kipkokentry{gâlibiyet}{\latupgayn\latuplam\latupbe}{None}{kok:gayn_lam_be}
\kipkokentry{gâlip}{\latupgayn\latuplam\latupbe}{K\rom{1}, Ed.}{kok:gayn_lam_be}
\kipkokentry{galiz}{\latupgayn\latuplam\latupza}{None}{kok:gayn_lam_za}
\kipkokentry{gam}{\latupgayn\latupmim\latupmim}{None}{kok:gayn_mim_mim}
\kipkokentry{gammaz}{\latupgayn\latupmim\latupze}{None}{kok:gayn_mim_ze}
\kipkokentry{gamze}{\latupgayn\latupmim\latupze}{None}{kok:gayn_mim_ze}
\kipkokentry{ganî}{\latupgayn\latupnun\latupye$^1$}{None}{kok:gayn_nun_ye1}
\kipkokentry{ganîmet}{\latupgayn\latupnun\latupmim}{None}{kok:gayn_nun_mim}
\kipkokentry{garâib}{\latupgayn\latupre\latupbe}{None}{kok:gayn_re_be}
\kipkokentry{garet}{\latupgayn\latupvav\latupre}{None}{kok:gayn_vav_re}
\kipkokentry{garez}{\latupgayn\latupre\latupdad}{None}{kok:gayn_re_dad}
\kipkokentry{gargara}{\latupgayn\latupre}{None}{kok:gayn_re}
\kipkokentry{garibân}{\latupgayn\latupre\latupbe}{None}{kok:gayn_re_be}
\kipkokentry{garip}{\latupgayn\latupre\latupbe}{None}{kok:gayn_re_be}
\kipkokentry{gark}{\latupgayn\latupre\latupkaf}{None}{kok:gayn_re_kaf}
\kipkokentry{garp}{\latupgayn\latupre\latupbe}{None}{kok:gayn_re_be}
\kipkokentry{gâsıp}{\latupgayn\latupsad\latupbe}{K\rom{1}, Ed.}{kok:gayn_sad_be}
\kipkokentry{gasil}{\latupgayn\latupsin\latuplam}{None}{kok:gayn_sin_lam}
\kipkokentry{gasp}{\latupgayn\latupsad\latupbe}{None}{kok:gayn_sad_be}
\kipkokentry{gassal}{\latupgayn\latupsin\latuplam}{None}{kok:gayn_sin_lam}
\kipkokentry{gavat}{\latupkaf\latupvav\latupdal}{None}{kok:kaf_vav_dal}
\kipkokentry{gavs}{\latupgayn\latupvav\latupsad}{None}{kok:gayn_vav_sad}
\kipkokentry{gaybûbet}{\latupgayn\latupye\latupbe}{None}{kok:gayn_ye_be}
\kipkokentry{gâye}{\latupgayn\latupye}{None}{kok:gayn_ye}
\kipkokentry{gâyet}{\latupgayn\latupye}{None}{kok:gayn_ye}
\kipkokentry{gayret}{\latupgayn\latupye\latupre}{None}{kok:gayn_ye_re}
\kipkokentry{gayri}{\latupgayn\latupye\latupre}{None}{kok:gayn_ye_re}
\kipkokentry{gayur}{\latupgayn\latupye\latupre}{None}{kok:gayn_ye_re}
\kipkokentry{gayya}{\latupgayn\latupvav\latupye}{None}{kok:gayn_vav_ye}
\kipkokentry{gayz}{\latupgayn\latupye\latupza}{None}{kok:gayn_ye_za}
\kipkokentry{gazâ}{\latupgayn\latupze\latupvav}{None}{kok:gayn_ze_vav}
\kipkokentry{gazap}{\latupgayn\latupdad\latupbe}{None}{kok:gayn_dad_be}
\kipkokentry{gazel}{\latupgayn\latupze\latuplam}{None}{kok:gayn_ze_lam}
\kipkokentry{gâzi}{\latupgayn\latupze\latupvav}{K\rom{1}, Ed.}{kok:gayn_ze_vav}
\kipkokentry{gebeş}{\latupkef\latupbe\latupshin}{None}{kok:kef_be_shin}
\kipkokentry{gıdâ}{\latupgayn\latupzel\latupvav}{None}{kok:gayn_zel_vav}
\kipkokentry{gılman}{\latupgayn\latuplam\latupmim}{None}{kok:gayn_lam_mim}
\kipkokentry{gınâ}{\latupgayn\latupnun\latupye$^1$}{None}{kok:gayn_nun_ye1}
\kipkokentry{gıpta}{\latupgayn\latupbe\latupta}{None}{kok:gayn_be_ta}
\kipkokentry{gıyab}{\latupgayn\latupye\latupbe}{None}{kok:gayn_ye_be}
\kipkokentry{gıybet}{\latupgayn\latupye\latupbe}{None}{kok:gayn_ye_be}
\kipkokentry{gubar}{\latupgayn\latupbe\latupre}{None}{kok:gayn_be_re}
\kipkokentry{gudde}{\latupgayn\latupdal\latupdal}{None}{kok:gayn_dal_dal}
\kipkokentry{gudûbet}{\latupgayn\latupdad\latupbe}{None}{kok:gayn_dad_be}
\kipkokentry{gufran}{\latupgayn\latupfe\latupre}{None}{kok:gayn_fe_re}
\kipkokentry{gûl}{\latupgayn\latupvav\latuplam}{None}{kok:gayn_vav_lam}
\kipkokentry{gulam}{\latupgayn\latuplam\latupmim}{None}{kok:gayn_lam_mim}
\kipkokentry{gulgule}{\latupgayn\latuplam}{None}{kok:gayn_lam}
\kipkokentry{gurbet}{\latupgayn\latupre\latupbe}{None}{kok:gayn_re_be}
\kipkokentry{gurebâ}{\latupgayn\latupre\latupbe}{None}{kok:gayn_re_be}
\kipkokentry{gurup}{\latupgayn\latupre\latupbe}{None}{kok:gayn_re_be}
\kipkokentry{gurur}{\latupgayn\latupre\latupre}{None}{kok:gayn_re_re}
\kipkokentry{gusül}{\latupgayn\latupsin\latuplam}{None}{kok:gayn_sin_lam}
\end{multicols}
\dictchapter{H}
\begin{multicols}{2}
\kipkokentry{habâset}{\latupxa\latupbe\latupthe}{None}{kok:xa_be_the}
\kipkokentry{habbe}{\latupha\latupbe\latupbe$^1$}{None}{kok:ha_be_be1}
\kipkokentry{haber}{\latupxa\latupbe\latupre}{None}{kok:xa_be_re}
\kipkokentry{habip}{\latupha\latupbe\latupbe$^2$}{None}{kok:ha_be_be2}
\kipkokentry{habis}{\latupxa\latupbe\latupthe}{None}{kok:xa_be_the}
\kipkokentry{hac}{\latupha\latupcim\latupcim}{None}{kok:ha_cim_cim}
\kipkokentry{hacamat}{\latupha\latupcim\latupmim}{None}{kok:ha_cim_mim}
\kipkokentry{hacer}{\latupha\latupcim\latupre$^2$}{None}{kok:ha_cim_re2}
\kipkokentry{hâcet}{\latupha\latupvav\latupcim}{None}{kok:ha_vav_cim}
\kipkokentry{hacim}{\latupha\latupcim\latupmim}{None}{kok:ha_cim_mim}
\kipkokentry{hâcip}{\latupha\latupcim\latupbe}{K\rom{1}, Ed.}{kok:ha_cim_be}
\kipkokentry{hacir}{\latupha\latupcim\latupre$^1$}{None}{kok:ha_cim_re1}
\kipkokentry{haciz}{\latupha\latupcim\latupze}{None}{kok:ha_cim_ze}
\kipkokentry{had}{\latupha\latupdal\latupdal}{None}{kok:ha_dal_dal}
\kipkokentry{hadde}{\latupha\latupdal\latupdal}{None}{kok:ha_dal_dal}
\kipkokentry{hademe}{\latupxa\latupdal\latupmim}{None}{kok:xa_dal_mim}
\kipkokentry{hadım}{\latupxa\latupdal\latupmim}{K\rom{1}, Ed.}{kok:xa_dal_mim}
\kipkokentry{hâdim}{\latupxa\latupdal\latupmim}{K\rom{1}, Ed.}{kok:xa_dal_mim}
\kipkokentry{hadis}{\latupha\latupdal\latupthe}{None}{kok:ha_dal_the}
\kipkokentry{hâdise}{\latupha\latupdal\latupthe}{None}{kok:ha_dal_the}
\kipkokentry{hafakan}{\latupxa\latupfe\latupkaf}{None}{kok:xa_fe_kaf}
\kipkokentry{hâfıza}{\latupha\latupfe\latupza}{None}{kok:ha_fe_za}
\kipkokentry{hâfız}{\latupha\latupfe\latupza}{K\rom{1}, Ed.}{kok:ha_fe_za}
\kipkokentry{hafî}{\latupxa\latupfe\latupye}{None}{kok:xa_fe_ye}
\kipkokentry{hafif}{\latupxa\latupfe\latupfe}{None}{kok:xa_fe_fe}
\kipkokentry{hafit}{\latupha\latupfe\latupdal}{None}{kok:ha_fe_dal}
\kipkokentry{hafiye}{\latupxa\latupfe\latupye}{None}{kok:xa_fe_ye}
\kipkokentry{hafriyat}{\latupha\latupfe\latupre}{None}{kok:ha_fe_re}
\kipkokentry{hâile}{\latuphe\latupvav\latuplam}{None}{kok:he_vav_lam}
\kipkokentry{hâin}{\latupxa\latupvav\latupnun}{K\rom{1}, Ed.}{kok:xa_vav_nun}
\kipkokentry{hâiz}{\latupha\latupvav\latupze}{K\rom{1}, Ed.}{kok:ha_vav_ze}
\kipkokentry{hak}{\latupha\latupkaf\latupkaf}{None}{kok:ha_kaf_kaf}
\kipkokentry{hak}{\latupha\latupkef\latupkef}{None}{kok:ha_kef_kef}
\kipkokentry{hakâret}{\latupha\latupkaf\latupre}{None}{kok:ha_kaf_re}
\kipkokentry{hakem}{\latupha\latupkef\latupmim}{None}{kok:ha_kef_mim}
\kipkokentry{hakîkat}{\latupha\latupkaf\latupkaf}{None}{kok:ha_kaf_kaf}
\kipkokentry{hakim}{\latupha\latupkef\latupmim}{None}{kok:ha_kef_mim}
\kipkokentry{hâkim}{\latupha\latupkef\latupmim}{K\rom{1}, Ed.}{kok:ha_kef_mim}
\kipkokentry{hâkir}{\latupha\latupkaf\latupre}{K\rom{1}, Ed.}{kok:ha_kaf_re}
\kipkokentry{hakkak}{\latupha\latupkef\latupkef}{None}{kok:ha_kef_kef}
\kipkokentry{hal}{\latupha\latupvav\latuplam}{None}{kok:ha_vav_lam}
\kipkokentry{hâlâ}{\latupha\latupvav\latuplam}{None}{kok:ha_vav_lam}
\kipkokentry{hal}{\latupha\latuplam\latuplam}{None}{kok:ha_lam_lam}
\kipkokentry{hal}{\latupxa\latuplam\latupayn}{None}{kok:xa_lam_ayn}
\kipkokentry{hala}{\latupxa\latupvav\latuplam}{None}{kok:xa_vav_lam}
\kipkokentry{halas}{\latupxa\latuplam\latupsad}{None}{kok:xa_lam_sad}
\kipkokentry{halayık}{\latupxa\latuplam\latupkaf}{None}{kok:xa_lam_kaf}
\kipkokentry{halef}{\latupxa\latuplam\latupfe}{None}{kok:xa_lam_fe}
\kipkokentry{halel}{\latupxa\latuplam\latuplam}{None}{kok:xa_lam_lam}
\kipkokentry{hâlen}{\latupha\latupvav\latuplam}{None}{kok:ha_vav_lam}
\kipkokentry{hâlet}{\latupha\latupvav\latuplam}{None}{kok:ha_vav_lam}
\kipkokentry{halhal}{\latupxa\latuplam}{None}{kok:xa_lam}
\kipkokentry{hâli}{\latupxa\latuplam\latupvav}{K\rom{1}, Ed.}{kok:xa_lam_vav}
\kipkokentry{haliç}{\latupxa\latuplam\latupcim}{None}{kok:xa_lam_cim}
\kipkokentry{halîfe}{\latupxa\latuplam\latupfe}{None}{kok:xa_lam_fe}
\kipkokentry{hâlik}{\latupxa\latuplam\latupkaf}{K\rom{1}, Ed.}{kok:xa_lam_kaf}
\kipkokentry{halim}{\latupha\latuplam\latupmim}{None}{kok:ha_lam_mim}
\kipkokentry{hâlis}{\latupxa\latuplam\latupsad}{K\rom{1}, Ed.}{kok:xa_lam_sad}
\kipkokentry{halîta}{\latupxa\latuplam\latupta}{None}{kok:xa_lam_ta}
\kipkokentry{halk}{\latupxa\latuplam\latupkaf}{None}{kok:xa_lam_kaf}
\kipkokentry{halka}{\latupha\latuplam\latupkaf}{None}{kok:ha_lam_kaf}
\kipkokentry{hallaç}{\latupha\latuplam\latupcim}{None}{kok:ha_lam_cim}
\kipkokentry{halt}{\latupxa\latuplam\latupta}{None}{kok:xa_lam_ta}
\kipkokentry{halvet}{\latupxa\latuplam\latupvav}{None}{kok:xa_lam_vav}
\kipkokentry{hamâkat}{\latupha\latupmim\latupkaf}{None}{kok:ha_mim_kaf}
\kipkokentry{hamal}{\latupha\latupmim\latuplam}{None}{kok:ha_mim_lam}
\kipkokentry{hamam}{\latupha\latupmim\latupmim}{None}{kok:ha_mim_mim}
\kipkokentry{hamâset}{\latupha\latupmim\latupsin}{None}{kok:ha_mim_sin}
\kipkokentry{hamd}{\latupha\latupmim\latupdal}{None}{kok:ha_mim_dal}
\kipkokentry{hâmi}{\latupha\latupmim\latupye}{K\rom{1}, Ed.}{kok:ha_mim_ye}
\kipkokentry{hâmil}{\latupha\latupmim\latuplam}{K\rom{1}, Ed.}{kok:ha_mim_lam}
\kipkokentry{hâmile}{\latupha\latupmim\latuplam}{None}{kok:ha_mim_lam}
\kipkokentry{hamiş}{\latuphe\latupmim\latupshin}{K\rom{1}, Ed.}{kok:he_mim_shin}
\kipkokentry{hamit}{\latupha\latupmim\latupdal}{None}{kok:ha_mim_dal}
\kipkokentry{hamiyet}{\latupha\latupmim\latupvav}{None}{kok:ha_mim_vav}
\kipkokentry{haml}{\latupha\latupmim\latuplam}{None}{kok:ha_mim_lam}
\kipkokentry{hamle}{\latupha\latupmim\latuplam}{None}{kok:ha_mim_lam}
\kipkokentry{hamr}{\latupxa\latupmim\latupre}{None}{kok:xa_mim_re}
\kipkokentry{hamra}{\latupha\latupmim\latupre}{None}{kok:ha_mim_re}
\kipkokentry{hamsin}{\latupxa\latupmim\latupsin}{None}{kok:xa_mim_sin}
\kipkokentry{hamûle}{\latupha\latupmim\latuplam}{None}{kok:ha_mim_lam}
\kipkokentry{hamur}{\latupxa\latupmim\latupre}{None}{kok:xa_mim_re}
\kipkokentry{hanif}{\latupha\latupnun\latupfe}{None}{kok:ha_nun_fe}
\kipkokentry{hap}{\latupha\latupbe\latupbe$^1$}{None}{kok:ha_be_be1}
\kipkokentry{hapis}{\latupha\latupbe\latupsin}{None}{kok:ha_be_sin}
\kipkokentry{har}{\latupha\latupre\latupre$^2$}{None}{kok:ha_re_re2}
\kipkokentry{harâbe}{\latupxa\latupre\latupbe}{None}{kok:xa_re_be}
\kipkokentry{haram}{\latupha\latupre\latupmim}{None}{kok:ha_re_mim}
\kipkokentry{harap}{\latupxa\latupre\latupbe}{None}{kok:xa_re_be}
\kipkokentry{harâret}{\latupha\latupre\latupre$^2$}{None}{kok:ha_re_re2}
\kipkokentry{harbi}{\latupha\latupre\latupbe}{None}{kok:ha_re_be}
\kipkokentry{harç}{\latupxa\latupre\latupcim}{None}{kok:xa_re_cim}
\kipkokentry{harekât}{\latupha\latupre\latupkef}{None}{kok:ha_re_kef}
\kipkokentry{hareke}{\latupha\latupre\latupkef}{None}{kok:ha_re_kef}
\kipkokentry{hareket}{\latupha\latupre\latupkef}{None}{kok:ha_re_kef}
\kipkokentry{harem}{\latupha\latupre\latupmim}{None}{kok:ha_re_mim}
\kipkokentry{harf}{\latupha\latupre\latupfe$^1$}{None}{kok:ha_re_fe1}
\kipkokentry{hâriciye}{\latupxa\latupre\latupcim}{None}{kok:xa_re_cim}
\kipkokentry{hâriç}{\latupxa\latupre\latupcim}{K\rom{1}, Ed.}{kok:xa_re_cim}
\kipkokentry{hârika}{\latupxa\latupre\latupkaf}{None}{kok:xa_re_kaf}
\kipkokentry{hârim}{\latupha\latupre\latupmim}{K\rom{1}, Ed.}{kok:ha_re_mim}
\kipkokentry{hâris}{\latupha\latupre\latupsad}{K\rom{1}, Ed.}{kok:ha_re_sad}
\kipkokentry{harp}{\latupha\latupre\latupbe}{None}{kok:ha_re_be}
\kipkokentry{hars}{\latupha\latupre\latupthe}{None}{kok:ha_re_the}
\kipkokentry{has}{\latupxa\latupsad\latupsad}{K\rom{1}, Ed.}{kok:xa_sad_sad}
\kipkokentry{hasan}{\latupha\latupsin\latupnun}{None}{kok:ha_sin_nun}
\kipkokentry{hasar}{\latupxa\latupsin\latupre}{None}{kok:xa_sin_re}
\kipkokentry{hasat}{\latupha\latupsad\latupdal}{None}{kok:ha_sad_dal}
\kipkokentry{hasenat}{\latupha\latupsin\latupnun}{None}{kok:ha_sin_nun}
\kipkokentry{hasep}{\latupha\latupsin\latupbe}{None}{kok:ha_sin_be}
\kipkokentry{haset}{\latupha\latupsin\latupdal}{None}{kok:ha_sin_dal}
\kipkokentry{hâsıla}{\latupha\latupsad\latuplam}{None}{kok:ha_sad_lam}
\kipkokentry{hâsıl}{\latupha\latupsad\latuplam}{K\rom{1}, Ed.}{kok:ha_sad_lam}
\kipkokentry{hâsılat}{\latupha\latupsad\latuplam}{None}{kok:ha_sad_lam}
\kipkokentry{hasım}{\latupxa\latupsad\latupmim}{None}{kok:xa_sad_mim}
\kipkokentry{hasır}{\latupha\latupsad\latupre}{None}{kok:ha_sad_re}
\kipkokentry{hâsip}{\latupha\latupsin\latupbe}{K\rom{1}, Ed.}{kok:ha_sin_be}
\kipkokentry{hasis}{\latupxa\latupsin\latupsin}{None}{kok:xa_sin_sin}
\kipkokentry{haslet}{\latupxa\latupsad\latuplam}{None}{kok:xa_sad_lam}
\kipkokentry{haspa}{\latupha\latupsad\latupbe}{None}{kok:ha_sad_be}
\kipkokentry{hasr}{\latupha\latupsad\latupre}{None}{kok:ha_sad_re}
\kipkokentry{hasret}{\latupha\latupsin\latupre}{None}{kok:ha_sin_re}
\kipkokentry{hassa}{\latupxa\latupsad\latupsad}{None}{kok:xa_sad_sad}
\kipkokentry{hassa}{\latupha\latupsin\latupsin}{None}{kok:ha_sin_sin}
\kipkokentry{hassas}{\latupha\latupsin\latupsin}{None}{kok:ha_sin_sin}
\kipkokentry{hasut}{\latupha\latupsin\latupdal}{None}{kok:ha_sin_dal}
\kipkokentry{haşarı}{\latupha\latupshin\latupre}{None}{kok:ha_shin_re}
\kipkokentry{haşefe}{\latupha\latupshin\latupfe}{None}{kok:ha_shin_fe}
\kipkokentry{haşema}{\latupha\latupshin\latupmim}{None}{kok:ha_shin_mim}
\kipkokentry{haşere}{\latupha\latupshin\latupre}{None}{kok:ha_shin_re}
\kipkokentry{haşhaş}{\latupxa\latupshin}{None}{kok:xa_shin}
\kipkokentry{haşır}{\latupha\latupshin\latupre}{None}{kok:ha_shin_re}
\kipkokentry{haşin}{\latupxa\latupshin\latupnun}{None}{kok:xa_shin_nun}
\kipkokentry{haşiv}{\latupha\latupshin\latupvav}{None}{kok:ha_shin_vav}
\kipkokentry{haşiye}{\latupha\latupshin\latupvav}{None}{kok:ha_shin_vav}
\kipkokentry{haşmet}{\latupha\latupshin\latupmim}{None}{kok:ha_shin_mim}
\kipkokentry{hat}{\latupxa\latupta\latupta}{None}{kok:xa_ta_ta}
\kipkokentry{hatâ}{\latupxa\latupta\latupalif}{None}{kok:xa_ta_alif}
\kipkokentry{hâtem}{\latupxa\latupte\latupmim}{None}{kok:xa_te_mim}
\kipkokentry{hâtıra}{\latupxa\latupta\latupre}{None}{kok:xa_ta_re}
\kipkokentry{hatır}{\latupxa\latupta\latupre}{K\rom{1}, Ed.}{kok:xa_ta_re}
\kipkokentry{hatim}{\latupxa\latupte\latupmim}{None}{kok:xa_te_mim}
\kipkokentry{hâtime}{\latupxa\latupte\latupmim}{None}{kok:xa_te_mim}
\kipkokentry{hatip}{\latupxa\latupta\latupbe}{None}{kok:xa_ta_be}
\kipkokentry{hatmî}{\latupxa\latupta\latupmim}{None}{kok:xa_ta_mim}
\kipkokentry{hattat}{\latupxa\latupta\latupta}{None}{kok:xa_ta_ta}
\kipkokentry{hatve}{\latupxa\latupta\latupvav}{None}{kok:xa_ta_vav}
\kipkokentry{havâdis}{\latupha\latupdal\latupthe}{None}{kok:ha_dal_the}
\kipkokentry{havâle}{\latupha\latupvav\latuplam}{None}{kok:ha_vav_lam}
\kipkokentry{havâli}{\latupha\latupvav\latuplam}{None}{kok:ha_vav_lam}
\kipkokentry{havâs}{\latupxa\latupsad\latupsad}{None}{kok:xa_sad_sad}
\kipkokentry{havf}{\latupxa\latupvav\latupfe}{None}{kok:xa_vav_fe}
\kipkokentry{hâvî}{\latupha\latupvav\latupye}{K\rom{1}, Ed.}{kok:ha_vav_ye}
\kipkokentry{havil}{\latuphe\latupvav\latuplam}{None}{kok:he_vav_lam}
\kipkokentry{havsala}{\latupha\latupsad\latuplam}{None}{kok:ha_sad_lam}
\kipkokentry{havuz}{\latupha\latupvav\latupdad}{None}{kok:ha_vav_dad}
\kipkokentry{havvâ}{\latupha\latupye\latuphe}{None}{kok:ha_ye_he}
\kipkokentry{havya}{\latuphe\latupvav\latupye$^1$}{None}{kok:he_vav_ye1}
\kipkokentry{havza}{\latupha\latupvav\latupze}{None}{kok:ha_vav_ze}
\kipkokentry{hayâl}{\latupxa\latupye\latuplam}{None}{kok:xa_ye_lam}
\kipkokentry{hayâlet}{\latupxa\latupye\latuplam}{None}{kok:xa_ye_lam}
\kipkokentry{hayât}{\latupha\latupvav\latupta}{None}{kok:ha_vav_ta}
\kipkokentry{hayıf}{\latupha\latupye\latupfe}{None}{kok:ha_ye_fe}
\kipkokentry{hayır}{\latupxa\latupye\latupre}{None}{kok:xa_ye_re}
\kipkokentry{hayıt}{\latupha\latupvav\latupta}{K\rom{1}, Ed.}{kok:ha_vav_ta}
\kipkokentry{hayran}{\latupha\latupye\latupre}{None}{kok:ha_ye_re}
\kipkokentry{hayrat}{\latupxa\latupye\latupre}{None}{kok:xa_ye_re}
\kipkokentry{hayret}{\latupha\latupye\latupre}{None}{kok:ha_ye_re}
\kipkokentry{haysiyet}{\latupha\latupye\latupthe}{None}{kok:ha_ye_the}
\kipkokentry{hayy}{\latupha\latupye\latupye}{K\rom{1}, Ed.}{kok:ha_ye_ye}
\kipkokentry{hayz}{\latupha\latupye\latupdad}{None}{kok:ha_ye_dad}
\kipkokentry{haz}{\latupha\latupza\latupza}{None}{kok:ha_za_za}
\kipkokentry{hazâkat}{\latupha\latupzel\latupkaf}{None}{kok:ha_zel_kaf}
\kipkokentry{hazar}{\latupha\latupdad\latupre}{None}{kok:ha_dad_re}
\kipkokentry{hazf}{\latupha\latupzel\latupfe}{None}{kok:ha_zel_fe}
\kipkokentry{hazım}{\latuphe\latupdad\latupmim}{None}{kok:he_dad_mim}
\kipkokentry{hazır}{\latupha\latupdad\latupre}{K\rom{1}, Ed.}{kok:ha_dad_re}
\kipkokentry{hazin}{\latupha\latupze\latupnun}{None}{kok:ha_ze_nun}
\kipkokentry{hazîne}{\latupxa\latupze\latupnun}{None}{kok:xa_ze_nun}
\kipkokentry{hazîre}{\latupha\latupza\latupre}{None}{kok:ha_za_re}
\kipkokentry{hâzirun}{\latupha\latupdad\latupre}{None}{kok:ha_dad_re}
\kipkokentry{hazne}{\latupxa\latupze\latupnun}{None}{kok:xa_ze_nun}
\kipkokentry{hazret}{\latupha\latupdad\latupre}{None}{kok:ha_dad_re}
\kipkokentry{hebâ}{\latuphe\latupbe\latupvav}{None}{kok:he_be_vav}
\kipkokentry{hece}{\latuphe\latupcim\latupvav}{None}{kok:he_cim_vav}
\kipkokentry{hecin}{\latuphe\latupcim\latupnun}{None}{kok:he_cim_nun}
\kipkokentry{hedef}{\latuphe\latupdal\latupfe}{None}{kok:he_dal_fe}
\kipkokentry{heder}{\latuphe\latupdal\latupre}{None}{kok:he_dal_re}
\kipkokentry{hediye}{\latuphe\latupdal\latupye}{None}{kok:he_dal_ye}
\kipkokentry{hekim}{\latupha\latupkef\latupmim}{None}{kok:ha_kef_mim}
\kipkokentry{helâ}{\latupxa\latuplam\latupvav}{None}{kok:xa_lam_vav}
\kipkokentry{helak}{\latuphe\latuplam\latupkaf}{None}{kok:he_lam_kaf}
\kipkokentry{helal}{\latupha\latuplam\latuplam}{None}{kok:ha_lam_lam}
\kipkokentry{helecan}{\latupxa\latuplam\latupcim}{None}{kok:xa_lam_cim}
\kipkokentry{hellim}{\latupha\latuplam\latupmim}{None}{kok:ha_lam_mim}
\kipkokentry{helme}{\latupha\latuplam\latupmim}{None}{kok:ha_lam_mim}
\kipkokentry{helva}{\latupha\latuplam\latupvav}{None}{kok:ha_lam_vav}
\kipkokentry{hemze}{\latuphe\latupmim\latupze}{None}{kok:he_mim_ze}
\kipkokentry{hendese}{\latuphe\latupnun\latupdal\latupsin}{None}{kok:he_nun_dal_sin}
\kipkokentry{herif}{\latupha\latupre\latupfe$^2$}{None}{kok:ha_re_fe2}
\kipkokentry{herîse}{\latuphe\latupre\latupsin}{None}{kok:he_re_sin}
\kipkokentry{hesap}{\latupha\latupsin\latupbe}{None}{kok:ha_sin_be}
\kipkokentry{hevâ}{\latuphe\latupvav\latupye$^2$}{None}{kok:he_vav_ye2}
\kipkokentry{heves}{\latuphe\latupvav\latupsin}{None}{kok:he_vav_sin}
\kipkokentry{heybe}{\latupha\latupkaf\latupbe}{None}{kok:ha_kaf_be}
\kipkokentry{heybet}{\latuphe\latupye\latupbe}{None}{kok:he_ye_be}
\kipkokentry{heyecan}{\latuphe\latupye\latupcim}{None}{kok:he_ye_cim}
\kipkokentry{heyelan}{\latuphe\latupye\latuplam}{None}{kok:he_ye_lam}
\kipkokentry{heyet}{\latuphe\latupye\latupalif}{None}{kok:he_ye_alif}
\kipkokentry{hezeyan}{\latuphe\latupzel\latupye}{None}{kok:he_zel_ye}
\kipkokentry{hezîmet}{\latuphe\latupze\latupmim$^2$}{None}{kok:he_ze_mim2}
\kipkokentry{hıfz}{\latupha\latupfe\latupza}{None}{kok:ha_fe_za}
\kipkokentry{hılat}{\latupxa\latuplam\latupayn}{None}{kok:xa_lam_ayn}
\kipkokentry{hınzır}{\latupxa\latupnun\latupze\latupre}{None}{kok:xa_nun_ze_re}
\kipkokentry{hırka}{\latupxa\latupre\latupkaf}{None}{kok:xa_re_kaf}
\kipkokentry{hırpânî}{\latupxa\latupre\latupbe}{None}{kok:xa_re_be}
\kipkokentry{hırs}{\latupha\latupre\latupsad}{None}{kok:ha_re_sad}
\kipkokentry{hıyânet}{\latupxa\latupvav\latupnun}{None}{kok:xa_vav_nun}
\kipkokentry{hızır}{\latupxa\latupdad\latupre}{None}{kok:xa_dad_re}
\kipkokentry{hızma}{\latupxa\latupze\latupmim}{None}{kok:xa_ze_mim}
\kipkokentry{hîbe}{\latupvav\latuphe\latupbe}{None}{kok:vav_he_be}
\kipkokentry{hicap}{\latupha\latupcim\latupbe}{None}{kok:ha_cim_be}
\kipkokentry{hiciv}{\latuphe\latupcim\latupvav}{None}{kok:he_cim_vav}
\kipkokentry{hicran}{\latuphe\latupcim\latupre}{None}{kok:he_cim_re}
\kipkokentry{hicret}{\latuphe\latupcim\latupre}{None}{kok:he_cim_re}
\kipkokentry{hicrî}{\latuphe\latupcim\latupre}{None}{kok:he_cim_re}
\kipkokentry{hidâyet}{\latuphe\latupdal\latupye}{None}{kok:he_dal_ye}
\kipkokentry{hiddet}{\latupha\latupdal\latupdal}{None}{kok:ha_dal_dal}
\kipkokentry{hidemat}{\latupxa\latupdal\latupmim}{None}{kok:xa_dal_mim}
\kipkokentry{hikâye}{\latupha\latupkef\latupye}{None}{kok:ha_kef_ye}
\kipkokentry{hikmet}{\latupha\latupkef\latupmim}{None}{kok:ha_kef_mim}
\kipkokentry{hilaf}{\latupxa\latuplam\latupfe}{None}{kok:xa_lam_fe}
\kipkokentry{hilâfet}{\latupxa\latuplam\latupfe}{None}{kok:xa_lam_fe}
\kipkokentry{hilal}{\latuphe\latuplam\latuplam}{None}{kok:he_lam_lam}
\kipkokentry{hile}{\latupha\latupvav\latuplam}{None}{kok:ha_vav_lam}
\kipkokentry{hilkat}{\latupxa\latuplam\latupkaf}{None}{kok:xa_lam_kaf}
\kipkokentry{hilm}{\latupha\latuplam\latupmim}{None}{kok:ha_lam_mim}
\kipkokentry{hilye}{\latupha\latuplam\latupvav}{None}{kok:ha_lam_vav}
\kipkokentry{himâye}{\latupha\latupmim\latupye}{None}{kok:ha_mim_ye}
\kipkokentry{himmet}{\latuphe\latupmim\latupmim}{None}{kok:he_mim_mim}
\kipkokentry{hîn}{\latupha\latupye\latupnun}{None}{kok:ha_ye_nun}
\kipkokentry{hirfet}{\latupha\latupre\latupfe$^1$}{None}{kok:ha_re_fe1}
\kipkokentry{his}{\latupha\latupsin\latupsin}{None}{kok:ha_sin_sin}
\kipkokentry{hisar}{\latupha\latupsad\latupre}{None}{kok:ha_sad_re}
\kipkokentry{hisse}{\latupha\latupsad\latupsad}{None}{kok:ha_sad_sad}
\kipkokentry{hitam}{\latupxa\latupte\latupmim}{None}{kok:xa_te_mim}
\kipkokentry{hitan}{\latupxa\latupte\latupnun}{None}{kok:xa_te_nun}
\kipkokentry{hitap}{\latupxa\latupta\latupbe}{None}{kok:xa_ta_be}
\kipkokentry{hizâ}{\latuphe\latupzel\latupvav}{None}{kok:he_zel_vav}
\kipkokentry{hizip}{\latupha\latupze\latupbe}{None}{kok:ha_ze_be}
\kipkokentry{hizmet}{\latupxa\latupdal\latupmim}{None}{kok:xa_dal_mim}
\kipkokentry{hokka}{\latupha\latupkaf\latupkaf}{None}{kok:ha_kaf_kaf}
\kipkokentry{hubûbat}{\latupha\latupbe\latupbe$^1$}{None}{kok:ha_be_be1}
\kipkokentry{hudut}{\latupha\latupdal\latupdal}{None}{kok:ha_dal_dal}
\kipkokentry{hukuk}{\latupha\latupkaf\latupkaf}{None}{kok:ha_kaf_kaf}
\kipkokentry{hulk}{\latupxa\latuplam\latupkaf}{None}{kok:xa_lam_kaf}
\kipkokentry{hulul}{\latupha\latuplam\latuplam}{None}{kok:ha_lam_lam}
\kipkokentry{hulus}{\latupxa\latuplam\latupsad}{None}{kok:xa_lam_sad}
\kipkokentry{hummâ}{\latupha\latupmim\latupmim}{None}{kok:ha_mim_mim}
\kipkokentry{hurâfe}{\latupxa\latupre\latupfe}{None}{kok:xa_re_fe}
\kipkokentry{hurda}{\latupxa\latupdal\latupayn}{None}{kok:xa_dal_ayn}
\kipkokentry{huruç}{\latupxa\latupre\latupcim}{None}{kok:xa_re_cim}
\kipkokentry{huruf}{\latupha\latupre\latupfe$^1$}{None}{kok:ha_re_fe1}
\kipkokentry{husul}{\latupha\latupsad\latuplam}{None}{kok:ha_sad_lam}
\kipkokentry{husûmet}{\latupxa\latupsad\latupmim}{None}{kok:xa_sad_mim}
\kipkokentry{husus}{\latupxa\latupsad\latupsad}{None}{kok:xa_sad_sad}
\kipkokentry{husye}{\latupxa\latupsad\latupye}{None}{kok:xa_sad_ye}
\kipkokentry{huşû}{\latupxa\latupshin\latupayn}{None}{kok:xa_shin_ayn}
\kipkokentry{huşûnet}{\latupxa\latupshin\latupnun}{None}{kok:xa_shin_nun}
\kipkokentry{hutbe}{\latupxa\latupta\latupbe}{None}{kok:xa_ta_be}
\kipkokentry{huzme}{\latupha\latupze\latupmim}{None}{kok:ha_ze_mim}
\kipkokentry{huzur}{\latupha\latupdad\latupre}{None}{kok:ha_dad_re}
\kipkokentry{hüccet}{\latupha\latupcim\latupcim}{None}{kok:ha_cim_cim}
\kipkokentry{hücre}{\latupha\latupcim\latupre$^1$}{None}{kok:ha_cim_re1}
\kipkokentry{hücum}{\latuphe\latupcim\latupmim}{None}{kok:he_cim_mim}
\kipkokentry{hükûmet}{\latupha\latupkef\latupmim}{None}{kok:ha_kef_mim}
\kipkokentry{hüküm}{\latupha\latupkef\latupmim}{None}{kok:ha_kef_mim}
\kipkokentry{hülâsa}{\latupxa\latuplam\latupsad}{None}{kok:xa_lam_sad}
\kipkokentry{hülle}{\latupha\latuplam\latuplam}{None}{kok:ha_lam_lam}
\kipkokentry{hünnap}{\latupayn\latupnun\latupbe}{None}{kok:ayn_nun_be}
\kipkokentry{hünsâ}{\latupxa\latupnun\latupthe}{None}{kok:xa_nun_the}
\kipkokentry{hür}{\latupha\latupre\latupre$^1$}{None}{kok:ha_re_re1}
\kipkokentry{hürmet}{\latupha\latupre\latupmim}{None}{kok:ha_re_mim}
\kipkokentry{hürriyet}{\latupha\latupre\latupre$^1$}{None}{kok:ha_re_re1}
\kipkokentry{hüseyin}{\latupha\latupsin\latupnun}{None}{kok:ha_sin_nun}
\kipkokentry{hüsn}{\latupha\latupsin\latupnun}{None}{kok:ha_sin_nun}
\kipkokentry{hüsran}{\latupxa\latupsin\latupre}{None}{kok:xa_sin_re}
\kipkokentry{hüzün}{\latupha\latupze\latupnun}{None}{kok:ha_ze_nun}
\kipkokentry{hüzzam}{\latuphe\latupze\latupmim$^1$}{None}{kok:he_ze_mim1}
\end{multicols}
\dictchapter{I}
\begin{multicols}{2}
\kipkokentry{ırk}{\latupayn\latupre\latupkaf$^1$}{None}{kok:ayn_re_kaf1}
\kipkokentry{ırz}{\latupayn\latupre\latupdad}{None}{kok:ayn_re_dad}
\kipkokentry{ıslah}{\latupsad\latuplam\latupha}{None}{kok:sad_lam_ha}
\kipkokentry{ıslâhat}{\latupsad\latuplam\latupha}{None}{kok:sad_lam_ha}
\kipkokentry{ısrar}{\latupsad\latupre\latupre$^2$}{None}{kok:sad_re_re2}
\kipkokentry{ıstılah}{\latupsad\latuplam\latupha}{None}{kok:sad_lam_ha}
\kipkokentry{ıstırap}{\latupdad\latupre\latupbe}{None}{kok:dad_re_be}
\kipkokentry{ıtır}{\latupayn\latupta\latupre}{None}{kok:ayn_ta_re}
\kipkokentry{ıtk}{\latupayn\latupte\latupkaf}{None}{kok:ayn_te_kaf}
\kipkokentry{ıtlak}{\latupta\latuplam\latupkaf}{None}{kok:ta_lam_kaf}
\end{multicols}
\dictchapter{İ}
\begin{multicols}{2}
\kipkokentry{iâde}{\latupayn\latupvav\latupdal}{None}{kok:ayn_vav_dal}
\kipkokentry{iâne}{\latupayn\latupvav\latupnun}{None}{kok:ayn_vav_nun}
\kipkokentry{iâşe}{\latupayn\latupye\latupshin}{None}{kok:ayn_ye_shin}
\kipkokentry{ibâdet}{\latupayn\latupbe\latupdal}{None}{kok:ayn_be_dal}
\kipkokentry{ibâre}{\latupayn\latupbe\latupre$^1$}{None}{kok:ayn_be_re1}
\kipkokentry{ibâret}{\latupayn\latupbe\latupre$^1$}{None}{kok:ayn_be_re1}
\kipkokentry{ibkâ}{\latupbe\latupkaf\latupye}{None}{kok:be_kaf_ye}
\kipkokentry{iblağ}{\latupbe\latuplam\latupgayn}{None}{kok:be_lam_gayn}
\kipkokentry{ibn}{\latupbe\latupnun}{None}{kok:be_nun}
\kipkokentry{ibne}{\latupbe\latupnun}{None}{kok:be_nun}
\kipkokentry{ibrâ}{\latupbe\latupre\latupalif}{None}{kok:be_re_alif}
\kipkokentry{ibraz}{\latupbe\latupre\latupze}{None}{kok:be_re_ze}
\kipkokentry{ibre}{\latupayn\latupbe\latupre$^2$}{None}{kok:ayn_be_re2}
\kipkokentry{ibret}{\latupayn\latupbe\latupre$^1$}{None}{kok:ayn_be_re1}
\kipkokentry{icâbet}{\latupcim\latupvav\latupbe}{None}{kok:cim_vav_be}
\kipkokentry{îcap}{\latupvav\latupcim\latupbe}{None}{kok:vav_cim_be}
\kipkokentry{îcar}{\latupalif\latupcim\latupre}{None}{kok:alif_cim_re}
\kipkokentry{îcat}{\latupvav\latupcim\latupdal}{None}{kok:vav_cim_dal}
\kipkokentry{icâzet}{\latupcim\latupvav\latupze}{None}{kok:cim_vav_ze}
\kipkokentry{icbar}{\latupcim\latupbe\latupre}{None}{kok:cim_be_re}
\kipkokentry{iclal}{\latupcim\latuplam\latuplam}{None}{kok:cim_lam_lam}
\kipkokentry{icmâ}{\latupcim\latupmim\latupayn}{None}{kok:cim_mim_ayn}
\kipkokentry{icmal}{\latupcim\latupmim\latuplam}{None}{kok:cim_mim_lam}
\kipkokentry{icrâ}{\latupcim\latupre\latupye}{None}{kok:cim_re_ye}
\kipkokentry{icraat}{\latupcim\latupre\latupye}{None}{kok:cim_re_ye}
\kipkokentry{içtihat}{\latupcim\latuphe\latupdal}{None}{kok:cim_he_dal}
\kipkokentry{içtimâ}{\latupcim\latupmim\latupayn}{None}{kok:cim_mim_ayn}
\kipkokentry{içtinap}{\latupcim\latupnun\latupbe}{None}{kok:cim_nun_be}
\kipkokentry{idâdî}{\latupayn\latupdal\latupdal}{None}{kok:ayn_dal_dal}
\kipkokentry{îdam}{\latupayn\latupdal\latupmim}{None}{kok:ayn_dal_mim}
\kipkokentry{idâme}{\latupdal\latupvav\latupmim}{None}{kok:dal_vav_mim}
\kipkokentry{idâre}{\latupdal\latupvav\latupre}{None}{kok:dal_vav_re}
\kipkokentry{iddiâ}{\latupdal\latupayn\latupvav}{None}{kok:dal_ayn_vav}
\kipkokentry{idgam}{\latupdal\latupgayn\latupmim}{None}{kok:dal_gayn_mim}
\kipkokentry{idman}{\latupdal\latupmim\latupnun}{None}{kok:dal_mim_nun}
\kipkokentry{idrak}{\latupdal\latupre\latupkef}{None}{kok:dal_re_kef}
\kipkokentry{idrar}{\latupdal\latupre\latupre}{None}{kok:dal_re_re}
\kipkokentry{îfâ}{\latupvav\latupfe\latupye}{None}{kok:vav_fe_ye}
\kipkokentry{ifâde}{\latupfe\latupye\latupdal}{None}{kok:fe_ye_dal}
\kipkokentry{iffet}{\latupayn\latupfe\latupfe}{None}{kok:ayn_fe_fe}
\kipkokentry{iflah}{\latupfe\latuplam\latupha}{None}{kok:fe_lam_ha}
\kipkokentry{iflas}{\latupfe\latuplam\latupsin}{None}{kok:fe_lam_sin}
\kipkokentry{ifnâ}{\latupfe\latupnun\latupye}{None}{kok:fe_nun_ye}
\kipkokentry{ifrağ}{\latupfe\latupre\latupgayn}{None}{kok:fe_re_gayn}
\kipkokentry{ifrat}{\latupfe\latupre\latupta}{None}{kok:fe_re_ta}
\kipkokentry{ifraz}{\latupfe\latupre\latupze}{None}{kok:fe_re_ze}
\kipkokentry{ifrâzat}{\latupfe\latupre\latupze}{None}{kok:fe_re_ze}
\kipkokentry{ifsat}{\latupfe\latupsin\latupdal}{None}{kok:fe_sin_dal}
\kipkokentry{ifşâ}{\latupfe\latupshin\latupvav}{None}{kok:fe_shin_vav}
\kipkokentry{iftar}{\latupfe\latupta\latupre}{None}{kok:fe_ta_re}
\kipkokentry{iftihar}{\latupfe\latupxa\latupre}{None}{kok:fe_xa_re}
\kipkokentry{iftirâ}{\latupfe\latupre\latupye}{None}{kok:fe_re_ye}
\kipkokentry{iğbirar}{\latupgayn\latupbe\latupre}{None}{kok:gayn_be_re}
\kipkokentry{iğfal}{\latupgayn\latupfe\latuplam}{None}{kok:gayn_fe_lam}
\kipkokentry{iğtişaş}{\latupgayn\latupshin\latupshin}{None}{kok:gayn_shin_shin}
\kipkokentry{iğvâ}{\latupgayn\latupvav\latupye}{None}{kok:gayn_vav_ye}
\kipkokentry{ihâle}{\latupha\latupvav\latuplam}{None}{kok:ha_vav_lam}
\kipkokentry{ihânet}{\latupxa\latupvav\latupnun}{None}{kok:xa_vav_nun}
\kipkokentry{ihâta}{\latupha\latupvav\latupta}{None}{kok:ha_vav_ta}
\kipkokentry{ihbar}{\latupxa\latupbe\latupre}{None}{kok:xa_be_re}
\kipkokentry{ihdas}{\latupha\latupdal\latupthe}{None}{kok:ha_dal_the}
\kipkokentry{ihkak}{\latupha\latupkaf\latupkaf}{None}{kok:ha_kaf_kaf}
\kipkokentry{ihlal}{\latupxa\latuplam\latuplam}{None}{kok:xa_lam_lam}
\kipkokentry{ihlas}{\latupxa\latuplam\latupsad}{None}{kok:xa_lam_sad}
\kipkokentry{ihmal}{\latuphe\latupmim\latuplam}{None}{kok:he_mim_lam}
\kipkokentry{ihrâcat}{\latupxa\latupre\latupcim}{None}{kok:xa_re_cim}
\kipkokentry{ihraç}{\latupxa\latupre\latupcim}{None}{kok:xa_re_cim}
\kipkokentry{ihram}{\latupha\latupre\latupmim}{None}{kok:ha_re_mim}
\kipkokentry{ihsan}{\latupha\latupsin\latupnun}{None}{kok:ha_sin_nun}
\kipkokentry{ihsas}{\latupha\latupsin\latupsin}{None}{kok:ha_sin_sin}
\kipkokentry{ihtâ}{\latupxa\latupta\latupalif}{None}{kok:xa_ta_alif}
\kipkokentry{ihtar}{\latupxa\latupta\latupre}{None}{kok:xa_ta_re}
\kipkokentry{ihtidâ}{\latuphe\latupdal\latupye}{None}{kok:he_dal_ye}
\kipkokentry{ihtikar}{\latupha\latupkef\latupre}{None}{kok:ha_kef_re}
\kipkokentry{ihtilaç}{\latupxa\latuplam\latupcim}{None}{kok:xa_lam_cim}
\kipkokentry{ihtilaf}{\latupxa\latuplam\latupfe}{None}{kok:xa_lam_fe}
\kipkokentry{ihtilal}{\latupxa\latuplam\latuplam}{None}{kok:xa_lam_lam}
\kipkokentry{ihtilat}{\latupxa\latuplam\latupta}{None}{kok:xa_lam_ta}
\kipkokentry{ihtimal}{\latupha\latupmim\latuplam}{None}{kok:ha_mim_lam}
\kipkokentry{ihtimam}{\latuphe\latupmim\latupmim}{None}{kok:he_mim_mim}
\kipkokentry{ihtirâ}{\latupxa\latupre\latupayn}{None}{kok:xa_re_ayn}
\kipkokentry{ihtiram}{\latupha\latupre\latupmim}{None}{kok:ha_re_mim}
\kipkokentry{ihtiras}{\latupha\latupre\latupsad}{None}{kok:ha_re_sad}
\kipkokentry{ihtisap}{\latupha\latupsin\latupbe}{None}{kok:ha_sin_be}
\kipkokentry{ihtisas}{\latupxa\latupsad\latupsad}{None}{kok:xa_sad_sad}
\kipkokentry{ihtişam}{\latupha\latupshin\latupmim}{None}{kok:ha_shin_mim}
\kipkokentry{ihtivâ}{\latupha\latupvav\latupye}{None}{kok:ha_vav_ye}
\kipkokentry{ihtiyaç}{\latupha\latupvav\latupcim}{None}{kok:ha_vav_cim}
\kipkokentry{ihtiyar}{\latupxa\latupye\latupre}{None}{kok:xa_ye_re}
\kipkokentry{ihtiyat}{\latupha\latupvav\latupta}{None}{kok:ha_vav_ta}
\kipkokentry{ihvan}{\latupalif\latupxa\latupvav}{None}{kok:alif_xa_vav}
\kipkokentry{ihzar}{\latupha\latupdad\latupre}{None}{kok:ha_dad_re}
\kipkokentry{îkâ}{\latupvav\latupkaf\latupayn}{None}{kok:vav_kaf_ayn}
\kipkokentry{ikâme}{\latupkaf\latupvav\latupmim}{None}{kok:kaf_vav_mim}
\kipkokentry{ikâmet}{\latupkaf\latupvav\latupmim}{None}{kok:kaf_vav_mim}
\kipkokentry{îkaz}{\latupye\latupkaf\latupza}{None}{kok:ye_kaf_za}
\kipkokentry{ikbal}{\latupkaf\latupbe\latuplam}{None}{kok:kaf_be_lam}
\kipkokentry{ikmal}{\latupkef\latupmim\latuplam}{None}{kok:kef_mim_lam}
\kipkokentry{iknâ}{\latupkaf\latupnun\latupayn}{None}{kok:kaf_nun_ayn}
\kipkokentry{ikrah}{\latupkef\latupre\latuphe}{None}{kok:kef_re_he}
\kipkokentry{ikram}{\latupkef\latupre\latupmim}{None}{kok:kef_re_mim}
\kipkokentry{ikrar}{\latupkaf\latupre\latupre}{None}{kok:kaf_re_re}
\kipkokentry{ikraz}{\latupkaf\latupre\latupdad}{None}{kok:kaf_re_dad}
\kipkokentry{iktâ}{\latupkaf\latupta\latupayn}{None}{kok:kaf_ta_ayn}
\kipkokentry{iktibas}{\latupkaf\latupbe\latupsin}{None}{kok:kaf_be_sin}
\kipkokentry{iktidar}{\latupkaf\latupdal\latupre}{None}{kok:kaf_dal_re}
\kipkokentry{iktifâ}{\latupkef\latupfe\latupvav}{None}{kok:kef_fe_vav}
\kipkokentry{iktiran}{\latupkaf\latupre\latupnun}{None}{kok:kaf_re_nun}
\kipkokentry{iktisap}{\latupkef\latupsin\latupbe}{None}{kok:kef_sin_be}
\kipkokentry{iktisat}{\latupkaf\latupsad\latupdal}{None}{kok:kaf_sad_dal}
\kipkokentry{iktizâ}{\latupkaf\latupdad\latupye}{None}{kok:kaf_dad_ye}
\kipkokentry{îlâ}{\latupayn\latuplam\latupvav}{None}{kok:ayn_lam_vav}
\kipkokentry{ilaç}{\latupayn\latuplam\latupcim}{None}{kok:ayn_lam_cim}
\kipkokentry{ilah}{\latupalif\latuplam\latuphe}{None}{kok:alif_lam_he}
\kipkokentry{ilâhiyet}{\latupalif\latuplam\latuphe}{None}{kok:alif_lam_he}
\kipkokentry{ilâm}{\latupayn\latuplam\latupmim}{None}{kok:ayn_lam_mim}
\kipkokentry{îlan}{\latupayn\latuplam\latupnun}{None}{kok:ayn_lam_nun}
\kipkokentry{ilâve}{\latupayn\latuplam\latupvav}{None}{kok:ayn_lam_vav}
\kipkokentry{ilgâ}{\latuplam\latupgayn\latupvav}{None}{kok:lam_gayn_vav}
\kipkokentry{ilhak}{\latuplam\latupha\latupkaf}{None}{kok:lam_ha_kaf}
\kipkokentry{ilham}{\latuplam\latuphe\latupmim}{None}{kok:lam_he_mim}
\kipkokentry{ilim}{\latupayn\latuplam\latupmim}{None}{kok:ayn_lam_mim}
\kipkokentry{ilkah}{\latuplam\latupkaf\latupha}{None}{kok:lam_kaf_ha}
\kipkokentry{illet}{\latupayn\latuplam\latuplam}{None}{kok:ayn_lam_lam}
\kipkokentry{illiyet}{\latupayn\latuplam\latuplam}{None}{kok:ayn_lam_lam}
\kipkokentry{iltibas}{\latuplam\latupbe\latupsin}{None}{kok:lam_be_sin}
\kipkokentry{iltifat}{\latuplam\latupfe\latupte}{None}{kok:lam_fe_te}
\kipkokentry{iltihak}{\latuplam\latupha\latupkaf}{None}{kok:lam_ha_kaf}
\kipkokentry{iltihap}{\latuplam\latuphe\latupbe}{None}{kok:lam_he_be}
\kipkokentry{iltimas}{\latuplam\latupmim\latupsin}{None}{kok:lam_mim_sin}
\kipkokentry{iltizam}{\latuplam\latupze\latupmim}{None}{kok:lam_ze_mim}
\kipkokentry{ilzam}{\latuplam\latupze\latupmim}{None}{kok:lam_ze_mim}
\kipkokentry{îmâ}{\latupvav\latupmim\latupalif}{None}{kok:vav_mim_alif}
\kipkokentry{îmal}{\latupayn\latupmim\latuplam}{None}{kok:ayn_mim_lam}
\kipkokentry{îmalat}{\latupayn\latupmim\latuplam}{None}{kok:ayn_mim_lam}
\kipkokentry{imâle}{\latupmim\latupye\latuplam}{None}{kok:mim_ye_lam}
\kipkokentry{imam}{\latupalif\latupmim\latupmim}{None}{kok:alif_mim_mim}
\kipkokentry{imâme}{\latupalif\latupmim\latupmim}{None}{kok:alif_mim_mim}
\kipkokentry{imân}{\latupalif\latupmim\latupnun}{None}{kok:alif_mim_nun}
\kipkokentry{îmar}{\latupayn\latupmim\latupre}{None}{kok:ayn_mim_re}
\kipkokentry{imâret}{\latupayn\latupmim\latupre}{None}{kok:ayn_mim_re}
\kipkokentry{imdat}{\latupmim\latupdal\latupdal$^1$}{None}{kok:mim_dal_dal1}
\kipkokentry{imhâ}{\latupmim\latupha\latupvav}{None}{kok:mim_ha_vav}
\kipkokentry{imkan}{\latupmim\latupkef\latupnun}{None}{kok:mim_kef_nun}
\kipkokentry{imlâ}{\latupmim\latuplam\latupvav}{None}{kok:mim_lam_vav}
\kipkokentry{imsak}{\latupmim\latupsin\latupkaf}{None}{kok:mim_sin_kaf}
\kipkokentry{imtihan}{\latupmim\latupha\latupnun}{None}{kok:mim_ha_nun}
\kipkokentry{imtinâ}{\latupmim\latupnun\latupayn}{None}{kok:mim_nun_ayn}
\kipkokentry{imtiyaz}{\latupmim\latupye\latupze}{None}{kok:mim_ye_ze}
\kipkokentry{imtizaç}{\latupmim\latupze\latupcim}{None}{kok:mim_ze_cim}
\kipkokentry{imzâ}{\latupmim\latupdad\latupye}{None}{kok:mim_dad_ye}
\kipkokentry{inam}{\latupnun\latupayn\latupmim}{None}{kok:nun_ayn_mim}
\kipkokentry{inat}{\latupayn\latupnun\latupdal}{None}{kok:ayn_nun_dal}
\kipkokentry{inâyet}{\latupayn\latupnun\latupye}{None}{kok:ayn_nun_ye}
\kipkokentry{inbisat}{\latupbe\latupsin\latupta}{None}{kok:be_sin_ta}
\kipkokentry{ind}{\latupayn\latupnun\latupdal}{None}{kok:ayn_nun_dal}
\kipkokentry{indifâ}{\latupdal\latupfe\latupayn}{None}{kok:dal_fe_ayn}
\kipkokentry{infak}{\latupnun\latupfe\latupkaf}{None}{kok:nun_fe_kaf}
\kipkokentry{infaz}{\latupnun\latupfe\latupzel}{None}{kok:nun_fe_zel}
\kipkokentry{infiâl}{\latupfe\latupayn\latuplam}{None}{kok:fe_ayn_lam}
\kipkokentry{infilak}{\latupfe\latuplam\latupkaf}{None}{kok:fe_lam_kaf}
\kipkokentry{infirat}{\latupfe\latupre\latupdal}{None}{kok:fe_re_dal}
\kipkokentry{infisah}{\latupfe\latupsin\latupxa}{None}{kok:fe_sin_xa}
\kipkokentry{infisal}{\latupfe\latupsad\latuplam}{None}{kok:fe_sad_lam}
\kipkokentry{inhâ}{\latupnun\latuphe\latupvav}{None}{kok:nun_he_vav}
\kipkokentry{inhidam}{\latuphe\latupdal\latupmim}{None}{kok:he_dal_mim}
\kipkokentry{inhilal}{\latupha\latuplam\latuplam}{None}{kok:ha_lam_lam}
\kipkokentry{inhiraf}{\latupha\latupre\latupfe$^1$}{None}{kok:ha_re_fe1}
\kipkokentry{inhisar}{\latupha\latupsad\latupre}{None}{kok:ha_sad_re}
\kipkokentry{inhitat}{\latupha\latupta\latupta}{None}{kok:ha_ta_ta}
\kipkokentry{inkar}{\latupnun\latupkef\latupre}{None}{kok:nun_kef_re}
\kipkokentry{inkıbaz}{\latupkaf\latupbe\latupdad}{None}{kok:kaf_be_dad}
\kipkokentry{inkılap}{\latupkaf\latuplam\latupbe$^1$}{None}{kok:kaf_lam_be1}
\kipkokentry{inkıraz}{\latupkaf\latupre\latupdad}{None}{kok:kaf_re_dad}
\kipkokentry{inkıtâ}{\latupkaf\latupta\latupayn}{None}{kok:kaf_ta_ayn}
\kipkokentry{inkıyat}{\latupkaf\latupye\latupdal}{None}{kok:kaf_ye_dal}
\kipkokentry{inkisar}{\latupkef\latupsin\latupre}{None}{kok:kef_sin_re}
\kipkokentry{inkişaf}{\latupkef\latupshin\latupfe}{None}{kok:kef_shin_fe}
\kipkokentry{insaf}{\latupnun\latupsad\latupfe}{None}{kok:nun_sad_fe}
\kipkokentry{insan}{\latupalif\latupnun\latupsin}{None}{kok:alif_nun_sin}
\kipkokentry{insicam}{\latupsin\latupcim\latupmim}{None}{kok:sin_cim_mim}
\kipkokentry{insiyak}{\latupsin\latupvav\latupkaf}{None}{kok:sin_vav_kaf}
\kipkokentry{inşâ}{\latupnun\latupshin\latupalif}{None}{kok:nun_shin_alif}
\kipkokentry{inşirah}{\latupshin\latupre\latupha}{None}{kok:shin_re_ha}
\kipkokentry{intaç}{\latupnun\latupte\latupcim}{None}{kok:nun_te_cim}
\kipkokentry{intâniye}{\latupnun\latupte\latupnun}{None}{kok:nun_te_nun}
\kipkokentry{intibâ}{\latupta\latupbe\latupayn}{None}{kok:ta_be_ayn}
\kipkokentry{intibah}{\latupnun\latupbe\latuphe}{None}{kok:nun_be_he}
\kipkokentry{intibak}{\latupta\latupbe\latupkaf}{None}{kok:ta_be_kaf}
\kipkokentry{intifâ}{\latupnun\latupfe\latupayn}{None}{kok:nun_fe_ayn}
\kipkokentry{intifâda}{\latupnun\latupfe\latupdad}{None}{kok:nun_fe_dad}
\kipkokentry{intihâ}{\latupnun\latuphe\latupvav}{None}{kok:nun_he_vav}
\kipkokentry{intihal}{\latupnun\latupha\latuplam}{None}{kok:nun_ha_lam}
\kipkokentry{intihap}{\latupnun\latupxa\latupbe}{None}{kok:nun_xa_be}
\kipkokentry{intihar}{\latupnun\latupha\latupre}{None}{kok:nun_ha_re}
\kipkokentry{intikal}{\latupnun\latupkaf\latuplam}{None}{kok:nun_kaf_lam}
\kipkokentry{intikam}{\latupnun\latupkaf\latupmim}{None}{kok:nun_kaf_mim}
\kipkokentry{intisap}{\latupnun\latupsin\latupbe}{None}{kok:nun_sin_be}
\kipkokentry{intişasr}{\latupnun\latupshin\latupre}{None}{kok:nun_shin_re}
\kipkokentry{intizam}{\latupnun\latupza\latupmim}{None}{kok:nun_za_mim}
\kipkokentry{intizar}{\latupnun\latupza\latupre}{None}{kok:nun_za_re}
\kipkokentry{inzal}{\latupnun\latupze\latuplam}{None}{kok:nun_ze_lam}
\kipkokentry{inzibat}{\latupza\latupbe\latupta}{None}{kok:za_be_ta}
\kipkokentry{inzivâ}{\latupze\latupvav\latupye}{None}{kok:ze_vav_ye}
\kipkokentry{iptal}{\latupbe\latupta\latuplam}{None}{kok:be_ta_lam}
\kipkokentry{iptidâ}{\latupbe\latupdal\latupalif}{None}{kok:be_dal_alif}
\kipkokentry{iptilâ}{\latupbe\latuplam\latupvav}{None}{kok:be_lam_vav}
\kipkokentry{irâde}{\latupre\latupvav\latupdal}{None}{kok:re_vav_dal}
\kipkokentry{îrap}{\latupayn\latupre\latupbe}{None}{kok:ayn_re_be}
\kipkokentry{irât}{\latupvav\latupre\latupdal}{None}{kok:vav_re_dal}
\kipkokentry{irfan}{\latupayn\latupre\latupfe}{None}{kok:ayn_re_fe}
\kipkokentry{irs}{\latupvav\latupre\latupthe}{None}{kok:vav_re_the}
\kipkokentry{irsâliye}{\latupre\latupsin\latuplam}{None}{kok:re_sin_lam}
\kipkokentry{irşat}{\latupre\latupshin\latupdal}{None}{kok:re_shin_dal}
\kipkokentry{irtibat}{\latupre\latupbe\latupta}{None}{kok:re_be_ta}
\kipkokentry{irticâ}{\latupre\latupcim\latupayn}{None}{kok:re_cim_ayn}
\kipkokentry{irtical}{\latupre\latupcim\latuplam}{None}{kok:re_cim_lam}
\kipkokentry{irtidat}{\latupre\latupdal\latupdal}{None}{kok:re_dal_dal}
\kipkokentry{irtifâ}{\latupre\latupfe\latupayn}{None}{kok:re_fe_ayn}
\kipkokentry{irtifak}{\latupre\latupfe\latupkaf}{None}{kok:re_fe_kaf}
\kipkokentry{irtihal}{\latupre\latupha\latuplam}{None}{kok:re_ha_lam}
\kipkokentry{irtikap}{\latupre\latupkef\latupbe}{None}{kok:re_kef_be}
\kipkokentry{irtişâ}{\latupre\latupshin\latupvav}{None}{kok:re_shin_vav}
\kipkokentry{isâbet}{\latupsad\latupvav\latupbe}{None}{kok:sad_vav_be}
\kipkokentry{isâle}{\latupsin\latupye\latuplam}{None}{kok:sin_ye_lam}
\kipkokentry{ishal}{\latupsin\latuphe\latuplam}{None}{kok:sin_he_lam}
\kipkokentry{isim}{\latupsin\latupmim}{None}{kok:sin_mim}
\kipkokentry{iskan}{\latupsin\latupkef\latupnun}{None}{kok:sin_kef_nun}
\kipkokentry{iskat}{\latupsin\latupkaf\latupta}{None}{kok:sin_kaf_ta}
\kipkokentry{islam}{\latupsin\latuplam\latupmim}{None}{kok:sin_lam_mim}
\kipkokentry{islâmiyet}{\latupsin\latuplam\latupmim}{None}{kok:sin_lam_mim}
\kipkokentry{ismet}{\latupayn\latupsad\latupmim}{None}{kok:ayn_sad_mim}
\kipkokentry{isnat}{\latupsin\latupnun\latupdal}{None}{kok:sin_nun_dal}
\kipkokentry{ispat}{\latupthe\latupbe\latupte}{None}{kok:the_be_te}
\kipkokentry{israf}{\latupsin\latupre\latupfe}{None}{kok:sin_re_fe}
\kipkokentry{istiap}{\latupvav\latupayn\latupbe}{None}{kok:vav_ayn_be}
\kipkokentry{istiâre}{\latupayn\latupvav\latupre}{None}{kok:ayn_vav_re}
\kipkokentry{istibdat}{\latupbe\latupdal\latupdal}{None}{kok:be_dal_dal}
\kipkokentry{istical}{\latupayn\latupcim\latuplam}{None}{kok:ayn_cim_lam}
\kipkokentry{isticvap}{\latupcim\latupvav\latupbe}{None}{kok:cim_vav_be}
\kipkokentry{istidâ}{\latupdal\latupayn\latupvav}{None}{kok:dal_ayn_vav}
\kipkokentry{istîdat}{\latupayn\latupdal\latupdal}{None}{kok:ayn_dal_dal}
\kipkokentry{istidlal}{\latupdal\latuplam\latuplam}{None}{kok:dal_lam_lam}
\kipkokentry{istifâ}{\latupayn\latupfe\latupvav}{None}{kok:ayn_fe_vav}
\kipkokentry{istifâde}{\latupfe\latupye\latupdal}{None}{kok:fe_ye_dal}
\kipkokentry{istifham}{\latupfe\latuphe\latupmim}{None}{kok:fe_he_mim}
\kipkokentry{istifrâ}{\latupfe\latupre\latupgayn}{None}{kok:fe_re_gayn}
\kipkokentry{istiğfar}{\latupgayn\latupfe\latupre}{None}{kok:gayn_fe_re}
\kipkokentry{istiğnâ}{\latupgayn\latupnun\latupye$^1$}{None}{kok:gayn_nun_ye1}
\kipkokentry{istiğrak}{\latupgayn\latupre\latupkaf}{None}{kok:gayn_re_kaf}
\kipkokentry{istihâle}{\latupha\latupvav\latuplam}{None}{kok:ha_vav_lam}
\kipkokentry{istihâre}{\latupxa\latupye\latupre}{None}{kok:xa_ye_re}
\kipkokentry{istihbârat}{\latupxa\latupbe\latupre}{None}{kok:xa_be_re}
\kipkokentry{istihdam}{\latupxa\latupdal\latupmim}{None}{kok:xa_dal_mim}
\kipkokentry{istihfaf}{\latupxa\latupfe\latupfe}{None}{kok:xa_fe_fe}
\kipkokentry{istihkak}{\latupha\latupkaf\latupkaf}{None}{kok:ha_kaf_kaf}
\kipkokentry{istihkam}{\latupha\latupkef\latupmim}{None}{kok:ha_kef_mim}
\kipkokentry{istihlak}{\latuphe\latuplam\latupkaf}{None}{kok:he_lam_kaf}
\kipkokentry{istihraç}{\latupxa\latupre\latupcim}{None}{kok:xa_re_cim}
\kipkokentry{istihsal}{\latupha\latupsad\latuplam}{None}{kok:ha_sad_lam}
\kipkokentry{istihsan}{\latupha\latupsin\latupnun}{None}{kok:ha_sin_nun}
\kipkokentry{istihzâ}{\latuphe\latupze\latupalif}{None}{kok:he_ze_alif}
\kipkokentry{istikâmet}{\latupkaf\latupvav\latupmim}{None}{kok:kaf_vav_mim}
\kipkokentry{istikbal}{\latupkaf\latupbe\latuplam}{None}{kok:kaf_be_lam}
\kipkokentry{istiklal}{\latupkaf\latuplam\latuplam}{None}{kok:kaf_lam_lam}
\kipkokentry{istikrah}{\latupkef\latupre\latuphe}{None}{kok:kef_re_he}
\kipkokentry{istikrar}{\latupkaf\latupre\latupre}{None}{kok:kaf_re_re}
\kipkokentry{istikraz}{\latupkaf\latupre\latupdad}{None}{kok:kaf_re_dad}
\kipkokentry{istikşaf}{\latupkef\latupshin\latupfe}{None}{kok:kef_shin_fe}
\kipkokentry{istilâ}{\latupvav\latuplam\latupye}{None}{kok:vav_lam_ye}
\kipkokentry{istilzam}{\latuplam\latupze\latupmim}{None}{kok:lam_ze_mim}
\kipkokentry{istimâ}{\latupsin\latupmim\latupayn}{None}{kok:sin_mim_ayn}
\kipkokentry{istimal}{\latupayn\latupmim\latuplam}{None}{kok:ayn_mim_lam}
\kipkokentry{istimdat}{\latupmim\latupdal\latupdal$^1$}{None}{kok:mim_dal_dal1}
\kipkokentry{istimlak}{\latupmim\latuplam\latupkef}{None}{kok:mim_lam_kef}
\kipkokentry{istimnâ}{\latupmim\latupnun\latupye$^2$}{None}{kok:mim_nun_ye2}
\kipkokentry{istinâbe}{\latupnun\latupvav\latupbe}{None}{kok:nun_vav_be}
\kipkokentry{istinaf}{\latupalif\latupnun\latupfe}{None}{kok:alif_nun_fe}
\kipkokentry{istinat}{\latupsin\latupnun\latupdal}{None}{kok:sin_nun_dal}
\kipkokentry{istinkaf}{\latupnun\latupkef\latupfe}{None}{kok:nun_kef_fe}
\kipkokentry{istinsah}{\latupnun\latupsin\latupxa}{None}{kok:nun_sin_xa}
\kipkokentry{istintak}{\latupnun\latupta\latupkaf}{None}{kok:nun_ta_kaf}
\kipkokentry{istirahat}{\latupre\latupvav\latupha}{None}{kok:re_vav_ha}
\kipkokentry{istirdat}{\latupre\latupdal\latupdal}{None}{kok:re_dal_dal}
\kipkokentry{istirham}{\latupre\latupha\latupmim}{None}{kok:re_ha_mim}
\kipkokentry{istiskal}{\latupthe\latupkaf\latuplam}{None}{kok:the_kaf_lam}
\kipkokentry{istismar}{\latupsin\latupmim\latupre$^2$}{None}{kok:sin_mim_re2}
\kipkokentry{istisnâ}{\latupthe\latupnun\latupye}{None}{kok:the_nun_ye}
\kipkokentry{istişâre}{\latupshin\latupvav\latupre}{None}{kok:shin_vav_re}
\kipkokentry{istitrat}{\latupta\latupre\latupdal}{None}{kok:ta_re_dal}
\kipkokentry{istivâ}{\latupsin\latupvav\latupye}{None}{kok:sin_vav_ye}
\kipkokentry{istizah}{\latupvav\latupdad\latupha}{None}{kok:vav_dad_ha}
\kipkokentry{isyan}{\latupayn\latupsad\latupye}{None}{kok:ayn_sad_ye}
\kipkokentry{işâret}{\latupshin\latupvav\latupre}{None}{kok:shin_vav_re}
\kipkokentry{işgal}{\latupshin\latupgayn\latuplam}{None}{kok:shin_gayn_lam}
\kipkokentry{işkil}{\latupshin\latupkef\latuplam$^1$}{None}{kok:shin_kef_lam1}
\kipkokentry{işret}{\latupayn\latupshin\latupre}{None}{kok:ayn_shin_re}
\kipkokentry{iştah}{\latupshin\latuphe\latupvav}{None}{kok:shin_he_vav}
\kipkokentry{iştigal}{\latupshin\latupgayn\latuplam}{None}{kok:shin_gayn_lam}
\kipkokentry{iştikak}{\latupshin\latupkaf\latupkaf}{None}{kok:shin_kaf_kaf}
\kipkokentry{iştirâ}{\latupshin\latupre\latupye}{None}{kok:shin_re_ye}
\kipkokentry{iştirak}{\latupshin\latupre\latupkef}{None}{kok:shin_re_kef}
\kipkokentry{iştiyak}{\latupshin\latupvav\latupkaf}{None}{kok:shin_vav_kaf}
\kipkokentry{îtâ}{\latupayn\latupta\latupvav}{None}{kok:ayn_ta_vav}
\kipkokentry{itaat}{\latupta\latupvav\latupayn}{None}{kok:ta_vav_ayn}
\kipkokentry{itfâ}{\latupta\latupfe\latupalif}{None}{kok:ta_fe_alif}
\kipkokentry{itfâiye}{\latupta\latupfe\latupalif}{None}{kok:ta_fe_alif}
\kipkokentry{ithaf}{\latupte\latupha\latupfe}{None}{kok:te_ha_fe}
\kipkokentry{ithal}{\latupdal\latupxa\latuplam}{None}{kok:dal_xa_lam}
\kipkokentry{ithâlat}{\latupdal\latupxa\latuplam}{None}{kok:dal_xa_lam}
\kipkokentry{itham}{\latupvav\latuphe\latupmim}{None}{kok:vav_he_mim}
\kipkokentry{îtibar}{\latupayn\latupbe\latupre$^1$}{None}{kok:ayn_be_re1}
\kipkokentry{îtidal}{\latupayn\latupdal\latuplam}{None}{kok:ayn_dal_lam}
\kipkokentry{îtikaf}{\latupayn\latupkef\latupfe}{None}{kok:ayn_kef_fe}
\kipkokentry{îtikat}{\latupayn\latupkaf\latupdal}{None}{kok:ayn_kaf_dal}
\kipkokentry{îtilaf}{\latupalif\latuplam\latupfe}{None}{kok:alif_lam_fe}
\kipkokentry{îtimat}{\latupayn\latupmim\latupdal}{None}{kok:ayn_mim_dal}
\kipkokentry{îtinâ}{\latupayn\latupnun\latupye}{None}{kok:ayn_nun_ye}
\kipkokentry{îtiraf}{\latupayn\latupre\latupfe}{None}{kok:ayn_re_fe}
\kipkokentry{îtiraz}{\latupayn\latupre\latupdad}{None}{kok:ayn_re_dad}
\kipkokentry{îtiyat}{\latupayn\latupvav\latupdal}{None}{kok:ayn_vav_dal}
\kipkokentry{îtizar}{\latupayn\latupzel\latupre}{None}{kok:ayn_zel_re}
\kipkokentry{itlaf}{\latupte\latuplam\latupfe}{None}{kok:te_lam_fe}
\kipkokentry{itmam}{\latupte\latupmim\latupmim}{None}{kok:te_mim_mim}
\kipkokentry{itminan}{\latupta\latupmim\latupalif\latupnun}{None}{kok:ta_mim_alif_nun}
\kipkokentry{ittifak}{\latupvav\latupfe\latupkaf}{None}{kok:vav_fe_kaf}
\kipkokentry{ittihat}{\latupvav\latupha\latupdal}{None}{kok:vav_ha_dal}
\kipkokentry{ittihaz}{\latupalif\latupxa\latupzel}{None}{kok:alif_xa_zel}
\kipkokentry{ittilâ}{\latupta\latuplam\latupayn}{None}{kok:ta_lam_ayn}
\kipkokentry{ittisal}{\latupvav\latupsad\latuplam}{None}{kok:vav_sad_lam}
\kipkokentry{ivaz}{\latupayn\latupvav\latupdad}{None}{kok:ayn_vav_dad}
\kipkokentry{izâbe}{\latupzel\latupvav\latupbe}{None}{kok:zel_vav_be}
\kipkokentry{izâfe}{\latupdad\latupye\latupfe}{None}{kok:dad_ye_fe}
\kipkokentry{izâfiyet}{\latupdad\latupye\latupfe}{None}{kok:dad_ye_fe}
\kipkokentry{izah}{\latupvav\latupdad\latupha}{None}{kok:vav_dad_ha}
\kipkokentry{izâle}{\latupze\latupvav\latuplam}{None}{kok:ze_vav_lam}
\kipkokentry{izan}{\latupzel\latupayn\latupnun}{None}{kok:zel_ayn_nun}
\kipkokentry{îzaz}{\latupayn\latupze\latupze}{None}{kok:ayn_ze_ze}
\kipkokentry{izdiham}{\latupze\latupha\latupmim}{None}{kok:ze_ha_mim}
\kipkokentry{izdivaç}{\latupze\latupvav\latupcim}{None}{kok:ze_vav_cim}
\kipkokentry{izhar}{\latupza\latuphe\latupre}{None}{kok:za_he_re}
\kipkokentry{izin}{\latupalif\latupzel\latupnun}{None}{kok:alif_zel_nun}
\kipkokentry{izmihlal}{\latupdad\latupha\latuplam}{None}{kok:dad_ha_lam}
\kipkokentry{izzet}{\latupayn\latupze\latupze}{None}{kok:ayn_ze_ze}
\end{multicols}
\dictchapter{K}
\begin{multicols}{2}
\kipkokentry{kabahat}{\latupkaf\latupbe\latupha}{None}{kok:kaf_be_ha}
\kipkokentry{kabala}{\latupkaf\latupbe\latuplam}{None}{kok:kaf_be_lam}
\kipkokentry{kâbe}{\latupkef\latupayn\latupbe}{None}{kok:kef_ayn_be}
\kipkokentry{kabız}{\latupkaf\latupbe\latupdad}{None}{kok:kaf_be_dad}
\kipkokentry{kabil}{\latupkaf\latupbe\latuplam}{None}{kok:kaf_be_lam}
\kipkokentry{kâbil}{\latupkaf\latupbe\latuplam}{K\rom{1}, Ed.}{kok:kaf_be_lam}
\kipkokentry{kabîle}{\latupkaf\latupbe\latuplam}{None}{kok:kaf_be_lam}
\kipkokentry{kâbiliyet}{\latupkaf\latupbe\latuplam}{None}{kok:kaf_be_lam}
\kipkokentry{kabir}{\latupkaf\latupbe\latupre}{None}{kok:kaf_be_re}
\kipkokentry{kabul}{\latupkaf\latupbe\latuplam}{None}{kok:kaf_be_lam}
\kipkokentry{kabza}{\latupkaf\latupbe\latupdad}{None}{kok:kaf_be_dad}
\kipkokentry{kabz}{\latupkaf\latupbe\latupdad}{None}{kok:kaf_be_dad}
\kipkokentry{kadar}{\latupkaf\latupdal\latupre}{None}{kok:kaf_dal_re}
\kipkokentry{kadayıf}{\latupkaf\latupta\latupfe}{None}{kok:kaf_ta_fe}
\kipkokentry{kadeh}{\latupkaf\latupdal\latupha}{None}{kok:kaf_dal_ha}
\kipkokentry{kadem}{\latupkaf\latupdal\latupmim}{None}{kok:kaf_dal_mim}
\kipkokentry{kademe}{\latupkaf\latupdal\latupmim}{None}{kok:kaf_dal_mim}
\kipkokentry{kader}{\latupkaf\latupdal\latupre}{None}{kok:kaf_dal_re}
\kipkokentry{kadı}{\latupkaf\latupdad\latupye}{K\rom{1}, Ed.}{kok:kaf_dad_ye}
\kipkokentry{kadife}{\latupkaf\latupta\latupfe}{None}{kok:kaf_ta_fe}
\kipkokentry{kadim}{\latupkaf\latupdal\latupmim}{None}{kok:kaf_dal_mim}
\kipkokentry{kadir}{\latupkaf\latupdal\latupre}{K\rom{1}, Ed.}{kok:kaf_dal_re}
\kipkokentry{kadir}{\latupkaf\latupdal\latupre}{None}{kok:kaf_dal_re}
\kipkokentry{kadit}{\latupkaf\latupdal\latupdal}{None}{kok:kaf_dal_dal}
\kipkokentry{kafa}{\latupkaf\latupfe\latupvav$^2$}{None}{kok:kaf_fe_vav2}
\kipkokentry{kafes}{\latupkaf\latupfe\latupsad}{None}{kok:kaf_fe_sad}
\kipkokentry{kaffe}{\latupkef\latupfe\latupfe}{None}{kok:kef_fe_fe}
\kipkokentry{kâfî}{\latupkef\latupfe\latupvav}{K\rom{1}, Ed.}{kok:kef_fe_vav}
\kipkokentry{kâfile}{\latupkaf\latupfe\latuplam}{None}{kok:kaf_fe_lam}
\kipkokentry{kâfir}{\latupkef\latupfe\latupre}{K\rom{1}, Ed.}{kok:kef_fe_re}
\kipkokentry{kâfiye}{\latupkaf\latupfe\latupvav$^1$}{None}{kok:kaf_fe_vav1}
\kipkokentry{kahhar}{\latupkaf\latuphe\latupre}{None}{kok:kaf_he_re}
\kipkokentry{kahır}{\latupkaf\latuphe\latupre}{None}{kok:kaf_he_re}
\kipkokentry{kâhin}{\latupkef\latuphe\latupnun}{K\rom{1}, Ed.}{kok:kef_he_nun}
\kipkokentry{kâhir}{\latupkaf\latuphe\latupre}{K\rom{1}, Ed.}{kok:kaf_he_re}
\kipkokentry{kahkaha}{\latupkaf\latuphe}{None}{kok:kaf_he}
\kipkokentry{kahpe}{\latupkaf\latupha\latupbe}{None}{kok:kaf_ha_be}
\kipkokentry{kaht}{\latupkaf\latupha\latupte}{None}{kok:kaf_ha_te}
\kipkokentry{kahve}{\latupkaf\latuphe\latupvav}{None}{kok:kaf_he_vav}
\kipkokentry{kâide}{\latupkaf\latupayn\latupdal}{None}{kok:kaf_ayn_dal}
\kipkokentry{kâim}{\latupkaf\latupvav\latupmim}{K\rom{1}, Ed.}{kok:kaf_vav_mim}
\kipkokentry{kâinat}{\latupkef\latupvav\latupnun}{None}{kok:kef_vav_nun}
\kipkokentry{kale}{\latupkaf\latuplam\latupayn$^1$}{None}{kok:kaf_lam_ayn1}
\kipkokentry{kalem}{\latupkaf\latuplam\latupmim}{None}{kok:kaf_lam_mim}
\kipkokentry{kalfa}{\latupxa\latuplam\latupfe}{None}{kok:xa_lam_fe}
\kipkokentry{kalp}{\latupkaf\latuplam\latupbe$^1$}{None}{kok:kaf_lam_be1}
\kipkokentry{kalp}{\latupkaf\latuplam\latupbe$^2$}{None}{kok:kaf_lam_be2}
\kipkokentry{kalye}{\latupkaf\latuplam\latupvav}{None}{kok:kaf_lam_vav}
\kipkokentry{kamer}{\latupkaf\latupmim\latupre$^1$}{None}{kok:kaf_mim_re1}
\kipkokentry{kamet}{\latupkaf\latupvav\latupmim}{None}{kok:kaf_vav_mim}
\kipkokentry{kâmil}{\latupkef\latupmim\latuplam}{K\rom{1}, Ed.}{kok:kef_mim_lam}
\kipkokentry{kanaat}{\latupkaf\latupnun\latupayn}{None}{kok:kaf_nun_ayn}
\kipkokentry{kâni}{\latupkaf\latupnun\latupayn}{K\rom{1}, Ed.}{kok:kaf_nun_ayn}
\kipkokentry{karâbet}{\latupkaf\latupre\latupbe$^1$}{None}{kok:kaf_re_be1}
\kipkokentry{karar}{\latupkaf\latupre\latupre}{None}{kok:kaf_re_re}
\kipkokentry{kâri}{\latupkaf\latupre\latupalif}{K\rom{1}, Ed.}{kok:kaf_re_alif}
\kipkokentry{karin}{\latupkaf\latupre\latupnun}{None}{kok:kaf_re_nun}
\kipkokentry{karîne}{\latupkaf\latupre\latupnun}{None}{kok:kaf_re_nun}
\kipkokentry{karye}{\latupkaf\latupre\latupye}{None}{kok:kaf_re_ye}
\kipkokentry{karz}{\latupkaf\latupre\latupdad}{None}{kok:kaf_re_dad}
\kipkokentry{kasaba}{\latupkaf\latupsad\latupbe}{None}{kok:kaf_sad_be}
\kipkokentry{kasap}{\latupkaf\latupsad\latupbe}{None}{kok:kaf_sad_be}
\kipkokentry{kasâvet}{\latupkaf\latupsin\latupvav}{None}{kok:kaf_sin_vav}
\kipkokentry{kasem}{\latupkaf\latupsin\latupmim}{None}{kok:kaf_sin_mim}
\kipkokentry{kâsım}{\latupkaf\latupsin\latupmim}{K\rom{1}, Ed.}{kok:kaf_sin_mim}
\kipkokentry{kasır}{\latupkaf\latupsin\latupre}{None}{kok:kaf_sin_re}
\kipkokentry{kasıt}{\latupkaf\latupsad\latupdal}{None}{kok:kaf_sad_dal}
\kipkokentry{kasîde}{\latupkaf\latupsad\latupdal}{None}{kok:kaf_sad_dal}
\kipkokentry{kasvet}{\latupkaf\latupsin\latupvav}{None}{kok:kaf_sin_vav}
\kipkokentry{kâşif}{\latupkef\latupshin\latupfe}{K\rom{1}, Ed.}{kok:kef_shin_fe}
\kipkokentry{kat}{\latupkaf\latupta\latupayn}{None}{kok:kaf_ta_ayn}
\kipkokentry{katar}{\latupkaf\latupta\latupre}{None}{kok:kaf_ta_re}
\kipkokentry{katî}{\latupkaf\latupta\latupayn}{None}{kok:kaf_ta_ayn}
\kipkokentry{kâtil}{\latupkaf\latupte\latuplam}{K\rom{1}, Ed.}{kok:kaf_te_lam}
\kipkokentry{katil}{\latupkaf\latupte\latuplam}{None}{kok:kaf_te_lam}
\kipkokentry{kâtip}{\latupkef\latupte\latupbe}{K\rom{1}, Ed.}{kok:kef_te_be}
\kipkokentry{katran}{\latupkaf\latupta\latupre}{None}{kok:kaf_ta_re}
\kipkokentry{katre}{\latupkaf\latupta\latupre}{None}{kok:kaf_ta_re}
\kipkokentry{kavaf}{\latupxa\latupfe\latupfe}{None}{kok:xa_fe_fe}
\kipkokentry{kaval}{\latupkaf\latupvav\latuplam}{None}{kok:kaf_vav_lam}
\kipkokentry{kavas}{\latupkaf\latupvav\latupsin}{None}{kok:kaf_vav_sin}
\kipkokentry{kavî}{\latupkaf\latupvav\latupye}{None}{kok:kaf_vav_ye}
\kipkokentry{kavil}{\latupkaf\latupvav\latuplam}{None}{kok:kaf_vav_lam}
\kipkokentry{kavim}{\latupkaf\latupvav\latupmim}{None}{kok:kaf_vav_mim}
\kipkokentry{kavis}{\latupkaf\latupvav\latupsin}{None}{kok:kaf_vav_sin}
\kipkokentry{kayıp}{\latupgayn\latupye\latupbe}{K\rom{1}, Ed.}{kok:gayn_ye_be}
\kipkokentry{kayıp}{\latupgayn\latupye\latupbe}{None}{kok:gayn_ye_be}
\kipkokentry{kayısı}{\latupkaf\latupye\latupsin}{None}{kok:kaf_ye_sin}
\kipkokentry{kayıt}{\latupkaf\latupye\latupdal}{None}{kok:kaf_ye_dal}
\kipkokentry{kâyme}{\latupkaf\latupvav\latupmim}{None}{kok:kaf_vav_mim}
\kipkokentry{kayyım}{\latupkaf\latupvav\latupmim}{None}{kok:kaf_vav_mim}
\kipkokentry{kayyum}{\latupkaf\latupvav\latupmim}{None}{kok:kaf_vav_mim}
\kipkokentry{kazâ}{\latupkaf\latupdad\latupye}{None}{kok:kaf_dad_ye}
\kipkokentry{kazaz}{\latupkaf\latupze\latupze}{None}{kok:kaf_ze_ze}
\kipkokentry{kâzip}{\latupkef\latupzel\latupbe}{K\rom{1}, Ed.}{kok:kef_zel_be}
\kipkokentry{kaziye}{\latupkaf\latupdad\latupye}{None}{kok:kaf_dad_ye}
\kipkokentry{kazûrat}{\latupkaf\latupzel\latupre}{None}{kok:kaf_zel_re}
\kipkokentry{kebap}{\latupkef\latupbe\latupbe}{None}{kok:kef_be_be}
\kipkokentry{kebir}{\latupkef\latupbe\latupre}{None}{kok:kef_be_re}
\kipkokentry{keder}{\latupkef\latupdal\latupre}{None}{kok:kef_dal_re}
\kipkokentry{kefâlet}{\latupkef\latupfe\latuplam}{None}{kok:kef_fe_lam}
\kipkokentry{kefâret}{\latupkef\latupfe\latupre}{None}{kok:kef_fe_re}
\kipkokentry{kefe}{\latupkef\latupfe\latupfe}{None}{kok:kef_fe_fe}
\kipkokentry{kefen}{\latupkef\latupfe\latupnun}{None}{kok:kef_fe_nun}
\kipkokentry{kefere}{\latupkef\latupfe\latupre}{None}{kok:kef_fe_re}
\kipkokentry{kefil}{\latupkef\latupfe\latuplam}{None}{kok:kef_fe_lam}
\kipkokentry{kehânet}{\latupkef\latuphe\latupnun}{None}{kok:kef_he_nun}
\kipkokentry{kelam}{\latupkef\latuplam\latupmim}{None}{kok:kef_lam_mim}
\kipkokentry{kelime}{\latupkef\latuplam\latupmim}{None}{kok:kef_lam_mim}
\kipkokentry{kemal}{\latupkef\latupmim\latuplam}{None}{kok:kef_mim_lam}
\kipkokentry{kemiyet}{\latupkef\latupmim\latupmim}{None}{kok:kef_mim_mim}
\kipkokentry{kenef}{\latupkef\latupnun\latupfe}{None}{kok:kef_nun_fe}
\kipkokentry{kenet}{\latupkef\latupnun\latupdal}{None}{kok:kef_nun_dal}
\kipkokentry{kerahat}{\latupkef\latupre\latuphe}{None}{kok:kef_re_he}
\kipkokentry{kerâmet}{\latupkef\latupre\latupmim}{None}{kok:kef_re_mim}
\kipkokentry{kere}{\latupkef\latupre\latupre}{None}{kok:kef_re_re}
\kipkokentry{kerem}{\latupkef\latupre\latupmim}{None}{kok:kef_re_mim}
\kipkokentry{kerh}{\latupkef\latupre\latuphe}{None}{kok:kef_re_he}
\kipkokentry{kerim}{\latupkef\latupre\latupmim}{None}{kok:kef_re_mim}
\kipkokentry{kerime}{\latupkef\latupre\latupmim}{None}{kok:kef_re_mim}
\kipkokentry{kerrat}{\latupkef\latupre\latupre}{None}{kok:kef_re_re}
\kipkokentry{kesâfet}{\latupkef\latupthe\latupfe}{None}{kok:kef_the_fe}
\kipkokentry{kesat}{\latupkef\latupsin\latupdal}{None}{kok:kef_sin_dal}
\kipkokentry{kesif}{\latupkef\latupthe\latupfe}{None}{kok:kef_the_fe}
\kipkokentry{kesir}{\latupkef\latupsin\latupre}{None}{kok:kef_sin_re}
\kipkokentry{kesp}{\latupkef\latupsin\latupbe}{None}{kok:kef_sin_be}
\kipkokentry{kesret}{\latupkef\latupthe\latupre}{None}{kok:kef_the_re}
\kipkokentry{keşif}{\latupkef\latupshin\latupfe}{None}{kok:kef_shin_fe}
\kipkokentry{keşif}{\latupkef\latupshin\latupshin}{None}{kok:kef_shin_shin}
\kipkokentry{keşşaf}{\latupkef\latupshin\latupfe}{None}{kok:kef_shin_fe}
\kipkokentry{ketebe}{\latupkef\latupte\latupbe}{None}{kok:kef_te_be}
\kipkokentry{ketum}{\latupkef\latupte\latupmim}{None}{kok:kef_te_mim}
\kipkokentry{kevâşe}{\latupkaf\latupvav\latupdal}{None}{kok:kaf_vav_dal}
\kipkokentry{kevser}{\latupkef\latupthe\latupre}{None}{kok:kef_the_re}
\kipkokentry{keyfiyet}{\latupkef\latupye\latupfe}{None}{kok:kef_ye_fe}
\kipkokentry{keyif}{\latupkef\latupye\latupfe}{None}{kok:kef_ye_fe}
\kipkokentry{kıbbe}{\latupkaf\latupbe\latupbe}{None}{kok:kaf_be_be}
\kipkokentry{kıble}{\latupkaf\latupbe\latuplam}{None}{kok:kaf_be_lam}
\kipkokentry{kıdem}{\latupkaf\latupdal\latupmim}{None}{kok:kaf_dal_mim}
\kipkokentry{kılıf}{\latupgayn\latuplam\latupfe}{None}{kok:gayn_lam_fe}
\kipkokentry{kımıl}{\latupkaf\latupmim\latuplam}{None}{kok:kaf_mim_lam}
\kipkokentry{kıraat}{\latupkaf\latupre\latupalif}{None}{kok:kaf_re_alif}
\kipkokentry{kırba}{\latupkaf\latupre\latupbe$^1$}{None}{kok:kaf_re_be1}
\kipkokentry{kısas}{\latupkaf\latupsad\latupsad}{None}{kok:kaf_sad_sad}
\kipkokentry{kısâs}{\latupkaf\latupsad\latupsad}{None}{kok:kaf_sad_sad}
\kipkokentry{kısım}{\latupkaf\latupsin\latupmim}{None}{kok:kaf_sin_mim}
\kipkokentry{kısmet}{\latupkaf\latupsin\latupmim}{None}{kok:kaf_sin_mim}
\kipkokentry{kıssa}{\latupkaf\latupsad\latupsad}{None}{kok:kaf_sad_sad}
\kipkokentry{kışır}{\latupkaf\latupshin\latupre}{None}{kok:kaf_shin_re}
\kipkokentry{kıta}{\latupkaf\latupta\latupayn}{None}{kok:kaf_ta_ayn}
\kipkokentry{kıtal}{\latupkaf\latupte\latuplam}{None}{kok:kaf_te_lam}
\kipkokentry{kıvam}{\latupkaf\latupvav\latupmim}{None}{kok:kaf_vav_mim}
\kipkokentry{kıyâfet}{\latupkaf\latupye\latupfe}{None}{kok:kaf_ye_fe}
\kipkokentry{kıyam}{\latupkaf\latupvav\latupmim}{None}{kok:kaf_vav_mim}
\kipkokentry{kıyâmet}{\latupkaf\latupvav\latupmim}{None}{kok:kaf_vav_mim}
\kipkokentry{kıyas}{\latupkaf\latupye\latupsin}{None}{kok:kaf_ye_sin}
\kipkokentry{kıymet}{\latupkaf\latupvav\latupmim}{None}{kok:kaf_vav_mim}
\kipkokentry{kibar}{\latupkef\latupbe\latupre}{None}{kok:kef_be_re}
\kipkokentry{kibir}{\latupkef\latupbe\latupre}{None}{kok:kef_be_re}
\kipkokentry{kifâyet}{\latupkef\latupfe\latupvav}{None}{kok:kef_fe_vav}
\kipkokentry{kinâye}{\latupkef\latupnun\latupvav}{None}{kok:kef_nun_vav}
\kipkokentry{kirâ}{\latupkef\latupre\latupye}{None}{kok:kef_re_ye}
\kipkokentry{kirâm}{\latupkef\latupre\latupmim}{None}{kok:kef_re_mim}
\kipkokentry{kispet}{\latupkef\latupsin\latupvav}{None}{kok:kef_sin_vav}
\kipkokentry{kisve}{\latupkef\latupsin\latupvav}{None}{kok:kef_sin_vav}
\kipkokentry{kitâbe}{\latupkef\latupte\latupbe}{None}{kok:kef_te_be}
\kipkokentry{kitâbet}{\latupkef\latupte\latupbe}{None}{kok:kef_te_be}
\kipkokentry{kitâbiyat}{\latupkef\latupte\latupbe}{None}{kok:kef_te_be}
\kipkokentry{kitap}{\latupkef\latupte\latupbe}{None}{kok:kef_te_be}
\kipkokentry{kitle}{\latupkef\latupte\latuplam}{None}{kok:kef_te_lam}
\kipkokentry{köle}{\latupgayn\latuplam\latupmim}{None}{kok:gayn_lam_mim}
\kipkokentry{kubbe}{\latupkaf\latupbe\latupbe}{None}{kok:kaf_be_be}
\kipkokentry{kubur}{\latupkaf\latupbe\latupre}{None}{kok:kaf_be_re}
\kipkokentry{kudret}{\latupkaf\latupdal\latupre}{None}{kok:kaf_dal_re}
\kipkokentry{kudüm}{\latupkaf\latupdal\latupmim}{None}{kok:kaf_dal_mim}
\kipkokentry{kudüs}{\latupkaf\latupdal\latupsin}{None}{kok:kaf_dal_sin}
\kipkokentry{kulp}{\latupkaf\latuplam\latupbe$^1$}{None}{kok:kaf_lam_be1}
\kipkokentry{kumar}{\latupkaf\latupmim\latupre$^2$}{None}{kok:kaf_mim_re2}
\kipkokentry{kumaş}{\latupkaf\latupmim\latupshin}{None}{kok:kaf_mim_shin}
\kipkokentry{kumkuma}{\latupkaf\latupmim}{None}{kok:kaf_mim}
\kipkokentry{kumru}{\latupkaf\latupmim\latupre$^1$}{None}{kok:kaf_mim_re1}
\kipkokentry{kurâ}{\latupkaf\latupre\latupye}{None}{kok:kaf_re_ye}
\kipkokentry{kurâ}{\latupkaf\latupre\latupayn}{None}{kok:kaf_re_ayn}
\kipkokentry{kuran}{\latupkaf\latupre\latupalif}{None}{kok:kaf_re_alif}
\kipkokentry{kurb}{\latupkaf\latupre\latupbe$^1$}{None}{kok:kaf_re_be1}
\kipkokentry{kurban}{\latupkaf\latupre\latupbe$^2$}{None}{kok:kaf_re_be2}
\kipkokentry{kurs}{\latupkaf\latupre\latupsad}{None}{kok:kaf_re_sad}
\kipkokentry{kurun}{\latupkaf\latupre\latupnun}{None}{kok:kaf_re_nun}
\kipkokentry{kuskus}{\latupkaf\latupsad}{None}{kok:kaf_sad}
\kipkokentry{kusur}{\latupkaf\latupsad\latupre}{None}{kok:kaf_sad_re}
\kipkokentry{kutsî}{\latupkaf\latupdal\latupsin}{None}{kok:kaf_dal_sin}
\kipkokentry{kutup}{\latupkaf\latupta\latupbe}{None}{kok:kaf_ta_be}
\kipkokentry{kutur}{\latupkaf\latupta\latupre}{None}{kok:kaf_ta_re}
\kipkokentry{kuvâ}{\latupkaf\latupvav\latupye}{None}{kok:kaf_vav_ye}
\kipkokentry{kuvve}{\latupkaf\latupvav\latupye}{None}{kok:kaf_vav_ye}
\kipkokentry{kuvvet}{\latupkaf\latupvav\latupye}{None}{kok:kaf_vav_ye}
\kipkokentry{kuyut}{\latupkaf\latupye\latupdal}{None}{kok:kaf_ye_dal}
\kipkokentry{kübrâ}{\latupkef\latupbe\latupre}{None}{kok:kef_be_re}
\kipkokentry{küffar}{\latupkef\latupfe\latupre}{None}{kok:kef_fe_re}
\kipkokentry{küfran}{\latupkef\latupfe\latupre}{None}{kok:kef_fe_re}
\kipkokentry{küfür}{\latupkef\latupfe\latupre}{None}{kok:kef_fe_re}
\kipkokentry{külfet}{\latupkef\latuplam\latupfe}{None}{kok:kef_lam_fe}
\kipkokentry{künefe}{\latupkef\latupnun\latupfe}{None}{kok:kef_nun_fe}
\kipkokentry{künh}{\latupkef\latupnun\latuphe}{None}{kok:kef_nun_he}
\kipkokentry{künye}{\latupkef\latupnun\latupvav}{None}{kok:kef_nun_vav}
\kipkokentry{küre}{\latupkef\latupre\latupvav}{None}{kok:kef_re_vav}
\kipkokentry{kürsî}{\latupkef\latupre\latupsin}{None}{kok:kef_re_sin}
\kipkokentry{kürsü}{\latupkef\latupre\latupsin}{None}{kok:kef_re_sin}
\kipkokentry{küsuf}{\latupkef\latupsin\latupfe}{None}{kok:kef_sin_fe}
\kipkokentry{küsur}{\latupkef\latupsin\latupre}{None}{kok:kef_sin_re}
\kipkokentry{kütle}{\latupkef\latupte\latuplam}{None}{kok:kef_te_lam}
\kipkokentry{küttab}{\latupkef\latupte\latupbe}{None}{kok:kef_te_be}
\kipkokentry{kütüphane}{\latupkef\latupte\latupbe}{None}{kok:kef_te_be}
\end{multicols}
\dictchapter{L}
\begin{multicols}{2}
\kipkokentry{labne}{\latuplam\latupbe\latupnun}{None}{kok:lam_be_nun}
\kipkokentry{lafız}{\latuplam\latupfe\latupza}{None}{kok:lam_fe_za}
\kipkokentry{lağv}{\latuplam\latupgayn\latupvav}{None}{kok:lam_gayn_vav}
\kipkokentry{lâhika}{\latuplam\latupha\latupkaf}{None}{kok:lam_ha_kaf}
\kipkokentry{lahit}{\latuplam\latupha\latupdal}{None}{kok:lam_ha_dal}
\kipkokentry{lahza}{\latuplam\latupha\latupza}{None}{kok:lam_ha_za}
\kipkokentry{lain}{\latuplam\latupayn\latupnun}{None}{kok:lam_ayn_nun}
\kipkokentry{lâkap}{\latuplam\latupkaf\latupbe}{None}{kok:lam_kaf_be}
\kipkokentry{lânet}{\latuplam\latupayn\latupnun}{None}{kok:lam_ayn_nun}
\kipkokentry{latif}{\latuplam\latupta\latupfe}{None}{kok:lam_ta_fe}
\kipkokentry{latife}{\latuplam\latupta\latupfe}{None}{kok:lam_ta_fe}
\kipkokentry{lâyık}{\latuplam\latupye\latupkaf}{K\rom{1}, Ed.}{kok:lam_ye_kaf}
\kipkokentry{lâyiha}{\latuplam\latupvav\latupha}{None}{kok:lam_vav_ha}
\kipkokentry{lâzım}{\latuplam\latupze\latupmim}{K\rom{1}, Ed.}{kok:lam_ze_mim}
\kipkokentry{leblebi}{\latuplam\latupbe}{None}{kok:lam_be}
\kipkokentry{lef}{\latuplam\latupfe\latupfe}{None}{kok:lam_fe_fe}
\kipkokentry{lehçe}{\latuplam\latuphe\latupcim}{None}{kok:lam_he_cim}
\kipkokentry{lehim}{\latuplam\latupha\latupmim}{None}{kok:lam_ha_mim}
\kipkokentry{letâfet}{\latuplam\latupta\latupfe}{None}{kok:lam_ta_fe}
\kipkokentry{levâzım}{\latuplam\latupze\latupmim}{None}{kok:lam_ze_mim}
\kipkokentry{levha}{\latuplam\latupvav\latupha}{None}{kok:lam_vav_ha}
\kipkokentry{levin}{\latuplam\latupvav\latupye}{K\rom{1}, Ed.}{kok:lam_vav_ye}
\kipkokentry{leylek}{\latuplam\latupkaf\latuplam\latupkaf}{None}{kok:lam_kaf_lam_kaf}
\kipkokentry{leylî}{\latuplam\latupye\latuplam}{None}{kok:lam_ye_lam}
\kipkokentry{leziz}{\latuplam\latupzel\latupzel}{None}{kok:lam_zel_zel}
\kipkokentry{lezzet}{\latuplam\latupzel\latupzel}{None}{kok:lam_zel_zel}
\kipkokentry{libas}{\latuplam\latupbe\latupsin}{None}{kok:lam_be_sin}
\kipkokentry{lif}{\latuplam\latupye\latupfe}{None}{kok:lam_ye_fe}
\kipkokentry{lisan}{\latuplam\latupsin\latupnun}{None}{kok:lam_sin_nun}
\kipkokentry{livâ}{\latuplam\latupvav\latupye}{None}{kok:lam_vav_ye}
\kipkokentry{livâta}{\latuplam\latupvav\latupta}{None}{kok:lam_vav_ta}
\kipkokentry{liyâkat}{\latuplam\latupye\latupkaf}{None}{kok:lam_ye_kaf}
\kipkokentry{lobut}{\latupnun\latupbe\latupta}{None}{kok:nun_be_ta}
\kipkokentry{lokma}{\latuplam\latupkaf\latupmim}{None}{kok:lam_kaf_mim}
\kipkokentry{lugat}{\latuplam\latupgayn\latupvav}{None}{kok:lam_gayn_vav}
\kipkokentry{lütfen}{\latuplam\latupta\latupfe}{None}{kok:lam_ta_fe}
\kipkokentry{lütuf}{\latuplam\latupta\latupfe}{None}{kok:lam_ta_fe}
\kipkokentry{lüzûcet}{\latuplam\latupze\latupcim}{None}{kok:lam_ze_cim}
\kipkokentry{lüzum}{\latuplam\latupze\latupmim}{None}{kok:lam_ze_mim}
\end{multicols}
\dictchapter{M}
\begin{multicols}{2}
\kipkokentry{maadâ}{\latupayn\latupdal\latupvav}{None}{kok:ayn_dal_vav}
\kipkokentry{maarif}{\latupayn\latupre\latupfe}{None}{kok:ayn_re_fe}
\kipkokentry{maaş}{\latupayn\latupye\latupshin}{None}{kok:ayn_ye_shin}
\kipkokentry{maaz}{\latupayn\latupvav\latupzel}{None}{kok:ayn_vav_zel}
\kipkokentry{mâbet}{\latupayn\latupbe\latupdal}{None}{kok:ayn_be_dal}
\kipkokentry{mâbut}{\latupayn\latupbe\latupdal}{K\rom{1}, Edl.}{kok:ayn_be_dal}
\kipkokentry{mâcerâ}{\latupcim\latupre\latupye}{None}{kok:cim_re_ye}
\kipkokentry{mâcit}{\latupmim\latupcim\latupdal}{K\rom{1}, Ed.}{kok:mim_cim_dal}
\kipkokentry{mâcun}{\latupayn\latupcim\latupnun}{K\rom{1}, Edl.}{kok:ayn_cim_nun}
\kipkokentry{madde}{\latupmim\latupdal\latupdal$^2$}{None}{kok:mim_dal_dal2}
\kipkokentry{mâdem}{\latupdal\latupvav\latupmim}{None}{kok:dal_vav_mim}
\kipkokentry{mâden}{\latupayn\latupdal\latupnun}{None}{kok:ayn_dal_nun}
\kipkokentry{mafsal}{\latupfe\latupsad\latuplam}{None}{kok:fe_sad_lam}
\kipkokentry{mağara}{\latupgayn\latupvav\latupre}{None}{kok:gayn_vav_re}
\kipkokentry{mağdur}{\latupgayn\latupdal\latupre}{K\rom{1}, Edl.}{kok:gayn_dal_re}
\kipkokentry{mağfiret}{\latupgayn\latupfe\latupre}{None}{kok:gayn_fe_re}
\kipkokentry{mağlup}{\latupgayn\latuplam\latupbe}{K\rom{1}, Edl.}{kok:gayn_lam_be}
\kipkokentry{mağrip}{\latupgayn\latupre\latupbe}{None}{kok:gayn_re_be}
\kipkokentry{mağrur}{\latupgayn\latupre\latupre}{K\rom{1}, Edl.}{kok:gayn_re_re}
\kipkokentry{mağşuş}{\latupgayn\latupshin\latupshin}{K\rom{1}, Edl.}{kok:gayn_shin_shin}
\kipkokentry{mahalle}{\latupha\latuplam\latuplam}{None}{kok:ha_lam_lam}
\kipkokentry{mahâret}{\latupmim\latuphe\latupre}{None}{kok:mim_he_re}
\kipkokentry{mahbup}{\latupha\latupbe\latupbe$^2$}{K\rom{1}, Edl.}{kok:ha_be_be2}
\kipkokentry{mahcur}{\latupha\latupcim\latupre$^1$}{K\rom{1}, Edl.}{kok:ha_cim_re1}
\kipkokentry{mahcuz}{\latupha\latupcim\latupze}{K\rom{1}, Edl.}{kok:ha_cim_ze}
\kipkokentry{mahçup}{\latupha\latupcim\latupbe}{K\rom{1}, Edl.}{kok:ha_cim_be}
\kipkokentry{mahdum}{\latupxa\latupdal\latupmim}{K\rom{1}, Edl.}{kok:xa_dal_mim}
\kipkokentry{mahdut}{\latupha\latupdal\latupdal}{K\rom{1}, Edl.}{kok:ha_dal_dal}
\kipkokentry{mahfaza}{\latupha\latupfe\latupza}{None}{kok:ha_fe_za}
\kipkokentry{mahfî}{\latupxa\latupfe\latupye}{K\rom{1}, Edl.}{kok:xa_fe_ye}
\kipkokentry{mahfil}{\latupha\latupfe\latuplam}{None}{kok:ha_fe_lam}
\kipkokentry{mahfuz}{\latupha\latupfe\latupza}{K\rom{1}, Edl.}{kok:ha_fe_za}
\kipkokentry{mâhir}{\latupmim\latuphe\latupre}{K\rom{1}, Ed.}{kok:mim_he_re}
\kipkokentry{mahkeme}{\latupha\latupkef\latupmim}{None}{kok:ha_kef_mim}
\kipkokentry{mahkum}{\latupha\latupkef\latupmim}{K\rom{1}, Edl.}{kok:ha_kef_mim}
\kipkokentry{mahlas}{\latupxa\latuplam\latupsad}{None}{kok:xa_lam_sad}
\kipkokentry{mahluk}{\latupxa\latuplam\latupkaf}{K\rom{1}, Edl.}{kok:xa_lam_kaf}
\kipkokentry{mahlukat}{\latupxa\latuplam\latupkaf}{None}{kok:xa_lam_kaf}
\kipkokentry{mahlul}{\latupha\latuplam\latuplam}{K\rom{1}, Edl.}{kok:ha_lam_lam}
\kipkokentry{mahmur}{\latupxa\latupmim\latupre}{K\rom{1}, Edl.}{kok:xa_mim_re}
\kipkokentry{mahmut}{\latupha\latupmim\latupdal}{K\rom{1}, Edl.}{kok:ha_mim_dal}
\kipkokentry{mahmuz}{\latuphe\latupmim\latupze}{None}{kok:he_mim_ze}
\kipkokentry{mahpes}{\latupha\latupbe\latupsin}{None}{kok:ha_be_sin}
\kipkokentry{mahpus}{\latupha\latupbe\latupsin}{K\rom{1}, Edl.}{kok:ha_be_sin}
\kipkokentry{mahreç}{\latupxa\latupre\latupcim}{None}{kok:xa_re_cim}
\kipkokentry{mahrem}{\latupha\latupre\latupmim}{None}{kok:ha_re_mim}
\kipkokentry{mahrukat}{\latupha\latupre\latupkaf}{None}{kok:ha_re_kaf}
\kipkokentry{mahrum}{\latupha\latupre\latupmim}{K\rom{1}, Edl.}{kok:ha_re_mim}
\kipkokentry{mahrut}{\latupxa\latupre\latupta}{K\rom{1}, Edl.}{kok:xa_re_ta}
\kipkokentry{mahsul}{\latupha\latupsad\latuplam}{K\rom{1}, Edl.}{kok:ha_sad_lam}
\kipkokentry{mahsup}{\latupha\latupsin\latupbe}{K\rom{1}, Edl.}{kok:ha_sin_be}
\kipkokentry{mahsur}{\latupha\latupsad\latupre}{K\rom{1}, Edl.}{kok:ha_sad_re}
\kipkokentry{mahsus}{\latupxa\latupsad\latupsad}{K\rom{1}, Edl.}{kok:xa_sad_sad}
\kipkokentry{mahşer}{\latupha\latupshin\latupre}{None}{kok:ha_shin_re}
\kipkokentry{mâhut}{\latupayn\latuphe\latupdal}{K\rom{1}, Edl.}{kok:ayn_he_dal}
\kipkokentry{mahv}{\latupmim\latupha\latupvav}{None}{kok:mim_ha_vav}
\kipkokentry{mahzen}{\latupxa\latupze\latupnun}{None}{kok:xa_ze_nun}
\kipkokentry{mahzun}{\latupha\latupze\latupnun}{K\rom{1}, Edl.}{kok:ha_ze_nun}
\kipkokentry{mahzur}{\latupha\latupzel\latupre}{K\rom{1}, Edl.}{kok:ha_zel_re}
\kipkokentry{mâil}{\latupmim\latupye\latuplam}{K\rom{1}, Ed.}{kok:mim_ye_lam}
\kipkokentry{maişet}{\latupayn\latupye\latupshin}{None}{kok:ayn_ye_shin}
\kipkokentry{makâle}{\latupkaf\latupvav\latuplam}{None}{kok:kaf_vav_lam}
\kipkokentry{makam}{\latupkaf\latupvav\latupmim}{None}{kok:kaf_vav_mim}
\kipkokentry{makara}{\latupbe\latupkef\latupre}{None}{kok:be_kef_re}
\kipkokentry{makas}{\latupkaf\latupsad\latupsad}{None}{kok:kaf_sad_sad}
\kipkokentry{makat}{\latupkaf\latupayn\latupdal}{None}{kok:kaf_ayn_dal}
\kipkokentry{makber}{\latupkaf\latupbe\latupre}{None}{kok:kaf_be_re}
\kipkokentry{makbul}{\latupkaf\latupbe\latuplam}{K\rom{1}, Edl.}{kok:kaf_be_lam}
\kipkokentry{makbuz}{\latupkaf\latupbe\latupdad}{K\rom{1}, Edl.}{kok:kaf_be_dad}
\kipkokentry{mâkes}{\latupayn\latupkef\latupsin}{None}{kok:ayn_kef_sin}
\kipkokentry{maklûbe}{\latupkaf\latuplam\latupbe$^1$}{None}{kok:kaf_lam_be1}
\kipkokentry{maksat}{\latupkaf\latupsad\latupdal}{None}{kok:kaf_sad_dal}
\kipkokentry{maksem}{\latupkaf\latupsin\latupmim}{None}{kok:kaf_sin_mim}
\kipkokentry{maksut}{\latupkaf\latupsad\latupdal}{K\rom{1}, Edl.}{kok:kaf_sad_dal}
\kipkokentry{maktû}{\latupkaf\latupta\latupayn}{K\rom{1}, Edl.}{kok:kaf_ta_ayn}
\kipkokentry{maktul}{\latupkaf\latupte\latuplam}{K\rom{1}, Edl.}{kok:kaf_te_lam}
\kipkokentry{mâkul}{\latupayn\latupkaf\latuplam}{K\rom{1}, Edl.}{kok:ayn_kaf_lam}
\kipkokentry{makûle}{\latupkaf\latupvav\latuplam}{None}{kok:kaf_vav_lam}
\kipkokentry{mal}{\latupmim\latupvav\latuplam}{None}{kok:mim_vav_lam}
\kipkokentry{mâlî}{\latupmim\latupvav\latuplam}{None}{kok:mim_vav_lam}
\kipkokentry{mâlik}{\latupmim\latuplam\latupkef}{K\rom{1}, Ed.}{kok:mim_lam_kef}
\kipkokentry{mâliye}{\latupmim\latupvav\latuplam}{None}{kok:mim_vav_lam}
\kipkokentry{mâlul}{\latupayn\latuplam\latuplam}{K\rom{1}, Edl.}{kok:ayn_lam_lam}
\kipkokentry{mâlum}{\latupayn\latuplam\latupmim}{K\rom{1}, Edl.}{kok:ayn_lam_mim}
\kipkokentry{malzeme}{\latuplam\latupze\latupmim}{None}{kok:lam_ze_mim}
\kipkokentry{mâmelek}{\latupmim\latuplam\latupkef}{None}{kok:mim_lam_kef}
\kipkokentry{mâmul}{\latupayn\latupmim\latuplam}{K\rom{1}, Edl.}{kok:ayn_mim_lam}
\kipkokentry{mâmur}{\latupayn\latupmim\latupre}{K\rom{1}, Edl.}{kok:ayn_mim_re}
\kipkokentry{mânâ}{\latupayn\latupnun\latupye}{K\rom{1}, Edl.}{kok:ayn_nun_ye}
\kipkokentry{mânen}{\latupayn\latupnun\latupye}{None}{kok:ayn_nun_ye}
\kipkokentry{mânevî}{\latupayn\latupnun\latupye}{None}{kok:ayn_nun_ye}
\kipkokentry{mangal}{\latupnun\latupkaf\latuplam}{None}{kok:nun_kaf_lam}
\kipkokentry{mangır}{\latupnun\latupkaf\latupre}{K\rom{1}, Edl.}{kok:nun_kaf_re}
\kipkokentry{mâniâ}{\latupmim\latupnun\latupayn}{None}{kok:mim_nun_ayn}
\kipkokentry{mânî}{\latupmim\latupnun\latupayn}{K\rom{1}, Ed.}{kok:mim_nun_ayn}
\kipkokentry{mânî}{\latupayn\latupnun\latupye}{None}{kok:ayn_nun_ye}
\kipkokentry{mansıp}{\latupnun\latupsad\latupbe}{None}{kok:nun_sad_be}
\kipkokentry{mansur}{\latupnun\latupsad\latupre$^1$}{K\rom{1}, Edl.}{kok:nun_sad_re1}
\kipkokentry{mantık}{\latupnun\latupta\latupkaf}{None}{kok:nun_ta_kaf}
\kipkokentry{manzara}{\latupnun\latupza\latupre}{None}{kok:nun_za_re}
\kipkokentry{manzum}{\latupnun\latupza\latupmim}{K\rom{1}, Edl.}{kok:nun_za_mim}
\kipkokentry{mapus}{\latupha\latupbe\latupsin}{K\rom{1}, Edl.}{kok:ha_be_sin}
\kipkokentry{maraba}{\latupre\latupbe\latupayn}{None}{kok:re_be_ayn}
\kipkokentry{maraza}{\latupayn\latupre\latupdad}{None}{kok:ayn_re_dad}
\kipkokentry{maraz}{\latupmim\latupre\latupdad}{None}{kok:mim_re_dad}
\kipkokentry{mârifet}{\latupayn\latupre\latupfe}{None}{kok:ayn_re_fe}
\kipkokentry{mariz}{\latupmim\latupre\latupdad}{None}{kok:mim_re_dad}
\kipkokentry{mars}{\latupmim\latupre\latupsin}{None}{kok:mim_re_sin}
\kipkokentry{mâruf}{\latupayn\latupre\latupfe}{K\rom{1}, Edl.}{kok:ayn_re_fe}
\kipkokentry{mâruz}{\latupayn\latupre\latupdad}{K\rom{1}, Edl.}{kok:ayn_re_dad}
\kipkokentry{masal}{\latupmim\latupthe\latuplam}{None}{kok:mim_the_lam}
\kipkokentry{masdar}{\latupsad\latupdal\latupre}{None}{kok:sad_dal_re}
\kipkokentry{maskara}{\latupsin\latupxa\latupre}{None}{kok:sin_xa_re}
\kipkokentry{maslahat}{\latupsad\latuplam\latupha}{None}{kok:sad_lam_ha}
\kipkokentry{masraf}{\latupsad\latupre\latupfe}{None}{kok:sad_re_fe}
\kipkokentry{mass}{\latupmim\latupsad\latupsad}{None}{kok:mim_sad_sad}
\kipkokentry{mastar}{\latupsin\latupta\latupre}{None}{kok:sin_ta_re}
\kipkokentry{mâsum}{\latupayn\latupsad\latupmim}{K\rom{1}, Edl.}{kok:ayn_sad_mim}
\kipkokentry{masun}{\latupsad\latupvav\latupnun}{K\rom{1}, Edl.}{kok:sad_vav_nun}
\kipkokentry{maşa}{\latupha\latupshin\latupshin}{None}{kok:ha_shin_shin}
\kipkokentry{mâşerî}{\latupayn\latupshin\latupre}{None}{kok:ayn_shin_re}
\kipkokentry{maşlah}{\latupshin\latuplam\latupha}{None}{kok:shin_lam_ha}
\kipkokentry{maşrapa}{\latupshin\latupre\latupbe}{None}{kok:shin_re_be}
\kipkokentry{maşrık}{\latupshin\latupre\latupkaf}{None}{kok:shin_re_kaf}
\kipkokentry{mâşuk}{\latupayn\latupshin\latupkaf}{K\rom{1}, Edl.}{kok:ayn_shin_kaf}
\kipkokentry{matbaa}{\latupta\latupbe\latupayn}{None}{kok:ta_be_ayn}
\kipkokentry{matbû}{\latupta\latupbe\latupayn}{K\rom{1}, Edl.}{kok:ta_be_ayn}
\kipkokentry{mâtem}{\latupalif\latupte\latupmim}{None}{kok:alif_te_mim}
\kipkokentry{matkap}{\latupthe\latupkaf\latupbe}{None}{kok:the_kaf_be}
\kipkokentry{matlâ}{\latupta\latuplam\latupayn}{None}{kok:ta_lam_ayn}
\kipkokentry{matlup}{\latupta\latuplam\latupbe}{K\rom{1}, Edl.}{kok:ta_lam_be}
\kipkokentry{matrah}{\latupta\latupre\latupha}{None}{kok:ta_re_ha}
\kipkokentry{matrak}{\latupta\latupre\latupkaf}{None}{kok:ta_re_kaf}
\kipkokentry{mâtuf}{\latupayn\latupta\latupfe}{K\rom{1}, Edl.}{kok:ayn_ta_fe}
\kipkokentry{maval}{\latupmim\latupvav\latuplam}{None}{kok:mim_vav_lam}
\kipkokentry{mâverâ}{\latupvav\latupre\latupye}{None}{kok:vav_re_ye}
\kipkokentry{mâyasıl}{\latupsin\latupye\latuplam}{None}{kok:sin_ye_lam}
\kipkokentry{mâyî}{\latupmim\latupye\latupayn}{K\rom{1}, Ed.}{kok:mim_ye_ayn}
\kipkokentry{mazarrat}{\latupdad\latupre\latupre}{None}{kok:dad_re_re}
\kipkokentry{mazbata}{\latupza\latupbe\latupta}{None}{kok:za_be_ta}
\kipkokentry{mazbut}{\latupza\latupbe\latupta}{K\rom{1}, Edl.}{kok:za_be_ta}
\kipkokentry{mâzeret}{\latupayn\latupzel\latupre}{None}{kok:ayn_zel_re}
\kipkokentry{mazgal}{\latupze\latupgayn\latuplam}{None}{kok:ze_gayn_lam}
\kipkokentry{mazhar}{\latupza\latuphe\latupre}{None}{kok:za_he_re}
\kipkokentry{mâzi}{\latupmim\latupdad\latupye}{K\rom{1}, Ed.}{kok:mim_dad_ye}
\kipkokentry{mazlum}{\latupza\latuplam\latupmim}{K\rom{1}, Edl.}{kok:za_lam_mim}
\kipkokentry{mazmun}{\latupdad\latupmim\latupnun}{K\rom{1}, Edl.}{kok:dad_mim_nun}
\kipkokentry{maznun}{\latupza\latupnun}{K\rom{1}, Edl.}{kok:za_nun}
\kipkokentry{mazruf}{\latupza\latupre\latupfe}{K\rom{1}, Edl.}{kok:za_re_fe}
\kipkokentry{mâzul}{\latupayn\latupze\latuplam}{K\rom{1}, Edl.}{kok:ayn_ze_lam}
\kipkokentry{mâzur}{\latupayn\latupzel\latupre}{K\rom{1}, Edl.}{kok:ayn_zel_re}
\kipkokentry{meal}{\latupalif\latupvav\latuplam}{None}{kok:alif_vav_lam}
\kipkokentry{meap}{\latupalif\latupvav\latupbe}{None}{kok:alif_vav_be}
\kipkokentry{mebdê}{\latupbe\latupdal\latupalif}{None}{kok:be_dal_alif}
\kipkokentry{meblağ}{\latupbe\latuplam\latupgayn}{None}{kok:be_lam_gayn}
\kipkokentry{mebrûre}{\latupbe\latupre\latupre}{None}{kok:be_re_re}
\kipkokentry{mebus}{\latupbe\latupayn\latupthe}{K\rom{1}, Edl.}{kok:be_ayn_the}
\kipkokentry{mebzul}{\latupbe\latupzel\latuplam}{K\rom{1}, Edl.}{kok:be_zel_lam}
\kipkokentry{mecal}{\latupcim\latupvav\latuplam}{None}{kok:cim_vav_lam}
\kipkokentry{mecaz}{\latupcim\latupvav\latupze}{None}{kok:cim_vav_ze}
\kipkokentry{mecbur}{\latupcim\latupbe\latupre}{K\rom{1}, Edl.}{kok:cim_be_re}
\kipkokentry{meccânen}{\latupmim\latupcim\latupnun}{None}{kok:mim_cim_nun}
\kipkokentry{mecîdiye}{\latupmim\latupcim\latupdal}{None}{kok:mim_cim_dal}
\kipkokentry{meclis}{\latupcim\latuplam\latupsin}{None}{kok:cim_lam_sin}
\kipkokentry{mecmuâ}{\latupcim\latupmim\latupayn}{None}{kok:cim_mim_ayn}
\kipkokentry{mecmû}{\latupcim\latupmim\latupayn}{K\rom{1}, Edl.}{kok:cim_mim_ayn}
\kipkokentry{mecnun}{\latupcim\latupnun\latupnun}{K\rom{1}, Edl.}{kok:cim_nun_nun}
\kipkokentry{mecrâ}{\latupcim\latupre\latupye}{None}{kok:cim_re_ye}
\kipkokentry{mecruh}{\latupcim\latupre\latupha}{K\rom{1}, Edl.}{kok:cim_re_ha}
\kipkokentry{meczup}{\latupcim\latupzel\latupbe}{K\rom{1}, Edl.}{kok:cim_zel_be}
\kipkokentry{meçhul}{\latupcim\latuphe\latuplam}{K\rom{1}, Edl.}{kok:cim_he_lam}
\kipkokentry{medâr}{\latupdal\latupvav\latupre}{None}{kok:dal_vav_re}
\kipkokentry{meddah}{\latupmim\latupdal\latuphe}{None}{kok:mim_dal_he}
\kipkokentry{medenî}{\latupmim\latupdal\latupnun}{None}{kok:mim_dal_nun}
\kipkokentry{medet}{\latupmim\latupdal\latupdal$^1$}{None}{kok:mim_dal_dal1}
\kipkokentry{medfun}{\latupdal\latupfe\latupnun}{K\rom{1}, Edl.}{kok:dal_fe_nun}
\kipkokentry{medhal}{\latupdal\latupxa\latuplam}{None}{kok:dal_xa_lam}
\kipkokentry{medîne}{\latupdal\latupye\latupnun$^1$}{None}{kok:dal_ye_nun1}
\kipkokentry{medrese}{\latupdal\latupre\latupsin}{None}{kok:dal_re_sin}
\kipkokentry{medyun}{\latupdal\latupye\latupnun$^1$}{K\rom{1}, Edl.}{kok:dal_ye_nun1}
\kipkokentry{mefhum}{\latupfe\latuphe\latupmim}{K\rom{1}, Edl.}{kok:fe_he_mim}
\kipkokentry{mefkûre}{\latupfe\latupkef\latupre}{None}{kok:fe_kef_re}
\kipkokentry{mefluç}{\latupfe\latuplam\latupcim}{K\rom{1}, Edl.}{kok:fe_lam_cim}
\kipkokentry{mefruşat}{\latupfe\latupre\latupshin}{None}{kok:fe_re_shin}
\kipkokentry{mefruz}{\latupfe\latupre\latupdad}{K\rom{1}, Edl.}{kok:fe_re_dad}
\kipkokentry{meftun}{\latupfe\latupte\latupnun}{K\rom{1}, Edl.}{kok:fe_te_nun}
\kipkokentry{meful}{\latupfe\latupayn\latuplam}{K\rom{1}, Edl.}{kok:fe_ayn_lam}
\kipkokentry{mehâbet}{\latuphe\latupye\latupbe}{None}{kok:he_ye_be}
\kipkokentry{mêhaz}{\latupalif\latupxa\latupzel}{None}{kok:alif_xa_zel}
\kipkokentry{mehdi}{\latuphe\latupdal\latupye}{K\rom{1}, Edl.}{kok:he_dal_ye}
\kipkokentry{mehel}{\latupha\latuplam\latuplam}{None}{kok:ha_lam_lam}
\kipkokentry{mehil}{\latupmim\latuphe\latuplam}{None}{kok:mim_he_lam}
\kipkokentry{mekan}{\latupkef\latupvav\latupnun}{None}{kok:kef_vav_nun}
\kipkokentry{mekkâre}{\latupkef\latupre\latupye}{None}{kok:kef_re_ye}
\kipkokentry{mekruh}{\latupkef\latupre\latuphe}{K\rom{1}, Edl.}{kok:kef_re_he}
\kipkokentry{meksefe}{\latupkef\latupthe\latupfe}{None}{kok:kef_the_fe}
\kipkokentry{mektep}{\latupkef\latupte\latupbe}{None}{kok:kef_te_be}
\kipkokentry{mektup}{\latupkef\latupte\latupbe}{K\rom{1}, Edl.}{kok:kef_te_be}
\kipkokentry{melâike}{\latupmim\latuplam\latupkef}{None}{kok:mim_lam_kef}
\kipkokentry{melal}{\latupmim\latuplam\latuplam$^2$}{None}{kok:mim_lam_lam2}
\kipkokentry{melâmi}{\latuplam\latupvav\latupmim}{None}{kok:lam_vav_mim}
\kipkokentry{melânet}{\latuplam\latupayn\latupnun}{None}{kok:lam_ayn_nun}
\kipkokentry{melbusat}{\latuplam\latupbe\latupsin}{None}{kok:lam_be_sin}
\kipkokentry{melce}{\latuplam\latupcim\latupalif}{None}{kok:lam_cim_alif}
\kipkokentry{meleke}{\latupmim\latuplam\latupkef}{None}{kok:mim_lam_kef}
\kipkokentry{melfuf}{\latuplam\latupfe\latupfe}{K\rom{1}, Edl.}{kok:lam_fe_fe}
\kipkokentry{melik}{\latupmim\latuplam\latupkef}{K\rom{1}, Ed.}{kok:mim_lam_kef}
\kipkokentry{meluf}{\latupalif\latuplam\latupfe}{K\rom{1}, Edl.}{kok:alif_lam_fe}
\kipkokentry{melul}{\latupmim\latuplam\latuplam$^2$}{None}{kok:mim_lam_lam2}
\kipkokentry{melun}{\latuplam\latupayn\latupnun}{K\rom{1}, Edl.}{kok:lam_ayn_nun}
\kipkokentry{memâlik}{\latupmim\latuplam\latupkef}{None}{kok:mim_lam_kef}
\kipkokentry{memat}{\latupmim\latupvav\latupte}{None}{kok:mim_vav_te}
\kipkokentry{memlaha}{\latupmim\latuplam\latupha}{None}{kok:mim_lam_ha}
\kipkokentry{memleket}{\latupmim\latuplam\latupkef}{None}{kok:mim_lam_kef}
\kipkokentry{memlük}{\latupmim\latuplam\latupkef}{K\rom{1}, Edl.}{kok:mim_lam_kef}
\kipkokentry{memnû}{\latupmim\latupnun\latupayn}{K\rom{1}, Edl.}{kok:mim_nun_ayn}
\kipkokentry{memnun}{\latupmim\latupnun\latupnun}{K\rom{1}, Edl.}{kok:mim_nun_nun}
\kipkokentry{memur}{\latupalif\latupmim\latupre}{K\rom{1}, Edl.}{kok:alif_mim_re}
\kipkokentry{men}{\latupmim\latupnun\latupayn}{None}{kok:mim_nun_ayn}
\kipkokentry{menbâ}{\latupnun\latupbe\latupayn}{None}{kok:nun_be_ayn}
\kipkokentry{mendebur}{\latupdal\latupbe\latupre}{None}{kok:dal_be_re}
\kipkokentry{menfâ}{\latupnun\latupfe\latupvav}{None}{kok:nun_fe_vav}
\kipkokentry{menfaat}{\latupnun\latupfe\latupayn}{None}{kok:nun_fe_ayn}
\kipkokentry{menfez}{\latupnun\latupfe\latupzel}{None}{kok:nun_fe_zel}
\kipkokentry{menfî}{\latupnun\latupfe\latupvav}{None}{kok:nun_fe_vav}
\kipkokentry{menfur}{\latupnun\latupfe\latupre}{K\rom{1}, Edl.}{kok:nun_fe_re}
\kipkokentry{menhus}{\latupnun\latupha\latupsin}{K\rom{1}, Edl.}{kok:nun_ha_sin}
\kipkokentry{meni}{\latupmim\latupnun\latupye$^2$}{None}{kok:mim_nun_ye2}
\kipkokentry{menkıbe}{\latupnun\latupkaf\latupbe}{None}{kok:nun_kaf_be}
\kipkokentry{menkul}{\latupnun\latupkaf\latuplam}{K\rom{1}, Edl.}{kok:nun_kaf_lam}
\kipkokentry{mensucat}{\latupnun\latupsin\latupcim}{None}{kok:nun_sin_cim}
\kipkokentry{mensup}{\latupnun\latupsin\latupbe}{K\rom{1}, Edl.}{kok:nun_sin_be}
\kipkokentry{mensur}{\latupnun\latupthe\latupre}{K\rom{1}, Edl.}{kok:nun_the_re}
\kipkokentry{menşe}{\latupnun\latupshin\latupalif}{None}{kok:nun_shin_alif}
\kipkokentry{menşur}{\latupnun\latupshin\latupre}{K\rom{1}, Edl.}{kok:nun_shin_re}
\kipkokentry{menzil}{\latupnun\latupze\latuplam}{None}{kok:nun_ze_lam}
\kipkokentry{mera}{\latupre\latupayn\latupye}{None}{kok:re_ayn_ye}
\kipkokentry{merak}{\latupre\latupkaf\latupkaf}{None}{kok:re_kaf_kaf}
\kipkokentry{meram}{\latupre\latupvav\latupmim}{None}{kok:re_vav_mim}
\kipkokentry{merâsim}{\latupre\latupsin\latupmim}{None}{kok:re_sin_mim}
\kipkokentry{merâtip}{\latupre\latupte\latupbe}{None}{kok:re_te_be}
\kipkokentry{merbut}{\latupre\latupbe\latupta}{K\rom{1}, Edl.}{kok:re_be_ta}
\kipkokentry{mercî}{\latupre\latupcim\latupayn}{None}{kok:re_cim_ayn}
\kipkokentry{meret}{\latupmim\latupre\latupdal}{None}{kok:mim_re_dal}
\kipkokentry{merhabâ}{\latupre\latupha\latupbe}{None}{kok:re_ha_be}
\kipkokentry{merhale}{\latupre\latupha\latuplam}{None}{kok:re_ha_lam}
\kipkokentry{merhamet}{\latupre\latupha\latupmim}{None}{kok:re_ha_mim}
\kipkokentry{merhum}{\latupre\latupha\latupmim}{K\rom{1}, Edl.}{kok:re_ha_mim}
\kipkokentry{merî}{\latupre\latupayn\latupye}{K\rom{1}, Edl.}{kok:re_ayn_ye}
\kipkokentry{merkep}{\latupre\latupkef\latupbe}{None}{kok:re_kef_be}
\kipkokentry{merkez}{\latupre\latupkef\latupze}{None}{kok:re_kef_ze}
\kipkokentry{mermi}{\latupre\latupmim\latupye}{K\rom{1}, Edl.}{kok:re_mim_ye}
\kipkokentry{mersiye}{\latupre\latupthe\latupvav}{None}{kok:re_the_vav}
\kipkokentry{mertebe}{\latupre\latupte\latupbe}{None}{kok:re_te_be}
\kipkokentry{mesâbe}{\latupthe\latupvav\latupbe}{None}{kok:the_vav_be}
\kipkokentry{mesâfe}{\latupsin\latupvav\latupfe}{None}{kok:sin_vav_fe}
\kipkokentry{mesağ}{\latupsin\latupvav\latupgayn}{None}{kok:sin_vav_gayn}
\kipkokentry{mesaha}{\latupsin\latupvav\latupha}{None}{kok:sin_vav_ha}
\kipkokentry{mesâi}{\latupsin\latupayn\latupye}{None}{kok:sin_ayn_ye}
\kipkokentry{mesâne}{\latupthe\latupvav\latupnun}{None}{kok:the_vav_nun}
\kipkokentry{mescit}{\latupsin\latupcim\latupdal}{None}{kok:sin_cim_dal}
\kipkokentry{mesel}{\latupmim\latupthe\latuplam}{None}{kok:mim_the_lam}
\kipkokentry{meselâ}{\latupmim\latupthe\latuplam}{None}{kok:mim_the_lam}
\kipkokentry{mesele}{\latupsin\latupalif\latuplam}{None}{kok:sin_alif_lam}
\kipkokentry{meserret}{\latupsin\latupre\latupre}{None}{kok:sin_re_re}
\kipkokentry{mesh}{\latupmim\latupsin\latupha}{None}{kok:mim_sin_ha}
\kipkokentry{mesih}{\latupmim\latupsin\latupha}{None}{kok:mim_sin_ha}
\kipkokentry{mesir}{\latupsin\latupye\latupre}{None}{kok:sin_ye_re}
\kipkokentry{mesîre}{\latupsin\latupye\latupre}{None}{kok:sin_ye_re}
\kipkokentry{meskûkat}{\latupsin\latupkef\latupkef}{None}{kok:sin_kef_kef}
\kipkokentry{meskun}{\latupsin\latupkef\latupnun}{K\rom{1}, Edl.}{kok:sin_kef_nun}
\kipkokentry{meslek}{\latupsin\latuplam\latupkef}{None}{kok:sin_lam_kef}
\kipkokentry{mesmû}{\latupsin\latupmim\latupayn}{K\rom{1}, Edl.}{kok:sin_mim_ayn}
\kipkokentry{mesnet}{\latupsin\latupnun\latupdal}{None}{kok:sin_nun_dal}
\kipkokentry{mesnevî}{\latupthe\latupnun\latupye}{None}{kok:the_nun_ye}
\kipkokentry{mesrur}{\latupsin\latupre\latupre}{K\rom{1}, Edl.}{kok:sin_re_re}
\kipkokentry{mestur}{\latupsin\latupte\latupre}{K\rom{1}, Edl.}{kok:sin_te_re}
\kipkokentry{mesul}{\latupsin\latupalif\latuplam}{K\rom{1}, Edl.}{kok:sin_alif_lam}
\kipkokentry{mesut}{\latupsin\latupayn\latupdal}{K\rom{1}, Edl.}{kok:sin_ayn_dal}
\kipkokentry{meşakkat}{\latupshin\latupkaf\latupkaf}{None}{kok:shin_kaf_kaf}
\kipkokentry{meşâle}{\latupshin\latupayn\latuplam}{None}{kok:shin_ayn_lam}
\kipkokentry{meşâyih}{\latupshin\latupye\latupxa}{None}{kok:shin_ye_xa}
\kipkokentry{meşcere}{\latupshin\latupcim\latupre}{None}{kok:shin_cim_re}
\kipkokentry{meşgale}{\latupshin\latupgayn\latuplam}{None}{kok:shin_gayn_lam}
\kipkokentry{meşgul}{\latupshin\latupgayn\latuplam}{K\rom{1}, Edl.}{kok:shin_gayn_lam}
\kipkokentry{meşhet}{\latupshin\latuphe\latupdal}{None}{kok:shin_he_dal}
\kipkokentry{meşhur}{\latupshin\latuphe\latupre}{K\rom{1}, Edl.}{kok:shin_he_re}
\kipkokentry{meşhut}{\latupshin\latuphe\latupdal}{K\rom{1}, Edl.}{kok:shin_he_dal}
\kipkokentry{meşîhat}{\latupshin\latupye\latupxa}{None}{kok:shin_ye_xa}
\kipkokentry{meşîme}{\latupshin\latupye\latupmim}{None}{kok:shin_ye_mim}
\kipkokentry{meşk}{\latupmim\latupshin\latupkaf}{None}{kok:mim_shin_kaf}
\kipkokentry{meşkuk}{\latupshin\latupkef\latupkef}{K\rom{1}, Edl.}{kok:shin_kef_kef}
\kipkokentry{meşrep}{\latupshin\latupre\latupbe}{None}{kok:shin_re_be}
\kipkokentry{meşrû}{\latupshin\latupre\latupayn}{K\rom{1}, Edl.}{kok:shin_re_ayn}
\kipkokentry{meşrubat}{\latupshin\latupre\latupbe}{None}{kok:shin_re_be}
\kipkokentry{meşruhat}{\latupshin\latupre\latupha}{None}{kok:shin_re_ha}
\kipkokentry{meşrut}{\latupshin\latupre\latupta}{K\rom{1}, Edl.}{kok:shin_re_ta}
\kipkokentry{meşrûtiyet}{\latupshin\latupre\latupta}{None}{kok:shin_re_ta}
\kipkokentry{meşum}{\latupshin\latupalif\latupmim}{K\rom{1}, Edl.}{kok:shin_alif_mim}
\kipkokentry{meşveret}{\latupshin\latupvav\latupre}{None}{kok:shin_vav_re}
\kipkokentry{met}{\latupmim\latupdal\latupdal$^1$}{None}{kok:mim_dal_dal1}
\kipkokentry{meta}{\latupmim\latupte\latupayn}{None}{kok:mim_te_ayn}
\kipkokentry{metânet}{\latupmim\latupte\latupnun}{None}{kok:mim_te_nun}
\kipkokentry{metbû}{\latupte\latupbe\latupayn}{K\rom{1}, Edl.}{kok:te_be_ayn}
\kipkokentry{meth}{\latupmim\latupdal\latuphe}{None}{kok:mim_dal_he}
\kipkokentry{metin}{\latupmim\latupte\latupnun}{None}{kok:mim_te_nun}
\kipkokentry{metin}{\latupmim\latupte\latupnun}{None}{kok:mim_te_nun}
\kipkokentry{metris}{\latupte\latupre\latupsin}{None}{kok:te_re_sin}
\kipkokentry{metruk}{\latupte\latupre\latupkef}{K\rom{1}, Edl.}{kok:te_re_kef}
\kipkokentry{mevâli}{\latupvav\latuplam\latupye}{None}{kok:vav_lam_ye}
\kipkokentry{mevâşi}{\latupmim\latupshin\latupye}{None}{kok:mim_shin_ye}
\kipkokentry{mevce}{\latupmim\latupvav\latupcim}{None}{kok:mim_vav_cim}
\kipkokentry{mevcut}{\latupvav\latupcim\latupdal}{K\rom{1}, Edl.}{kok:vav_cim_dal}
\kipkokentry{mevduat}{\latupvav\latupdal\latupayn}{None}{kok:vav_dal_ayn}
\kipkokentry{mevhibe}{\latupvav\latuphe\latupbe}{None}{kok:vav_he_be}
\kipkokentry{mevhum}{\latupvav\latuphe\latupmim}{K\rom{1}, Edl.}{kok:vav_he_mim}
\kipkokentry{mevkî}{\latupvav\latupkaf\latupayn}{None}{kok:vav_kaf_ayn}
\kipkokentry{mevkuf}{\latupvav\latupkaf\latupfe}{K\rom{1}, Edl.}{kok:vav_kaf_fe}
\kipkokentry{mevkûte}{\latupvav\latupkaf\latupte}{None}{kok:vav_kaf_te}
\kipkokentry{mevlâ}{\latupvav\latuplam\latupye}{None}{kok:vav_lam_ye}
\kipkokentry{mevlit}{\latupvav\latuplam\latupdal}{None}{kok:vav_lam_dal}
\kipkokentry{mevsim}{\latupvav\latupsin\latupmim}{None}{kok:vav_sin_mim}
\kipkokentry{mevsuk}{\latupvav\latupthe\latupkaf}{K\rom{1}, Edl.}{kok:vav_the_kaf}
\kipkokentry{mevt}{\latupmim\latupvav\latupte}{None}{kok:mim_vav_te}
\kipkokentry{mevtâ}{\latupmim\latupvav\latupte}{None}{kok:mim_vav_te}
\kipkokentry{mevut}{\latupvav\latupayn\latupdal}{K\rom{1}, Edl.}{kok:vav_ayn_dal}
\kipkokentry{mevzî}{\latupvav\latupdad\latupayn}{None}{kok:vav_dad_ayn}
\kipkokentry{mevzû}{\latupvav\latupdad\latupayn}{K\rom{1}, Edl.}{kok:vav_dad_ayn}
\kipkokentry{mevzuat}{\latupvav\latupdad\latupayn}{None}{kok:vav_dad_ayn}
\kipkokentry{mevzun}{\latupvav\latupze\latupnun}{K\rom{1}, Edl.}{kok:vav_ze_nun}
\kipkokentry{meyil}{\latupmim\latupye\latuplam}{None}{kok:mim_ye_lam}
\kipkokentry{meymenet}{\latupye\latupmim\latupnun}{None}{kok:ye_mim_nun}
\kipkokentry{meyus}{\latupye\latupalif\latupsin}{K\rom{1}, Edl.}{kok:ye_alif_sin}
\kipkokentry{meyyal}{\latupmim\latupye\latuplam}{None}{kok:mim_ye_lam}
\kipkokentry{meyyit}{\latupmim\latupvav\latupte}{None}{kok:mim_vav_te}
\kipkokentry{mezâlim}{\latupza\latuplam\latupmim}{None}{kok:za_lam_mim}
\kipkokentry{mezar}{\latupze\latupvav\latupre}{None}{kok:ze_vav_re}
\kipkokentry{mezât}{\latupze\latupye\latupdal}{None}{kok:ze_ye_dal}
\kipkokentry{mezbaha}{\latupzel\latupbe\latupha}{None}{kok:zel_be_ha}
\kipkokentry{mezbele}{\latupze\latupbe\latuplam}{None}{kok:ze_be_lam}
\kipkokentry{mezc}{\latupmim\latupze\latupcim}{None}{kok:mim_ze_cim}
\kipkokentry{mezellet}{\latupzel\latuplam\latuplam}{None}{kok:zel_lam_lam}
\kipkokentry{mezgit}{\latupze\latupye\latupte}{None}{kok:ze_ye_te}
\kipkokentry{mezhep}{\latupzel\latuphe\latupbe$^1$}{None}{kok:zel_he_be1}
\kipkokentry{meziyet}{\latupmim\latupze\latupye}{None}{kok:mim_ze_ye}
\kipkokentry{mezkur}{\latupzel\latupkef\latupre$^1$}{K\rom{1}, Edl.}{kok:zel_kef_re1}
\kipkokentry{mezmur}{\latupze\latupmim\latupre$^1$}{K\rom{1}, Edl.}{kok:ze_mim_re1}
\kipkokentry{mezrâ}{\latupzel\latupre\latupayn}{None}{kok:zel_re_ayn}
\kipkokentry{mêzun}{\latupalif\latupzel\latupnun}{K\rom{1}, Edl.}{kok:alif_zel_nun}
\kipkokentry{mıntıka}{\latupnun\latupte\latupkaf}{None}{kok:nun_te_kaf}
\kipkokentry{mısrâ}{\latupsad\latupre\latupayn}{None}{kok:sad_re_ayn}
\kipkokentry{mızrak}{\latupze\latupre\latupkaf}{None}{kok:ze_re_kaf}
\kipkokentry{mızrap}{\latupdad\latupre\latupbe}{None}{kok:dad_re_be}
\kipkokentry{miat}{\latupayn\latupvav\latupdal}{None}{kok:ayn_vav_dal}
\kipkokentry{mibzer}{\latupbe\latupze\latupre}{None}{kok:be_ze_re}
\kipkokentry{mîde}{\latupmim\latupayn\latupdal}{None}{kok:mim_ayn_dal}
\kipkokentry{miftah}{\latupfe\latupte\latupha}{None}{kok:fe_te_ha}
\kipkokentry{miğfer}{\latupgayn\latupfe\latupre}{None}{kok:gayn_fe_re}
\kipkokentry{mihenk}{\latupha\latupkef\latupkef}{None}{kok:ha_kef_kef}
\kipkokentry{mihnet}{\latupmim\latupha\latupnun}{None}{kok:mim_ha_nun}
\kipkokentry{mihrak}{\latupha\latupre\latupkaf}{None}{kok:ha_re_kaf}
\kipkokentry{mihrap}{\latupha\latupre\latupbe}{None}{kok:ha_re_be}
\kipkokentry{mihver}{\latupha\latupvav\latupre}{None}{kok:ha_vav_re}
\kipkokentry{mikâp}{\latupkef\latupayn\latupbe}{None}{kok:kef_ayn_be}
\kipkokentry{miktar}{\latupkaf\latupdal\latupre}{None}{kok:kaf_dal_re}
\kipkokentry{mikyas}{\latupkaf\latupye\latupsin}{None}{kok:kaf_ye_sin}
\kipkokentry{milel}{\latupmim\latuplam\latuplam$^1$}{None}{kok:mim_lam_lam1}
\kipkokentry{millet}{\latupmim\latuplam\latuplam$^1$}{None}{kok:mim_lam_lam1}
\kipkokentry{mîmar}{\latupayn\latupmim\latupre}{None}{kok:ayn_mim_re}
\kipkokentry{minâre}{\latupnun\latupvav\latupre}{None}{kok:nun_vav_re}
\kipkokentry{minber}{\latupnun\latupbe\latupre}{None}{kok:nun_be_re}
\kipkokentry{minnet}{\latupmim\latupnun\latupnun}{None}{kok:mim_nun_nun}
\kipkokentry{minval}{\latupnun\latupvav\latuplam}{None}{kok:nun_vav_lam}
\kipkokentry{mîraç}{\latupayn\latupre\latupcim}{None}{kok:ayn_re_cim}
\kipkokentry{mîras}{\latupvav\latupre\latupthe}{None}{kok:vav_re_the}
\kipkokentry{mirât}{\latupre\latupalif\latupye}{None}{kok:re_alif_ye}
\kipkokentry{misâfir}{\latupsin\latupfe\latupre}{None}{kok:sin_fe_re}
\kipkokentry{mîsak}{\latupvav\latupthe\latupkaf}{None}{kok:vav_the_kaf}
\kipkokentry{misal}{\latupmim\latupthe\latuplam}{None}{kok:mim_the_lam}
\kipkokentry{misil}{\latupmim\latupthe\latuplam}{None}{kok:mim_the_lam}
\kipkokentry{miskal}{\latupthe\latupkaf\latuplam}{None}{kok:the_kaf_lam}
\kipkokentry{misvak}{\latupsin\latupvav\latupkef}{None}{kok:sin_vav_kef}
\kipkokentry{mîyar}{\latupayn\latupye\latupre$^1$}{None}{kok:ayn_ye_re1}
\kipkokentry{mizaç}{\latupmim\latupze\latupcim}{None}{kok:mim_ze_cim}
\kipkokentry{mizah}{\latupmim\latupze\latupha}{None}{kok:mim_ze_ha}
\kipkokentry{mîzan}{\latupvav\latupze\latupnun}{None}{kok:vav_ze_nun}
\kipkokentry{molla}{\latupvav\latuplam\latupye}{None}{kok:vav_lam_ye}
\kipkokentry{muaccel}{\latupayn\latupcim\latuplam}{None}{kok:ayn_cim_lam}
\kipkokentry{muâdelet}{\latupayn\latupdal\latuplam}{None}{kok:ayn_dal_lam}
\kipkokentry{muâdil}{\latupayn\latupdal\latuplam}{None}{kok:ayn_dal_lam}
\kipkokentry{muaf}{\latupayn\latupfe\latupvav}{None}{kok:ayn_fe_vav}
\kipkokentry{muâhede}{\latupayn\latuphe\latupdal}{None}{kok:ayn_he_dal}
\kipkokentry{muâheze}{\latupalif\latupxa\latupzel}{None}{kok:alif_xa_zel}
\kipkokentry{muahhar}{\latupalif\latupxa\latupre}{None}{kok:alif_xa_re}
\kipkokentry{muallak}{\latupayn\latuplam\latupkaf}{None}{kok:ayn_lam_kaf}
\kipkokentry{muallim}{\latupayn\latuplam\latupmim}{None}{kok:ayn_lam_mim}
\kipkokentry{muâmele}{\latupayn\latupmim\latuplam}{None}{kok:ayn_mim_lam}
\kipkokentry{muammâ}{\latupayn\latupmim\latupye}{None}{kok:ayn_mim_ye}
\kipkokentry{muannit}{\latupayn\latupnun\latupdal}{None}{kok:ayn_nun_dal}
\kipkokentry{muârız}{\latupayn\latupre\latupdad}{None}{kok:ayn_re_dad}
\kipkokentry{muarrep}{\latupayn\latupre\latupbe}{None}{kok:ayn_re_be}
\kipkokentry{muâsır}{\latupayn\latupsad\latupre$^1$}{None}{kok:ayn_sad_re1}
\kipkokentry{muâşaka}{\latupayn\latupshin\latupkaf}{None}{kok:ayn_shin_kaf}
\kipkokentry{muâşeret}{\latupayn\latupshin\latupre}{None}{kok:ayn_shin_re}
\kipkokentry{muâvenet}{\latupayn\latupvav\latupnun}{None}{kok:ayn_vav_nun}
\kipkokentry{muâvin}{\latupayn\latupvav\latupnun}{None}{kok:ayn_vav_nun}
\kipkokentry{muâyede}{\latupayn\latupye\latupdal}{None}{kok:ayn_ye_dal}
\kipkokentry{muâyene}{\latupayn\latupye\latupnun}{None}{kok:ayn_ye_nun}
\kipkokentry{muayyen}{\latupayn\latupye\latupnun}{None}{kok:ayn_ye_nun}
\kipkokentry{muazzam}{\latupayn\latupza\latupmim}{None}{kok:ayn_za_mim}
\kipkokentry{muazzep}{\latupayn\latupzel\latupbe}{None}{kok:ayn_zel_be}
\kipkokentry{muazzez}{\latupayn\latupze\latupze}{None}{kok:ayn_ze_ze}
\kipkokentry{mûcip}{\latupvav\latupcim\latupbe}{None}{kok:vav_cim_be}
\kipkokentry{mûcit}{\latupvav\latupcim\latupdal}{None}{kok:vav_cim_dal}
\kipkokentry{mûcize}{\latupayn\latupcim\latupze}{None}{kok:ayn_cim_ze}
\kipkokentry{mûdî}{\latupvav\latupdal\latupayn}{None}{kok:vav_dal_ayn}
\kipkokentry{mufassal}{\latupfe\latupsad\latuplam}{None}{kok:fe_sad_lam}
\kipkokentry{mugâlata}{\latupgayn\latuplam\latupta}{None}{kok:gayn_lam_ta}
\kipkokentry{mugannî}{\latupgayn\latupnun\latupye$^2$}{None}{kok:gayn_nun_ye2}
\kipkokentry{mugâyir}{\latupgayn\latupye\latupre}{None}{kok:gayn_ye_re}
\kipkokentry{muğber}{\latupgayn\latupbe\latupre}{None}{kok:gayn_be_re}
\kipkokentry{muğlak}{\latupgayn\latuplam\latupkaf}{None}{kok:gayn_lam_kaf}
\kipkokentry{muhabbet}{\latupha\latupbe\latupbe$^2$}{None}{kok:ha_be_be2}
\kipkokentry{muhâberat}{\latupxa\latupbe\latupre}{None}{kok:xa_be_re}
\kipkokentry{muhâbere}{\latupxa\latupbe\latupre}{None}{kok:xa_be_re}
\kipkokentry{muhâbir}{\latupxa\latupbe\latupre}{None}{kok:xa_be_re}
\kipkokentry{muhâceret}{\latuphe\latupcim\latupre}{None}{kok:he_cim_re}
\kipkokentry{muhâcim}{\latuphe\latupcim\latupmim}{None}{kok:he_cim_mim}
\kipkokentry{muhâcir}{\latuphe\latupcim\latupre}{None}{kok:he_cim_re}
\kipkokentry{muhâfaza}{\latupha\latupfe\latupza}{None}{kok:ha_fe_za}
\kipkokentry{muhaffef}{\latupxa\latupfe\latupfe}{None}{kok:xa_fe_fe}
\kipkokentry{muhâfız}{\latupha\latupfe\latupza}{None}{kok:ha_fe_za}
\kipkokentry{muhâkeme}{\latupha\latupkef\latupmim}{None}{kok:ha_kef_mim}
\kipkokentry{muhakkak}{\latupha\latupkaf\latupkaf}{None}{kok:ha_kaf_kaf}
\kipkokentry{muhakkik}{\latupha\latupkaf\latupkaf}{None}{kok:ha_kaf_kaf}
\kipkokentry{muhâl}{\latupha\latupvav\latuplam}{None}{kok:ha_vav_lam}
\kipkokentry{muhâlefet}{\latupxa\latuplam\latupfe}{None}{kok:xa_lam_fe}
\kipkokentry{muhâlif}{\latupxa\latuplam\latupfe}{None}{kok:xa_lam_fe}
\kipkokentry{muhallebi}{\latupha\latuplam\latupbe}{None}{kok:ha_lam_be}
\kipkokentry{muhammara}{\latupha\latupmim\latupre}{None}{kok:ha_mim_re}
\kipkokentry{muhammen}{\latupxa\latupmim\latupnun}{None}{kok:xa_mim_nun}
\kipkokentry{Muhammet}{\latupha\latupmim\latupdal}{None}{kok:ha_mim_dal}
\kipkokentry{muhârebe}{\latupha\latupre\latupbe}{None}{kok:ha_re_be}
\kipkokentry{muharip}{\latupha\latupre\latupbe}{None}{kok:ha_re_be}
\kipkokentry{muharref}{\latupha\latupre\latupfe$^1$}{None}{kok:ha_re_fe1}
\kipkokentry{muharrer}{\latupha\latupre\latupre$^1$}{None}{kok:ha_re_re1}
\kipkokentry{muharrik}{\latupha\latupre\latupkef}{None}{kok:ha_re_kef}
\kipkokentry{muharrir}{\latupha\latupre\latupre$^1$}{None}{kok:ha_re_re1}
\kipkokentry{muhâsara}{\latupha\latupsad\latupre}{None}{kok:ha_sad_re}
\kipkokentry{muhâsebe}{\latupha\latupsin\latupbe}{None}{kok:ha_sin_be}
\kipkokentry{muhâsım}{\latupxa\latupsad\latupmim}{None}{kok:xa_sad_mim}
\kipkokentry{muhâsip}{\latupha\latupsin\latupbe}{None}{kok:ha_sin_be}
\kipkokentry{muhâtap}{\latupxa\latupta\latupbe}{None}{kok:xa_ta_be}
\kipkokentry{muhâtara}{\latupxa\latupta\latupre}{None}{kok:xa_ta_re}
\kipkokentry{muhâvere}{\latupha\latupvav\latupre}{None}{kok:ha_vav_re}
\kipkokentry{muhâvir}{\latupha\latupvav\latupre}{None}{kok:ha_vav_re}
\kipkokentry{muhayyel}{\latupxa\latupye\latuplam}{None}{kok:xa_ye_lam}
\kipkokentry{muhayyer}{\latupxa\latupye\latupre}{None}{kok:xa_ye_re}
\kipkokentry{muhayyile}{\latupxa\latupye\latuplam}{None}{kok:xa_ye_lam}
\kipkokentry{muhbir}{\latupxa\latupbe\latupre}{None}{kok:xa_be_re}
\kipkokentry{muhdes}{\latupha\latupdal\latupthe}{None}{kok:ha_dal_the}
\kipkokentry{muhik}{\latupha\latupkaf\latupkaf}{None}{kok:ha_kaf_kaf}
\kipkokentry{muhip}{\latupha\latupbe\latupbe$^2$}{None}{kok:ha_be_be2}
\kipkokentry{mûhit}{\latupha\latupvav\latupta}{None}{kok:ha_vav_ta}
\kipkokentry{muhkem}{\latupha\latupkef\latupmim}{None}{kok:ha_kef_mim}
\kipkokentry{muhlis}{\latupxa\latuplam\latupsad}{None}{kok:xa_lam_sad}
\kipkokentry{muhrip}{\latupxa\latupre\latupbe}{None}{kok:xa_re_be}
\kipkokentry{muhsin}{\latupha\latupsin\latupnun}{None}{kok:ha_sin_nun}
\kipkokentry{muhtaç}{\latupha\latupvav\latupcim}{None}{kok:ha_vav_cim}
\kipkokentry{muhtar}{\latupxa\latupye\latupre}{None}{kok:xa_ye_re}
\kipkokentry{muhtasar}{\latupxa\latupsad\latupre}{None}{kok:xa_sad_re}
\kipkokentry{muhtekir}{\latupha\latupkef\latupre}{None}{kok:ha_kef_re}
\kipkokentry{muhtelif}{\latupxa\latuplam\latupfe}{None}{kok:xa_lam_fe}
\kipkokentry{muhtelit}{\latupxa\latuplam\latupta}{None}{kok:xa_lam_ta}
\kipkokentry{muhtemel}{\latupha\latupmim\latuplam}{None}{kok:ha_mim_lam}
\kipkokentry{muhterem}{\latupha\latupre\latupmim}{None}{kok:ha_re_mim}
\kipkokentry{muhteris}{\latupha\latupre\latupsad}{None}{kok:ha_re_sad}
\kipkokentry{muhtesip}{\latupha\latupsin\latupbe}{None}{kok:ha_sin_be}
\kipkokentry{muhteşem}{\latupha\latupshin\latupmim}{None}{kok:ha_shin_mim}
\kipkokentry{muhtevâ}{\latupha\latupvav\latupye}{None}{kok:ha_vav_ye}
\kipkokentry{muhtevî}{\latupha\latupvav\latupye}{None}{kok:ha_vav_ye}
\kipkokentry{muhtıra}{\latupxa\latupta\latupre}{None}{kok:xa_ta_re}
\kipkokentry{mûin}{\latupayn\latupvav\latupnun}{None}{kok:ayn_vav_nun}
\kipkokentry{mukâbele}{\latupkaf\latupbe\latuplam}{None}{kok:kaf_be_lam}
\kipkokentry{mukâbil}{\latupkaf\latupbe\latuplam}{None}{kok:kaf_be_lam}
\kipkokentry{mukaddem}{\latupkaf\latupdal\latupmim}{None}{kok:kaf_dal_mim}
\kipkokentry{mukadder}{\latupkaf\latupdal\latupre}{None}{kok:kaf_dal_re}
\kipkokentry{mukaddes}{\latupkaf\latupdal\latupsin}{None}{kok:kaf_dal_sin}
\kipkokentry{mukaddime}{\latupkaf\latupdal\latupmim}{None}{kok:kaf_dal_mim}
\kipkokentry{mukallit}{\latupkaf\latuplam\latupdal}{None}{kok:kaf_lam_dal}
\kipkokentry{mukârenet}{\latupkaf\latupre\latupnun}{None}{kok:kaf_re_nun}
\kipkokentry{mukarrer}{\latupkaf\latupre\latupre}{None}{kok:kaf_re_re}
\kipkokentry{mukâtaa}{\latupkaf\latupta\latupayn}{None}{kok:kaf_ta_ayn}
\kipkokentry{mukâtele}{\latupkaf\latupte\latuplam}{None}{kok:kaf_te_lam}
\kipkokentry{mukâvele}{\latupkaf\latupvav\latuplam}{None}{kok:kaf_vav_lam}
\kipkokentry{mukâvemet}{\latupkaf\latupvav\latupmim}{None}{kok:kaf_vav_mim}
\kipkokentry{mukâvim}{\latupkaf\latupvav\latupmim}{None}{kok:kaf_vav_mim}
\kipkokentry{mukavvâ}{\latupkaf\latupvav\latupye}{None}{kok:kaf_vav_ye}
\kipkokentry{mukâyese}{\latupkaf\latupye\latupsin}{None}{kok:kaf_ye_sin}
\kipkokentry{mukayyet}{\latupkaf\latupye\latupdal}{None}{kok:kaf_ye_dal}
\kipkokentry{mukim}{\latupkaf\latupvav\latupmim}{None}{kok:kaf_vav_mim}
\kipkokentry{muktedir}{\latupkaf\latupdal\latupre}{None}{kok:kaf_dal_re}
\kipkokentry{muktezâ}{\latupkaf\latupdad\latupye}{None}{kok:kaf_dad_ye}
\kipkokentry{mûmin}{\latupalif\latupmim\latupnun}{None}{kok:alif_mim_nun}
\kipkokentry{mûnis}{\latupalif\latupnun\latupsin}{None}{kok:alif_nun_sin}
\kipkokentry{muntazam}{\latupnun\latupza\latupmim}{None}{kok:nun_za_mim}
\kipkokentry{muntazır}{\latupnun\latupza\latupre}{None}{kok:nun_za_re}
\kipkokentry{munzam}{\latupdad\latupmim\latupmim}{None}{kok:dad_mim_mim}
\kipkokentry{murabbâ}{\latupre\latupbe\latupvav}{None}{kok:re_be_vav}
\kipkokentry{murabbâ}{\latupre\latupbe\latupayn}{None}{kok:re_be_ayn}
\kipkokentry{murahhas}{\latupre\latupxa\latupsad}{None}{kok:re_xa_sad}
\kipkokentry{murâkebe}{\latupre\latupkaf\latupbe}{None}{kok:re_kaf_be}
\kipkokentry{murâkıp}{\latupre\latupkaf\latupbe}{None}{kok:re_kaf_be}
\kipkokentry{murassa}{\latupre\latupsad\latupayn}{None}{kok:re_sad_ayn}
\kipkokentry{murat}{\latupre\latupvav\latupdal}{None}{kok:re_vav_dal}
\kipkokentry{mûris}{\latupvav\latupre\latupthe}{None}{kok:vav_re_the}
\kipkokentry{musâhebe}{\latupsad\latupha\latupbe}{None}{kok:sad_ha_be}
\kipkokentry{musahhih}{\latupsad\latupha\latupha}{None}{kok:sad_ha_ha}
\kipkokentry{musâhip}{\latupsad\latupha\latupbe}{None}{kok:sad_ha_be}
\kipkokentry{musakka}{\latupsin\latupkaf\latupye}{None}{kok:sin_kaf_ye}
\kipkokentry{musâlaha}{\latupsad\latuplam\latupha}{None}{kok:sad_lam_ha}
\kipkokentry{musalla}{\latupsad\latuplam\latupvav}{None}{kok:sad_lam_vav}
\kipkokentry{musallat}{\latupsin\latuplam\latupta}{None}{kok:sin_lam_ta}
\kipkokentry{musannif}{\latupsad\latupnun\latupfe}{None}{kok:sad_nun_fe}
\kipkokentry{mushaf}{\latupsad\latupha\latupfe}{None}{kok:sad_ha_fe}
\kipkokentry{musır}{\latupsad\latupre\latupre$^2$}{None}{kok:sad_re_re2}
\kipkokentry{musîbet}{\latupsad\latupvav\latupbe}{None}{kok:sad_vav_be}
\kipkokentry{mustarip}{\latupdad\latupre\latupbe}{None}{kok:dad_re_be}
\kipkokentry{mustazâf}{\latupdad\latupayn\latupfe}{None}{kok:dad_ayn_fe}
\kipkokentry{muşamba}{\latupshin\latupmim\latupayn}{None}{kok:shin_mim_ayn}
\kipkokentry{mûtâ}{\latupayn\latupta\latupvav}{None}{kok:ayn_ta_vav}
\kipkokentry{mutaassıp}{\latupayn\latupsad\latupbe}{None}{kok:ayn_sad_be}
\kipkokentry{mutâbakat}{\latupta\latupbe\latupkaf}{None}{kok:ta_be_kaf}
\kipkokentry{mutâbık}{\latupta\latupbe\latupkaf}{None}{kok:ta_be_kaf}
\kipkokentry{mutantan}{\latupta\latupnun}{None}{kok:ta_nun}
\kipkokentry{mutasarrıf}{\latupsad\latupre\latupfe}{None}{kok:sad_re_fe}
\kipkokentry{mutasavver}{\latupsad\latupvav\latupre$^1$}{None}{kok:sad_vav_re1}
\kipkokentry{mutasavvıf}{\latupsad\latupvav\latupfe}{None}{kok:sad_vav_fe}
\kipkokentry{mûtat}{\latupayn\latupvav\latupdal}{None}{kok:ayn_vav_dal}
\kipkokentry{mûteber}{\latupayn\latupbe\latupre$^1$}{None}{kok:ayn_be_re1}
\kipkokentry{mûtedil}{\latupayn\latupdal\latuplam}{None}{kok:ayn_dal_lam}
\kipkokentry{mûtemet}{\latupayn\latupmim\latupdal}{None}{kok:ayn_mim_dal}
\kipkokentry{mutenâ}{\latupayn\latupnun\latupye}{None}{kok:ayn_nun_ye}
\kipkokentry{mûtezile}{\latupayn\latupze\latuplam}{None}{kok:ayn_ze_lam}
\kipkokentry{mutfak}{\latupta\latupbe\latupxa}{None}{kok:ta_be_xa}
\kipkokentry{mutî}{\latupta\latupvav\latupayn}{None}{kok:ta_vav_ayn}
\kipkokentry{mutlak}{\latupta\latuplam\latupkaf}{None}{kok:ta_lam_kaf}
\kipkokentry{mutlâkiyet}{\latupta\latuplam\latupkaf}{None}{kok:ta_lam_kaf}
\kipkokentry{mutmain}{\latupta\latupmim\latupalif\latupnun}{None}{kok:ta_mim_alif_nun}
\kipkokentry{mutrip}{\latupta\latupre\latupbe}{None}{kok:ta_re_be}
\kipkokentry{muttakî}{\latupvav\latupkaf\latupye}{None}{kok:vav_kaf_ye}
\kipkokentry{muttalî}{\latupta\latuplam\latupayn}{None}{kok:ta_lam_ayn}
\kipkokentry{muttalip}{\latupta\latuplam\latupbe}{None}{kok:ta_lam_be}
\kipkokentry{muttasıl}{\latupvav\latupsad\latuplam}{None}{kok:vav_sad_lam}
\kipkokentry{muvâcehe}{\latupvav\latupcim\latuphe}{None}{kok:vav_cim_he}
\kipkokentry{muvâfakat}{\latupvav\latupfe\latupkaf}{None}{kok:vav_fe_kaf}
\kipkokentry{muvaffak}{\latupvav\latupfe\latupkaf}{None}{kok:vav_fe_kaf}
\kipkokentry{muvâfık}{\latupvav\latupfe\latupkaf}{None}{kok:vav_fe_kaf}
\kipkokentry{muvahhit}{\latupvav\latupha\latupdal}{None}{kok:vav_ha_dal}
\kipkokentry{muvakkat}{\latupvav\latupkaf\latupte}{None}{kok:vav_kaf_te}
\kipkokentry{muvakkit}{\latupvav\latupkaf\latupte}{None}{kok:vav_kaf_te}
\kipkokentry{muvâsalat}{\latupvav\latupsad\latuplam}{None}{kok:vav_sad_lam}
\kipkokentry{muvâsıl}{\latupvav\latupsad\latuplam}{None}{kok:vav_sad_lam}
\kipkokentry{muvâzaa}{\latupvav\latupdad\latupayn}{None}{kok:vav_dad_ayn}
\kipkokentry{muvâzene}{\latupvav\latupze\latupnun}{None}{kok:vav_ze_nun}
\kipkokentry{muvâzi}{\latupvav\latupze\latupye}{None}{kok:vav_ze_ye}
\kipkokentry{muvazzaf}{\latupvav\latupza\latupfe}{None}{kok:vav_za_fe}
\kipkokentry{muzaf}{\latupdad\latupye\latupfe}{None}{kok:dad_ye_fe}
\kipkokentry{muzaffer}{\latupza\latupfe\latupre}{None}{kok:za_fe_re}
\kipkokentry{muzır}{\latupdad\latupre\latupre}{None}{kok:dad_re_re}
\kipkokentry{mûzip}{\latupayn\latupzel\latupbe}{None}{kok:ayn_zel_be}
\kipkokentry{mübâdele}{\latupbe\latupdal\latuplam}{None}{kok:be_dal_lam}
\kipkokentry{mübâdil}{\latupbe\latupdal\latuplam}{None}{kok:be_dal_lam}
\kipkokentry{mübah}{\latupbe\latupvav\latupha}{None}{kok:be_vav_ha}
\kipkokentry{mübalağa}{\latupbe\latuplam\latupgayn}{None}{kok:be_lam_gayn}
\kipkokentry{mübârek}{\latupbe\latupre\latupkef}{None}{kok:be_re_kef}
\kipkokentry{mübâreze}{\latupbe\latupre\latupze}{None}{kok:be_re_ze}
\kipkokentry{mübâriz}{\latupbe\latupre\latupze}{None}{kok:be_re_ze}
\kipkokentry{mübâşir}{\latupbe\latupshin\latupre}{None}{kok:be_shin_re}
\kipkokentry{mübâyaa}{\latupbe\latupye\latupayn}{None}{kok:be_ye_ayn}
\kipkokentry{mübeccel}{\latupbe\latupcim\latuplam}{None}{kok:be_cim_lam}
\kipkokentry{müberrâ}{\latupbe\latupre\latupalif}{None}{kok:be_re_alif}
\kipkokentry{mübin}{\latupbe\latupye\latupnun}{None}{kok:be_ye_nun}
\kipkokentry{mücâdele}{\latupcim\latupdal\latuplam}{None}{kok:cim_dal_lam}
\kipkokentry{mücâhede}{\latupcim\latuphe\latupdal}{None}{kok:cim_he_dal}
\kipkokentry{mücâhit}{\latupcim\latuphe\latupdal}{None}{kok:cim_he_dal}
\kipkokentry{mücâvir}{\latupcim\latupvav\latupre}{None}{kok:cim_vav_re}
\kipkokentry{mücâzat}{\latupcim\latupze\latupye}{None}{kok:cim_ze_ye}
\kipkokentry{mücbir}{\latupcim\latupbe\latupre}{None}{kok:cim_be_re}
\kipkokentry{müceddit}{\latupcim\latupdal\latupdal$^1$}{None}{kok:cim_dal_dal1}
\kipkokentry{mücehhez}{\latupcim\latuphe\latupze}{None}{kok:cim_he_ze}
\kipkokentry{mücellit}{\latupcim\latuplam\latupdal}{None}{kok:cim_lam_dal}
\kipkokentry{mücerret}{\latupcim\latupre\latupdal}{None}{kok:cim_re_dal}
\kipkokentry{mücessem}{\latupcim\latupsin\latupmim}{None}{kok:cim_sin_mim}
\kipkokentry{mücevher}{\latupcim\latupvav\latuphe\latupre}{None}{kok:cim_vav_he_re}
\kipkokentry{mücrim}{\latupcim\latupre\latupmim}{None}{kok:cim_re_mim}
\kipkokentry{müçtehit}{\latupcim\latuphe\latupdal}{None}{kok:cim_he_dal}
\kipkokentry{müdâfaa}{\latupdal\latupfe\latupayn}{None}{kok:dal_fe_ayn}
\kipkokentry{müdâfî}{\latupdal\latupfe\latupayn}{None}{kok:dal_fe_ayn}
\kipkokentry{müdâhale}{\latupdal\latupxa\latuplam}{None}{kok:dal_xa_lam}
\kipkokentry{müdâhene}{\latupdal\latuphe\latupnun}{None}{kok:dal_he_nun}
\kipkokentry{müdâhil}{\latupdal\latupxa\latuplam}{None}{kok:dal_xa_lam}
\kipkokentry{müdâvim}{\latupdal\latupvav\latupmim}{None}{kok:dal_vav_mim}
\kipkokentry{müddeî}{\latupdal\latupayn\latupvav}{None}{kok:dal_ayn_vav}
\kipkokentry{müddet}{\latupmim\latupdal\latupdal$^1$}{None}{kok:mim_dal_dal1}
\kipkokentry{müdebbir}{\latupdal\latupbe\latupre}{None}{kok:dal_be_re}
\kipkokentry{müdellel}{\latupdal\latuplam\latuplam}{None}{kok:dal_lam_lam}
\kipkokentry{müderris}{\latupdal\latupre\latupsin}{None}{kok:dal_re_sin}
\kipkokentry{müdevver}{\latupdal\latupvav\latupre}{None}{kok:dal_vav_re}
\kipkokentry{müdrik}{\latupdal\latupre\latupkef}{None}{kok:dal_re_kef}
\kipkokentry{müdür}{\latupdal\latupvav\latupre}{None}{kok:dal_vav_re}
\kipkokentry{müebbet}{\latupalif\latupbe\latupdal}{None}{kok:alif_be_dal}
\kipkokentry{müeccel}{\latupalif\latupcim\latuplam}{None}{kok:alif_cim_lam}
\kipkokentry{müeddep}{\latupalif\latupdal\latupbe}{None}{kok:alif_dal_be}
\kipkokentry{müellif}{\latupalif\latuplam\latupfe}{None}{kok:alif_lam_fe}
\kipkokentry{müennes}{\latupnun\latupsin\latupvav}{None}{kok:nun_sin_vav}
\kipkokentry{müesses}{\latupalif\latupsin\latupsin}{None}{kok:alif_sin_sin}
\kipkokentry{müessese}{\latupalif\latupsin\latupsin}{None}{kok:alif_sin_sin}
\kipkokentry{müessif}{\latupalif\latupsin\latupfe}{None}{kok:alif_sin_fe}
\kipkokentry{müessir}{\latupalif\latupthe\latupre}{None}{kok:alif_the_re}
\kipkokentry{müeyyide}{\latupalif\latupye\latupdal}{None}{kok:alif_ye_dal}
\kipkokentry{müezzin}{\latupalif\latupzel\latupnun}{None}{kok:alif_zel_nun}
\kipkokentry{müfessir}{\latupfe\latupsin\latupre}{None}{kok:fe_sin_re}
\kipkokentry{müfettiş}{\latupfe\latupte\latupshin}{None}{kok:fe_te_shin}
\kipkokentry{müfit}{\latupfe\latupye\latupdal}{None}{kok:fe_ye_dal}
\kipkokentry{müflis}{\latupfe\latuplam\latupsin}{None}{kok:fe_lam_sin}
\kipkokentry{müfredat}{\latupfe\latupre\latupdal}{None}{kok:fe_re_dal}
\kipkokentry{müfreze}{\latupfe\latupre\latupze}{None}{kok:fe_re_ze}
\kipkokentry{müfrit}{\latupfe\latupre\latupta}{None}{kok:fe_re_ta}
\kipkokentry{müfsit}{\latupfe\latupsin\latupdal}{None}{kok:fe_sin_dal}
\kipkokentry{müftehir}{\latupfe\latupxa\latupre}{None}{kok:fe_xa_re}
\kipkokentry{müfteri}{\latupfe\latupre\latupye}{None}{kok:fe_re_ye}
\kipkokentry{müftü}{\latupfe\latupte\latupvav}{None}{kok:fe_te_vav}
\kipkokentry{mühendis}{\latuphe\latupnun\latupdal\latupsin}{None}{kok:he_nun_dal_sin}
\kipkokentry{müheyyâ}{\latuphe\latupye\latupalif}{None}{kok:he_ye_alif}
\kipkokentry{mühim}{\latuphe\latupmim\latupmim}{None}{kok:he_mim_mim}
\kipkokentry{mühimmat}{\latuphe\latupmim\latupmim}{None}{kok:he_mim_mim}
\kipkokentry{mühlet}{\latupmim\latuphe\latuplam}{None}{kok:mim_he_lam}
\kipkokentry{mühtedî}{\latuphe\latupdal\latupye}{None}{kok:he_dal_ye}
\kipkokentry{mükâfat}{\latupkef\latupfe\latupvav}{None}{kok:kef_fe_vav}
\kipkokentry{mükâleme}{\latupkef\latuplam\latupmim}{None}{kok:kef_lam_mim}
\kipkokentry{mükedder}{\latupkef\latupdal\latupre}{None}{kok:kef_dal_re}
\kipkokentry{mükellef}{\latupkef\latuplam\latupfe}{None}{kok:kef_lam_fe}
\kipkokentry{mükemmel}{\latupkef\latupmim\latuplam}{None}{kok:kef_mim_lam}
\kipkokentry{mükerrem}{\latupkef\latupre\latupmim}{None}{kok:kef_re_mim}
\kipkokentry{mükerrer}{\latupkef\latupre\latupre}{None}{kok:kef_re_re}
\kipkokentry{mükrim}{\latupkef\latupre\latupmim}{None}{kok:kef_re_mim}
\kipkokentry{müktesep}{\latupkef\latupsin\latupbe}{None}{kok:kef_sin_be}
\kipkokentry{mülâhaza}{\latuplam\latupha\latupza}{None}{kok:lam_ha_za}
\kipkokentry{mülâkat}{\latuplam\latupkaf\latupye}{None}{kok:lam_kaf_ye}
\kipkokentry{mülâyim}{\latuplam\latupalif\latupmim}{None}{kok:lam_alif_mim}
\kipkokentry{mülâzım}{\latuplam\latupze\latupmim}{None}{kok:lam_ze_mim}
\kipkokentry{mülevves}{\latuplam\latupvav\latupthe}{None}{kok:lam_vav_the}
\kipkokentry{mülgâ}{\latuplam\latupgayn\latupvav}{None}{kok:lam_gayn_vav}
\kipkokentry{mülhak}{\latuplam\latupha\latupkaf}{None}{kok:lam_ha_kaf}
\kipkokentry{mülhem}{\latuplam\latuphe\latupmim}{None}{kok:lam_he_mim}
\kipkokentry{mülhit}{\latuplam\latupha\latupdal}{None}{kok:lam_ha_dal}
\kipkokentry{mülk}{\latupmim\latuplam\latupkef}{None}{kok:mim_lam_kef}
\kipkokentry{mülkî}{\latupmim\latuplam\latupkef}{None}{kok:mim_lam_kef}
\kipkokentry{mülkiyet}{\latupmim\latuplam\latupkef}{None}{kok:mim_lam_kef}
\kipkokentry{mültecî}{\latuplam\latupcim\latupalif}{None}{kok:lam_cim_alif}
\kipkokentry{mültefit}{\latuplam\latupfe\latupte}{None}{kok:lam_fe_te}
\kipkokentry{mültezim}{\latuplam\latupze\latupmim}{None}{kok:lam_ze_mim}
\kipkokentry{mümâsil}{\latupmim\latupthe\latuplam}{None}{kok:mim_the_lam}
\kipkokentry{mümbit}{\latupnun\latupbe\latupte}{None}{kok:nun_be_te}
\kipkokentry{mümessil}{\latupmim\latupthe\latuplam}{None}{kok:mim_the_lam}
\kipkokentry{mümeyyiz}{\latupmim\latupye\latupze}{None}{kok:mim_ye_ze}
\kipkokentry{mümkün}{\latupmim\latupkef\latupnun}{None}{kok:mim_kef_nun}
\kipkokentry{mümtaz}{\latupmim\latupye\latupze}{None}{kok:mim_ye_ze}
\kipkokentry{mümtenî}{\latupmim\latupnun\latupayn}{None}{kok:mim_nun_ayn}
\kipkokentry{mümzî}{\latupmim\latupdad\latupye}{None}{kok:mim_dad_ye}
\kipkokentry{münâcat}{\latupnun\latupcim\latupvav}{None}{kok:nun_cim_vav}
\kipkokentry{münâdi}{\latupnun\latupdal\latupvav}{None}{kok:nun_dal_vav}
\kipkokentry{münâfıl}{\latupnun\latupfe\latupkaf}{None}{kok:nun_fe_kaf}
\kipkokentry{münâkaşa}{\latupnun\latupkaf\latupshin}{None}{kok:nun_kaf_shin}
\kipkokentry{münâsebet}{\latupnun\latupsin\latupbe}{None}{kok:nun_sin_be}
\kipkokentry{münâsip}{\latupnun\latupsin\latupbe}{None}{kok:nun_sin_be}
\kipkokentry{münâvebe}{\latupnun\latupvav\latupbe}{None}{kok:nun_vav_be}
\kipkokentry{münâzaa}{\latupnun\latupze\latupayn}{None}{kok:nun_ze_ayn}
\kipkokentry{münâzara}{\latupnun\latupza\latupre}{None}{kok:nun_za_re}
\kipkokentry{müncer}{\latupcim\latupre\latupre}{None}{kok:cim_re_re}
\kipkokentry{müncî}{\latupnun\latupcim\latupvav}{None}{kok:nun_cim_vav}
\kipkokentry{mündemiç}{\latupdal\latupmim\latupcim}{None}{kok:dal_mim_cim}
\kipkokentry{münderecat}{\latupdal\latupre\latupcim$^1$}{None}{kok:dal_re_cim1}
\kipkokentry{müneccim}{\latupnun\latupcim\latupmim}{None}{kok:nun_cim_mim}
\kipkokentry{münekkit}{\latupnun\latupkaf\latupdal}{None}{kok:nun_kaf_dal}
\kipkokentry{münevver}{\latupnun\latupvav\latupre}{None}{kok:nun_vav_re}
\kipkokentry{münezzeh}{\latupnun\latupze\latuphe}{None}{kok:nun_ze_he}
\kipkokentry{münferit}{\latupfe\latupre\latupdal}{None}{kok:fe_re_dal}
\kipkokentry{münfesih}{\latupfe\latupsin\latupxa}{None}{kok:fe_sin_xa}
\kipkokentry{münhal}{\latupha\latuplam\latuplam}{None}{kok:ha_lam_lam}
\kipkokentry{münhâsır}{\latupha\latupsad\latupre}{None}{kok:ha_sad_re}
\kipkokentry{münim}{\latupnun\latupayn\latupmim}{None}{kok:nun_ayn_mim}
\kipkokentry{münkalip}{\latupkaf\latuplam\latupbe$^1$}{None}{kok:kaf_lam_be1}
\kipkokentry{münkir}{\latupnun\latupkef\latupre}{None}{kok:nun_kef_re}
\kipkokentry{münşeat}{\latupnun\latupshin\latupalif}{None}{kok:nun_shin_alif}
\kipkokentry{müntehâ}{\latupnun\latuphe\latupvav}{None}{kok:nun_he_vav}
\kipkokentry{müntehip}{\latupnun\latupxa\latupbe}{None}{kok:nun_xa_be}
\kipkokentry{müntesip}{\latupnun\latupsin\latupbe}{None}{kok:nun_sin_be}
\kipkokentry{münteşir}{\latupnun\latupshin\latupre}{None}{kok:nun_shin_re}
\kipkokentry{münzevî}{\latupze\latupvav\latupye}{None}{kok:ze_vav_ye}
\kipkokentry{müphem}{\latupbe\latuphe\latupmim}{None}{kok:be_he_mim}
\kipkokentry{müptedî}{\latupbe\latupdal\latupalif}{None}{kok:be_dal_alif}
\kipkokentry{müptelâ}{\latupbe\latuplam\latupvav}{None}{kok:be_lam_vav}
\kipkokentry{müptezel}{\latupbe\latupzel\latuplam}{None}{kok:be_zel_lam}
\kipkokentry{mürâcaat}{\latupre\latupcim\latupayn}{None}{kok:re_cim_ayn}
\kipkokentry{mürafaa}{\latupre\latupfe\latupayn}{None}{kok:re_fe_ayn}
\kipkokentry{mürâi}{\latupre\latupalif\latupye}{None}{kok:re_alif_ye}
\kipkokentry{mürebbî}{\latupre\latupbe\latupvav}{None}{kok:re_be_vav}
\kipkokentry{müreccah}{\latupre\latupcim\latupha}{None}{kok:re_cim_ha}
\kipkokentry{müreffeh}{\latupre\latupfe\latuphe}{None}{kok:re_fe_he}
\kipkokentry{mürekkep}{\latupre\latupkef\latupbe}{None}{kok:re_kef_be}
\kipkokentry{mürettep}{\latupre\latupte\latupbe}{None}{kok:re_te_be}
\kipkokentry{mürettip}{\latupre\latupte\latupbe}{None}{kok:re_te_be}
\kipkokentry{mürevviç}{\latupre\latupvav\latupcim}{None}{kok:re_vav_cim}
\kipkokentry{mürit}{\latupre\latupvav\latupdal}{None}{kok:re_vav_dal}
\kipkokentry{mürsel}{\latupre\latupsin\latuplam}{None}{kok:re_sin_lam}
\kipkokentry{mürşit}{\latupre\latupshin\latupdal}{None}{kok:re_shin_dal}
\kipkokentry{mürtecî}{\latupre\latupcim\latupayn}{None}{kok:re_cim_ayn}
\kipkokentry{mürted}{\latupre\latupdal\latupdal}{None}{kok:re_dal_dal}
\kipkokentry{mürur}{\latupmim\latupre\latupre}{None}{kok:mim_re_re}
\kipkokentry{mürüvvet}{\latupmim\latupre\latupvav}{None}{kok:mim_re_vav}
\kipkokentry{müsaade}{\latupsin\latupayn\latupdal}{None}{kok:sin_ayn_dal}
\kipkokentry{müsâbaka}{\latupsin\latupbe\latupkaf}{None}{kok:sin_be_kaf}
\kipkokentry{müsâbık}{\latupsin\latupbe\latupkaf}{None}{kok:sin_be_kaf}
\kipkokentry{müsâdeme}{\latupsad\latupdal\latupmim}{None}{kok:sad_dal_mim}
\kipkokentry{müsâdere}{\latupsad\latupdal\latupre}{None}{kok:sad_dal_re}
\kipkokentry{müsâit}{\latupsin\latupayn\latupdal}{None}{kok:sin_ayn_dal}
\kipkokentry{müsâleme}{\latupsin\latuplam\latupmim}{None}{kok:sin_lam_mim}
\kipkokentry{müsâmaha}{\latupsin\latupmim\latupha}{None}{kok:sin_mim_ha}
\kipkokentry{müsâmere}{\latupsin\latupmim\latupre$^1$}{None}{kok:sin_mim_re1}
\kipkokentry{müsâvat}{\latupsin\latupvav\latupye}{None}{kok:sin_vav_ye}
\kipkokentry{müsâvi}{\latupsin\latupvav\latupye}{None}{kok:sin_vav_ye}
\kipkokentry{müsebbip}{\latupsin\latupbe\latupbe}{None}{kok:sin_be_be}
\kipkokentry{müseccel}{\latupsin\latupcim\latuplam}{None}{kok:sin_cim_lam}
\kipkokentry{müseddes}{\latupsin\latupdal\latupsin}{None}{kok:sin_dal_sin}
\kipkokentry{müsekkin}{\latupsin\latupkef\latupnun}{None}{kok:sin_kef_nun}
\kipkokentry{müsellah}{\latupsin\latuplam\latupha}{None}{kok:sin_lam_ha}
\kipkokentry{müsellem}{\latupsin\latuplam\latupmim}{None}{kok:sin_lam_mim}
\kipkokentry{müselles}{\latupthe\latuplam\latupthe}{None}{kok:the_lam_the}
\kipkokentry{müsemmâ}{\latupsin\latupmim\latupye}{None}{kok:sin_mim_ye}
\kipkokentry{müshil}{\latupsin\latuphe\latuplam}{None}{kok:sin_he_lam}
\kipkokentry{müskirat}{\latupsin\latupkef\latupre}{None}{kok:sin_kef_re}
\kipkokentry{müslim}{\latupsin\latuplam\latupmim}{None}{kok:sin_lam_mim}
\kipkokentry{müsmir}{\latupsin\latupmim\latupre$^2$}{None}{kok:sin_mim_re2}
\kipkokentry{müsnet}{\latupsin\latupnun\latupdal}{None}{kok:sin_nun_dal}
\kipkokentry{müspet}{\latupthe\latupbe\latupte}{None}{kok:the_be_te}
\kipkokentry{müsrif}{\latupsin\latupre\latupfe}{None}{kok:sin_re_fe}
\kipkokentry{müstacel}{\latupayn\latupcim\latuplam}{None}{kok:ayn_cim_lam}
\kipkokentry{müstafî}{\latupayn\latupfe\latupvav}{None}{kok:ayn_fe_vav}
\kipkokentry{müstağnî}{\latupgayn\latupnun\latupye$^1$}{None}{kok:gayn_nun_ye1}
\kipkokentry{müstağrak}{\latupgayn\latupre\latupkaf}{None}{kok:gayn_re_kaf}
\kipkokentry{müstahdem}{\latupxa\latupdal\latupmim}{None}{kok:xa_dal_mim}
\kipkokentry{müstahkem}{\latupha\latupkef\latupmim}{None}{kok:ha_kef_mim}
\kipkokentry{müstahsil}{\latupha\latupsad\latuplam}{None}{kok:ha_sad_lam}
\kipkokentry{müstahzar}{\latupha\latupdad\latupre}{None}{kok:ha_dad_re}
\kipkokentry{müstakbel}{\latupkaf\latupbe\latuplam}{None}{kok:kaf_be_lam}
\kipkokentry{müstakil}{\latupkaf\latuplam\latuplam}{None}{kok:kaf_lam_lam}
\kipkokentry{müstakim}{\latupkaf\latupvav\latupmim}{None}{kok:kaf_vav_mim}
\kipkokentry{müstamel}{\latupayn\latupmim\latuplam}{None}{kok:ayn_mim_lam}
\kipkokentry{müstantik}{\latupnun\latupta\latupkaf}{None}{kok:nun_ta_kaf}
\kipkokentry{müstatil}{\latupta\latupvav\latuplam}{None}{kok:ta_vav_lam}
\kipkokentry{müstear}{\latupayn\latupvav\latupre}{None}{kok:ayn_vav_re}
\kipkokentry{müstebit}{\latupbe\latupdal\latupdal}{None}{kok:be_dal_dal}
\kipkokentry{müstecap}{\latupcim\latupvav\latupbe}{None}{kok:cim_vav_be}
\kipkokentry{müstecir}{\latupalif\latupcim\latupre}{None}{kok:alif_cim_re}
\kipkokentry{müstefit}{\latupfe\latupye\latupdal}{None}{kok:fe_ye_dal}
\kipkokentry{müstefreşe}{\latupfe\latupre\latupshin}{None}{kok:fe_re_shin}
\kipkokentry{müstehak}{\latupha\latupkaf\latupkaf}{None}{kok:ha_kaf_kaf}
\kipkokentry{müstehcen}{\latuphe\latupcim\latupnun}{None}{kok:he_cim_nun}
\kipkokentry{müstehlik}{\latuphe\latuplam\latupkaf}{None}{kok:he_lam_kaf}
\kipkokentry{müstehzî}{\latuphe\latupze\latupalif}{None}{kok:he_ze_alif}
\kipkokentry{müstekreh}{\latupkef\latupre\latuphe}{None}{kok:kef_re_he}
\kipkokentry{müstemleke}{\latupmim\latuplam\latupkef}{None}{kok:mim_lam_kef}
\kipkokentry{müstenit}{\latupsin\latupnun\latupdal}{None}{kok:sin_nun_dal}
\kipkokentry{müstensih}{\latupnun\latupsin\latupxa}{None}{kok:nun_sin_xa}
\kipkokentry{müsterih}{\latupre\latupvav\latupha}{None}{kok:re_vav_ha}
\kipkokentry{müstesnâ}{\latupthe\latupnun\latupye}{None}{kok:the_nun_ye}
\kipkokentry{müsteşâr}{\latupshin\latupvav\latupre}{None}{kok:shin_vav_re}
\kipkokentry{müsteşrik}{\latupshin\latupre\latupkaf}{None}{kok:shin_re_kaf}
\kipkokentry{müstevlî}{\latupvav\latuplam\latupye}{None}{kok:vav_lam_ye}
\kipkokentry{müstezât}{\latupze\latupye\latupdal}{None}{kok:ze_ye_dal}
\kipkokentry{müsvedde}{\latupsin\latupvav\latupdal}{None}{kok:sin_vav_dal}
\kipkokentry{müşâbih}{\latupshin\latupbe\latupha}{None}{kok:shin_be_ha}
\kipkokentry{müşâhede}{\latupshin\latuphe\latupdal}{None}{kok:shin_he_dal}
\kipkokentry{müşahhas}{\latupshin\latupxa\latupsad}{None}{kok:shin_xa_sad}
\kipkokentry{müşâhit}{\latupshin\latuphe\latupdal}{None}{kok:shin_he_dal}
\kipkokentry{müşâvere}{\latupshin\latupvav\latupre}{None}{kok:shin_vav_re}
\kipkokentry{müşâvir}{\latupshin\latupvav\latupre}{None}{kok:shin_vav_re}
\kipkokentry{müşebbeh}{\latupshin\latupbe\latupha}{None}{kok:shin_be_ha}
\kipkokentry{müşerref}{\latupshin\latupre\latupfe}{None}{kok:shin_re_fe}
\kipkokentry{müşevveş}{\latupshin\latupvav\latupshin}{None}{kok:shin_vav_shin}
\kipkokentry{müşevvik}{\latupshin\latupvav\latupkaf}{None}{kok:shin_vav_kaf}
\kipkokentry{müşfik}{\latupshin\latupfe\latupkaf}{None}{kok:shin_fe_kaf}
\kipkokentry{müşir}{\latupshin\latupvav\latupre}{None}{kok:shin_vav_re}
\kipkokentry{müşkül}{\latupshin\latupkef\latuplam$^1$}{None}{kok:shin_kef_lam1}
\kipkokentry{müşrik}{\latupshin\latupre\latupkef}{None}{kok:shin_re_kef}
\kipkokentry{müştâk}{\latupshin\latupvav\latupkaf}{None}{kok:shin_vav_kaf}
\kipkokentry{müştak}{\latupshin\latupkaf\latupvav$^2$}{None}{kok:shin_kaf_vav2}
\kipkokentry{müştak}{\latupshin\latupkaf\latupkaf}{None}{kok:shin_kaf_kaf}
\kipkokentry{müştekî}{\latupshin\latupkaf\latupvav$^2$}{None}{kok:shin_kaf_vav2}
\kipkokentry{müştemilat}{\latupshin\latupmim\latuplam}{None}{kok:shin_mim_lam}
\kipkokentry{müşterek}{\latupshin\latupre\latupkef}{None}{kok:shin_re_kef}
\kipkokentry{müşteri}{\latupshin\latupre\latupye}{None}{kok:shin_re_ye}
\kipkokentry{mütâlaa}{\latupta\latuplam\latupayn}{None}{kok:ta_lam_ayn}
\kipkokentry{mütâreke}{\latupte\latupre\latupkef}{None}{kok:te_re_kef}
\kipkokentry{müteaddit}{\latupayn\latupdal\latupdal}{None}{kok:ayn_dal_dal}
\kipkokentry{müteahhit}{\latupayn\latuphe\latupdal}{None}{kok:ayn_he_dal}
\kipkokentry{müteâkip}{\latupayn\latupkaf\latupbe}{None}{kok:ayn_kaf_be}
\kipkokentry{müteal}{\latupayn\latuplam\latupvav}{None}{kok:ayn_lam_vav}
\kipkokentry{müteallik}{\latupayn\latuplam\latupkaf}{None}{kok:ayn_lam_kaf}
\kipkokentry{mütearife}{\latupayn\latupre\latupfe}{None}{kok:ayn_re_fe}
\kipkokentry{mütebahhir}{\latupbe\latupha\latupre$^1$}{None}{kok:be_ha_re1}
\kipkokentry{mütebâki}{\latupbe\latupkaf\latupye}{None}{kok:be_kaf_ye}
\kipkokentry{mütebessim}{\latupbe\latupsin\latupmim}{None}{kok:be_sin_mim}
\kipkokentry{mütecânis}{\latupcim\latupnun\latupsin}{None}{kok:cim_nun_sin}
\kipkokentry{mütecâviz}{\latupcim\latupvav\latupze}{None}{kok:cim_vav_ze}
\kipkokentry{mütecessis}{\latupcim\latupsin\latupsin}{None}{kok:cim_sin_sin}
\kipkokentry{mütedeyyin}{\latupdal\latupye\latupnun$^2$}{None}{kok:dal_ye_nun2}
\kipkokentry{müteessif}{\latupalif\latupsin\latupfe}{None}{kok:alif_sin_fe}
\kipkokentry{müteessir}{\latupalif\latupthe\latupre}{None}{kok:alif_the_re}
\kipkokentry{mütefekkir}{\latupfe\latupkef\latupre}{None}{kok:fe_kef_re}
\kipkokentry{müteferrik}{\latupfe\latupre\latupkaf}{None}{kok:fe_re_kaf}
\kipkokentry{mütefessih}{\latupfe\latupsin\latupxa}{None}{kok:fe_sin_xa}
\kipkokentry{mütegallibe}{\latupgayn\latuplam\latupbe}{None}{kok:gayn_lam_be}
\kipkokentry{mütehakkim}{\latupha\latupkef\latupmim}{None}{kok:ha_kef_mim}
\kipkokentry{mütehammil}{\latupha\latupmim\latuplam}{None}{kok:ha_mim_lam}
\kipkokentry{müteharrik}{\latupha\latupre\latupkef}{None}{kok:ha_re_kef}
\kipkokentry{mütehassıs}{\latupxa\latupsad\latupsad}{None}{kok:xa_sad_sad}
\kipkokentry{mütehassis}{\latupha\latupsin\latupsin}{None}{kok:ha_sin_sin}
\kipkokentry{mütehavvil}{\latupha\latupvav\latuplam}{None}{kok:ha_vav_lam}
\kipkokentry{mütekâbil}{\latupkaf\latupbe\latuplam}{None}{kok:kaf_be_lam}
\kipkokentry{mütekâit}{\latupkaf\latupayn\latupdal}{None}{kok:kaf_ayn_dal}
\kipkokentry{mütekâmil}{\latupkef\latupmim\latuplam}{None}{kok:kef_mim_lam}
\kipkokentry{mütekebbir}{\latupkef\latupbe\latupre}{None}{kok:kef_be_re}
\kipkokentry{mütekellim}{\latupkef\latuplam\latupmim}{None}{kok:kef_lam_mim}
\kipkokentry{mütelezziz}{\latuplam\latupzel\latupzel}{None}{kok:lam_zel_zel}
\kipkokentry{mütemâdi}{\latupmim\latupdal\latupye}{None}{kok:mim_dal_ye}
\kipkokentry{mütemâyil}{\latupmim\latupye\latuplam}{None}{kok:mim_ye_lam}
\kipkokentry{mütemâyiz}{\latupmim\latupye\latupze}{None}{kok:mim_ye_ze}
\kipkokentry{mütemeddin}{\latupmim\latupdal\latupnun}{None}{kok:mim_dal_nun}
\kipkokentry{mütemmim}{\latupte\latupmim\latupmim}{None}{kok:te_mim_mim}
\kipkokentry{mütenâhî}{\latupnun\latuphe\latupvav}{None}{kok:nun_he_vav}
\kipkokentry{mütenâsip}{\latupnun\latupsin\latupbe}{None}{kok:nun_sin_be}
\kipkokentry{müterâdif}{\latupre\latupdal\latupfe}{None}{kok:re_dal_fe}
\kipkokentry{müterâkim}{\latupre\latupkef\latupmim}{None}{kok:re_kef_mim}
\kipkokentry{mütercim}{\latupte\latupre\latupcim\latupmim}{None}{kok:te_re_cim_mim}
\kipkokentry{mütereddit}{\latupre\latupdal\latupdal}{None}{kok:re_dal_dal}
\kipkokentry{mütesâdif}{\latupsad\latupdal\latupfe$^1$}{None}{kok:sad_dal_fe1}
\kipkokentry{müteselsil}{\latupsin\latuplam}{None}{kok:sin_lam}
\kipkokentry{müteşâir}{\latupshin\latupayn\latupre}{None}{kok:shin_ayn_re}
\kipkokentry{müteşebbis}{\latupshin\latupbe\latupthe}{None}{kok:shin_be_the}
\kipkokentry{müteşekkil}{\latupshin\latupkef\latuplam$^2$}{None}{kok:shin_kef_lam2}
\kipkokentry{müteşekkir}{\latupshin\latupkef\latupre}{None}{kok:shin_kef_re}
\kipkokentry{mütevâzı}{\latupvav\latupdad\latupayn}{None}{kok:vav_dad_ayn}
\kipkokentry{müteveccih}{\latupvav\latupcim\latuphe}{None}{kok:vav_cim_he}
\kipkokentry{müteveffâ}{\latupvav\latupfe\latupye}{None}{kok:vav_fe_ye}
\kipkokentry{mütevekkil}{\latupvav\latupkef\latuplam}{None}{kok:vav_kef_lam}
\kipkokentry{mütevelli}{\latupvav\latuplam\latupye}{None}{kok:vav_lam_ye}
\kipkokentry{mütevellit}{\latupvav\latuplam\latupdal}{None}{kok:vav_lam_dal}
\kipkokentry{müteyakkız}{\latupye\latupkaf\latupza}{None}{kok:ye_kaf_za}
\kipkokentry{müthiş}{\latupdal\latuphe\latupshin}{None}{kok:dal_he_shin}
\kipkokentry{müttefik}{\latupvav\latupfe\latupkaf}{None}{kok:vav_fe_kaf}
\kipkokentry{müttehit}{\latupvav\latupha\latupdal}{None}{kok:vav_ha_dal}
\kipkokentry{müvekkil}{\latupvav\latupkef\latuplam}{None}{kok:vav_kef_lam}
\kipkokentry{müvellet}{\latupvav\latuplam\latupdal}{None}{kok:vav_lam_dal}
\kipkokentry{müverrih}{\latupvav\latupre\latupxa}{None}{kok:vav_re_xa}
\kipkokentry{müvezzî}{\latupvav\latupze\latupayn}{None}{kok:vav_ze_ayn}
\kipkokentry{müyesser}{\latupye\latupsin\latupre}{None}{kok:ye_sin_re}
\kipkokentry{müzâheret}{\latupza\latuphe\latupre}{None}{kok:za_he_re}
\kipkokentry{müzâhir}{\latupza\latuphe\latupre}{None}{kok:za_he_re}
\kipkokentry{müzâkere}{\latupzel\latupkef\latupre$^1$}{None}{kok:zel_kef_re1}
\kipkokentry{müzâyaka}{\latupdad\latupye\latupkaf}{None}{kok:dad_ye_kaf}
\kipkokentry{müzâyede}{\latupze\latupye\latupdal}{None}{kok:ze_ye_dal}
\kipkokentry{müzehhep}{\latupzel\latuphe\latupbe$^2$}{None}{kok:zel_he_be2}
\kipkokentry{müzekker}{\latupzel\latupkef\latupre$^2$}{None}{kok:zel_kef_re2}
\kipkokentry{müzekkere}{\latupzel\latupkef\latupre$^1$}{None}{kok:zel_kef_re1}
\kipkokentry{müzevvir}{\latupze\latupvav\latupre}{None}{kok:ze_vav_re}
\kipkokentry{müzeyyen}{\latupze\latupye\latupnun}{None}{kok:ze_ye_nun}
\kipkokentry{müzmin}{\latupze\latupmim\latupnun}{None}{kok:ze_mim_nun}
\end{multicols}
\dictchapter{N}
\begin{multicols}{2}
\kipkokentry{naaş}{\latupnun\latupayn\latupshin}{None}{kok:nun_ayn_shin}
\kipkokentry{nabız}{\latupnun\latupbe\latupdad}{None}{kok:nun_be_dad}
\kipkokentry{nâdim}{\latupnun\latupdal\latupmim}{K\rom{1}, Ed.}{kok:nun_dal_mim}
\kipkokentry{nâdir}{\latupnun\latupdal\latupre}{K\rom{1}, Ed.}{kok:nun_dal_re}
\kipkokentry{nafaka}{\latupnun\latupfe\latupkaf}{None}{kok:nun_fe_kaf}
\kipkokentry{nâfi}{\latupnun\latupfe\latupayn}{K\rom{1}, Ed.}{kok:nun_fe_ayn}
\kipkokentry{nâfile}{\latupnun\latupfe\latuplam}{None}{kok:nun_fe_lam}
\kipkokentry{nâfiz}{\latupnun\latupfe\latupzel}{K\rom{1}, Ed.}{kok:nun_fe_zel}
\kipkokentry{nağme}{\latupnun\latupgayn\latupmim}{None}{kok:nun_gayn_mim}
\kipkokentry{nahıl}{\latupnun\latupxa\latuplam}{None}{kok:nun_xa_lam}
\kipkokentry{nahif}{\latupnun\latupha\latupfe}{None}{kok:nun_ha_fe}
\kipkokentry{nahiv}{\latupnun\latupha\latupvav}{None}{kok:nun_ha_vav}
\kipkokentry{nâhiye}{\latupnun\latupha\latupvav}{None}{kok:nun_ha_vav}
\kipkokentry{nâil}{\latupnun\latupye\latuplam}{K\rom{1}, Ed.}{kok:nun_ye_lam}
\kipkokentry{nâip}{\latupnun\latupvav\latupbe}{K\rom{1}, Ed.}{kok:nun_vav_be}
\kipkokentry{nakarat}{\latupnun\latupkaf\latupre}{None}{kok:nun_kaf_re}
\kipkokentry{nâkıs}{\latupnun\latupkaf\latupsad}{K\rom{1}, Ed.}{kok:nun_kaf_sad}
\kipkokentry{nakıs}{\latupnun\latupkaf\latupsad}{None}{kok:nun_kaf_sad}
\kipkokentry{nakış}{\latupnun\latupkaf\latupshin}{None}{kok:nun_kaf_shin}
\kipkokentry{nakil}{\latupnun\latupkaf\latuplam}{None}{kok:nun_kaf_lam}
\kipkokentry{nakip}{\latupnun\latupkaf\latupbe}{None}{kok:nun_kaf_be}
\kipkokentry{nakîse}{\latupnun\latupkaf\latupsad}{None}{kok:nun_kaf_sad}
\kipkokentry{nakit}{\latupnun\latupkaf\latupdal}{None}{kok:nun_kaf_dal}
\kipkokentry{nakkâre}{\latupnun\latupkaf\latupre}{None}{kok:nun_kaf_re}
\kipkokentry{nakkaş}{\latupnun\latupkaf\latupshin}{None}{kok:nun_kaf_shin}
\kipkokentry{nakz}{\latupnun\latupkaf\latupdad}{None}{kok:nun_kaf_dad}
\kipkokentry{nal}{\latupnun\latupayn\latuplam}{None}{kok:nun_ayn_lam}
\kipkokentry{nalın}{\latupnun\latupayn\latuplam}{None}{kok:nun_ayn_lam}
\kipkokentry{nar}{\latupnun\latupvav\latupre}{None}{kok:nun_vav_re}
\kipkokentry{nâra}{\latupnun\latupayn\latupre}{None}{kok:nun_ayn_re}
\kipkokentry{nas}{\latupnun\latupsad\latupsad}{None}{kok:nun_sad_sad}
\kipkokentry{nasır}{\latupnun\latupsad\latupre$^2$}{None}{kok:nun_sad_re2}
\kipkokentry{nasîhat}{\latupnun\latupsad\latupha}{None}{kok:nun_sad_ha}
\kipkokentry{nasip}{\latupnun\latupsad\latupbe}{None}{kok:nun_sad_be}
\kipkokentry{nasip}{\latupnun\latupsad\latupbe}{None}{kok:nun_sad_be}
\kipkokentry{nasrânî}{\latupnun\latupsad\latupre$^1$}{None}{kok:nun_sad_re1}
\kipkokentry{nâşi}{\latupnun\latupshin\latupalif}{K\rom{1}, Ed.}{kok:nun_shin_alif}
\kipkokentry{nâtıka}{\latupnun\latupta\latupkaf}{None}{kok:nun_ta_kaf}
\kipkokentry{nâtır}{\latupnun\latupta\latupre}{K\rom{1}, Ed.}{kok:nun_ta_re}
\kipkokentry{nazar}{\latupnun\latupza\latupre}{None}{kok:nun_za_re}
\kipkokentry{nazariyat}{\latupnun\latupza\latupre}{None}{kok:nun_za_re}
\kipkokentry{nazariye}{\latupnun\latupza\latupre}{None}{kok:nun_za_re}
\kipkokentry{nâzım}{\latupnun\latupza\latupmim}{K\rom{1}, Ed.}{kok:nun_za_mim}
\kipkokentry{nazım}{\latupnun\latupza\latupmim}{None}{kok:nun_za_mim}
\kipkokentry{nâzır}{\latupnun\latupza\latupre}{K\rom{1}, Ed.}{kok:nun_za_re}
\kipkokentry{nâzil}{\latupnun\latupze\latuplam}{K\rom{1}, Ed.}{kok:nun_ze_lam}
\kipkokentry{nazîre}{\latupnun\latupza\latupre}{None}{kok:nun_za_re}
\kipkokentry{nebat}{\latupnun\latupbe\latupte}{None}{kok:nun_be_te}
\kipkokentry{nebbaş}{\latupnun\latupbe\latupshin}{None}{kok:nun_be_shin}
\kipkokentry{nebi}{\latupnun\latupbe\latupvav}{None}{kok:nun_be_vav}
\kipkokentry{nebze}{\latupnun\latupbe\latupzel}{None}{kok:nun_be_zel}
\kipkokentry{necâset}{\latupnun\latupcim\latupsin}{None}{kok:nun_cim_sin}
\kipkokentry{necat}{\latupnun\latupcim\latupvav}{None}{kok:nun_cim_vav}
\kipkokentry{neccar}{\latupnun\latupcim\latupre}{None}{kok:nun_cim_re}
\kipkokentry{necim}{\latupnun\latupcim\latupmim}{None}{kok:nun_cim_mim}
\kipkokentry{necip}{\latupnun\latupcim\latupbe}{None}{kok:nun_cim_be}
\kipkokentry{necis}{\latupnun\latupcim\latupsin}{None}{kok:nun_cim_sin}
\kipkokentry{nedâmet}{\latupnun\latupdal\latupmim}{None}{kok:nun_dal_mim}
\kipkokentry{nedîme}{\latupnun\latupdal\latupmim}{None}{kok:nun_dal_mim}
\kipkokentry{nedret}{\latupnun\latupdal\latupre}{None}{kok:nun_dal_re}
\kipkokentry{nefâset}{\latupnun\latupfe\latupsin}{None}{kok:nun_fe_sin}
\kipkokentry{nefer}{\latupnun\latupfe\latupre}{None}{kok:nun_fe_re}
\kipkokentry{nefes}{\latupnun\latupfe\latupsin}{None}{kok:nun_fe_sin}
\kipkokentry{nefis}{\latupnun\latupfe\latupsin}{None}{kok:nun_fe_sin}
\kipkokentry{nefis}{\latupnun\latupfe\latupsin}{None}{kok:nun_fe_sin}
\kipkokentry{nefiy}{\latupnun\latupfe\latupvav}{None}{kok:nun_fe_vav}
\kipkokentry{nefret}{\latupnun\latupfe\latupre}{None}{kok:nun_fe_re}
\kipkokentry{neft}{\latupnun\latupfe\latupte}{None}{kok:nun_fe_te}
\kipkokentry{nehârî}{\latupnun\latuphe\latupre$^2$}{None}{kok:nun_he_re2}
\kipkokentry{nehir}{\latupnun\latuphe\latupre$^1$}{None}{kok:nun_he_re1}
\kipkokentry{nekâhet}{\latupnun\latupkaf\latuphe}{None}{kok:nun_kaf_he}
\kipkokentry{nekre}{\latupnun\latupkef\latupre}{None}{kok:nun_kef_re}
\kipkokentry{nemâ}{\latupnun\latupmim\latupvav}{None}{kok:nun_mim_vav}
\kipkokentry{nesak}{\latupnun\latupsin\latupkaf}{None}{kok:nun_sin_kaf}
\kipkokentry{nesep}{\latupnun\latupsin\latupbe}{None}{kok:nun_sin_be}
\kipkokentry{nesih}{\latupnun\latupsin\latupxa}{None}{kok:nun_sin_xa}
\kipkokentry{nesil}{\latupnun\latupsin\latuplam}{None}{kok:nun_sin_lam}
\kipkokentry{nesim}{\latupnun\latupsin\latupmim}{None}{kok:nun_sin_mim}
\kipkokentry{nesir}{\latupnun\latupthe\latupre}{None}{kok:nun_the_re}
\kipkokentry{neşe}{\latupnun\latupshin\latupta}{None}{kok:nun_shin_ta}
\kipkokentry{neşet}{\latupnun\latupshin\latupalif}{None}{kok:nun_shin_alif}
\kipkokentry{neşid}{\latupnun\latupshin\latupdal}{None}{kok:nun_shin_dal}
\kipkokentry{neşir}{\latupnun\latupshin\latupre}{None}{kok:nun_shin_re}
\kipkokentry{netîce}{\latupnun\latupte\latupcim}{None}{kok:nun_te_cim}
\kipkokentry{nevâle}{\latupnun\latupvav\latuplam}{None}{kok:nun_vav_lam}
\kipkokentry{nevâzil}{\latupnun\latupze\latuplam}{None}{kok:nun_ze_lam}
\kipkokentry{nevî}{\latupnun\latupvav\latupayn}{None}{kok:nun_vav_ayn}
\kipkokentry{nevir}{\latupnun\latupvav\latupre}{None}{kok:nun_vav_re}
\kipkokentry{nezâfet}{\latupnun\latupza\latupfe}{None}{kok:nun_za_fe}
\kipkokentry{nezâret}{\latupnun\latupza\latupre}{None}{kok:nun_za_re}
\kipkokentry{nezih}{\latupnun\latupze\latuphe}{None}{kok:nun_ze_he}
\kipkokentry{nezir}{\latupnun\latupzel\latupre}{None}{kok:nun_zel_re}
\kipkokentry{nezle}{\latupnun\latupze\latuplam}{None}{kok:nun_ze_lam}
\kipkokentry{nısıf}{\latupnun\latupsad\latupfe}{None}{kok:nun_sad_fe}
\kipkokentry{nidâ}{\latupnun\latupdal\latupye}{None}{kok:nun_dal_ye}
\kipkokentry{nifak}{\latupnun\latupfe\latupkaf}{None}{kok:nun_fe_kaf}
\kipkokentry{nihaî}{\latupnun\latuphe\latupvav}{None}{kok:nun_he_vav}
\kipkokentry{nihâyet}{\latupnun\latuphe\latupvav}{None}{kok:nun_he_vav}
\kipkokentry{nihrir}{\latupnun\latuphe\latupre$^2$}{None}{kok:nun_he_re2}
\kipkokentry{nîmet}{\latupnun\latupayn\latupmim}{None}{kok:nun_ayn_mim}
\kipkokentry{nisâ}{\latupnun\latupsin\latupvav}{None}{kok:nun_sin_vav}
\kipkokentry{nisap}{\latupnun\latupsad\latupbe}{None}{kok:nun_sad_be}
\kipkokentry{nispet}{\latupnun\latupsin\latupbe}{None}{kok:nun_sin_be}
\kipkokentry{nisvan}{\latupnun\latupsin\latupvav}{None}{kok:nun_sin_vav}
\kipkokentry{nisyan}{\latupnun\latupsin\latupye}{None}{kok:nun_sin_ye}
\kipkokentry{niyâbet}{\latupnun\latupvav\latupbe}{None}{kok:nun_vav_be}
\kipkokentry{niyet}{\latupnun\latupvav\latupye}{None}{kok:nun_vav_ye}
\kipkokentry{nizâ}{\latupnun\latupze\latupayn}{None}{kok:nun_ze_ayn}
\kipkokentry{nizam}{\latupnun\latupza\latupmim}{None}{kok:nun_za_mim}
\kipkokentry{noksan}{\latupnun\latupkaf\latupsad}{None}{kok:nun_kaf_sad}
\kipkokentry{nokta}{\latupnun\latupkaf\latupta}{None}{kok:nun_kaf_ta}
\kipkokentry{nöbet}{\latupnun\latupvav\latupbe}{None}{kok:nun_vav_be}
\kipkokentry{nur}{\latupnun\latupvav\latupre}{None}{kok:nun_vav_re}
\kipkokentry{nush}{\latupnun\latupsad\latupha}{None}{kok:nun_sad_ha}
\kipkokentry{nusret}{\latupnun\latupsad\latupre$^1$}{None}{kok:nun_sad_re1}
\kipkokentry{nutfe}{\latupnun\latupta\latupfe}{None}{kok:nun_ta_fe}
\kipkokentry{nutuk}{\latupnun\latupta\latupkaf}{None}{kok:nun_ta_kaf}
\kipkokentry{nübüvvet}{\latupnun\latupbe\latupvav}{None}{kok:nun_be_vav}
\kipkokentry{nücum}{\latupnun\latupcim\latupmim}{None}{kok:nun_cim_mim}
\kipkokentry{nüfus}{\latupnun\latupfe\latupsin}{None}{kok:nun_fe_sin}
\kipkokentry{nüfuz}{\latupnun\latupfe\latupzel}{None}{kok:nun_fe_zel}
\kipkokentry{nüks}{\latupnun\latupkef\latupsin}{None}{kok:nun_kef_sin}
\kipkokentry{nükte}{\latupnun\latupkef\latupte}{None}{kok:nun_kef_te}
\kipkokentry{nüsha}{\latupnun\latupsin\latupxa}{None}{kok:nun_sin_xa}
\kipkokentry{nüve}{\latupnun\latupvav\latupye}{None}{kok:nun_vav_ye}
\kipkokentry{nüzul}{\latupnun\latupze\latuplam}{None}{kok:nun_ze_lam}
\end{multicols}
\dictchapter{Ö}
\begin{multicols}{2}
\kipkokentry{ömür}{\latupayn\latupmim\latupre}{None}{kok:ayn_mim_re}
\kipkokentry{örf}{\latupayn\latupre\latupfe}{None}{kok:ayn_re_fe}
\kipkokentry{öşür}{\latupayn\latupshin\latupre}{None}{kok:ayn_shin_re}
\kipkokentry{özür}{\latupayn\latupzel\latupre}{None}{kok:ayn_zel_re}
\end{multicols}
\dictchapter{P}
\begin{multicols}{2}
\kipkokentry{pusat}{\latupbe\latupsin\latupta}{None}{kok:be_sin_ta}
\end{multicols}
\dictchapter{R}
\begin{multicols}{2}
\kipkokentry{rab}{\latupre\latupbe\latupbe}{None}{kok:re_be_be}
\kipkokentry{râbıta}{\latupre\latupbe\latupta}{None}{kok:re_be_ta}
\kipkokentry{râcî}{\latupre\latupcim\latupayn}{K\rom{1}, Ed.}{kok:re_cim_ayn}
\kipkokentry{racim}{\latupre\latupcim\latupmim}{None}{kok:re_cim_mim}
\kipkokentry{radde}{\latupre\latupdal\latupdal}{None}{kok:re_dal_dal}
\kipkokentry{raf}{\latupre\latupfe\latupfe}{None}{kok:re_fe_fe}
\kipkokentry{rafızî}{\latupre\latupfe\latupdad}{None}{kok:re_fe_dad}
\kipkokentry{râgıp}{\latupre\latupgayn\latupbe}{K\rom{1}, Ed.}{kok:re_gayn_be}
\kipkokentry{rağbet}{\latupre\latupgayn\latupbe}{None}{kok:re_gayn_be}
\kipkokentry{rağmen}{\latupre\latupgayn\latupmim}{None}{kok:re_gayn_mim}
\kipkokentry{rahat}{\latupre\latupvav\latupha}{None}{kok:re_vav_ha}
\kipkokentry{rahim}{\latupre\latupha\latupmim}{None}{kok:re_ha_mim}
\kipkokentry{rahim}{\latupre\latupha\latupmim}{None}{kok:re_ha_mim}
\kipkokentry{râhip}{\latupre\latuphe\latupbe}{K\rom{1}, Ed.}{kok:re_he_be}
\kipkokentry{rahle}{\latupre\latupha\latuplam}{None}{kok:re_ha_lam}
\kipkokentry{rahmân}{\latupre\latupha\latupmim}{None}{kok:re_ha_mim}
\kipkokentry{rahmet}{\latupre\latupha\latupmim}{None}{kok:re_ha_mim}
\kipkokentry{rahmet}{\latupre\latuphe\latupmim}{None}{kok:re_he_mim}
\kipkokentry{raiyet}{\latupre\latupayn\latupye}{None}{kok:re_ayn_ye}
\kipkokentry{rakam}{\latupre\latupkaf\latupmim}{None}{kok:re_kaf_mim}
\kipkokentry{rakı}{\latupayn\latupre\latupkaf$^2$}{None}{kok:ayn_re_kaf2}
\kipkokentry{râkım}{\latupre\latupkaf\latupmim}{K\rom{1}, Ed.}{kok:re_kaf_mim}
\kipkokentry{rakik}{\latupre\latupkaf\latupkaf}{None}{kok:re_kaf_kaf}
\kipkokentry{rakip}{\latupre\latupkaf\latupbe}{None}{kok:re_kaf_be}
\kipkokentry{rakkas}{\latupre\latupkaf\latupsad}{None}{kok:re_kaf_sad}
\kipkokentry{raks}{\latupre\latupkaf\latupsad}{None}{kok:re_kaf_sad}
\kipkokentry{ramak}{\latupre\latupmim\latupkaf}{None}{kok:re_mim_kaf}
\kipkokentry{ramazan}{\latupre\latupmim\latupdad}{None}{kok:re_mim_dad}
\kipkokentry{râmiz}{\latupre\latupmim\latupze}{K\rom{1}, Ed.}{kok:re_mim_ze}
\kipkokentry{rânâ}{\latupre\latupayn\latupnun}{None}{kok:re_ayn_nun}
\kipkokentry{rapt}{\latupre\latupbe\latupta}{None}{kok:re_be_ta}
\kipkokentry{raptiye}{\latupre\latupbe\latupta}{None}{kok:re_be_ta}
\kipkokentry{rasat}{\latupre\latupsad\latupdal}{None}{kok:re_sad_dal}
\kipkokentry{râşe}{\latupre\latupayn\latupshin}{None}{kok:re_ayn_shin}
\kipkokentry{râvî}{\latupre\latupvav\latupye}{K\rom{1}, Ed.}{kok:re_vav_ye}
\kipkokentry{ravza}{\latupre\latupvav\latupdad}{None}{kok:re_vav_dad}
\kipkokentry{râyiç}{\latupre\latupvav\latupcim}{K\rom{1}, Ed.}{kok:re_vav_cim}
\kipkokentry{râyihâ}{\latupre\latupvav\latupha}{None}{kok:re_vav_ha}
\kipkokentry{râzı}{\latupre\latupdad\latupvav}{K\rom{1}, Ed.}{kok:re_dad_vav}
\kipkokentry{reâyâ}{\latupre\latupayn\latupye}{None}{kok:re_ayn_ye}
\kipkokentry{recm}{\latupre\latupcim\latupmim}{None}{kok:re_cim_mim}
\kipkokentry{recüliyet}{\latupre\latupcim\latuplam}{None}{kok:re_cim_lam}
\kipkokentry{redif}{\latupre\latupdal\latupfe}{None}{kok:re_dal_fe}
\kipkokentry{ref}{\latupre\latupfe\latupayn}{None}{kok:re_fe_ayn}
\kipkokentry{refah}{\latupre\latupfe\latuphe}{None}{kok:re_fe_he}
\kipkokentry{refâkat}{\latupre\latupfe\latupkaf}{None}{kok:re_fe_kaf}
\kipkokentry{refik}{\latupre\latupfe\latupkaf}{None}{kok:re_fe_kaf}
\kipkokentry{refika}{\latupre\latupfe\latupkaf}{None}{kok:re_fe_kaf}
\kipkokentry{regâip}{\latupre\latupgayn\latupbe}{None}{kok:re_gayn_be}
\kipkokentry{rehâvet}{\latupre\latupxa\latupvav}{None}{kok:re_xa_vav}
\kipkokentry{rehin}{\latupre\latuphe\latupnun}{None}{kok:re_he_nun}
\kipkokentry{rehîne}{\latupre\latuphe\latupnun}{None}{kok:re_he_nun}
\kipkokentry{reis}{\latupre\latupalif\latupsin}{None}{kok:re_alif_sin}
\kipkokentry{rekâbet}{\latupre\latupkaf\latupbe}{None}{kok:re_kaf_be}
\kipkokentry{rekat}{\latupre\latupkaf\latupayn}{None}{kok:re_kaf_ayn}
\kipkokentry{remil}{\latupre\latupmim\latuplam}{None}{kok:re_mim_lam}
\kipkokentry{remiz}{\latupre\latupmim\latupze}{None}{kok:re_mim_ze}
\kipkokentry{remzi}{\latupre\latupmim\latupze}{None}{kok:re_mim_ze}
\kipkokentry{resen}{\latupre\latupalif\latupsin}{None}{kok:re_alif_sin}
\kipkokentry{resif}{\latupre\latupsad\latupfe}{None}{kok:re_sad_fe}
\kipkokentry{resim}{\latupre\latupsin\latupmim}{None}{kok:re_sin_mim}
\kipkokentry{ressam}{\latupre\latupsin\latupmim}{None}{kok:re_sin_mim}
\kipkokentry{resul}{\latupre\latupsin\latuplam}{None}{kok:re_sin_lam}
\kipkokentry{reşit}{\latupre\latupshin\latupdal}{None}{kok:re_shin_dal}
\kipkokentry{ret}{\latupre\latupdal\latupdal}{None}{kok:re_dal_dal}
\kipkokentry{revaç}{\latupre\latupvav\latupcim}{None}{kok:re_vav_cim}
\kipkokentry{revak}{\latupre\latupvav\latupkaf}{None}{kok:re_vav_kaf}
\kipkokentry{rey}{\latupre\latupalif\latupye}{None}{kok:re_alif_ye}
\kipkokentry{reyhan}{\latupre\latupvav\latupha}{None}{kok:re_vav_ha}
\kipkokentry{rezâlet}{\latupre\latupzel\latuplam}{None}{kok:re_zel_lam}
\kipkokentry{reze}{\latupre\latupze\latupze}{None}{kok:re_ze_ze}
\kipkokentry{rezil}{\latupre\latupzel\latuplam}{None}{kok:re_zel_lam}
\kipkokentry{rızâ}{\latupre\latupdad\latupvav}{None}{kok:re_dad_vav}
\kipkokentry{rızk}{\latupre\latupze\latupkaf}{None}{kok:re_ze_kaf}
\kipkokentry{riâyet}{\latupre\latupayn\latupye}{None}{kok:re_ayn_ye}
\kipkokentry{ribat}{\latupre\latupbe\latupta}{None}{kok:re_be_ta}
\kipkokentry{ricâ}{\latupre\latupcim\latupvav}{None}{kok:re_cim_vav}
\kipkokentry{rical}{\latupre\latupcim\latuplam}{None}{kok:re_cim_lam}
\kipkokentry{ricat}{\latupre\latupcim\latupayn}{None}{kok:re_cim_ayn}
\kipkokentry{rikap}{\latupre\latupkef\latupbe}{None}{kok:re_kef_be}
\kipkokentry{rikkat}{\latupre\latupkaf\latupkaf}{None}{kok:re_kaf_kaf}
\kipkokentry{risâle}{\latupre\latupsin\latuplam}{None}{kok:re_sin_lam}
\kipkokentry{rivâyet}{\latupre\latupvav\latupye}{None}{kok:re_vav_ye}
\kipkokentry{riyâ}{\latupre\latupalif\latupye}{None}{kok:re_alif_ye}
\kipkokentry{riyâset}{\latupre\latupalif\latupsin}{None}{kok:re_alif_sin}
\kipkokentry{riyâziye}{\latupre\latupvav\latupdad}{None}{kok:re_vav_dad}
\kipkokentry{rubâî}{\latupre\latupbe\latupayn}{None}{kok:re_be_ayn}
\kipkokentry{rubu}{\latupre\latupbe\latupayn}{None}{kok:re_be_ayn}
\kipkokentry{ruh}{\latupre\latupvav\latupha}{None}{kok:re_vav_ha}
\kipkokentry{ruhban}{\latupre\latuphe\latupbe}{None}{kok:re_he_be}
\kipkokentry{ruhsat}{\latupre\latupxa\latupsad}{None}{kok:re_xa_sad}
\kipkokentry{rumuz}{\latupre\latupmim\latupze}{None}{kok:re_mim_ze}
\kipkokentry{rutûbet}{\latupre\latupta\latupbe}{None}{kok:re_ta_be}
\kipkokentry{rücû}{\latupre\latupcim\latupayn}{None}{kok:re_cim_ayn}
\kipkokentry{rüçhan}{\latupre\latupcim\latupha}{None}{kok:re_cim_ha}
\kipkokentry{rükn}{\latupre\latupkef\latupnun}{None}{kok:re_kef_nun}
\kipkokentry{rükû}{\latupre\latupkaf\latupayn}{None}{kok:re_kaf_ayn}
\kipkokentry{rüsum}{\latupre\latupsin\latupmim}{None}{kok:re_sin_mim}
\kipkokentry{rüşeym}{\latupre\latupshin\latupmim}{None}{kok:re_shin_mim}
\kipkokentry{rüşt}{\latupre\latupshin\latupdal}{None}{kok:re_shin_dal}
\kipkokentry{rüşvet}{\latupre\latupshin\latupvav}{None}{kok:re_shin_vav}
\kipkokentry{rütbe}{\latupre\latupte\latupbe}{None}{kok:re_te_be}
\kipkokentry{rüyâ}{\latupre\latupalif\latupye}{None}{kok:re_alif_ye}
\kipkokentry{rüyet}{\latupre\latupalif\latupye}{None}{kok:re_alif_ye}
\end{multicols}
\dictchapter{S}
\begin{multicols}{2}
\kipkokentry{saadet}{\latupsin\latupayn\latupdal}{None}{kok:sin_ayn_dal}
\kipkokentry{saat}{\latupvav\latupsin\latupayn}{None}{kok:vav_sin_ayn}
\kipkokentry{saba}{\latupsad\latupbe\latupvav}{None}{kok:sad_be_vav}
\kipkokentry{sabah}{\latupsad\latupbe\latupha}{None}{kok:sad_be_ha}
\kipkokentry{sâbıka}{\latupsin\latupbe\latupkaf}{None}{kok:sin_be_kaf}
\kipkokentry{sâbık}{\latupsin\latupbe\latupkaf}{K\rom{1}, Ed.}{kok:sin_be_kaf}
\kipkokentry{sabır}{\latupsad\latupbe\latupre}{None}{kok:sad_be_re}
\kipkokentry{sabî}{\latupsad\latupbe\latupvav}{None}{kok:sad_be_vav}
\kipkokentry{sâbit}{\latupthe\latupbe\latupte}{K\rom{1}, Ed.}{kok:the_be_te}
\kipkokentry{sâcit}{\latupsin\latupcim\latupdal}{K\rom{1}, Ed.}{kok:sin_cim_dal}
\kipkokentry{sadâ}{\latupsad\latupdal\latupye}{None}{kok:sad_dal_ye}
\kipkokentry{sadaka}{\latupsad\latupdal\latupkaf}{None}{kok:sad_dal_kaf}
\kipkokentry{sadâkat}{\latupsad\latupdal\latupkaf}{None}{kok:sad_dal_kaf}
\kipkokentry{sadâret}{\latupsad\latupdal\latupre}{None}{kok:sad_dal_re}
\kipkokentry{sadet}{\latupsad\latupdal\latupdal}{None}{kok:sad_dal_dal}
\kipkokentry{sâdık}{\latupsad\latupdal\latupkaf}{K\rom{1}, Ed.}{kok:sad_dal_kaf}
\kipkokentry{sâdır}{\latupsad\latupdal\latupre}{K\rom{1}, Ed.}{kok:sad_dal_re}
\kipkokentry{sadme}{\latupsad\latupdal\latupmim}{None}{kok:sad_dal_mim}
\kipkokentry{sadr}{\latupsad\latupdal\latupre}{None}{kok:sad_dal_re}
\kipkokentry{saf}{\latupsad\latupfe\latupfe}{None}{kok:sad_fe_fe}
\kipkokentry{safâ}{\latupsad\latupfe\latupvav}{None}{kok:sad_fe_vav}
\kipkokentry{safahat}{\latupsad\latupfe\latupha}{None}{kok:sad_fe_ha}
\kipkokentry{saffet}{\latupsad\latupfe\latupvav}{None}{kok:sad_fe_vav}
\kipkokentry{safha}{\latupsad\latupfe\latupha}{None}{kok:sad_fe_ha}
\kipkokentry{sâfi}{\latupsad\latupfe\latupvav}{K\rom{1}, Ed.}{kok:sad_fe_vav}
\kipkokentry{safra}{\latupsad\latupfe\latupre$^1$}{None}{kok:sad_fe_re1}
\kipkokentry{safsata}{\latupsin\latupfe\latupsad\latupta}{None}{kok:sin_fe_sad_ta}
\kipkokentry{sağir}{\latupsad\latupgayn\latupre}{None}{kok:sad_gayn_re}
\kipkokentry{saha}{\latupsin\latupvav\latupha}{None}{kok:sin_vav_ha}
\kipkokentry{sahabe}{\latupsad\latupha\latupbe}{None}{kok:sad_ha_be}
\kipkokentry{sahaf}{\latupsad\latupha\latupfe}{None}{kok:sad_ha_fe}
\kipkokentry{sahan}{\latupsad\latupha\latupnun}{None}{kok:sad_ha_nun}
\kipkokentry{sahî}{\latupsad\latupha\latupha}{None}{kok:sad_ha_ha}
\kipkokentry{sâhife}{\latupsad\latupha\latupfe}{None}{kok:sad_ha_fe}
\kipkokentry{sâhil}{\latupsin\latupha\latuplam}{K\rom{1}, Ed.}{kok:sin_ha_lam}
\kipkokentry{sâhip}{\latupsad\latupha\latupbe}{K\rom{1}, Ed.}{kok:sad_ha_be}
\kipkokentry{sahn}{\latupsad\latupha\latupnun}{None}{kok:sad_ha_nun}
\kipkokentry{sahne}{\latupsad\latupha\latupnun}{None}{kok:sad_ha_nun}
\kipkokentry{sahrâ}{\latupsad\latupha\latupre}{None}{kok:sad_ha_re}
\kipkokentry{sahur}{\latupsin\latupha\latupre$^1$}{None}{kok:sin_ha_re1}
\kipkokentry{sâik}{\latupsin\latupvav\latupkaf}{K\rom{1}, Ed.}{kok:sin_vav_kaf}
\kipkokentry{sâir}{\latupsin\latupalif\latupre}{K\rom{1}, Ed.}{kok:sin_alif_re}
\kipkokentry{saka}{\latupsin\latupkaf\latupye}{None}{kok:sin_kaf_ye}
\kipkokentry{sakat}{\latupsin\latupkaf\latupta}{None}{kok:sin_kaf_ta}
\kipkokentry{sakatat}{\latupsin\latupkaf\latupta}{None}{kok:sin_kaf_ta}
\kipkokentry{sâkıt}{\latupsin\latupkaf\latupta}{K\rom{1}, Ed.}{kok:sin_kaf_ta}
\kipkokentry{sâki}{\latupsin\latupkaf\latupye}{K\rom{1}, Ed.}{kok:sin_kaf_ye}
\kipkokentry{sakil}{\latupthe\latupkaf\latuplam}{None}{kok:the_kaf_lam}
\kipkokentry{sakim}{\latupsin\latupkaf\latupmim}{None}{kok:sin_kaf_mim}
\kipkokentry{sâkin}{\latupsin\latupkef\latupnun}{K\rom{1}, Ed.}{kok:sin_kef_nun}
\kipkokentry{salâ}{\latupsad\latuplam\latupvav}{None}{kok:sad_lam_vav}
\kipkokentry{salâbet}{\latupsad\latuplam\latupbe}{None}{kok:sad_lam_be}
\kipkokentry{salah}{\latupsad\latuplam\latupha}{None}{kok:sad_lam_ha}
\kipkokentry{salâhiyet}{\latupsad\latuplam\latupha}{None}{kok:sad_lam_ha}
\kipkokentry{salavat}{\latupsad\latuplam\latupvav}{None}{kok:sad_lam_vav}
\kipkokentry{salh}{\latupsin\latuplam\latupxa}{None}{kok:sin_lam_xa}
\kipkokentry{salı}{\latupthe\latuplam\latupthe}{None}{kok:the_lam_the}
\kipkokentry{sâlih}{\latupsad\latuplam\latupha}{K\rom{1}, Ed.}{kok:sad_lam_ha}
\kipkokentry{sâlik}{\latupsin\latuplam\latupkef}{K\rom{1}, Ed.}{kok:sin_lam_kef}
\kipkokentry{sâlim}{\latupsin\latuplam\latupmim}{K\rom{1}, Ed.}{kok:sin_lam_mim}
\kipkokentry{salip}{\latupsad\latuplam\latupbe}{None}{kok:sad_lam_be}
\kipkokentry{sâlise}{\latupthe\latuplam\latupthe}{None}{kok:the_lam_the}
\kipkokentry{saltanat}{\latupsin\latuplam\latupta\latupnun}{None}{kok:sin_lam_ta_nun}
\kipkokentry{sam}{\latupsin\latupmim\latupmim}{K\rom{1}, Ed.}{kok:sin_mim_mim}
\kipkokentry{samîmî}{\latupsad\latupmim\latupmim}{None}{kok:sad_mim_mim}
\kipkokentry{sanat}{\latupsad\latupnun\latupayn}{None}{kok:sad_nun_ayn}
\kipkokentry{sanâyî}{\latupsad\latupnun\latupayn}{None}{kok:sad_nun_ayn}
\kipkokentry{sânî}{\latupthe\latupnun\latupye}{K\rom{1}, Ed.}{kok:the_nun_ye}
\kipkokentry{sâniye}{\latupthe\latupnun\latupye}{None}{kok:the_nun_ye}
\kipkokentry{sarâ}{\latupsad\latupre\latupayn}{None}{kok:sad_re_ayn}
\kipkokentry{saraç}{\latupsin\latupre\latupcim}{None}{kok:sin_re_cim}
\kipkokentry{sarâhat}{\latupsad\latupre\latupha}{None}{kok:sad_re_ha}
\kipkokentry{sarf}{\latupsad\latupre\latupfe}{None}{kok:sad_re_fe}
\kipkokentry{sârî}{\latupsin\latupre\latupye}{K\rom{1}, Ed.}{kok:sin_re_ye}
\kipkokentry{sarih}{\latupsad\latupre\latupha}{None}{kok:sad_re_ha}
\kipkokentry{sarraf}{\latupsad\latupre\latupfe}{None}{kok:sad_re_fe}
\kipkokentry{satıh}{\latupsin\latupta\latupha}{None}{kok:sin_ta_ha}
\kipkokentry{satır}{\latupsin\latupta\latupre}{None}{kok:sin_ta_re}
\kipkokentry{satır}{\latupsin\latupta\latupre}{None}{kok:sin_ta_re}
\kipkokentry{satvet}{\latupsad\latupte\latupvav}{None}{kok:sad_te_vav}
\kipkokentry{savlet}{\latupsad\latupvav\latuplam}{None}{kok:sad_vav_lam}
\kipkokentry{savt}{\latupsad\latupvav\latupte}{None}{kok:sad_vav_te}
\kipkokentry{sây}{\latupsin\latupayn\latupye}{None}{kok:sin_ayn_ye}
\kipkokentry{sayfa}{\latupsad\latupha\latupfe}{None}{kok:sad_ha_fe}
\kipkokentry{sayha}{\latupsad\latupye\latupha}{None}{kok:sad_ye_ha}
\kipkokentry{sebat}{\latupthe\latupbe\latupte}{None}{kok:the_be_te}
\kipkokentry{sebep}{\latupsin\latupbe\latupbe}{None}{kok:sin_be_be}
\kipkokentry{sebil}{\latupsin\latupbe\latuplam}{None}{kok:sin_be_lam}
\kipkokentry{sebt}{\latupsin\latupbe\latupte}{None}{kok:sin_be_te}
\kipkokentry{seccâde}{\latupsin\latupcim\latupdal}{None}{kok:sin_cim_dal}
\kipkokentry{secde}{\latupsin\latupcim\latupdal}{None}{kok:sin_cim_dal}
\kipkokentry{secî}{\latupsin\latupcim\latupayn}{None}{kok:sin_cim_ayn}
\kipkokentry{seciye}{\latupsin\latupcim\latupvav}{None}{kok:sin_cim_vav}
\kipkokentry{sedef}{\latupsad\latupdal\latupfe$^2$}{None}{kok:sad_dal_fe2}
\kipkokentry{sedir}{\latupsad\latupdal\latupre}{None}{kok:sad_dal_re}
\kipkokentry{sefahat}{\latupsin\latupfe\latuphe}{None}{kok:sin_fe_he}
\kipkokentry{sefâin}{\latupsin\latupfe\latupnun}{None}{kok:sin_fe_nun}
\kipkokentry{sefâlet}{\latupsin\latupfe\latuplam}{None}{kok:sin_fe_lam}
\kipkokentry{sefâret}{\latupsin\latupfe\latupre}{None}{kok:sin_fe_re}
\kipkokentry{sefer}{\latupsin\latupfe\latupre}{None}{kok:sin_fe_re}
\kipkokentry{sefih}{\latupsin\latupfe\latuphe}{None}{kok:sin_fe_he}
\kipkokentry{sefil}{\latupsin\latupfe\latuplam}{None}{kok:sin_fe_lam}
\kipkokentry{sefîne}{\latupsin\latupfe\latupnun}{None}{kok:sin_fe_nun}
\kipkokentry{sefir}{\latupsin\latupfe\latupre}{None}{kok:sin_fe_re}
\kipkokentry{seher}{\latupsin\latupha\latupre$^1$}{None}{kok:sin_ha_re1}
\kipkokentry{sehiv}{\latupsin\latuphe\latupvav}{None}{kok:sin_he_vav}
\kipkokentry{sekte}{\latupsin\latupkef\latupte}{None}{kok:sin_kef_te}
\kipkokentry{sel}{\latupsin\latupye\latuplam}{None}{kok:sin_ye_lam}
\kipkokentry{selâle}{\latupshin\latuplam\latuplam}{None}{kok:shin_lam_lam}
\kipkokentry{selam}{\latupsin\latuplam\latupmim}{None}{kok:sin_lam_mim}
\kipkokentry{selâmet}{\latupsin\latuplam\latupmim}{None}{kok:sin_lam_mim}
\kipkokentry{selâse}{\latupthe\latuplam\latupthe}{None}{kok:the_lam_the}
\kipkokentry{selâtin}{\latupsin\latuplam\latupta\latupnun}{None}{kok:sin_lam_ta_nun}
\kipkokentry{selef}{\latupsin\latuplam\latupfe}{None}{kok:sin_lam_fe}
\kipkokentry{selefî}{\latupsin\latuplam\latupfe}{None}{kok:sin_lam_fe}
\kipkokentry{selem}{\latupsin\latuplam\latupmim}{None}{kok:sin_lam_mim}
\kipkokentry{selim}{\latupsin\latuplam\latupmim}{None}{kok:sin_lam_mim}
\kipkokentry{semâ}{\latupsin\latupmim\latupvav}{None}{kok:sin_mim_vav}
\kipkokentry{semâ}{\latupsin\latupmim\latupayn}{None}{kok:sin_mim_ayn}
\kipkokentry{semah}{\latupsin\latupmim\latupayn}{None}{kok:sin_mim_ayn}
\kipkokentry{semâî}{\latupsin\latupmim\latupayn}{None}{kok:sin_mim_ayn}
\kipkokentry{semavî}{\latupsin\latupmim\latupvav}{None}{kok:sin_mim_vav}
\kipkokentry{semere}{\latupsin\latupmim\latupre$^2$}{None}{kok:sin_mim_re2}
\kipkokentry{semt}{\latupsin\latupmim\latupte}{None}{kok:sin_mim_te}
\kipkokentry{senâ}{\latupthe\latupnun\latupye}{None}{kok:the_nun_ye}
\kipkokentry{sene}{\latupthe\latupnun\latupye}{None}{kok:the_nun_ye}
\kipkokentry{sene}{\latupsin\latupnun\latupalif}{None}{kok:sin_nun_alif}
\kipkokentry{senet}{\latupsin\latupnun\latupdal}{None}{kok:sin_nun_dal}
\kipkokentry{seniye}{\latupsin\latupnun\latupvav}{None}{kok:sin_nun_vav}
\kipkokentry{serap}{\latupsin\latupre\latupbe}{None}{kok:sin_re_be}
\kipkokentry{serd}{\latupsin\latupre\latupdal}{None}{kok:sin_re_dal}
\kipkokentry{serî}{\latupsin\latupre\latupayn}{None}{kok:sin_re_ayn}
\kipkokentry{servet}{\latupthe\latupre\latupye}{None}{kok:the_re_ye}
\kipkokentry{set}{\latupsin\latupdal\latupdal}{None}{kok:sin_dal_dal}
\kipkokentry{setr}{\latupsin\latupte\latupre}{None}{kok:sin_te_re}
\kipkokentry{setre}{\latupsin\latupte\latupre}{None}{kok:sin_te_re}
\kipkokentry{settar}{\latupsin\latupte\latupre}{None}{kok:sin_te_re}
\kipkokentry{sevap}{\latupthe\latupvav\latupbe}{None}{kok:the_vav_be}
\kipkokentry{sevdâ}{\latupsin\latupvav\latupdal}{None}{kok:sin_vav_dal}
\kipkokentry{sevir}{\latupthe\latupvav\latupre}{None}{kok:the_vav_re}
\kipkokentry{seviye}{\latupsin\latupvav\latupye}{None}{kok:sin_vav_ye}
\kipkokentry{sevk}{\latupsin\latupvav\latupkaf}{None}{kok:sin_vav_kaf}
\kipkokentry{seyahat}{\latupsin\latupye\latupha}{None}{kok:sin_ye_ha}
\kipkokentry{seyf}{\latupsin\latupye\latupfe}{None}{kok:sin_ye_fe}
\kipkokentry{seyfiye}{\latupsin\latupye\latupfe}{None}{kok:sin_ye_fe}
\kipkokentry{seyir}{\latupsin\latupye\latupre}{None}{kok:sin_ye_re}
\kipkokentry{seyis}{\latupsin\latupvav\latupsin}{K\rom{1}, Ed.}{kok:sin_vav_sin}
\kipkokentry{seyran}{\latupsin\latupye\latupre}{None}{kok:sin_ye_re}
\kipkokentry{seyyah}{\latupsin\latupye\latupha}{None}{kok:sin_ye_ha}
\kipkokentry{seyyal}{\latupsin\latupye\latuplam}{None}{kok:sin_ye_lam}
\kipkokentry{seyyânen}{\latupsin\latupvav\latupye}{None}{kok:sin_vav_ye}
\kipkokentry{seyyar}{\latupsin\latupye\latupre}{None}{kok:sin_ye_re}
\kipkokentry{seyyare}{\latupsin\latupye\latupre}{None}{kok:sin_ye_re}
\kipkokentry{sıdk}{\latupsad\latupdal\latupkaf}{None}{kok:sad_dal_kaf}
\kipkokentry{sıfat}{\latupvav\latupsad\latupfe}{None}{kok:vav_sad_fe}
\kipkokentry{sıfır}{\latupsad\latupfe\latupre$^2$}{None}{kok:sad_fe_re2}
\kipkokentry{sıhhat}{\latupsad\latupha\latupha}{None}{kok:sad_ha_ha}
\kipkokentry{sıhrî}{\latupsad\latuphe\latupre}{None}{kok:sad_he_re}
\kipkokentry{sıklet}{\latupthe\latupkaf\latuplam}{None}{kok:the_kaf_lam}
\kipkokentry{sıla}{\latupvav\latupsad\latuplam}{None}{kok:vav_sad_lam}
\kipkokentry{sınâî}{\latupsad\latupnun\latupayn}{None}{kok:sad_nun_ayn}
\kipkokentry{sınıf}{\latupsad\latupnun\latupfe}{None}{kok:sad_nun_fe}
\kipkokentry{sır}{\latupsin\latupre\latupre}{None}{kok:sin_re_re}
\kipkokentry{sırat}{\latupsad\latupre\latupta}{None}{kok:sad_re_ta}
\kipkokentry{sırf}{\latupsad\latupre\latupfe}{None}{kok:sad_re_fe}
\kipkokentry{sıska}{\latupsin\latupkaf\latupye}{None}{kok:sin_kaf_ye}
\kipkokentry{sıyânet}{\latupsad\latupvav\latupnun}{None}{kok:sad_vav_nun}
\kipkokentry{sıyga}{\latupsad\latupvav\latupgayn}{None}{kok:sad_vav_gayn}
\kipkokentry{sibak}{\latupsin\latupbe\latupkaf}{None}{kok:sin_be_kaf}
\kipkokentry{sicil}{\latupsin\latupcim\latuplam}{None}{kok:sin_cim_lam}
\kipkokentry{siftah}{\latupfe\latupte\latupha}{None}{kok:fe_te_ha}
\kipkokentry{sihir}{\latupsin\latupha\latupre$^3$}{None}{kok:sin_ha_re3}
\kipkokentry{sikke}{\latupsin\latupkef\latupkef}{None}{kok:sin_kef_kef}
\kipkokentry{silah}{\latupsin\latuplam\latupha}{None}{kok:sin_lam_ha}
\kipkokentry{silk}{\latupsin\latuplam\latupkef}{None}{kok:sin_lam_kef}
\kipkokentry{silsile}{\latupsin\latuplam}{None}{kok:sin_lam}
\kipkokentry{sin}{\latupsin\latupnun\latupnun}{None}{kok:sin_nun_nun}
\kipkokentry{sirâyet}{\latupsin\latupre\latupye}{None}{kok:sin_re_ye}
\kipkokentry{sîret}{\latupsin\latupye\latupre}{None}{kok:sin_ye_re}
\kipkokentry{sirkat}{\latupsin\latupre\latupkaf}{None}{kok:sin_re_kaf}
\kipkokentry{siyak}{\latupsin\latupvav\latupkaf}{None}{kok:sin_vav_kaf}
\kipkokentry{siyâsa}{\latupsin\latupvav\latupsin}{None}{kok:sin_vav_sin}
\kipkokentry{siyâset}{\latupsin\latupvav\latupsin}{None}{kok:sin_vav_sin}
\kipkokentry{siyer}{\latupsin\latupye\latupre}{None}{kok:sin_ye_re}
\kipkokentry{sofa}{\latupsad\latupfe\latupfe}{None}{kok:sad_fe_fe}
\kipkokentry{sofra}{\latupsin\latupfe\latupre}{None}{kok:sin_fe_re}
\kipkokentry{sohbet}{\latupsad\latupha\latupbe}{None}{kok:sad_ha_be}
\kipkokentry{sokak}{\latupze\latupkaf\latupkaf}{None}{kok:ze_kaf_kaf}
\kipkokentry{sual}{\latupsin\latupalif\latuplam}{None}{kok:sin_alif_lam}
\kipkokentry{sudur}{\latupsad\latupdal\latupre}{None}{kok:sad_dal_re}
\kipkokentry{sûfî}{\latupsad\latupvav\latupfe}{None}{kok:sad_vav_fe}
\kipkokentry{suhûlet}{\latupsin\latuphe\latuplam}{None}{kok:sin_he_lam}
\kipkokentry{sui}{\latupsin\latupvav\latupalif}{None}{kok:sin_vav_alif}
\kipkokentry{sukuk}{\latupsad\latupkef\latupkef}{None}{kok:sad_kef_kef}
\kipkokentry{sukut}{\latupsin\latupkaf\latupta}{None}{kok:sin_kaf_ta}
\kipkokentry{sulh}{\latupsad\latuplam\latupha}{None}{kok:sad_lam_ha}
\kipkokentry{sulp}{\latupsad\latuplam\latupbe}{None}{kok:sad_lam_be}
\kipkokentry{sulta}{\latupsin\latuplam\latupta}{None}{kok:sin_lam_ta}
\kipkokentry{sultan}{\latupsin\latuplam\latupta}{None}{kok:sin_lam_ta}
\kipkokentry{sunî}{\latupsad\latupnun\latupayn}{None}{kok:sad_nun_ayn}
\kipkokentry{sur}{\latupsin\latupvav\latupre}{None}{kok:sin_vav_re}
\kipkokentry{sûr}{\latupsad\latupvav\latupre$^2$}{None}{kok:sad_vav_re2}
\kipkokentry{sûre}{\latupsin\latupvav\latupre}{None}{kok:sin_vav_re}
\kipkokentry{sûret}{\latupsad\latupvav\latupre$^1$}{None}{kok:sad_vav_re1}
\kipkokentry{sübut}{\latupthe\latupbe\latupte}{None}{kok:the_be_te}
\kipkokentry{sübyan}{\latupsad\latupbe\latupvav}{None}{kok:sad_be_vav}
\kipkokentry{süfla}{\latupsin\latupfe\latuplam}{None}{kok:sin_fe_lam}
\kipkokentry{süflî}{\latupsin\latupfe\latuplam}{None}{kok:sin_fe_lam}
\kipkokentry{sükun}{\latupsin\latupkef\latupnun}{None}{kok:sin_kef_nun}
\kipkokentry{sükûnet}{\latupsin\latupkef\latupnun}{None}{kok:sin_kef_nun}
\kipkokentry{sükut}{\latupsin\latupkef\latupte}{None}{kok:sin_kef_te}
\kipkokentry{sülâle}{\latupsin\latuplam\latuplam}{None}{kok:sin_lam_lam}
\kipkokentry{sülasî}{\latupthe\latuplam\latupthe}{None}{kok:the_lam_the}
\kipkokentry{süluk}{\latupsin\latuplam\latupkef}{None}{kok:sin_lam_kef}
\kipkokentry{sülüs}{\latupthe\latuplam\latupthe}{None}{kok:the_lam_the}
\kipkokentry{sünnet}{\latupsin\latupnun\latupnun}{None}{kok:sin_nun_nun}
\kipkokentry{sünnî}{\latupsin\latupnun\latupnun}{None}{kok:sin_nun_nun}
\kipkokentry{süphan}{\latupsin\latupbe\latupha}{None}{kok:sin_be_ha}
\kipkokentry{sürâhî}{\latupsad\latupre\latupha}{None}{kok:sad_re_ha}
\kipkokentry{sürat}{\latupsin\latupre\latupayn}{None}{kok:sin_re_ayn}
\kipkokentry{süreyyâ}{\latupthe\latupre\latupye}{None}{kok:the_re_ye}
\kipkokentry{sürre}{\latupsad\latupre\latupre$^1$}{None}{kok:sad_re_re1}
\kipkokentry{sürur}{\latupsin\latupre\latupre}{None}{kok:sin_re_re}
\kipkokentry{sütre}{\latupsin\latupte\latupre}{None}{kok:sin_te_re}
\end{multicols}
\dictchapter{Ş}
\begin{multicols}{2}
\kipkokentry{şâban}{\latupshin\latupayn\latupbe}{None}{kok:shin_ayn_be}
\kipkokentry{şafak}{\latupshin\latupfe\latupkaf}{None}{kok:shin_fe_kaf}
\kipkokentry{şahıs}{\latupshin\latupxa\latupsad}{None}{kok:shin_xa_sad}
\kipkokentry{şâhika}{\latupshin\latuphe\latupkaf}{None}{kok:shin_he_kaf}
\kipkokentry{şâhit}{\latupshin\latuphe\latupdal}{K\rom{1}, Ed.}{kok:shin_he_dal}
\kipkokentry{şahrem}{\latupshin\latupre\latupha}{None}{kok:shin_re_ha}
\kipkokentry{şâibe}{\latupshin\latupvav\latupbe}{None}{kok:shin_vav_be}
\kipkokentry{şâir}{\latupshin\latupayn\latupre}{K\rom{1}, Ed.}{kok:shin_ayn_re}
\kipkokentry{şâiya}{\latupshin\latupye\latupayn}{None}{kok:shin_ye_ayn}
\kipkokentry{şaka}{\latupshin\latupkaf\latupvav$^1$}{None}{kok:shin_kaf_vav1}
\kipkokentry{şakak}{\latupshin\latupkaf\latupkaf}{None}{kok:shin_kaf_kaf}
\kipkokentry{şaki}{\latupshin\latupkaf\latupvav$^1$}{None}{kok:shin_kaf_vav1}
\kipkokentry{şamata}{\latupshin\latupmim\latupte}{None}{kok:shin_mim_te}
\kipkokentry{şâmil}{\latupshin\latupmim\latuplam}{K\rom{1}, Ed.}{kok:shin_mim_lam}
\kipkokentry{şan}{\latupshin\latupalif\latupnun}{None}{kok:shin_alif_nun}
\kipkokentry{şap}{\latupshin\latupbe\latupbe}{None}{kok:shin_be_be}
\kipkokentry{şapka}{\latupshin\latupbe\latupkef}{None}{kok:shin_be_kef}
\kipkokentry{şarap}{\latupshin\latupre\latupbe}{None}{kok:shin_re_be}
\kipkokentry{şark}{\latupshin\latupre\latupkaf}{None}{kok:shin_re_kaf}
\kipkokentry{şart}{\latupshin\latupre\latupta}{None}{kok:shin_re_ta}
\kipkokentry{şaşaa}{\latupshin\latupayn}{None}{kok:shin_ayn}
\kipkokentry{şatafat}{\latupshin\latupta\latupfe}{None}{kok:shin_ta_fe}
\kipkokentry{şatır}{\latupshin\latupta\latupre}{K\rom{1}, Ed.}{kok:shin_ta_re}
\kipkokentry{şâyî}{\latupshin\latupye\latupayn}{K\rom{1}, Ed.}{kok:shin_ye_ayn}
\kipkokentry{şeâmet}{\latupshin\latupalif\latupmim}{None}{kok:shin_alif_mim}
\kipkokentry{şebek}{\latupshin\latupbe\latupkef}{None}{kok:shin_be_kef}
\kipkokentry{şebeke}{\latupshin\latupbe\latupkef}{None}{kok:shin_be_kef}
\kipkokentry{şecaat}{\latupshin\latupcim\latupayn}{None}{kok:shin_cim_ayn}
\kipkokentry{şecere}{\latupshin\latupcim\latupre}{None}{kok:shin_cim_re}
\kipkokentry{şedde}{\latupshin\latupdal\latupdal}{None}{kok:shin_dal_dal}
\kipkokentry{şedit}{\latupshin\latupdal\latupdal}{None}{kok:shin_dal_dal}
\kipkokentry{şefaat}{\latupshin\latupfe\latupayn}{None}{kok:shin_fe_ayn}
\kipkokentry{şeffaf}{\latupshin\latupfe\latupfe}{None}{kok:shin_fe_fe}
\kipkokentry{şefkat}{\latupshin\latupfe\latupkaf}{None}{kok:shin_fe_kaf}
\kipkokentry{şehâdet}{\latupshin\latuphe\latupdal}{None}{kok:shin_he_dal}
\kipkokentry{şehir}{\latupshin\latuphe\latupre}{None}{kok:shin_he_re}
\kipkokentry{şehir}{\latupshin\latuphe\latupre}{None}{kok:shin_he_re}
\kipkokentry{şehit}{\latupshin\latuphe\latupdal}{None}{kok:shin_he_dal}
\kipkokentry{şehlâ}{\latupshin\latuphe\latuplam}{None}{kok:shin_he_lam}
\kipkokentry{şehvet}{\latupshin\latuphe\latupvav}{None}{kok:shin_he_vav}
\kipkokentry{şek}{\latupshin\latupkef\latupkef}{None}{kok:shin_kef_kef}
\kipkokentry{şekâvet}{\latupshin\latupkaf\latupvav$^1$}{None}{kok:shin_kaf_vav1}
\kipkokentry{şekil}{\latupshin\latupkef\latuplam$^2$}{None}{kok:shin_kef_lam2}
\kipkokentry{şekvâ}{\latupshin\latupkaf\latupvav$^2$}{None}{kok:shin_kaf_vav2}
\kipkokentry{şemâil}{\latupshin\latupmim\latuplam}{None}{kok:shin_mim_lam}
\kipkokentry{şemâme}{\latupshin\latupmim\latupmim}{None}{kok:shin_mim_mim}
\kipkokentry{şems}{\latupshin\latupmim\latupsin}{None}{kok:shin_mim_sin}
\kipkokentry{şemsiye}{\latupshin\latupmim\latupsin}{None}{kok:shin_mim_sin}
\kipkokentry{şenaat}{\latupshin\latupnun\latupayn}{None}{kok:shin_nun_ayn}
\kipkokentry{şenî}{\latupshin\latupnun\latupayn}{None}{kok:shin_nun_ayn}
\kipkokentry{şeniyet}{\latupshin\latupalif\latupnun}{None}{kok:shin_alif_nun}
\kipkokentry{şer}{\latupshin\latupre\latupayn}{None}{kok:shin_re_ayn}
\kipkokentry{şer}{\latupshin\latupre\latupre}{None}{kok:shin_re_re}
\kipkokentry{şerâit}{\latupshin\latupre\latupta}{None}{kok:shin_re_ta}
\kipkokentry{şerbet}{\latupshin\latupre\latupbe}{None}{kok:shin_re_be}
\kipkokentry{şeref}{\latupshin\latupre\latupfe}{None}{kok:shin_re_fe}
\kipkokentry{şerefe}{\latupshin\latupre\latupfe}{None}{kok:shin_re_fe}
\kipkokentry{şerh}{\latupshin\latupre\latupha}{None}{kok:shin_re_ha}
\kipkokentry{şeriat}{\latupshin\latupre\latupayn}{None}{kok:shin_re_ayn}
\kipkokentry{şerif}{\latupshin\latupre\latupfe}{None}{kok:shin_re_fe}
\kipkokentry{şerik}{\latupshin\latupre\latupkef}{None}{kok:shin_re_kef}
\kipkokentry{şerir}{\latupshin\latupre\latupre}{None}{kok:shin_re_re}
\kipkokentry{şerit}{\latupshin\latupre\latupta}{None}{kok:shin_re_ta}
\kipkokentry{şetâret}{\latupshin\latupta\latupre}{None}{kok:shin_ta_re}
\kipkokentry{şevk}{\latupshin\latupvav\latupkaf}{None}{kok:shin_vav_kaf}
\kipkokentry{şevket}{\latupshin\latupvav\latupkef}{None}{kok:shin_vav_kef}
\kipkokentry{şey}{\latupshin\latupye\latupalif}{None}{kok:shin_ye_alif}
\kipkokentry{şeyh}{\latupshin\latupye\latupxa}{None}{kok:shin_ye_xa}
\kipkokentry{şeytan}{\latupshin\latupye\latupta}{None}{kok:shin_ye_ta}
\kipkokentry{şık}{\latupshin\latupkaf\latupkaf}{None}{kok:shin_kaf_kaf}
\kipkokentry{şiar}{\latupshin\latupayn\latupre}{None}{kok:shin_ayn_re}
\kipkokentry{şiddet}{\latupshin\latupdal\latupdal}{None}{kok:shin_dal_dal}
\kipkokentry{şifâ}{\latupshin\latupfe\latupvav}{None}{kok:shin_fe_vav}
\kipkokentry{şifâhî}{\latupshin\latupfe\latuphe}{None}{kok:shin_fe_he}
\kipkokentry{Şii}{\latupshin\latupye\latupayn}{None}{kok:shin_ye_ayn}
\kipkokentry{şiir}{\latupshin\latupayn\latupre}{None}{kok:shin_ayn_re}
\kipkokentry{şikâyet}{\latupshin\latupkaf\latupvav$^2$}{None}{kok:shin_kaf_vav2}
\kipkokentry{şirâ}{\latupshin\latupre\latupye}{None}{kok:shin_re_ye}
\kipkokentry{şirk}{\latupshin\latupre\latupkef}{None}{kok:shin_re_kef}
\kipkokentry{şirket}{\latupshin\latupre\latupkef}{None}{kok:shin_re_kef}
\kipkokentry{şirret}{\latupshin\latupre\latupre}{None}{kok:shin_re_re}
\kipkokentry{şom}{\latupshin\latupalif\latupmim}{None}{kok:shin_alif_mim}
\kipkokentry{şöbiyet}{\latupshin\latupbe\latupayn}{None}{kok:shin_be_ayn}
\kipkokentry{şöhret}{\latupshin\latuphe\latupre}{None}{kok:shin_he_re}
\kipkokentry{şuâ}{\latupshin\latupayn\latupayn}{None}{kok:shin_ayn_ayn}
\kipkokentry{şuârâ}{\latupshin\latupayn\latupre}{None}{kok:shin_ayn_re}
\kipkokentry{şûbe}{\latupshin\latupayn\latupbe}{None}{kok:shin_ayn_be}
\kipkokentry{şufâ}{\latupshin\latupfe\latupayn}{None}{kok:shin_fe_ayn}
\kipkokentry{şûle}{\latupshin\latupayn\latuplam}{None}{kok:shin_ayn_lam}
\kipkokentry{şûrâ}{\latupshin\latupvav\latupre}{None}{kok:shin_vav_re}
\kipkokentry{şurup}{\latupshin\latupre\latupbe}{None}{kok:shin_re_be}
\kipkokentry{şuur}{\latupshin\latupayn\latupre}{None}{kok:shin_ayn_re}
\kipkokentry{şühedâ}{\latupshin\latuphe\latupdal}{None}{kok:shin_he_dal}
\kipkokentry{şükran}{\latupshin\latupkef\latupre}{None}{kok:shin_kef_re}
\kipkokentry{şükür}{\latupshin\latupkef\latupre}{None}{kok:shin_kef_re}
\kipkokentry{şümul}{\latupshin\latupmim\latuplam}{None}{kok:shin_mim_lam}
\kipkokentry{şüphe}{\latupshin\latupbe\latupha}{None}{kok:shin_be_ha}
\kipkokentry{şürekâ}{\latupshin\latupre\latupkef}{None}{kok:shin_re_kef}
\kipkokentry{şüyû}{\latupshin\latupye\latupayn}{None}{kok:shin_ye_ayn}
\end{multicols}
\dictchapter{T}
\begin{multicols}{2}
\kipkokentry{taaccüp}{\latupayn\latupcim\latupbe}{None}{kok:ayn_cim_be}
\kipkokentry{taaddî}{\latupayn\latupdal\latupvav}{None}{kok:ayn_dal_vav}
\kipkokentry{taahhüt}{\latupayn\latuphe\latupdal}{None}{kok:ayn_he_dal}
\kipkokentry{taalluk}{\latupayn\latuplam\latupkaf}{None}{kok:ayn_lam_kaf}
\kipkokentry{taam}{\latupta\latupayn\latupmim}{None}{kok:ta_ayn_mim}
\kipkokentry{taammüt}{\latupayn\latupmim\latupdal}{None}{kok:ayn_mim_dal}
\kipkokentry{taarruz}{\latupayn\latupre\latupdad}{None}{kok:ayn_re_dad}
\kipkokentry{taassup}{\latupayn\latupsad\latupbe}{None}{kok:ayn_sad_be}
\kipkokentry{taayyün}{\latupayn\latupye\latupnun}{None}{kok:ayn_ye_nun}
\kipkokentry{tab}{\latupta\latupbe\latupayn}{None}{kok:ta_be_ayn}
\kipkokentry{tabâbet}{\latupta\latupbe\latupbe}{None}{kok:ta_be_be}
\kipkokentry{tabahhur}{\latupbe\latupxa\latupre}{None}{kok:be_xa_re}
\kipkokentry{tabak}{\latupta\latupbe\latupkaf}{None}{kok:ta_be_kaf}
\kipkokentry{tabaka}{\latupta\latupbe\latupkaf}{None}{kok:ta_be_kaf}
\kipkokentry{tabak}{\latupdal\latupbe\latupgayn}{None}{kok:dal_be_gayn}
\kipkokentry{tabasbus}{\latupbe\latupsad}{None}{kok:be_sad}
\kipkokentry{tâbi}{\latupte\latupbe\latupayn}{K\rom{1}, Ed.}{kok:te_be_ayn}
\kipkokentry{tabiat}{\latupta\latupbe\latupayn}{None}{kok:ta_be_ayn}
\kipkokentry{tabii}{\latupta\latupbe\latupayn}{None}{kok:ta_be_ayn}
\kipkokentry{tabip}{\latupta\latupbe\latupbe}{None}{kok:ta_be_be}
\kipkokentry{tâbir}{\latupayn\latupbe\latupre$^1$}{K\rom{2}, Eyl.}{kok:ayn_be_re1}
\kipkokentry{tabla}{\latupta\latupbe\latuplam}{None}{kok:ta_be_lam}
\kipkokentry{tabya}{\latupayn\latupbe\latupye}{None}{kok:ayn_be_ye}
\kipkokentry{tâcir}{\latupte\latupcim\latupre}{K\rom{1}, Ed.}{kok:te_cim_re}
\kipkokentry{tâciz}{\latupayn\latupcim\latupze}{K\rom{2}, Eyl.}{kok:ayn_cim_ze}
\kipkokentry{tâdat}{\latupayn\latupdal\latupdal}{K\rom{2}, Eyl.}{kok:ayn_dal_dal}
\kipkokentry{tâdil}{\latupayn\latupdal\latuplam}{K\rom{2}, Eyl.}{kok:ayn_dal_lam}
\kipkokentry{tâdilat}{\latupayn\latupdal\latuplam}{None}{kok:ayn_dal_lam}
\kipkokentry{tafsilat}{\latupfe\latupsad\latuplam}{None}{kok:fe_sad_lam}
\kipkokentry{tağşiş}{\latupgayn\latupshin\latupshin}{K\rom{2}, Eyl.}{kok:gayn_shin_shin}
\kipkokentry{tâğut}{\latupta\latupgayn\latupte}{None}{kok:ta_gayn_te}
\kipkokentry{tağyir}{\latupgayn\latupye\latupre}{K\rom{2}, Eyl.}{kok:gayn_ye_re}
\kipkokentry{tahakkuk}{\latupha\latupkaf\latupkaf}{None}{kok:ha_kaf_kaf}
\kipkokentry{tahakküm}{\latupha\latupkef\latupmim}{None}{kok:ha_kef_mim}
\kipkokentry{tahammül}{\latupha\latupmim\latuplam}{None}{kok:ha_mim_lam}
\kipkokentry{tahâret}{\latupta\latuphe\latupre}{None}{kok:ta_he_re}
\kipkokentry{taharrî}{\latupha\latupre\latupye}{None}{kok:ha_re_ye}
\kipkokentry{tahattur}{\latupxa\latupta\latupre}{None}{kok:xa_ta_re}
\kipkokentry{tahavvül}{\latupha\latupvav\latuplam}{None}{kok:ha_vav_lam}
\kipkokentry{tahayyül}{\latupxa\latupye\latuplam}{None}{kok:xa_ye_lam}
\kipkokentry{tahcir}{\latuphe\latupcim\latupre}{K\rom{2}, Eyl.}{kok:he_cim_re}
\kipkokentry{tahdit}{\latupha\latupdal\latupdal}{K\rom{2}, Eyl.}{kok:ha_dal_dal}
\kipkokentry{tahfif}{\latupxa\latupfe\latupfe}{K\rom{2}, Eyl.}{kok:xa_fe_fe}
\kipkokentry{tahıl}{\latupdal\latupxa\latuplam}{None}{kok:dal_xa_lam}
\kipkokentry{tahin}{\latupta\latupha\latupnun}{None}{kok:ta_ha_nun}
\kipkokentry{tâhir}{\latupta\latuphe\latupre}{K\rom{1}, Ed.}{kok:ta_he_re}
\kipkokentry{tahkik}{\latupha\latupkaf\latupkaf}{K\rom{2}, Eyl.}{kok:ha_kaf_kaf}
\kipkokentry{tahkim}{\latupha\latupkef\latupmim}{K\rom{2}, Eyl.}{kok:ha_kef_mim}
\kipkokentry{tahkimat}{\latupha\latupkef\latupmim}{None}{kok:ha_kef_mim}
\kipkokentry{tahkir}{\latupha\latupkaf\latupre}{K\rom{2}, Eyl.}{kok:ha_kaf_re}
\kipkokentry{tahkiye}{\latupha\latupkef\latupye}{None}{kok:ha_kef_ye}
\kipkokentry{tahlil}{\latupha\latuplam\latuplam}{K\rom{2}, Eyl.}{kok:ha_lam_lam}
\kipkokentry{tahlisiye}{\latupxa\latuplam\latupsad}{None}{kok:xa_lam_sad}
\kipkokentry{tahliye}{\latupxa\latuplam\latupvav}{None}{kok:xa_lam_vav}
\kipkokentry{tahmin}{\latupxa\latupmim\latupnun}{K\rom{2}, Eyl.}{kok:xa_mim_nun}
\kipkokentry{tahmis}{\latupxa\latupmim\latupsin}{K\rom{2}, Eyl.}{kok:xa_mim_sin}
\kipkokentry{tahnit}{\latupha\latupnun\latupta}{K\rom{2}, Eyl.}{kok:ha_nun_ta}
\kipkokentry{tahribat}{\latupxa\latupre\latupbe}{None}{kok:xa_re_be}
\kipkokentry{tahrif}{\latupha\latupre\latupfe$^1$}{K\rom{2}, Eyl.}{kok:ha_re_fe1}
\kipkokentry{tahrik}{\latupha\latupre\latupkef}{K\rom{2}, Eyl.}{kok:ha_re_kef}
\kipkokentry{tahrip}{\latupxa\latupre\latupbe}{K\rom{2}, Eyl.}{kok:xa_re_be}
\kipkokentry{tahrir}{\latupha\latupre\latupre$^1$}{K\rom{2}, Eyl.}{kok:ha_re_re1}
\kipkokentry{tahriş}{\latupha\latupre\latupshin}{K\rom{2}, Eyl.}{kok:ha_re_shin}
\kipkokentry{tahsil}{\latupha\latupsad\latuplam}{K\rom{2}, Eyl.}{kok:ha_sad_lam}
\kipkokentry{tahsin}{\latupha\latupsin\latupnun}{K\rom{2}, Eyl.}{kok:ha_sin_nun}
\kipkokentry{tahsis}{\latupxa\latupsad\latupsad}{K\rom{2}, Eyl.}{kok:xa_sad_sad}
\kipkokentry{tahsisat}{\latupxa\latupsad\latupsad}{None}{kok:xa_sad_sad}
\kipkokentry{tahşidat}{\latupha\latupshin\latupdal}{None}{kok:ha_shin_dal}
\kipkokentry{tahtie}{\latupxa\latupta\latupalif}{K\rom{2}, Eyl.}{kok:xa_ta_alif}
\kipkokentry{tahvil}{\latupha\latupvav\latuplam}{K\rom{2}, Eyl.}{kok:ha_vav_lam}
\kipkokentry{tâife}{\latupta\latupvav\latupfe}{None}{kok:ta_vav_fe}
\kipkokentry{takaddüm}{\latupkaf\latupdal\latupmim}{None}{kok:kaf_dal_mim}
\kipkokentry{takallus}{\latupkaf\latuplam\latupsad}{None}{kok:kaf_lam_sad}
\kipkokentry{takas}{\latupkaf\latupsad\latupsad}{None}{kok:kaf_sad_sad}
\kipkokentry{tâkat}{\latupta\latupvav\latupkaf}{None}{kok:ta_vav_kaf}
\kipkokentry{takaza}{\latupkaf\latupdad\latupye}{None}{kok:kaf_dad_ye}
\kipkokentry{takbih}{\latupkaf\latupbe\latupha}{K\rom{2}, Eyl.}{kok:kaf_be_ha}
\kipkokentry{takdim}{\latupkaf\latupdal\latupmim}{K\rom{2}, Eyl.}{kok:kaf_dal_mim}
\kipkokentry{takdir}{\latupkaf\latupdal\latupre}{K\rom{2}, Eyl.}{kok:kaf_dal_re}
\kipkokentry{takdis}{\latupkaf\latupdal\latupsin}{K\rom{2}, Eyl.}{kok:kaf_dal_sin}
\kipkokentry{tâkip}{\latupayn\latupkaf\latupbe}{K\rom{2}, Eyl.}{kok:ayn_kaf_be}
\kipkokentry{takiye}{\latupvav\latupkaf\latupye}{None}{kok:vav_kaf_ye}
\kipkokentry{takke}{\latupvav\latupkaf\latupye}{None}{kok:vav_kaf_ye}
\kipkokentry{takla}{\latupkaf\latuplam\latupayn$^2$}{K\rom{2}, Eyl.}{kok:kaf_lam_ayn2}
\kipkokentry{taklit}{\latupkaf\latuplam\latupdal}{K\rom{2}, Eyl.}{kok:kaf_lam_dal}
\kipkokentry{takrîben}{\latupkaf\latupre\latupbe$^1$}{None}{kok:kaf_re_be1}
\kipkokentry{takrir}{\latupkaf\latupre\latupre}{K\rom{2}, Eyl.}{kok:kaf_re_re}
\kipkokentry{takriz}{\latupkaf\latupre\latupdad}{K\rom{2}, Eyl.}{kok:kaf_re_dad}
\kipkokentry{taksim}{\latupkaf\latupsin\latupmim}{K\rom{2}, Eyl.}{kok:kaf_sin_mim}
\kipkokentry{taksîmat}{\latupkaf\latupsin\latupmim}{None}{kok:kaf_sin_mim}
\kipkokentry{taksir}{\latupkaf\latupsad\latupre}{K\rom{2}, Eyl.}{kok:kaf_sad_re}
\kipkokentry{taksit}{\latupkaf\latupsin\latupta}{K\rom{2}, Eyl.}{kok:kaf_sin_ta}
\kipkokentry{taktil}{\latupkaf\latupte\latuplam}{K\rom{2}, Eyl.}{kok:kaf_te_lam}
\kipkokentry{takvâ}{\latupvav\latupkaf\latupye}{K\rom{2}, Eyl.}{kok:vav_kaf_ye}
\kipkokentry{takvim}{\latupkaf\latupvav\latupmim}{K\rom{2}, Eyl.}{kok:kaf_vav_mim}
\kipkokentry{takviye}{\latupkaf\latupvav\latupye}{None}{kok:kaf_vav_ye}
\kipkokentry{takyidat}{\latupkaf\latupye\latupdal}{None}{kok:kaf_ye_dal}
\kipkokentry{talak}{\latupta\latuplam\latupkaf}{None}{kok:ta_lam_kaf}
\kipkokentry{talâkat}{\latupta\latuplam\latupkaf}{None}{kok:ta_lam_kaf}
\kipkokentry{talebe}{\latupta\latuplam\latupbe}{None}{kok:ta_lam_be}
\kipkokentry{talep}{\latupta\latuplam\latupbe}{None}{kok:ta_lam_be}
\kipkokentry{tâli}{\latupte\latuplam\latupvav}{K\rom{1}, Ed.}{kok:te_lam_vav}
\kipkokentry{tâlih}{\latupta\latuplam\latupayn}{K\rom{1}, Ed.}{kok:ta_lam_ayn}
\kipkokentry{tâlik}{\latupayn\latuplam\latupkaf}{K\rom{2}, Eyl.}{kok:ayn_lam_kaf}
\kipkokentry{tâlim}{\latupayn\latuplam\latupmim}{K\rom{2}, Eyl.}{kok:ayn_lam_mim}
\kipkokentry{tâlip}{\latupta\latuplam\latupbe}{K\rom{1}, Ed.}{kok:ta_lam_be}
\kipkokentry{taltif}{\latuplam\latupta\latupfe}{K\rom{2}, Eyl.}{kok:lam_ta_fe}
\kipkokentry{tam}{\latupte\latupmim\latupmim}{K\rom{1}, Ed.}{kok:te_mim_mim}
\kipkokentry{tamah}{\latupta\latupmim\latupayn}{None}{kok:ta_mim_ayn}
\kipkokentry{tamam}{\latupte\latupmim\latupmim}{None}{kok:te_mim_mim}
\kipkokentry{tâmim}{\latupayn\latupmim\latupmim}{K\rom{2}, Eyl.}{kok:ayn_mim_mim}
\kipkokentry{tâmir}{\latupayn\latupmim\latupre}{K\rom{2}, Eyl.}{kok:ayn_mim_re}
\kipkokentry{tanassur}{\latupnun\latupsad\latupre$^1$}{None}{kok:nun_sad_re1}
\kipkokentry{tantana}{\latupta\latupnun}{None}{kok:ta_nun}
\kipkokentry{tanzim}{\latupnun\latupza\latupmim}{K\rom{2}, Eyl.}{kok:nun_za_mim}
\kipkokentry{tanzimat}{\latupnun\latupza\latupmim}{None}{kok:nun_za_mim}
\kipkokentry{taraf}{\latupta\latupre\latupfe}{None}{kok:ta_re_fe}
\kipkokentry{tarafeyn}{\latupta\latupre\latupfe}{None}{kok:ta_re_fe}
\kipkokentry{tarassut}{\latupre\latupsad\latupdal}{None}{kok:re_sad_dal}
\kipkokentry{tarâvet}{\latupta\latupre\latupvav}{None}{kok:ta_re_vav}
\kipkokentry{tarh}{\latupta\latupre\latupha}{None}{kok:ta_re_ha}
\kipkokentry{târif}{\latupayn\latupre\latupfe}{K\rom{2}, Eyl.}{kok:ayn_re_fe}
\kipkokentry{târife}{\latupayn\latupre\latupfe}{None}{kok:ayn_re_fe}
\kipkokentry{târih}{\latupvav\latupre\latupxa}{K\rom{2}, Eyl.}{kok:vav_re_xa}
\kipkokentry{tarik}{\latupta\latupre\latupkaf}{None}{kok:ta_re_kaf}
\kipkokentry{tarikat}{\latupta\latupre\latupkaf}{None}{kok:ta_re_kaf}
\kipkokentry{târiz}{\latupayn\latupre\latupdad}{K\rom{2}, Eyl.}{kok:ayn_re_dad}
\kipkokentry{tart}{\latupta\latupre\latupdal}{None}{kok:ta_re_dal}
\kipkokentry{tarz}{\latupta\latupre\latupze}{None}{kok:ta_re_ze}
\kipkokentry{tas}{\latupta\latupvav\latupsin}{None}{kok:ta_vav_sin}
\kipkokentry{tasallut}{\latupsin\latuplam\latupta}{None}{kok:sin_lam_ta}
\kipkokentry{tasarruf}{\latupsad\latupre\latupfe}{None}{kok:sad_re_fe}
\kipkokentry{tasavvuf}{\latupsad\latupvav\latupfe}{None}{kok:sad_vav_fe}
\kipkokentry{tasavvur}{\latupsad\latupvav\latupre$^1$}{None}{kok:sad_vav_re1}
\kipkokentry{tasdik}{\latupsad\latupdal\latupkaf}{K\rom{2}, Eyl.}{kok:sad_dal_kaf}
\kipkokentry{tasfiye}{\latupsad\latupfe\latupvav}{None}{kok:sad_fe_vav}
\kipkokentry{tashih}{\latupsad\latupha\latupha}{K\rom{2}, Eyl.}{kok:sad_ha_ha}
\kipkokentry{tasnif}{\latupsad\latupnun\latupfe}{K\rom{2}, Eyl.}{kok:sad_nun_fe}
\kipkokentry{tasrif}{\latupsad\latupre\latupfe}{K\rom{2}, Eyl.}{kok:sad_re_fe}
\kipkokentry{tasrih}{\latupsad\latupre\latupha}{K\rom{2}, Eyl.}{kok:sad_re_ha}
\kipkokentry{tasvip}{\latupsad\latupvav\latupbe}{K\rom{2}, Eyl.}{kok:sad_vav_be}
\kipkokentry{tasvir}{\latupsad\latupvav\latupre$^1$}{K\rom{2}, Eyl.}{kok:sad_vav_re1}
\kipkokentry{tatbik}{\latupta\latupbe\latupkaf}{K\rom{2}, Eyl.}{kok:ta_be_kaf}
\kipkokentry{tatbikat}{\latupta\latupbe\latupkaf}{None}{kok:ta_be_kaf}
\kipkokentry{tâtil}{\latupayn\latupta\latuplam}{K\rom{2}, Eyl.}{kok:ayn_ta_lam}
\kipkokentry{tatmin}{\latupta\latupmim\latupnun}{K\rom{2}, Eyl.}{kok:ta_mim_nun}
\kipkokentry{tâun}{\latupta\latupayn\latupnun}{None}{kok:ta_ayn_nun}
\kipkokentry{tav}{\latupta\latupvav\latupayn}{None}{kok:ta_vav_ayn}
\kipkokentry{tavaf}{\latupta\latupvav\latupfe}{None}{kok:ta_vav_fe}
\kipkokentry{tavassut}{\latupvav\latupsin\latupta}{None}{kok:vav_sin_ta}
\kipkokentry{tavattun}{\latupvav\latupta\latupnun}{None}{kok:vav_ta_nun}
\kipkokentry{tavır}{\latupta\latupvav\latupre}{None}{kok:ta_vav_re}
\kipkokentry{tavil}{\latupta\latupvav\latuplam}{None}{kok:ta_vav_lam}
\kipkokentry{tâviz}{\latupayn\latupvav\latupdad}{K\rom{2}, Eyl.}{kok:ayn_vav_dad}
\kipkokentry{tavsif}{\latupvav\latupsad\latupfe}{K\rom{2}, Eyl.}{kok:vav_sad_fe}
\kipkokentry{tavsiye}{\latupvav\latupsad\latupye}{None}{kok:vav_sad_ye}
\kipkokentry{tavzih}{\latupvav\latupdad\latupha}{K\rom{2}, Eyl.}{kok:vav_dad_ha}
\kipkokentry{tayf}{\latupta\latupye\latupfe}{None}{kok:ta_ye_fe}
\kipkokentry{tâyin}{\latupayn\latupye\latupnun}{K\rom{2}, Eyl.}{kok:ayn_ye_nun}
\kipkokentry{tayyâre}{\latupta\latupye\latupre}{None}{kok:ta_ye_re}
\kipkokentry{tayyip}{\latupta\latupye\latupbe}{None}{kok:ta_ye_be}
\kipkokentry{tazammun}{\latupdad\latupmim\latupnun}{None}{kok:dad_mim_nun}
\kipkokentry{tâzim}{\latupayn\latupza\latupmim}{K\rom{2}, Eyl.}{kok:ayn_za_mim}
\kipkokentry{tâzip}{\latupayn\latupzel\latupbe}{K\rom{2}, Eyl.}{kok:ayn_zel_be}
\kipkokentry{tâzir}{\latupayn\latupze\latupre}{K\rom{2}, Eyl.}{kok:ayn_ze_re}
\kipkokentry{tâziye}{\latupayn\latupze\latupye}{None}{kok:ayn_ze_ye}
\kipkokentry{tâziz}{\latupayn\latupze\latupze}{K\rom{2}, Eyl.}{kok:ayn_ze_ze}
\kipkokentry{tazmin}{\latupdad\latupmim\latupnun}{K\rom{2}, Eyl.}{kok:dad_mim_nun}
\kipkokentry{tazminat}{\latupdad\latupmim\latupnun}{None}{kok:dad_mim_nun}
\kipkokentry{tazyik}{\latupdad\latupye\latupkaf}{K\rom{2}, Eyl.}{kok:dad_ye_kaf}
\kipkokentry{teali}{\latupayn\latuplam\latupvav}{None}{kok:ayn_lam_vav}
\kipkokentry{teâmül}{\latupayn\latupmim\latuplam}{None}{kok:ayn_mim_lam}
\kipkokentry{teâti}{\latupayn\latupta\latupvav}{None}{kok:ayn_ta_vav}
\kipkokentry{teâvün}{\latupayn\latupvav\latupnun}{None}{kok:ayn_vav_nun}
\kipkokentry{tebaa}{\latupte\latupbe\latupayn}{None}{kok:te_be_ayn}
\kipkokentry{tebârüz}{\latupbe\latupre\latupze}{None}{kok:be_re_ze}
\kipkokentry{tebcil}{\latupbe\latupcim\latuplam}{K\rom{2}, Eyl.}{kok:be_cim_lam}
\kipkokentry{tebdil}{\latupbe\latupdal\latuplam}{K\rom{2}, Eyl.}{kok:be_dal_lam}
\kipkokentry{tebeddül}{\latupbe\latupdal\latuplam}{None}{kok:be_dal_lam}
\kipkokentry{tebellüğ}{\latupbe\latuplam\latupgayn}{None}{kok:be_lam_gayn}
\kipkokentry{teberrû}{\latupbe\latupre\latupayn}{None}{kok:be_re_ayn}
\kipkokentry{tebessüm}{\latupbe\latupsin\latupmim}{None}{kok:be_sin_mim}
\kipkokentry{teblîgat}{\latupbe\latuplam\latupgayn}{None}{kok:be_lam_gayn}
\kipkokentry{tebliğ}{\latupbe\latuplam\latupgayn}{K\rom{2}, Eyl.}{kok:be_lam_gayn}
\kipkokentry{tebrik}{\latupbe\latupre\latupkef}{K\rom{2}, Eyl.}{kok:be_re_kef}
\kipkokentry{tebşir}{\latupbe\latupshin\latupre}{K\rom{2}, Eyl.}{kok:be_shin_re}
\kipkokentry{tecâhül}{\latupcim\latuphe\latuplam}{None}{kok:cim_he_lam}
\kipkokentry{tecânüs}{\latupcim\latupnun\latupsin}{None}{kok:cim_nun_sin}
\kipkokentry{tecâvüz}{\latupcim\latupvav\latupze}{None}{kok:cim_vav_ze}
\kipkokentry{tecdit}{\latupcim\latupdal\latupdal$^1$}{K\rom{2}, Eyl.}{kok:cim_dal_dal1}
\kipkokentry{teceddüt}{\latupcim\latupdal\latupdal$^1$}{None}{kok:cim_dal_dal1}
\kipkokentry{tecellî}{\latupcim\latuplam\latupvav}{None}{kok:cim_lam_vav}
\kipkokentry{tecerrüt}{\latupcim\latupre\latupdal}{None}{kok:cim_re_dal}
\kipkokentry{tecessüm}{\latupcim\latupsin\latupmim}{None}{kok:cim_sin_mim}
\kipkokentry{tecessüs}{\latupcim\latupsin\latupsin}{None}{kok:cim_sin_sin}
\kipkokentry{tecezzî}{\latupcim\latupze\latupalif}{None}{kok:cim_ze_alif}
\kipkokentry{têcil}{\latupalif\latupcim\latuplam}{K\rom{2}, Eyl.}{kok:alif_cim_lam}
\kipkokentry{tecrit}{\latupcim\latupre\latupdal}{K\rom{2}, Eyl.}{kok:cim_re_dal}
\kipkokentry{tecrübe}{\latupcim\latupre\latupbe}{None}{kok:cim_re_be}
\kipkokentry{tecvit}{\latupcim\latupvav\latupdal}{K\rom{2}, Eyl.}{kok:cim_vav_dal}
\kipkokentry{tecviz}{\latupcim\latupvav\latupze}{K\rom{2}, Eyl.}{kok:cim_vav_ze}
\kipkokentry{tecziye}{\latupcim\latupze\latupye}{None}{kok:cim_ze_ye}
\kipkokentry{teçhiz}{\latupcim\latuphe\latupze}{K\rom{2}, Eyl.}{kok:cim_he_ze}
\kipkokentry{teçhizat}{\latupcim\latuphe\latupze}{None}{kok:cim_he_ze}
\kipkokentry{tedârik}{\latupdal\latupre\latupkef}{None}{kok:dal_re_kef}
\kipkokentry{tedâvî}{\latupdal\latupvav\latupye}{None}{kok:dal_vav_ye}
\kipkokentry{tedâvül}{\latupdal\latupvav\latuplam}{None}{kok:dal_vav_lam}
\kipkokentry{tedbir}{\latupdal\latupbe\latupre}{K\rom{2}, Eyl.}{kok:dal_be_re}
\kipkokentry{tedhiş}{\latupdal\latuphe\latupshin}{K\rom{2}, Eyl.}{kok:dal_he_shin}
\kipkokentry{têdip}{\latupalif\latupdal\latupbe}{K\rom{2}, Eyl.}{kok:alif_dal_be}
\kipkokentry{tediye}{\latupalif\latupdal\latupye}{None}{kok:alif_dal_ye}
\kipkokentry{tedricen}{\latupdal\latupre\latupcim$^1$}{None}{kok:dal_re_cim1}
\kipkokentry{tedris}{\latupdal\latupre\latupsin}{K\rom{2}, Eyl.}{kok:dal_re_sin}
\kipkokentry{tedrisat}{\latupdal\latupre\latupsin}{None}{kok:dal_re_sin}
\kipkokentry{tedvir}{\latupdal\latupvav\latupre}{K\rom{2}, Eyl.}{kok:dal_vav_re}
\kipkokentry{teehhül}{\latupalif\latuphe\latuplam}{None}{kok:alif_he_lam}
\kipkokentry{teehhür}{\latupalif\latupxa\latupre}{None}{kok:alif_xa_re}
\kipkokentry{teemmül}{\latupalif\latupmim\latuplam}{None}{kok:alif_mim_lam}
\kipkokentry{teenni}{\latupalif\latupnun\latupye}{None}{kok:alif_nun_ye}
\kipkokentry{teessüf}{\latupalif\latupsin\latupfe}{None}{kok:alif_sin_fe}
\kipkokentry{teessür}{\latupalif\latupthe\latupre}{None}{kok:alif_the_re}
\kipkokentry{teessüs}{\latupalif\latupsin\latupsin}{None}{kok:alif_sin_sin}
\kipkokentry{tefe}{\latupdal\latupfe\latupfe}{None}{kok:dal_fe_fe}
\kipkokentry{tefekkür}{\latupfe\latupkef\latupre}{None}{kok:fe_kef_re}
\kipkokentry{teferruat}{\latupfe\latupre\latupayn}{None}{kok:fe_re_ayn}
\kipkokentry{tefessüh}{\latupfe\latupsin\latupxa}{None}{kok:fe_sin_xa}
\kipkokentry{tefhim}{\latupfe\latuphe\latupmim}{K\rom{2}, Eyl.}{kok:fe_he_mim}
\kipkokentry{tefrika}{\latupfe\latupre\latupkaf}{None}{kok:fe_re_kaf}
\kipkokentry{tefrik}{\latupfe\latupre\latupkaf}{K\rom{2}, Eyl.}{kok:fe_re_kaf}
\kipkokentry{tefriş}{\latupfe\latupre\latupshin}{K\rom{2}, Eyl.}{kok:fe_re_shin}
\kipkokentry{tefrit}{\latupfe\latupre\latupta}{K\rom{2}, Eyl.}{kok:fe_re_ta}
\kipkokentry{tefsir}{\latupfe\latupsin\latupre}{K\rom{2}, Eyl.}{kok:fe_sin_re}
\kipkokentry{teftiş}{\latupfe\latupte\latupshin}{K\rom{2}, Eyl.}{kok:fe_te_shin}
\kipkokentry{tefviz}{\latupfe\latupvav\latupdad}{K\rom{2}, Eyl.}{kok:fe_vav_dad}
\kipkokentry{tegâfül}{\latupgayn\latupfe\latuplam}{None}{kok:gayn_fe_lam}
\kipkokentry{tegallüp}{\latupgayn\latuplam\latupbe}{None}{kok:gayn_lam_be}
\kipkokentry{tegannî}{\latupgayn\latupnun\latupye$^2$}{None}{kok:gayn_nun_ye2}
\kipkokentry{tehdit}{\latuphe\latupdal\latupdal}{K\rom{2}, Eyl.}{kok:he_dal_dal}
\kipkokentry{tehevvür}{\latuphe\latupvav\latupre}{None}{kok:he_vav_re}
\kipkokentry{têhir}{\latupalif\latupxa\latupre}{K\rom{2}, Eyl.}{kok:alif_xa_re}
\kipkokentry{tehlike}{\latuphe\latuplam\latupkaf}{None}{kok:he_lam_kaf}
\kipkokentry{tekâbül}{\latupkaf\latupbe\latuplam}{None}{kok:kaf_be_lam}
\kipkokentry{tekâlif}{\latupkef\latuplam\latupfe}{None}{kok:kef_lam_fe}
\kipkokentry{tekâmül}{\latupkef\latupmim\latuplam}{None}{kok:kef_mim_lam}
\kipkokentry{tekâsüf}{\latupkef\latupthe\latupfe}{None}{kok:kef_the_fe}
\kipkokentry{tekâüt}{\latupkaf\latupayn\latupdal}{None}{kok:kaf_ayn_dal}
\kipkokentry{tekbir}{\latupkef\latupbe\latupre}{K\rom{2}, Eyl.}{kok:kef_be_re}
\kipkokentry{tekdir}{\latupkef\latupdal\latupre}{K\rom{2}, Eyl.}{kok:kef_dal_re}
\kipkokentry{tekebbür}{\latupkef\latupbe\latupre}{None}{kok:kef_be_re}
\kipkokentry{tekeffül}{\latupkef\latupfe\latuplam}{None}{kok:kef_fe_lam}
\kipkokentry{tekellüf}{\latupkef\latuplam\latupfe}{None}{kok:kef_lam_fe}
\kipkokentry{tekellüm}{\latupkef\latuplam\latupmim}{None}{kok:kef_lam_mim}
\kipkokentry{tekemmül}{\latupkef\latupmim\latuplam}{None}{kok:kef_mim_lam}
\kipkokentry{tekerrür}{\latupkef\latupre\latupre}{None}{kok:kef_re_re}
\kipkokentry{tekevvün}{\latupkef\latupvav\latupnun}{None}{kok:kef_vav_nun}
\kipkokentry{tekfin}{\latupkef\latupfe\latupnun}{K\rom{2}, Eyl.}{kok:kef_fe_nun}
\kipkokentry{tekfir}{\latupkef\latupfe\latupre}{K\rom{2}, Eyl.}{kok:kef_fe_re}
\kipkokentry{têkit}{\latupalif\latupkef\latupdal}{K\rom{2}, Eyl.}{kok:alif_kef_dal}
\kipkokentry{tekke}{\latupvav\latupkef\latupalif}{None}{kok:vav_kef_alif}
\kipkokentry{teklif}{\latupkef\latuplam\latupfe}{K\rom{2}, Eyl.}{kok:kef_lam_fe}
\kipkokentry{tekmil}{\latupkef\latupmim\latuplam}{K\rom{2}, Eyl.}{kok:kef_mim_lam}
\kipkokentry{tekrar}{\latupkef\latupre\latupre}{None}{kok:kef_re_re}
\kipkokentry{tekris}{\latupkef\latupre\latupsin}{K\rom{2}, Eyl.}{kok:kef_re_sin}
\kipkokentry{teksif}{\latupkef\latupthe\latupfe}{K\rom{2}, Eyl.}{kok:kef_the_fe}
\kipkokentry{teksir}{\latupkef\latupthe\latupre}{K\rom{2}, Eyl.}{kok:kef_the_re}
\kipkokentry{tekvin}{\latupkef\latupvav\latupnun}{K\rom{2}, Eyl.}{kok:kef_vav_nun}
\kipkokentry{tekzip}{\latupkef\latupzel\latupbe}{K\rom{2}, Eyl.}{kok:kef_zel_be}
\kipkokentry{telaffuz}{\latuplam\latupfe\latupza}{None}{kok:lam_fe_za}
\kipkokentry{telâfi}{\latuplam\latupfe\latupvav}{None}{kok:lam_fe_vav}
\kipkokentry{telakkî}{\latuplam\latupkaf\latupye}{None}{kok:lam_kaf_ye}
\kipkokentry{telaş}{\latuplam\latupshin\latupye}{None}{kok:lam_shin_ye}
\kipkokentry{telef}{\latupte\latuplam\latupfe}{None}{kok:te_lam_fe}
\kipkokentry{telezzüz}{\latuplam\latupzel\latupzel}{None}{kok:lam_zel_zel}
\kipkokentry{telif}{\latupalif\latuplam\latupfe}{K\rom{2}, Eyl.}{kok:alif_lam_fe}
\kipkokentry{telin}{\latuplam\latupayn\latupnun}{K\rom{2}, Eyl.}{kok:lam_ayn_nun}
\kipkokentry{telkin}{\latuplam\latupkaf\latupnun}{K\rom{2}, Eyl.}{kok:lam_kaf_nun}
\kipkokentry{tellak}{\latupdal\latuplam\latupkaf}{None}{kok:dal_lam_kaf}
\kipkokentry{tellal}{\latupdal\latuplam\latuplam}{None}{kok:dal_lam_lam}
\kipkokentry{telmih}{\latuplam\latupmim\latupha}{K\rom{2}, Eyl.}{kok:lam_mim_ha}
\kipkokentry{temâdi}{\latupmim\latupdal\latupye}{None}{kok:mim_dal_ye}
\kipkokentry{temas}{\latupmim\latupsin\latupsin}{None}{kok:mim_sin_sin}
\kipkokentry{temâşâ}{\latupmim\latupshin\latupye}{None}{kok:mim_shin_ye}
\kipkokentry{temâyül}{\latupmim\latupye\latuplam}{None}{kok:mim_ye_lam}
\kipkokentry{temâyüz}{\latupmim\latupye\latupze}{None}{kok:mim_ye_ze}
\kipkokentry{tembih}{\latupnun\latupbe\latuphe}{K\rom{2}, Eyl.}{kok:nun_be_he}
\kipkokentry{temcit}{\latupmim\latupcim\latupdal}{K\rom{2}, Eyl.}{kok:mim_cim_dal}
\kipkokentry{temdit}{\latupmim\latupdal\latupdal$^1$}{K\rom{2}, Eyl.}{kok:mim_dal_dal1}
\kipkokentry{temeddün}{\latupmim\latupdal\latupnun}{None}{kok:mim_dal_nun}
\kipkokentry{temellük}{\latupmim\latuplam\latupkef}{None}{kok:mim_lam_kef}
\kipkokentry{temennî}{\latupmim\latupnun\latupye$^1$}{None}{kok:mim_nun_ye1}
\kipkokentry{temerküz}{\latupmim\latupre\latupkef\latupze}{None}{kok:mim_re_kef_ze}
\kipkokentry{temerrüt}{\latupmim\latupre\latupdal}{None}{kok:mim_re_dal}
\kipkokentry{temessül}{\latupmim\latupthe\latuplam}{None}{kok:mim_the_lam}
\kipkokentry{temettü}{\latupmim\latupte\latupayn}{None}{kok:mim_te_ayn}
\kipkokentry{têmin}{\latupalif\latupmim\latupnun}{K\rom{2}, Eyl.}{kok:alif_mim_nun}
\kipkokentry{temiz}{\latupmim\latupye\latupze}{K\rom{2}, Eyl.}{kok:mim_ye_ze}
\kipkokentry{temkin}{\latupmim\latupkef\latupnun}{K\rom{2}, Eyl.}{kok:mim_kef_nun}
\kipkokentry{temlik}{\latupmim\latuplam\latupkef}{K\rom{2}, Eyl.}{kok:mim_lam_kef}
\kipkokentry{temmet}{\latupte\latupmim\latupmim}{None}{kok:te_mim_mim}
\kipkokentry{temrin}{\latupmim\latupre\latupnun}{K\rom{2}, Eyl.}{kok:mim_re_nun}
\kipkokentry{temsil}{\latupmim\latupthe\latuplam}{K\rom{2}, Eyl.}{kok:mim_the_lam}
\kipkokentry{temyiz}{\latupmim\latupye\latupze}{K\rom{2}, Eyl.}{kok:mim_ye_ze}
\kipkokentry{tenâkuz}{\latupnun\latupkaf\latupdad}{None}{kok:nun_kaf_dad}
\kipkokentry{tenâsüh}{\latupnun\latupsin\latupxa}{None}{kok:nun_sin_xa}
\kipkokentry{tenâsül}{\latupnun\latupsin\latuplam}{None}{kok:nun_sin_lam}
\kipkokentry{tenâsüp}{\latupnun\latupsin\latupbe}{None}{kok:nun_sin_be}
\kipkokentry{teneffüs}{\latupnun\latupfe\latupsin}{None}{kok:nun_fe_sin}
\kipkokentry{tenevvür}{\latupnun\latupvav\latupre}{None}{kok:nun_vav_re}
\kipkokentry{tenezzüh}{\latupnun\latupze\latuphe}{None}{kok:nun_ze_he}
\kipkokentry{tenezzül}{\latupnun\latupze\latuplam}{None}{kok:nun_ze_lam}
\kipkokentry{tenfiz}{\latupnun\latupfe\latupzel}{K\rom{2}, Eyl.}{kok:nun_fe_zel}
\kipkokentry{tenfiz}{\latupnun\latupfe\latupdad}{K\rom{2}, Eyl.}{kok:nun_fe_dad}
\kipkokentry{tenkil}{\latupnun\latupkef\latuplam}{K\rom{2}, Eyl.}{kok:nun_kef_lam}
\kipkokentry{tenkis}{\latupnun\latupkaf\latupsad}{K\rom{2}, Eyl.}{kok:nun_kaf_sad}
\kipkokentry{tenkit}{\latupnun\latupkaf\latupdal}{K\rom{2}, Eyl.}{kok:nun_kaf_dal}
\kipkokentry{tensikat}{\latupnun\latupsin\latupkaf}{None}{kok:nun_sin_kaf}
\kipkokentry{tensip}{\latupnun\latupsin\latupbe}{K\rom{2}, Eyl.}{kok:nun_sin_be}
\kipkokentry{tenvir}{\latupnun\latupvav\latupre}{K\rom{2}, Eyl.}{kok:nun_vav_re}
\kipkokentry{tenzih}{\latupnun\latupze\latuphe}{K\rom{2}, Eyl.}{kok:nun_ze_he}
\kipkokentry{tenzil}{\latupnun\latupze\latuplam}{K\rom{2}, Eyl.}{kok:nun_ze_lam}
\kipkokentry{terakki}{\latupre\latupkaf\latupye}{None}{kok:re_kaf_ye}
\kipkokentry{terâvih}{\latupre\latupvav\latupha}{None}{kok:re_vav_ha}
\kipkokentry{terbiye}{\latupre\latupbe\latupvav}{None}{kok:re_be_vav}
\kipkokentry{tercih}{\latupre\latupcim\latupha}{K\rom{2}, Eyl.}{kok:re_cim_ha}
\kipkokentry{tercii}{\latupre\latupcim\latupayn}{K\rom{2}, Eyl.}{kok:re_cim_ayn}
\kipkokentry{tercüman}{\latupte\latupre\latupcim\latupmim}{None}{kok:te_re_cim_mim}
\kipkokentry{tercüme}{\latupte\latupre\latupcim\latupmim}{None}{kok:te_re_cim_mim}
\kipkokentry{tereddî}{\latupre\latupdal\latupye}{None}{kok:re_dal_ye}
\kipkokentry{tereddüt}{\latupre\latupdal\latupdal}{None}{kok:re_dal_dal}
\kipkokentry{tereke}{\latupte\latupre\latupkef}{None}{kok:te_re_kef}
\kipkokentry{terennüm}{\latupre\latupnun\latupmim}{None}{kok:re_nun_mim}
\kipkokentry{terettüp}{\latupre\latupte\latupbe}{None}{kok:re_te_be}
\kipkokentry{terfî}{\latupre\latupfe\latupayn}{K\rom{2}, Eyl.}{kok:re_fe_ayn}
\kipkokentry{terhin}{\latupre\latuphe\latupnun}{K\rom{2}, Eyl.}{kok:re_he_nun}
\kipkokentry{terhis}{\latupre\latupxa\latupsad}{K\rom{2}, Eyl.}{kok:re_xa_sad}
\kipkokentry{terk}{\latupte\latupre\latupkef}{None}{kok:te_re_kef}
\kipkokentry{terkin}{\latupre\latupkaf\latupnun}{K\rom{2}, Eyl.}{kok:re_kaf_nun}
\kipkokentry{terkip}{\latupre\latupkef\latupbe}{K\rom{2}, Eyl.}{kok:re_kef_be}
\kipkokentry{tertip}{\latupre\latupte\latupbe}{K\rom{2}, Eyl.}{kok:re_te_be}
\kipkokentry{terviç}{\latupre\latupvav\latupcim}{K\rom{2}, Eyl.}{kok:re_vav_cim}
\kipkokentry{terziye}{\latupre\latupdad\latupvav}{None}{kok:re_dad_vav}
\kipkokentry{tesâdüf}{\latupsad\latupdal\latupfe$^1$}{None}{kok:sad_dal_fe1}
\kipkokentry{tesânüt}{\latupsin\latupnun\latupdal}{None}{kok:sin_nun_dal}
\kipkokentry{tescil}{\latupsin\latupcim\latuplam}{K\rom{2}, Eyl.}{kok:sin_cim_lam}
\kipkokentry{tesellî}{\latupsin\latuplam\latupvav}{None}{kok:sin_lam_vav}
\kipkokentry{tesellüm}{\latupsin\latuplam\latupmim}{None}{kok:sin_lam_mim}
\kipkokentry{teselsül}{\latupsin\latuplam}{None}{kok:sin_lam}
\kipkokentry{tesettür}{\latupsin\latupte\latupre}{None}{kok:sin_te_re}
\kipkokentry{teshil}{\latupsin\latuphe\latuplam}{K\rom{2}, Eyl.}{kok:sin_he_lam}
\kipkokentry{teshir}{\latupsin\latupha\latupre$^2$}{K\rom{2}, Eyl.}{kok:sin_ha_re2}
\kipkokentry{têsir}{\latupalif\latupthe\latupre}{K\rom{2}, Eyl.}{kok:alif_the_re}
\kipkokentry{têsis}{\latupalif\latupsin\latupsin}{K\rom{2}, Eyl.}{kok:alif_sin_sin}
\kipkokentry{teskin}{\latupsin\latupkef\latupnun}{K\rom{2}, Eyl.}{kok:sin_kef_nun}
\kipkokentry{teslim}{\latupsin\latuplam\latupmim}{K\rom{2}, Eyl.}{kok:sin_lam_mim}
\kipkokentry{teslis}{\latupthe\latuplam\latupthe}{K\rom{2}, Eyl.}{kok:the_lam_the}
\kipkokentry{tesmiye}{\latupsin\latupmim\latupye}{None}{kok:sin_mim_ye}
\kipkokentry{tesniye}{\latupthe\latupnun\latupye}{None}{kok:the_nun_ye}
\kipkokentry{tespih}{\latupsin\latupbe\latupha}{K\rom{2}, Eyl.}{kok:sin_be_ha}
\kipkokentry{tespit}{\latupthe\latupbe\latupte}{K\rom{2}, Eyl.}{kok:the_be_te}
\kipkokentry{tesviye}{\latupsin\latupvav\latupye}{None}{kok:sin_vav_ye}
\kipkokentry{teşbih}{\latupshin\latupbe\latupha}{K\rom{2}, Eyl.}{kok:shin_be_ha}
\kipkokentry{teşcî}{\latupshin\latupcim\latupayn}{K\rom{2}, Eyl.}{kok:shin_cim_ayn}
\kipkokentry{teşebbüs}{\latupshin\latupbe\latupthe}{None}{kok:shin_be_the}
\kipkokentry{teşekkül}{\latupshin\latupkef\latuplam$^2$}{None}{kok:shin_kef_lam2}
\kipkokentry{teşekkür}{\latupshin\latupkef\latupre}{None}{kok:shin_kef_re}
\kipkokentry{teşerrüf}{\latupshin\latupre\latupfe}{None}{kok:shin_re_fe}
\kipkokentry{teşevvüş}{\latupshin\latupvav\latupshin}{None}{kok:shin_vav_shin}
\kipkokentry{teşhir}{\latupshin\latuphe\latupre}{K\rom{2}, Eyl.}{kok:shin_he_re}
\kipkokentry{teşhis}{\latupshin\latupxa\latupsad}{K\rom{2}, Eyl.}{kok:shin_xa_sad}
\kipkokentry{teşkil}{\latupshin\latupkef\latuplam$^2$}{K\rom{2}, Eyl.}{kok:shin_kef_lam2}
\kipkokentry{teşmil}{\latupshin\latupmim\latuplam}{K\rom{2}, Eyl.}{kok:shin_mim_lam}
\kipkokentry{teşrî}{\latupshin\latupre\latupayn}{K\rom{2}, Eyl.}{kok:shin_re_ayn}
\kipkokentry{teşrif}{\latupshin\latupre\latupfe}{K\rom{2}, Eyl.}{kok:shin_re_fe}
\kipkokentry{teşrih}{\latupshin\latupre\latupha}{K\rom{2}, Eyl.}{kok:shin_re_ha}
\kipkokentry{teşrik}{\latupshin\latupre\latupkef}{K\rom{2}, Eyl.}{kok:shin_re_kef}
\kipkokentry{teşvik}{\latupshin\latupvav\latupkaf}{K\rom{2}, Eyl.}{kok:shin_vav_kaf}
\kipkokentry{teşviş}{\latupshin\latupvav\latupshin}{K\rom{2}, Eyl.}{kok:shin_vav_shin}
\kipkokentry{teşyî}{\latupshin\latupye\latupayn}{K\rom{2}, Eyl.}{kok:shin_ye_ayn}
\kipkokentry{tetebbû}{\latupte\latupbe\latupayn}{None}{kok:te_be_ayn}
\kipkokentry{tetkik}{\latupdal\latupkaf\latupkaf}{K\rom{2}, Eyl.}{kok:dal_kaf_kaf}
\kipkokentry{tevâfuk}{\latupvav\latupfe\latupkaf}{None}{kok:vav_fe_kaf}
\kipkokentry{tevakkuf}{\latupvav\latupkaf\latupfe}{None}{kok:vav_kaf_fe}
\kipkokentry{tevârih}{\latupvav\latupre\latupxa}{None}{kok:vav_re_xa}
\kipkokentry{tevârüs}{\latupvav\latupre\latupthe}{None}{kok:vav_re_the}
\kipkokentry{tevâtür}{\latupvav\latupte\latupre}{None}{kok:vav_te_re}
\kipkokentry{tevâzu}{\latupvav\latupdad\latupayn}{None}{kok:vav_dad_ayn}
\kipkokentry{tevcih}{\latupvav\latupcim\latuphe}{K\rom{2}, Eyl.}{kok:vav_cim_he}
\kipkokentry{tevdî}{\latupvav\latupdal\latupayn}{K\rom{2}, Eyl.}{kok:vav_dal_ayn}
\kipkokentry{teveccüh}{\latupvav\latupcim\latuphe}{None}{kok:vav_cim_he}
\kipkokentry{tevekkeli}{\latupvav\latupkef\latuplam}{None}{kok:vav_kef_lam}
\kipkokentry{tevekkül}{\latupvav\latupkef\latuplam}{None}{kok:vav_kef_lam}
\kipkokentry{tevellüt}{\latupvav\latuplam\latupdal}{None}{kok:vav_lam_dal}
\kipkokentry{tevessû}{\latupvav\latupsin\latupayn}{None}{kok:vav_sin_ayn}
\kipkokentry{tevessül}{\latupvav\latupsin\latuplam}{None}{kok:vav_sin_lam}
\kipkokentry{tevhid}{\latupvav\latupha\latupdal}{K\rom{2}, Eyl.}{kok:vav_ha_dal}
\kipkokentry{têvil}{\latupalif\latupvav\latuplam}{K\rom{2}, Eyl.}{kok:alif_vav_lam}
\kipkokentry{tevkif}{\latupvav\latupkaf\latupfe}{K\rom{2}, Eyl.}{kok:vav_kaf_fe}
\kipkokentry{tevkil}{\latupvav\latupkef\latuplam}{K\rom{2}, Eyl.}{kok:vav_kef_lam}
\kipkokentry{tevlit}{\latupvav\latuplam\latupdal}{K\rom{2}, Eyl.}{kok:vav_lam_dal}
\kipkokentry{tevriye}{\latupvav\latupre\latupye}{None}{kok:vav_re_ye}
\kipkokentry{tevsî}{\latupvav\latupsin\latupayn}{K\rom{2}, Eyl.}{kok:vav_sin_ayn}
\kipkokentry{tevsik}{\latupvav\latupthe\latupkaf}{K\rom{2}, Eyl.}{kok:vav_the_kaf}
\kipkokentry{tevzî}{\latupvav\latupze\latupayn}{K\rom{2}, Eyl.}{kok:vav_ze_ayn}
\kipkokentry{teyakkuz}{\latupye\latupkaf\latupza}{None}{kok:ye_kaf_za}
\kipkokentry{teyemmüm}{\latupye\latupmim\latupmim}{None}{kok:ye_mim_mim}
\kipkokentry{teyit}{\latupalif\latupye\latupdal}{K\rom{2}, Eyl.}{kok:alif_ye_dal}
\kipkokentry{tezâhür}{\latupza\latuphe\latupre}{None}{kok:za_he_re}
\kipkokentry{tezat}{\latupdad\latupdal\latupdal}{None}{kok:dad_dal_dal}
\kipkokentry{tezekkür}{\latupzel\latupkef\latupre$^1$}{None}{kok:zel_kef_re1}
\kipkokentry{tezhip}{\latupzel\latuphe\latupbe$^2$}{K\rom{2}, Eyl.}{kok:zel_he_be2}
\kipkokentry{tezkere}{\latupzel\latupkef\latupre$^1$}{None}{kok:zel_kef_re1}
\kipkokentry{tezkiye}{\latupze\latupkaf\latupvav}{None}{kok:ze_kaf_vav}
\kipkokentry{tezvirat}{\latupze\latupvav\latupre}{None}{kok:ze_vav_re}
\kipkokentry{tezyif}{\latupze\latupye\latupfe}{K\rom{2}, Eyl.}{kok:ze_ye_fe}
\kipkokentry{tezyin}{\latupze\latupye\latupnun}{K\rom{2}, Eyl.}{kok:ze_ye_nun}
\kipkokentry{tezyit}{\latupze\latupye\latupdal}{K\rom{2}, Eyl.}{kok:ze_ye_dal}
\kipkokentry{tıfıl}{\latupta\latupfe\latuplam}{None}{kok:ta_fe_lam}
\kipkokentry{tıp}{\latupta\latupbe\latupbe}{None}{kok:ta_be_be}
\kipkokentry{tıpkı}{\latupta\latupbe\latupkaf}{None}{kok:ta_be_kaf}
\kipkokentry{tıynet}{\latupta\latupye\latupnun}{None}{kok:ta_ye_nun}
\kipkokentry{ticâret}{\latupte\latupcim\latupre}{None}{kok:te_cim_re}
\kipkokentry{tiftik}{\latupfe\latupte\latupkef}{K\rom{2}, Eyl.}{kok:fe_te_kef}
\kipkokentry{tilâvet}{\latupte\latuplam\latupvav}{None}{kok:te_lam_vav}
\kipkokentry{timsal}{\latupmim\latupthe\latuplam}{None}{kok:mim_the_lam}
\kipkokentry{tirit}{\latupthe\latupre\latupdal}{None}{kok:the_re_dal}
\kipkokentry{tortu}{\latupdal\latupre\latupdal}{None}{kok:dal_re_dal}
\kipkokentry{töhmet}{\latupvav\latuphe\latupmim}{None}{kok:vav_he_mim}
\kipkokentry{tövbe}{\latupte\latupvav\latupbe}{None}{kok:te_vav_be}
\kipkokentry{tûbâ}{\latupta\latupye\latupbe}{None}{kok:ta_ye_be}
\kipkokentry{tûfan}{\latupta\latupvav\latupfe}{None}{kok:ta_vav_fe}
\kipkokentry{tufeylî}{\latupta\latupfe\latuplam}{None}{kok:ta_fe_lam}
\kipkokentry{tuğyan}{\latupta\latupgayn\latupye}{None}{kok:ta_gayn_ye}
\kipkokentry{tuhaf}{\latupte\latupha\latupfe}{None}{kok:te_ha_fe}
\kipkokentry{tuhafiye}{\latupte\latupha\latupfe}{None}{kok:te_ha_fe}
\kipkokentry{tûl}{\latupta\latupvav\latuplam}{None}{kok:ta_vav_lam}
\kipkokentry{tuluat}{\latupta\latuplam\latupayn}{None}{kok:ta_lam_ayn}
\kipkokentry{tüccar}{\latupte\latupcim\latupre}{None}{kok:te_cim_re}
\kipkokentry{türap}{\latupte\latupre\latupbe}{None}{kok:te_re_be}
\kipkokentry{türbe}{\latupte\latupre\latupbe}{None}{kok:te_re_be}
\end{multicols}
\dictchapter{U}
\begin{multicols}{2}
\kipkokentry{ubudiyet}{\latupayn\latupbe\latupdal}{None}{kok:ayn_be_dal}
\kipkokentry{ucûbe}{\latupayn\latupcim\latupbe}{None}{kok:ayn_cim_be}
\kipkokentry{ufuk}{\latupalif\latupfe\latupkaf}{None}{kok:alif_fe_kaf}
\kipkokentry{ufûnet}{\latupayn\latupfe\latupnun}{None}{kok:ayn_fe_nun}
\kipkokentry{uhde}{\latupayn\latuphe\latupdal}{None}{kok:ayn_he_dal}
\kipkokentry{uhrevî}{\latupalif\latupxa\latupre}{None}{kok:alif_xa_re}
\kipkokentry{uhuvvet}{\latupalif\latupxa\latupvav}{None}{kok:alif_xa_vav}
\kipkokentry{ukalâ}{\latupayn\latupkaf\latuplam}{None}{kok:ayn_kaf_lam}
\kipkokentry{ukbâ}{\latupayn\latupkaf\latupbe}{None}{kok:ayn_kaf_be}
\kipkokentry{ukde}{\latupayn\latupkaf\latupdal}{None}{kok:ayn_kaf_dal}
\kipkokentry{ulemâ}{\latupayn\latuplam\latupmim}{None}{kok:ayn_lam_mim}
\kipkokentry{ulûfe}{\latupayn\latuplam\latupfe}{None}{kok:ayn_lam_fe}
\kipkokentry{ulûhiyet}{\latupalif\latuplam\latuphe}{None}{kok:alif_lam_he}
\kipkokentry{ulum}{\latupayn\latuplam\latupmim}{None}{kok:ayn_lam_mim}
\kipkokentry{ulvî}{\latupayn\latuplam\latupvav}{None}{kok:ayn_lam_vav}
\kipkokentry{ulyâ}{\latupayn\latuplam\latupvav}{None}{kok:ayn_lam_vav}
\kipkokentry{umde}{\latupayn\latupmim\latupdal}{None}{kok:ayn_mim_dal}
\kipkokentry{umran}{\latupayn\latupmim\latupre}{None}{kok:ayn_mim_re}
\kipkokentry{umre}{\latupayn\latupmim\latupre}{None}{kok:ayn_mim_re}
\kipkokentry{umum}{\latupayn\latupmim\latupmim}{None}{kok:ayn_mim_mim}
\kipkokentry{umur}{\latupalif\latupmim\latupre}{None}{kok:alif_mim_re}
\kipkokentry{unsur}{\latupayn\latupnun\latupsad\latupre}{None}{kok:ayn_nun_sad_re}
\kipkokentry{usâre}{\latupayn\latupsad\latupre$^2$}{None}{kok:ayn_sad_re2}
\kipkokentry{usul}{\latupalif\latupsad\latuplam}{None}{kok:alif_sad_lam}
\kipkokentry{uşşak}{\latupayn\latupshin\latupkaf}{None}{kok:ayn_shin_kaf}
\kipkokentry{uzlet}{\latupayn\latupze\latuplam}{None}{kok:ayn_ze_lam}
\kipkokentry{uzuv}{\latupayn\latupdad\latupvav}{None}{kok:ayn_dad_vav}
\end{multicols}
\dictchapter{Ü}
\begin{multicols}{2}
\kipkokentry{ücret}{\latupalif\latupcim\latupre}{None}{kok:alif_cim_re}
\kipkokentry{ülfet}{\latupalif\latuplam\latupfe}{None}{kok:alif_lam_fe}
\kipkokentry{ümerâ}{\latupalif\latupmim\latupre}{None}{kok:alif_mim_re}
\kipkokentry{ümmet}{\latupalif\latupmim\latupmim}{None}{kok:alif_mim_mim}
\kipkokentry{ümmî}{\latupalif\latupmim\latupmim}{None}{kok:alif_mim_mim}
\kipkokentry{ünsiyet}{\latupalif\latupnun\latupsin}{None}{kok:alif_nun_sin}
\kipkokentry{üryan}{\latupayn\latupre\latupye}{None}{kok:ayn_re_ye}
\kipkokentry{üs}{\latupalif\latupsin\latupsin}{None}{kok:alif_sin_sin}
\end{multicols}
\dictchapter{V}
\begin{multicols}{2}
\kipkokentry{vaat}{\latupvav\latupayn\latupdal}{None}{kok:vav_ayn_dal}
\kipkokentry{vaaz}{\latupvav\latupayn\latupza}{None}{kok:vav_ayn_za}
\kipkokentry{vâcip}{\latupvav\latupcim\latupbe}{K\rom{1}, Ed.}{kok:vav_cim_be}
\kipkokentry{vâde}{\latupvav\latupayn\latupdal}{None}{kok:vav_ayn_dal}
\kipkokentry{vâdi}{\latupvav\latupdal\latupye}{K\rom{1}, Ed.}{kok:vav_dal_ye}
\kipkokentry{vahâmet}{\latupvav\latupxa\latupmim}{None}{kok:vav_xa_mim}
\kipkokentry{vahdet}{\latupvav\latupha\latupdal}{None}{kok:vav_ha_dal}
\kipkokentry{vahim}{\latupvav\latupxa\latupmim}{None}{kok:vav_xa_mim}
\kipkokentry{vahiy}{\latupvav\latupha\latupye}{None}{kok:vav_ha_ye}
\kipkokentry{vahşî}{\latupvav\latupha\latupshin}{None}{kok:vav_ha_shin}
\kipkokentry{vâiz}{\latupvav\latupayn\latupza}{K\rom{1}, Ed.}{kok:vav_ayn_za}
\kipkokentry{vakâ}{\latupvav\latupkaf\latupayn}{None}{kok:vav_kaf_ayn}
\kipkokentry{vakar}{\latupvav\latupkaf\latupre}{None}{kok:vav_kaf_re}
\kipkokentry{vâkıa}{\latupvav\latupkaf\latupayn}{None}{kok:vav_kaf_ayn}
\kipkokentry{vâkıf}{\latupvav\latupkaf\latupfe}{K\rom{1}, Ed.}{kok:vav_kaf_fe}
\kipkokentry{vakıf}{\latupvav\latupkaf\latupfe}{None}{kok:vav_kaf_fe}
\kipkokentry{vâkî}{\latupvav\latupkaf\latupayn}{K\rom{1}, Ed.}{kok:vav_kaf_ayn}
\kipkokentry{vakit}{\latupvav\latupkaf\latupte}{None}{kok:vav_kaf_te}
\kipkokentry{vâkur}{\latupvav\latupkaf\latupre}{None}{kok:vav_kaf_re}
\kipkokentry{vâli}{\latupvav\latuplam\latupye}{K\rom{1}, Ed.}{kok:vav_lam_ye}
\kipkokentry{vâlide}{\latupvav\latuplam\latupdal}{None}{kok:vav_lam_dal}
\kipkokentry{varaka}{\latupvav\latupre\latupkaf}{None}{kok:vav_re_kaf}
\kipkokentry{varak}{\latupvav\latupre\latupkaf}{None}{kok:vav_re_kaf}
\kipkokentry{vâridat}{\latupvav\latupre\latupdal}{None}{kok:vav_re_dal}
\kipkokentry{vâris}{\latupvav\latupre\latupthe}{K\rom{1}, Ed.}{kok:vav_re_the}
\kipkokentry{vârit}{\latupvav\latupre\latupdal}{K\rom{1}, Ed.}{kok:vav_re_dal}
\kipkokentry{varta}{\latupvav\latupre\latupta}{None}{kok:vav_re_ta}
\kipkokentry{vasat}{\latupvav\latupsin\latupta}{None}{kok:vav_sin_ta}
\kipkokentry{vasıf}{\latupvav\latupsad\latupfe}{None}{kok:vav_sad_fe}
\kipkokentry{vâsıl}{\latupvav\latupsad\latuplam}{K\rom{1}, Ed.}{kok:vav_sad_lam}
\kipkokentry{vâsıta}{\latupvav\latupsin\latupta}{None}{kok:vav_sin_ta}
\kipkokentry{vasî}{\latupvav\latupsin\latupayn}{None}{kok:vav_sin_ayn}
\kipkokentry{vâsî}{\latupvav\latupsad\latupye}{K\rom{1}, Ed.}{kok:vav_sad_ye}
\kipkokentry{vasiyet}{\latupvav\latupsad\latupye}{None}{kok:vav_sad_ye}
\kipkokentry{vatan}{\latupvav\latupta\latupnun}{None}{kok:vav_ta_nun}
\kipkokentry{vaz}{\latupvav\latupdad\latupayn}{None}{kok:vav_dad_ayn}
\kipkokentry{vâzıh}{\latupvav\latupdad\latupha}{K\rom{1}, Ed.}{kok:vav_dad_ha}
\kipkokentry{vazife}{\latupvav\latupza\latupfe}{None}{kok:vav_za_fe}
\kipkokentry{vaziyet}{\latupvav\latupdad\latupayn}{None}{kok:vav_dad_ayn}
\kipkokentry{vebâ}{\latupvav\latupbe\latupalif}{None}{kok:vav_be_alif}
\kipkokentry{vebal}{\latupvav\latupbe\latuplam}{None}{kok:vav_be_lam}
\kipkokentry{vecd}{\latupvav\latupcim\latupdal}{None}{kok:vav_cim_dal}
\kipkokentry{vecîbe}{\latupvav\latupcim\latupbe}{None}{kok:vav_cim_be}
\kipkokentry{vecih}{\latupvav\latupcim\latuphe}{None}{kok:vav_cim_he}
\kipkokentry{veciz}{\latupvav\latupcim\latupze}{None}{kok:vav_cim_ze}
\kipkokentry{veçhe}{\latupvav\latupcim\latuphe}{None}{kok:vav_cim_he}
\kipkokentry{vedâ}{\latupvav\latupdal\latupayn}{None}{kok:vav_dal_ayn}
\kipkokentry{vediâ}{\latupvav\latupdal\latupayn}{None}{kok:vav_dal_ayn}
\kipkokentry{vedut}{\latupvav\latupdal\latupdal}{None}{kok:vav_dal_dal}
\kipkokentry{vefâ}{\latupvav\latupfe\latupye}{None}{kok:vav_fe_ye}
\kipkokentry{vefat}{\latupvav\latupfe\latupye}{None}{kok:vav_fe_ye}
\kipkokentry{vehim}{\latupvav\latuphe\latupmim}{None}{kok:vav_he_mim}
\kipkokentry{vekâlet}{\latupvav\latupkef\latuplam}{None}{kok:vav_kef_lam}
\kipkokentry{vekâyî}{\latupvav\latupkaf\latupayn}{None}{kok:vav_kaf_ayn}
\kipkokentry{vekil}{\latupvav\latupkef\latuplam}{None}{kok:vav_kef_lam}
\kipkokentry{velâdet}{\latupvav\latuplam\latupdal}{None}{kok:vav_lam_dal}
\kipkokentry{velâyet}{\latupvav\latuplam\latupye}{None}{kok:vav_lam_ye}
\kipkokentry{velet}{\latupvav\latuplam\latupdal}{None}{kok:vav_lam_dal}
\kipkokentry{veli}{\latupvav\latuplam\latupye}{None}{kok:vav_lam_ye}
\kipkokentry{velut}{\latupvav\latuplam\latupdal}{None}{kok:vav_lam_dal}
\kipkokentry{velvele}{\latupvav\latuplam}{None}{kok:vav_lam}
\kipkokentry{verâset}{\latupvav\latupre\latupthe}{None}{kok:vav_re_the}
\kipkokentry{verem}{\latupvav\latupre\latupmim}{None}{kok:vav_re_mim}
\kipkokentry{verese}{\latupvav\latupre\latupthe}{None}{kok:vav_re_the}
\kipkokentry{vesâik}{\latupvav\latupthe\latupkaf}{None}{kok:vav_the_kaf}
\kipkokentry{vesâit}{\latupvav\latupsin\latupta}{None}{kok:vav_sin_ta}
\kipkokentry{vesâyet}{\latupvav\latupsad\latupye}{None}{kok:vav_sad_ye}
\kipkokentry{vesika}{\latupvav\latupthe\latupkaf}{None}{kok:vav_the_kaf}
\kipkokentry{vesîle}{\latupvav\latupsin\latuplam}{None}{kok:vav_sin_lam}
\kipkokentry{vesvese}{\latupvav\latupsin}{None}{kok:vav_sin}
\kipkokentry{vetîre}{\latupvav\latupte\latupre}{None}{kok:vav_te_re}
\kipkokentry{vezâret}{\latupvav\latupze\latupre}{None}{kok:vav_ze_re}
\kipkokentry{vezin}{\latupvav\latupze\latupnun}{None}{kok:vav_ze_nun}
\kipkokentry{vezir}{\latupvav\latupze\latupre}{None}{kok:vav_ze_re}
\kipkokentry{vezne}{\latupvav\latupze\latupnun}{None}{kok:vav_ze_nun}
\kipkokentry{vicâhen}{\latupvav\latupcim\latuphe}{None}{kok:vav_cim_he}
\kipkokentry{vicdan}{\latupvav\latupcim\latupdal}{None}{kok:vav_cim_dal}
\kipkokentry{vikâye}{\latupvav\latupkaf\latupye}{None}{kok:vav_kaf_ye}
\kipkokentry{vilâyet}{\latupvav\latuplam\latupye}{None}{kok:vav_lam_ye}
\kipkokentry{visal}{\latupvav\latupsad\latuplam}{None}{kok:vav_sad_lam}
\kipkokentry{vukû}{\latupvav\latupkaf\latupayn}{None}{kok:vav_kaf_ayn}
\kipkokentry{vukuat}{\latupvav\latupkaf\latupayn}{None}{kok:vav_kaf_ayn}
\kipkokentry{vukuf}{\latupvav\latupkaf\latupfe}{None}{kok:vav_kaf_fe}
\kipkokentry{vuslat}{\latupvav\latupsad\latuplam}{None}{kok:vav_sad_lam}
\kipkokentry{vuzuh}{\latupvav\latupdad\latupha}{None}{kok:vav_dad_ha}
\kipkokentry{vücut}{\latupvav\latupcim\latupdal}{None}{kok:vav_cim_dal}
\kipkokentry{vükelâ}{\latupvav\latupkef\latuplam}{None}{kok:vav_kef_lam}
\kipkokentry{vürut}{\latupvav\latupre\latupdal}{None}{kok:vav_re_dal}
\kipkokentry{vüsat}{\latupvav\latupsin\latupayn}{None}{kok:vav_sin_ayn}
\kipkokentry{vüzerâ}{\latupvav\latupze\latupre}{None}{kok:vav_ze_re}
\end{multicols}
\dictchapter{Y}
\begin{multicols}{2}
\kipkokentry{yakînen}{\latupye\latupkaf\latupnun}{None}{kok:ye_kaf_nun}
\kipkokentry{yâni}{\latupayn\latupnun\latupye}{None}{kok:ayn_nun_ye}
\kipkokentry{yed}{\latupye\latupdal\latupdal}{None}{kok:ye_dal_dal}
\kipkokentry{yeis}{\latupye\latupalif\latupsin}{None}{kok:ye_alif_sin}
\kipkokentry{yekun}{\latupkef\latupvav\latupnun}{None}{kok:kef_vav_nun}
\kipkokentry{yemenî}{\latupye\latupmim\latupnun}{None}{kok:ye_mim_nun}
\kipkokentry{yemin}{\latupye\latupmim\latupnun}{None}{kok:ye_mim_nun}
\kipkokentry{yesâri}{\latupye\latupsin\latupre}{None}{kok:ye_sin_re}
\kipkokentry{yetim}{\latupye\latupte\latupmim}{None}{kok:ye_te_mim}
\kipkokentry{yevm}{\latupye\latupvav\latupmim}{None}{kok:ye_vav_mim}
\kipkokentry{yevmiye}{\latupye\latupvav\latupmim}{None}{kok:ye_vav_mim}
\kipkokentry{yulaf}{\latupayn\latuplam\latupfe}{None}{kok:ayn_lam_fe}
\end{multicols}
\dictchapter{Z}
\begin{multicols}{2}
\kipkokentry{zaaf}{\latupdad\latupayn\latupfe}{None}{kok:dad_ayn_fe}
\kipkokentry{zâbıta}{\latupza\latupbe\latupta}{None}{kok:za_be_ta}
\kipkokentry{zabıt}{\latupza\latupbe\latupta}{None}{kok:za_be_ta}
\kipkokentry{zabit}{\latupza\latupbe\latupta}{K\rom{1}, Ed.}{kok:za_be_ta}
\kipkokentry{zafer}{\latupza\latupfe\latupre}{None}{kok:za_fe_re}
\kipkokentry{zafiyet}{\latupdad\latupayn\latupfe}{None}{kok:dad_ayn_fe}
\kipkokentry{zâhir}{\latupza\latuphe\latupre}{K\rom{1}, Ed.}{kok:za_he_re}
\kipkokentry{zahire}{\latupzel\latupxa\latupre}{None}{kok:zel_xa_re}
\kipkokentry{zâhit}{\latupze\latuphe\latupdal}{K\rom{1}, Ed.}{kok:ze_he_dal}
\kipkokentry{zahmet}{\latupze\latupha\latupmim}{None}{kok:ze_ha_mim}
\kipkokentry{zâil}{\latupze\latupvav\latuplam}{K\rom{1}, Ed.}{kok:ze_vav_lam}
\kipkokentry{zaim}{\latupze\latupayn\latupmim}{None}{kok:ze_ayn_mim}
\kipkokentry{zâit}{\latupze\latupye\latupdal}{K\rom{1}, Ed.}{kok:ze_ye_dal}
\kipkokentry{zâkir}{\latupzel\latupkef\latupre$^1$}{K\rom{1}, Ed.}{kok:zel_kef_re1}
\kipkokentry{zakkum}{\latupze\latupkaf\latupmim}{None}{kok:ze_kaf_mim}
\kipkokentry{zakkum}{\latupze\latupkaf\latupmim}{None}{kok:ze_kaf_mim}
\kipkokentry{zâlim}{\latupza\latuplam\latupmim}{K\rom{1}, Ed.}{kok:za_lam_mim}
\kipkokentry{zam}{\latupdad\latupmim\latupmim}{None}{kok:dad_mim_mim}
\kipkokentry{zamir}{\latupdad\latupmim\latupre}{None}{kok:dad_mim_re}
\kipkokentry{zamk}{\latupsad\latupmim\latupgayn}{None}{kok:sad_mim_gayn}
\kipkokentry{zamme}{\latupdad\latupmim\latupmim}{None}{kok:dad_mim_mim}
\kipkokentry{zan}{\latupza\latupnun}{None}{kok:za_nun}
\kipkokentry{zanaat}{\latupsad\latupnun\latupayn}{None}{kok:sad_nun_ayn}
\kipkokentry{zaptiye}{\latupza\latupbe\latupta}{None}{kok:za_be_ta}
\kipkokentry{zarâfet}{\latupza\latupre\latupfe}{None}{kok:za_re_fe}
\kipkokentry{zarar}{\latupdad\latupre\latupre}{None}{kok:dad_re_re}
\kipkokentry{zarf}{\latupza\latupre\latupfe}{None}{kok:za_re_fe}
\kipkokentry{zarif}{\latupza\latupre\latupfe}{None}{kok:za_re_fe}
\kipkokentry{zaruret}{\latupdad\latupre\latupre}{None}{kok:dad_re_re}
\kipkokentry{zât}{\latupzel\latupvav}{None}{kok:zel_vav}
\kipkokentry{zâten}{\latupzel\latupvav}{None}{kok:zel_vav}
\kipkokentry{zavallı}{\latupze\latupvav\latuplam}{None}{kok:ze_vav_lam}
\kipkokentry{zâviye}{\latupze\latupvav\latupye}{None}{kok:ze_vav_ye}
\kipkokentry{zayıf}{\latupdad\latupayn\latupfe}{None}{kok:dad_ayn_fe}
\kipkokentry{zâyi}{\latupdad\latupye\latupayn}{K\rom{1}, Ed.}{kok:dad_ye_ayn}
\kipkokentry{zeâmet}{\latupze\latupayn\latupmim}{None}{kok:ze_ayn_mim}
\kipkokentry{zebur}{\latupze\latupbe\latupre}{None}{kok:ze_be_re}
\kipkokentry{zecir}{\latupze\latupcim\latupre}{None}{kok:ze_cim_re}
\kipkokentry{zehâp}{\latupzel\latuphe\latupbe$^1$}{None}{kok:zel_he_be1}
\kipkokentry{zehrâ}{\latupze\latuphe\latupre}{None}{kok:ze_he_re}
\kipkokentry{zekâ}{\latupzel\latupkef\latupye}{None}{kok:zel_kef_ye}
\kipkokentry{zekat}{\latupze\latupkaf\latupvav}{None}{kok:ze_kaf_vav}
\kipkokentry{zeker}{\latupzel\latupkef\latupre$^2$}{None}{kok:zel_kef_re2}
\kipkokentry{zekî}{\latupzel\latupkef\latupye}{None}{kok:zel_kef_ye}
\kipkokentry{zelil}{\latupzel\latuplam\latuplam}{None}{kok:zel_lam_lam}
\kipkokentry{zelzele}{\latupze\latuplam}{None}{kok:ze_lam}
\kipkokentry{zem}{\latupzel\latupmim\latupmim}{None}{kok:zel_mim_mim}
\kipkokentry{zemzem}{\latupze\latupmim}{None}{kok:ze_mim}
\kipkokentry{zemzeme}{\latupze\latupmim}{None}{kok:ze_mim}
\kipkokentry{zerk}{\latupze\latupre\latupkaf}{None}{kok:ze_re_kaf}
\kipkokentry{zerre}{\latupzel\latupre\latupre}{None}{kok:zel_re_re}
\kipkokentry{zevâhir}{\latupza\latuphe\latupre}{None}{kok:za_he_re}
\kipkokentry{zeval}{\latupze\latupvav\latuplam}{None}{kok:ze_vav_lam}
\kipkokentry{zevat}{\latupzel\latupvav}{None}{kok:zel_vav}
\kipkokentry{zevce}{\latupze\latupvav\latupcim}{None}{kok:ze_vav_cim}
\kipkokentry{zevç}{\latupze\latupvav\latupcim}{None}{kok:ze_vav_cim}
\kipkokentry{zevk}{\latupzel\latupvav\latupkaf}{None}{kok:zel_vav_kaf}
\kipkokentry{zeyil}{\latupzel\latupye\latuplam}{None}{kok:zel_ye_lam}
\kipkokentry{zeytin}{\latupze\latupye\latupte}{None}{kok:ze_ye_te}
\kipkokentry{zıbın}{\latupze\latupbe\latupnun}{None}{kok:ze_be_nun}
\kipkokentry{zıl}{\latupza\latuplam\latuplam}{None}{kok:za_lam_lam}
\kipkokentry{zımn}{\latupdad\latupmim\latupnun}{None}{kok:dad_mim_nun}
\kipkokentry{zıt}{\latupdad\latupdal\latupdal}{None}{kok:dad_dal_dal}
\kipkokentry{zibil}{\latupze\latupbe\latuplam}{None}{kok:ze_be_lam}
\kipkokentry{zifaf}{\latupze\latupfe\latupfe}{None}{kok:ze_fe_fe}
\kipkokentry{zifir}{\latupzel\latupbe\latupre}{None}{kok:zel_be_re}
\kipkokentry{zihin}{\latupzel\latuphe\latupnun}{None}{kok:zel_he_nun}
\kipkokentry{zihniyet}{\latupzel\latuphe\latupnun}{None}{kok:zel_he_nun}
\kipkokentry{zikir}{\latupzel\latupkef\latupre$^1$}{None}{kok:zel_kef_re1}
\kipkokentry{zillet}{\latupzel\latuplam\latuplam}{None}{kok:zel_lam_lam}
\kipkokentry{zimmet}{\latupzel\latupmim\latupmim}{None}{kok:zel_mim_mim}
\kipkokentry{zimmi}{\latupzel\latupmim\latupmim}{None}{kok:zel_mim_mim}
\kipkokentry{zinâ}{\latupze\latupnun\latupye}{None}{kok:ze_nun_ye}
\kipkokentry{ziraat}{\latupzel\latupre\latupayn}{None}{kok:zel_re_ayn}
\kipkokentry{zirve}{\latupzel\latupre\latupvav}{None}{kok:zel_re_vav}
\kipkokentry{ziyâ}{\latupdad\latupvav\latupalif}{None}{kok:dad_vav_alif}
\kipkokentry{ziyâ}{\latupdad\latupye\latupayn}{None}{kok:dad_ye_ayn}
\kipkokentry{ziyâde}{\latupze\latupye\latupdal}{None}{kok:ze_ye_dal}
\kipkokentry{ziyâfet}{\latupdad\latupye\latupfe}{None}{kok:dad_ye_fe}
\kipkokentry{ziyâret}{\latupze\latupvav\latupre}{None}{kok:ze_vav_re}
\kipkokentry{ziynet}{\latupze\latupye\latupnun}{None}{kok:ze_ye_nun}
\kipkokentry{zuhur}{\latupza\latuphe\latupre}{None}{kok:za_he_re}
\kipkokentry{zulmet}{\latupza\latuplam\latupmim}{None}{kok:za_lam_mim}
\kipkokentry{zulüm}{\latupza\latuplam\latupmim}{None}{kok:za_lam_mim}
\kipkokentry{zübde}{\latupze\latupbe\latupdal}{None}{kok:ze_be_dal}
\kipkokentry{züccâciye}{\latupze\latupcim\latupcim}{None}{kok:ze_cim_cim}
\kipkokentry{zühal}{\latupze\latupha\latuplam}{None}{kok:ze_ha_lam}
\kipkokentry{zührevî}{\latupze\latuphe\latupre}{None}{kok:ze_he_re}
\kipkokentry{züht}{\latupze\latuphe\latupdal}{None}{kok:ze_he_dal}
\kipkokentry{zühul}{\latupzel\latuphe\latuplam}{None}{kok:zel_he_lam}
\kipkokentry{zül}{\latupzel\latuplam\latuplam}{None}{kok:zel_lam_lam}
\kipkokentry{zümre}{\latupze\latupmim\latupre$^2$}{None}{kok:ze_mim_re2}
\kipkokentry{zürefâ}{\latupza\latupre\latupfe}{None}{kok:za_re_fe}
\kipkokentry{zürriyet}{\latupzel\latupre\latupre}{None}{kok:zel_re_re}
\end{multicols}

% \end{multicols}

\restoregeometry
\recalctypearea


% \chapter{Fars Köklü Kelimelerin Kuralları}
\section{Fars Köklü Ekler}





\end{document}
